%! TEX root = 00.tex

It was also shown that it could be made with generation
a number of classes linearly proportionate to~$|A|$.

\subsection*{Partial Binary functions}
The next example elaborated the previous, and showed
how can a partial binary function~$f:R⨉S↛T$
can be encoded for any finite sets~$R,S,T$.

It is shown that for any~$r∈R, s∈S, t∈T$
the statement
\[
  \cc{t tmp = f.r().s();}
\]
will type check if and only if~$f(r,s)=t$.

Again, the number of generated types and their sizes (number of methods)
is linearly proportionate to~$|R|+|S|+|T|$.

\subsubsection*{Compile Time Stack data Structure}
Advancing toward more complicated and infinite structures,
a compile-time stack was introduced, to meet this end,
\Java Generics was employed.
It was shown, that for any finite set~$Γ$ of stack symbol a set of \Java types
could be automatically generated, implementing
operations~\textbf{pop$()$},and~\textbf{push$γ()$} for any~$γ∈Γ$ such that for
any set of operations~\cc{op$₁$…op$ₖ$} the sequence of calls\[
\cc{Stack.empty.op$₁()$….op$ₖ()$;} \] will type check if and only if the stack
was not popped more than it was pushed for any prefix of calls.  A~\textbf{top()}
operation was also showed, that returns the type of the current top of the
stack.

\subsubsection*{Peeping into the stack}
Finally, a pattern-matching like example was shown,
using overloading in \Java to ‟peep” into type arguments of a
generic type.
Given a ‟stack” type with two type parameters: top of the stack element
and rest of the stack, we can create a function that given such stack,
returns type \cc{Peep} that has the type argument of the top of the
input stack, i.e, we ‟peeped” into the stack.
\[
  \begin{array}{*1l}
    \cc{Stack<$γ₁$,Stack<$γ₂$,Empty>> \_1 ;}⏎
    \cc{Peep<$γ₁$,Stack<$γ₁$,Stack<$γ₂$,Empty>> \_2 = peep(x);}
  \end{array}
\]
