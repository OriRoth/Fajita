In previous chapters, we've seen the \RLLp, a jump automata based recognizer
that emulates the classic \LLp. The \RLLp uses the JSM, a special data
structure supporting jump and push operations in constant amount of work, it
also uses some pre-computed tables ($\Function Consolidate(·,·)$,~$\Function
Jumps(·)$,~$\Function Δ(·,·)$…).

In this chapter we will show can such data structure be encoded using the \Java
type checker, and the generics mechanism in particular.

\section{Encoding of the Stack Items}

As we seen in \cref{section:toolkit}, the stack symbols are encoded to types.
The main stack of the \RLLp ($S₀$) uses items as stack symbols, thus,
each grammar item is encoded into a \Java type.

For example, of the grammar rule \[
  \<Definitions> ::= \<Labels> \<Constants> \<Nested>
\]
taken from our running example in \cref{figure:running},
four derived \RLLp items will be encoded
\begin{equation*}
\begin{array}{l}
  \left[\<Definitions>::=·\<Labels> \<Constants> \<Nested>\right]\hfill⏎
  \left[\<Definitions>::= \<Labels>·\<Constants> \<Nested>\right]\hfill⏎
  \left[\<Definitions>::= \<Labels> \<Constants>·\<Nested>\right]\hfill⏎
  \left[\<Definitions>::= \<Labels> \<Constants> \<Nested>·\right]\hfill⏎
\end{array}.
\end{equation*}

The notation we use from now on, is that the name of the class is
the nonterminal at the left-hand side of the item concatenated with the
dot index. In out example the types will be
\begin{figure}[ht]
    \caption[Type encoding of \RLLp items]
      {\label{figure:item-encoding}
      Type encoding of items derived from rule \[
        \<Definitions> ::= \<Labels> \<Constants> \<Nested>
      \] (taken from the \Pascal fragment BNF defined in
      \cref{figure:running}).
      The index of the dot is denoted by the number in the class name.}
    \javaInput[minipage,width=\linewidth,left=-5ex]{pascal.item.encoding.listing}
  \end{figure}

\section{Encoding the JSM}

The \RLLp uses the JSM to support jump operations.
During it's run, the \RLLp uses the JSM's information if~$\Function Δ (·,·)$
determines a jump operation. Observe that this might happen only when the
input symbol is in~$\Function Follow(\<A>)$ set, \<A> being the left-hand
side of the top of the stack item.

For example, when recognizing the \Pascal fragment grammar, if the top of
the stack is e.g., \[
  i = \left[\<Definitions>::= \<Labels> \<Constants> \<Nested>·\right]
\] then for each input symbol from~$\Function Follow(\<Definitions>)$
(that contains \cc{begin} and \cc{procedure}), the \RLLp will
perform a jump operation ( the only exception is when~$\$$ is in the follow
set, which in this case the \RLLp accepts).

Since the JSM's state can only be known at ‟runtime” of the \RLLp
(being compile time of the fluent API), the JSM must be encoded with
\Java generics. Using out observation regarding the~$\Function
Follow(·)$ set, each item \[
  [\<A> ::=α·β]
\] will have~$k$ type parameters,~$k$ being the size of the
corresponding~$\Function Follow(\<A>)$ set - for item~$i$,~$k=1$.

The JSM's hash table is encoded with the type parameters of each type,
when ‟querying” this hash table is done by using the name of this
parameter.
The second operation of the JSM is push operation. It will be discussed in
the next sections.

\section{Encoding \texorpdfstring{Δ}{}}

Recall that~$Δ$ is the main prediction table of the \RLLp.
With the main loop step in \cref{algorithm:rll-parser}, it determines
whether the automaton ‟accept”s, ‟reject”s, ‟jump”s or ‟push”es.

\begin{description}
  \item[‟Reject”] as we seen in the jDPDA encoding (\cref{section:ecoop}), encoding of ‟reject”
operation in case the next input symbol is~$t$ summarizes in \emph{not} adding
any method for~$t$, this way, trying to invoke~$t$'s method will cause a
type-check error.

\item[‟Accept”] operation can only occur if the end-of-input symbol,~$\$$, was
  seen. Encoding ‟accept” operation then, is by adding method~$\$()$ to an
  item that can be followed by~$\$$ (there is more to this issue than meets the
  eye, and we will revise it later in
  \cref{section:endable}).

\item[‟Jump”] encoding a ‟jump” operation upon seeing input symbol~$t$ means
  popping multiple times from the stack and reverting to a former state.

  This operation is enabled by the JSM. Querying it is done by using the
  correct type parameter. Performing a~$\Function jump(t)$ operation is simply
  returning~$t$'s type parameter. The actual parameter will be the state of
  the JSM after a jump operation.

  \cref{figure:jump-encoding} shows an example of jump operations with our
  \Pascal fragment grammar example.
  \begin{figure}[ht]
    \caption[Type encoding of jump operations]
      {\label{figure:jump-encoding}
      Type encoding example of jump operations where \[
        i = [\<Constant> ::= \;\cc{;} \;· \;]
      \] is the item at top of the stack ( taken from the \Pascal
      fragment BNF defined in \cref{figure:running}).
      Methods \cc{semi\_t()}, \cc{begin\_t()} and \cc{procedure\_t()}
      represent the terminals ‟\cc{;}”, ‟\cc{begin}” and
      ‟\cc{procedure}” respectively.}
    \javaInput[minipage,width=\linewidth,left=-5ex]{pascal.jump.encoding.listing}
  \end{figure}

Item~$i$ in \cref{figure:jump-encoding} is ready to reduce,
thus,~$\FunctionΔ(i,t) = \textrm{jump}(t)$ for every terminal~$t$
from\[
  \Function Follow(\<Constant>) = ❴ \cc{;},\cc{procedure},\cc{begin}❵.
\]
In all of these cases, the implementation of the ‟jump” operation with \Java
    generics is expressed by the return type of the methods, specifically,
  the corresponding type argument will be returned.

\item[‟Push”] lastly, the ‟push” operation is the most complicated one as it
  needs to express the full complexity of the JSM :~$S₀$,~$S₁$ and the
  coordination between them. The coordination between~$S₀$ and~$S₁$ is
  expressed in \Java with the actual parameters of items.

  \cref{figure:push-encoding} shows how the coordination between~$S₀$
  and~$S₁$ is encoded using \Java generics.

  \begin{figure}[ht]
    \caption[Type encoding of push operations]
      {\label{figure:push-encoding}
        Type encoding example of~$\Function push(\Function Consolidate(i,t))$
        where \[
          i = [ \<Procedure>::= \cc{procedure} \; \cc{id} \;·\;
                \<Parameters> \;\cc{;} \;\<Definitions> \;\<Body> \;]
        \] and~$t=\cc{()}$.
        Method \cc{pair\_t()} represents the terminal ‟\cc{()}”.
        Type \cc{Procedure4} represents the item~$i$ with dot after
        \cc{;} and type \cc{Parameters1} represents item~$ [ \<Parameter> ::=
        \cc{()}·]$.
     }
    \javaInput[minipage,width=\linewidth,left=-5ex]{pascal.push.encoding.listing}
  \end{figure}
  \cref{figure:push-encoding} presents the encoding of item \[
  i = [ \<Procedure>::= \cc{procedure} \; \cc{id} \;·\;
  \<Parameters> \;\cc{;} \;\<Definitions> \;\<Body> \;] .
  \]

  Upon seeing terminal~$t=\cc{()}$, the \RLLp decides to push the value\[
    \Function Consolidate(i,t) = ❴
      \begin{array}{l}
      \left[ \<Parameters> ::= \cc{()} \;·\; \right]
⏎
       \left[ \<Procedure>::= \cc{procedure} \; \cc{id} \;·\;
        \<Parameters> \;\cc{;} \;\<Definitions> \;\<Body> \; \right]
      \end{array}
      ❵ ,
    \]
    when \<Parameter>'s item is the new item at the top of the stack. Sure
    enough, the return type of method~$\cc{pair_t()}$ in
    \cref{figure:push-encoding} is \cc{Parameter1}. Note that the actual
    parameters (i.e., the instantiation of generic types) encode the JSM's
    ‟runtime” state. Consider now what should be pushed to~$S₁$ (for
    brevity, we only show~\cc{()}'s entry in the partial maps). Since \[
      \Function Jumps(\left[ \<Parameters> ::= \cc{()} \;·\; \right])
      = ❴ ❵ ,
    \] it does not effect on~$S₁$, and accordingly, on the actual parameters,
    while \[
      \Function Jumps(\left[ \<Procedure>::= \cc{procedure} \; \cc{id}
      \;·\; \<Parameters> \;\cc{;} \;\<Definitions> \;\<Body> \; \right])[\cc{()}] =
    \]\[
      \left[ \<Procedure>::= \cc{procedure} \; \cc{id}
      \<Parameters> \;\cc{;} \;·\; \;\<Definitions> \;\<Body> \; \right]
    \] means that if the \RLLp will perform~$\Function jump(\cc{;})$, the
    operation will return the type~$Procedure4$ (defined in
    \cref{figure:push-encoding}).

    Returning to our example (\cref{figure:push-encoding}),
    \cc{Parameters1}'s actual parameter is \cc{Procedure4} as it should
    be the type return upon jumping on ‟\cc{;}”, and that since we
    have no additional information regarding the state of the JSM,
    \cc{Procedure4}'s actual parameters are given from \cc{Procedure2}
    formal parameters.
  \end{description}

\section{Small Implementation Details}
It is worth mentioning that for technical convenience, we did a small
transformation on the input grammar, as being done by LR parsers.
A new start symbol~$\<S'>$ is added to the grammar while for each old
start symbol~$\<S>$, a rule~$\<S'> ::= \<S>$ is introduced.

The reason is to prevent cases where the start symbol appears on the
right-hand side of derivations, without constraining the use of \Fajita.
In these cases, the~$\Function Follow(·)$ set of the start symbol
might not contain~$\$$ exclusively, which means that the derived items
will have type parameters. This situation is awkward as the correct instantiation
of these items will always have \cc{Error} types as arguments.
