\section{Conclusions}

\section{Future Work}
\subsection{Parsing}
A problem related to that of recognizing a formal language, is that of parsing,
i.e., creating, for input which is within the language, a parse tree according
to the language's grammar.  In the domain of fluent APIs, the distinction
between recognition and parsing is in fact the distinction between compile time
and runtime.  Before a program is run, the compiler checks whether the fluent
API call is legal, and code completion tools will only suggest legal extensions
of a current call chain.

In contrast, a parse tree can only be created at runtime, as the fluent API can
either or compile, or not compile, but cannot do much more.

There are two possible ways to parse at runtime with fluent APIs.  The first is
by creating the parse-tree iteratively, where each method invocations in the
call chain adding more components to this tree (again, it's computed during
runtime).

The second way is by generating this tree in ‟batch” mode: this is done by
maintaining a \emph{fluent-call-list} which starts empty and grows at runtime
by having each method invoked add to it a record storing the method's name and
values of its parameters.  The list is completed at the end of the
fluent-call-list, at which point it is fed to an appropriate parser that
converts it into a parse tree (or even an AST).

An interesting hypothesis we have, is that the parse tree can be created at
  runtime without running any additional parsing algorithm.
We believe it can be done because in LL parsing once we see a new rule, we 
know it is the next derivation in the leftmost derivation of the string.

The main question we wish to leave the reader with is:
\begin{quote}
  Can we ALWAYS parse without running ANY parsing method on runtime? 
\end{quote}

\subsection{The Endable Symbols Pitfall}
\label{section:endable}

Consider the nonterminal \<Body> from out running example
(\cref{figure:running}),
since \<Body> appears as the last symbol in a rule of a start symbol,~$\$$ is
in $\Function Follow(\<Body>)$.

In that case, the type encoding item \[
  i = [ \<Body> ::= \cc{begin} \; \cc{end} \; \cdot ] 
\] should have a method~\cc{\$()} concluding the API chain.

However, \<Body> appears in a non-endable context in \<Procedure>'s rule (this
context is not endable as~$\$$ is not in~$\Function Follow(\<Procedure>)$) and
since we have only one type encoding for item~$i$, a user of the fluent API
could legally conclude the API chain even when the call chain is not legal.

We believe this problem can be solved with some grammar tweaking.

\subsection{The Recursive JSM Encoding Pitfall}
In the way the JSMs are encoded in \Java, the~$S_1$ stack of partial mapping
is encoded as if the JSM have for each terminal a JSM snapshot that represents
the state of the JSM in case of jumping. 

In the case of right recursion, the JSM is requested to keep a snapshot of
itself (which includes another snapshot of itself, etc.).  It can be thought of
as the popping of the jump and the ``push after jump'' cancel each other
perfectly, to remain in the same state.  Thus, the matching encoding of the JSM
is a type, that tries to have itself as one of its type arguments.

We think this problem can be solved with two possible solutions:
\begin{enumerate}
  \item Upon recognizing this state, remove the recursive formal parameter, and the method
    that performs the jump will simply return the type of ``\kk{this}''
  \item \Java allows recursive types definitions as in the following example 
    \begin{quote}
      \javaInput[minipage,width=\linewidth]{../Fragments/java-enums.java}
    \end{quote}
\end{enumerate}

\subsection{Generating a Realtime Parser from an LR Grammar}
LR grammar are much more expressive than LL grammar, as they 
can express the syntax of most modern programming languages.

The LR parser is much more complex as every state of the parser holds all the
possibilities for parsing from this point (this is due to the parser's rightmost
derivation property).

We know from \cref{chapter:ecoop} that it can theoretically be done,
but we would like a practical algorithm.
