We now turn to the proof of \cref{Theorem:Gil-Levy}, which will be showing a type encoding for the DPDA that recognizes a certain DCFG language.
The proof is by reduction to type encoding of SDPDAs, which are simpler version of deterministic pushdown automata.

\subsection{Reduction}
The difficulty in type-encoding of general DPDAs is that of~$ε$-moves:
A DPDA is allowed to make a transition into a different state,
  pop one or more stack elements, and then, pop, push or even both pop and push, any number of stack elements,
  all without consuming a single input symbol.

In contrast, our type encoding relies on the encoding of an input string~$α∈Σ^*$, one symbol at a time, as a sequence of method calls.
A step of computation occurs in the domain of \Java types only at method calls. There are no-provisions for encoding~$ε$-moves, however determistic
  they are.
The following definition is of a deterministic pushdown automata without such moves.

\begin{Definition}
  \label{Definition:SDPDA}
  \slshape
  A \emph{\textbf Simplified \textbf Deterministic \textbf Pushdown \textbf Automaton} (SDPDA) of order~$k$ is
    a quintuple~$⟨Γ,Q,q₀,A,δ⟩$,
  where~$Γ$,~$Q$,~$q₀∈Q$,~$A⊆Q$ are precisely as in \cref{Definition:DPDA}.
  The signature of function~$δ$, the \emph{generalized partial transition function}
  of an SDPDA of order~$k$, is however~$δ: Q⨉\left(Γ∪❴\vdash❵\right)ᵏ⨉Σ↛Q⨉Γ^*$.
  \par
  An SDPDA begins as a DPDA. At each step it examines~$σ∈Σ$,
    the next input symbol,~$q∈Q$, the current state,
    and~$γ₁,⋯,γₖ∈Γ$, the~$k$ top most stack elements,
  \par
  If the stack is empty, or~$q∈A$, the automaton stops in accepting.
  If the transition function is undefined, i.e.,~$δ(q,γ₁,…,γₖ,σ)=⊥$ the automaton
    stops in rejection.
  Otherwise, and since there are no~$ε$-moves, there are~$q'∈Q$ and~$ζ∈Γ^*$
    such that~$δ(q,γ₁,…,γₖ,σ)=(q',ζ)$.
  The automaton then pops~$γ₁$ through~$γₖ$, pushes the sequence~$ζ$, and
    moves to state~$q'$.
\end{Definition}

We say that
a \emph{configuration}~$(q,γ)$,~$q∈Q$,~$γ∈Γ$,
  is an \emph{$ε$-configuration} if~$δ(q,γ,ε)=(q',ζ)$ for
    some~$q'∈Q$ and a sequence of stack elements~$ζ$,
Recall that by definition of DPDAs,~$δ(q,γ,σ)=⊥$ for
  all~$σ∈Σ$, i.e., there is precisely one transition
  leading out of an~$ε$-configuration.

Our objective is to remove all~$ε$-configurations,
  replacing these with the non-$ε$ configuration that
  they (eventually) lead to.
This is easy to do when~$|ζ|≥1$, in which
  case the top of the stack is known to be~$γ'$,~$γ'$ being the first
    element of~$ζ$; the configuration at the end of transition
    would be then be~$(q',γ')$, which in its turn may, or may not be,
    an~$ε$-configuration.

However, if~$ζ=ε$, we do not know which element resides at the top
  of the stack.
The transition from the current configuration can thus lead to \emph{multiple} configurations,
  depending on the actual element left at the top of the stack after popping~$γ$.
The definition of a configuration thus must be changed to include the \emph{two} top
  most stack elements.
But, even this would fail if the next configuration is also an~$ε$-configuration that pushes no elements.

To be able to compute the closure of~$ε$-configuration, we shall
  therefore include with the configuration the~$k$ top most stack elements,
  where~$k$ is defined to be large enough so that
  the top of the stack must never be guessed while going
  through~$ε$-configurations.

To compute~$k$ we need to compute the largest possible overdraft
  from the stack along a series of transitions through~$ε$-configuration.
To do so, define the following directed weighted graph:
Nodes in this graph are the~$|Q|·|Γ|$ distinct configurations
  of the form~$(q,γ)$,~$q∈Q$,~$γ∈Γ$.
Let a node~$(q,γ)$ be an~$ε$-configuration for which~$δ(q,γ,ε)=(q',ζ)$.
Then, there is an edge from~$(q,γ)$ to all~$|Γ|$ nodes of the form~$(q',γ')$,~$γ'∈Γ$, i.e.,
  to all possible ‟guesses” of the stack element beneath~$γ$.
Each such edge takes the weight~$1-|ζ|$ to denote the fact
  that the total ‟charge” to the stack is~$1-|ζ|$:
    one stack element (specifically~$γ$) is popped,
    while~$|ζ|$ (which could also be zero), are pushed.
The value of $k$ is simply the heaviest path in this graph. 

In the former graph, cycles might occour.
Sadly, we cannot solve those cycles, since arbitrary number of symbols might be removed from the 
  stack, and therefore, the required~$\epsilon$-closure cannot be computed.
As shown by someone~\cite{i:need:to::find:it}, DPDAs with no $\epsilon$-moves are less expressive than a normal DPDAs,
  and therefore, not all DPDA can be computed.
  
Now, we shall describe the transition function of an SDPDA in the terms of a DPDA (assuming the former graph is DAG).
Let~$M=⟨Q,Γ,q₀,A,δ⟩$ be a DPDA, our goal is to describe~$S_m=⟨Q_m,Γ,q_{0m},A_m,δ_m⟩$ 
  of order k, the corresponding SDPDA.
Notice that the stack elements are the same in both.
The other components of $S_m$ are constructed as follows:
\begin{itemize}
 \item $Q_m$ is the set~$Q⨉\left(Γ∪❴\vdash❵\right)ᵏ$ 
 \item $q_{0m}$ is the state $(q₀,\vdash^{k-1}\gamma)$ where~$\gamma$ is the stack start element of~$Γ$.
 \item $A_m$ is the set~$❴(q,ζ)\in Q_m | q \in A❵$
 \item for $\delta(q,\gamma,\sigma) = (q',\zeta)$
 \begin{itemize}
  \item if $\sigma\neq\epsilon$ than for all $(q,\zeta'\gamma')$ such that $\zeta'\gamma \in \Gamma^k$
    we define $\delta_m(q,\zeta'\gamma,\sigma)=(q',\zeta'\zeta)$
  \item if $\sigma = \epsilon$ then~$\delta_m(q,\zeta,\epsilon)=(q',\zeta')$ when~$(q',\zeta')$ 
    is the~$\epsilon$-closure on~$(q,\zeta)$.
    The~$\epsilon$-closure of~$(q,\zeta)$ is the series of consecutive $\epsilon$-transitions from $\delta$
    that end with a single ``input consuming'' transition. 
    This computation is possible due to the definition of~$k$, that assures us, that during this ``static''
    computition on~$\delta$, we will at all times know the top of the stack, and therefore, this computition
    is viable.
 \end{itemize}

\end{itemize}

\subsection{Type Encoding}
\begin{Theorem}
  \label{Theorem:SDPDA}
  For every SDPDA~$a$ there exists a set~$J_a$ of \Java type definitions, such that
  the command \[
    \cc{A _ = M.build $\textsf{java}(α)$.\$()};
  \]
  compiles against~$Jₘ$ if an only if~$α$ is the language recognized by~$M$.
\end{Theorem}

The remainder of this section is dedicated to the proof of \cref{Theorem:SDPDA}.
In particular, we describe how~$J_M$ is constructed from the
  specification of~$M$.

We will use the following notation for
The stack contents~$⊥ξ₁ξ₁ξ₂ξ₁$,

\begin{figure}[H]
\begin{JAVA}
class Stack<Head extends ¢$Γ$¢, Rest extends Stack<?,?> > {¢¢
  Head head;
  Rest rest;
  Stack(Head head, Rest rest) {¢¢ this.head = head; this.rest = rest;}
  Rest pop() {¢¢ return rest; };
}
\end{JAVA}
For each state~$ξ∈Q$, we generate a \Java class~$ξ$,
\begin{JAVA}
class ¢$ξ$¢ <T extends S> extends S<T> {¢¢
  ¢$ξ$¢(T t) {¢¢ super(t); }
  // ¢…¢
}
\end{JAVA}
\end{figure}
In addition, we define a special class~$\vdash$ to designate the empty stack.
\begin{JAVA}
class ¢$\vdash$¢ extends Stack<¢$Γ$¢, ¢$\vdash$¢> {¢¢
  ¢$\vdash$ ¢() {¢¢ super(null); }
  ¢$\vdash$ ¢pop() {¢¢ throw new RunTimeException(); }
}
\end{JAVA}
The stack contents~$⊥ξ₁ξ₁ξ₂ξ₁$,
where~$ξ₁,ξ₂∈Q$ are stack elements,
is represented by the following type
\begin{JAVA}
  ¢$ξ₁$¢ < ¢$ξ₂$¢ < ¢$ξ₁$¢ < ¢$⊥$ > > >
\end{JAVA}
This is a general concept for implementing an unbounded stack with \Java's type system,
that will be extended in the future.

\subsection{Encoding states of pushdown automaton with \Java generics}
The generalized transition function \cref{Equation:generalized:transition}
  takes~$k+2$ arguments: a state~$q∈Q$, an input symbol~$σ∈Σ$
    and~$k$ stack elements drawn from~$Γ$.
To capture the full behavior of the automaton, the transition function
  must be able to pass along, in one way or another, the full contents of the stack.

Since the scheme described above can only be applied for binary functions,
  we shall pack together the state~$q$, the stack, and the~$k$ top most
  elements of it into a single type.
The next input symbol,~$σ∈Σ$, the remaining argument of function~$δ$,
  is encoded as a method.

The type part of the encoding is obtained by instantiating a generic type as follows:
Let~\cc{Q} be the abstract class that represents~$Q$, and let~\cc{q} be the concrete class that
  implements~\cc{Q} for an automaton state~$q∈Q$.
Then, to accommodate the extra~$k$ parameters of~$δ$, we add~$k$ generic parameters
  to class~$Q$ and to every class~\cc{q} that implements it.
Yet another such parameter is added for representing stack contents.

Consider the case~$k=2$.
\begin{figure}
  \begin{JAVA}
abstract class Q<S, ¢$γ₁$¢, ¢$γ₂$¢
  \end{JAVA}
\end{figure}

\begin{figure}
  \caption{\label{Figure:SDPDA:hierarchy}%
    Type hierarchy of the type encoding of a simple SDPDA.
  }
  \begin{adjustbox}{}
    \input ../Figures/automaton.tikz
  \end{adjustbox}
\end{figure}

\begin{figure}
  \caption{\label{Figure:SDPDA:example}%
    A type encoding of a simple pushdown automaton.
  }
  \javaInput[minipage,width=\linewidth]{spda.listing}
\end{figure}
As might be expected, this type is obtained
The
The

To encode this function within
