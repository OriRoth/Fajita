On a first sight, the proof of \cref{Theorem:Gil-Levy} could follow the techniques
  sketched in \cref{Section:toolkit} to type encode a DPDA (\cref{Definition:DPDA}).
The partial transition function~$δ$ may be type encoded as in \cref{Figure:simple-binary},
and the stack data structure of a DPDA can be encoded as in \cref{Figure:stack-encoding}.

The techniques however fail with~$ε$-transitions,
  which allow the automaton to move between an unbounded number of
  configurations and maneuver the stack in a non-trivial manner,
  without making any progress on the input.
The fault in the scheme lies with compile time computation being carried out
  by the~$\textsf{java}(σ)()$ functions, each converting
  their receiver type to the type of the receiver of the next call in the chain.
We are not aware of a \Java type encoding which makes
  it possible to convert an input type into an output type, where
  the output is a computed from the input by an unbounded number o steps.†{%
    With the presumption that the \Java compiler halts for all inputs (a presumption that does
    not hold for e.g., \CC, and was never proved for \Java), the claim that there is no \Java type encoding for all DPDAs can be proved:
 Employing~$ε$-transitions, it is easy to construct an automaton~$A^∞$ that
  never halts on any input.
A type encoding of~$A^∞$ creates programs that send the compiler in an infinite loop.
}

The literature speaks of finite-delay DPDAs, in which the number
  of consecutive~$ε$-transitions is uniformly bounded and even of
  realtime DPDAs in which this bound is 0, i.e., no~$ε$-transitions.
Our proof relies on a special kind of realtime automata,
  described by Courcelle~\cite{Courcelle:77}.

\begin{Definition}[Simple-Jump Single-State Realtime Deterministic Pushdown Automaton]
  \label{Definition:JDPDA}
  \slshape
  A \textit{simple-jump, single-state, realtime deterministic pushdown automaton}
  (jDPDA, for short) is a triplet~$⟨Γ,γ₁,δ⟩$
  where~$Γ$ is a set of stack elements,~$γ₁∈Γ$ is the initial stack element,
  and~$δ$ is the \emph{partial transition function},~$δ:Γ⨉Σ↛Γ^*∪j(Γ)$,
  \[
    j(Γ) = ❴ \textit{instruction \textup{\textsf{jump}}}(γ) \; | \;γ∈Γ❵.
  \]
  A configuration of a jDPDA is some~$c∈Γ^*$ representing the stack contents.
  Initially, the stack holds~$γ₁$ only.
  For technical reasons, assume that the input terminates with~$\$ \not∈Σ$, a special end-of-file character.
  \begin{itemize}
    \item At each step a jDPDA examines~$γ$,
    the element at the top of the stack,
    and~$σ∈Σ$, the next input character,
    and executes the following:
          \begin{quote}
            \begin{enumerate}
              \item consume~$σ$
              \item pop~$γ$
              \item if~$δ(γ,σ)=ζ$,~$ζ∈Γ^*$, the automaton pushes~$ζ$ into the stack.
              \item if~$δ(γ,σ)=\textsf{jump}(γ')$,~$γ'∈Γ$, then the automaton repetitively
                    pops stack elements up-to and including the first occurrence of~$γ$.
            \end{enumerate}
          \end{quote}
    \item If the next character is~$\$$, then the automaton may reject or accept (but not do nothing else),
          depending on the value of~$γ$.
  \end{itemize}
  In addition, the automaton rejects if~$δ(γ,σ) =⊥$ (i.e., undefined), or if it encounters
  an empty stack (either at the beginning of a step or on course
  of a \textsf{jump operation}).
\end{Definition}

\begin{wraptable}r{33ex}
  \caption{\label{Table:A} The transition function of a jDPDA~$A$,~$Σ=❴σ₁,σ₂❵$,~$Γ=❴γ₁,γ₂❵$}
  \begin{tabular}{c|ll}
             & \hfill~$γ₁$ & \hfill~$γ₂$⏎
    \midrule
$σ₁$ & $\textsf{push}(γ₁,γ₁)$ & $\hfill ⊥$⏎
$σ₂$ & $\textsf{push}(γ₂,γ₂)$ & $\textsf{jump}(γ₁)$⏎
$\$$ & $\textsf{reject}$ & $\textsf{accept}$⏎
  \end{tabular}
\end{wraptable}

As it turns out, every DCFG language is recognized by some jDPDA, and conversely,
  every language accepted by a jDPDA
  is a DCFG language~\cite{Courcelle:77}.
The proof of \cref{Theorem:Gil-Levy} is therefore reduced to type-encoding of a given jDPDA\@.
Towards this end, we employ the type-encoding techniques developed above, and, in particular, the jump-stack data structure (\cref{Figure:jump}).

Henceforth, let~$k =|Σ|$,~$ℓ=|Σ|$.
The simple~$k=2$,~$ℓ=2$ jDPDA~$A$ defined in \cref{Table:A} will serve as our running example.
Let~$L$ be the language~$L$ recognized by~$A$.%
†{%
  Incidentally,
\[
L = \left❴ σ₁^{n₁}σ₂^{2m₁} σ₁^{n₂}σ₂^{2m₂} ⋯ σ₁^{n_r}σ₂^{2m_r} \; | \; r≥0∧∀r', 1≤r'≤r⇒\sum_{i=1}^{r'} nᵢ≥\sum_{i=1}^{r'} mᵢ\right❵
\]
which is clearly not-regular; the equivalence of DCFG and jDPDAs means also that there is a non-ambiguous, deterministic
BNF for~$L$; neither the BNF nor the above representation are material for the proof.
}

\subsection{Main Types}
Generation of a type encoding for a jDPDA starts with two empty types for sets~$L$,~$Σ^*$,
  where~$L$ represents the languages accepted by the jDPDA and~$Σ^*$ represents all words:
\begin{quote}
  \javaInput[left=-2ex,minipage,width=54ex]{proof.headers.listing}
\end{quote}
(The full type encoding is in \cref{Figure:A} below; to streamline the reading, we bring
  excerpts as necessary.)

A configuration is encoded by a generic type~\cc{C}.
Essentially,~\cc{C} is a representation of the stack,
  but~$2k+1$ other parameters are required:
\begin{itemize}
  \item \cc{Rest} a type encoding of the stack after a pop,
  \item~$k$ types, named \cc{J$γ$1}, … ,\cc{J$γ${}$k$}, encoding the stack
        after~$\textsf{jump}(γ₁)$, … ,$\textsf{jump}(γₖ)$, and,
  \item~$k$ types, named \cc{JR$γ$1}, … ,\cc{JR$γ${}$k$}, encoding the type of \cc{Rest}
    after~$\textsf{jump}{γ₁}$,…~$\textsf{jump}(γₖ)$.
\end{itemize}
All instantiations of~\cc{C} must make sure that actual parameters are properly constrained,
so as to ensure that they are (the type version of) pointers into the actual stack.

In the running example,~\cc{C} is defined as:
\begin{quote}
  \javaInput[left=0ex,minipage,width=63ex]{proof.configuration.listing}
\end{quote}
This excerpt shows also classes~\cc{E} and~\cc{¤} which encode (as in \cref{Figure:jump})
  the empty and the error configurations.

Type~\cc{C} defines~$ℓ+1$ functions (3 in the example), one for each possible input character.
Since~\cc{C} encodes an abstract configuration, return types of functions in it
  are appropriate defaults which intentionally fail to emulate the automaton's execution.
  The return type of \cc{\$()} is \cc{ΣΣ} (rejection);
  the transition functions \cc{$σ$1()}, … \cc{$σ${}$ℓ$()}, return the raw type~\cc{C}.

\subsection{Top-of-Stack Types}

Types \cc{C$γ$1}, … ,\cc{C$γ${}$k$}, specializing~\cc{C},
  encode stacks whose top element is~$γ₁$, … ,$γₖ$.
In~$A$ there are two of these:
\begin{quote}
  \javaInput[left=-4ex,minipage,width=45ex]{proof.many.listing}
\end{quote}
This excerpt shows inter-dependency in the actual parameters to the type~\cc{C}, e.g.,
  the instantiation of~\cc{C} in \cc{C$γ$1} employs only three
    independent types: \cc{Rest}, \cc{J$γ$1}, and \cc{J$γ$2}.
Selection of the parameters enforces the intended constraints of each of these.
Think, for example, on a stack obtained by pushing \cc{$γ$1} on \cc{Rest}.
In such a stack, the result of~$\textsf{jump}(γ₁)$ must be \cc{Rest},
  and so, another parameter is redundant.

In~$A$, types \cc{C$γ$1} and \cc{C$γ$2} take three parameters;
in general ‟Top of Stack" types take~$k+1$ parameters:
\cc{Rest} and the results of~$\cc{Rest.}\textsf{jump}(γᵢ)$,~$i=1,…,k$.
It is the duty of the generated code to instantiate these types
  only with parameters \cc{Rest},\cc{JR$γ$1}, … ,\cc{JR$γ${}$k$} that satisfy
  these implicit demands.

\begin{wrapfigure}[14]r{45ex}
  \caption{\label{Figure:chain} Accepting and non-accepting call chains with the
  type encoding of jDPDA~$A$ (as defined in \cref{Table:A})}
  \javaInput[left=-2ex,minipage,width=45ex]{proof.cases.listing}
\end{wrapfigure}
Note that these~$k+1$ parameters are sufficient for instantiating the base type~\cc{C}, since
  the remaining~$\textsf{jump}(γᵢ)$ parameters of~\cc{C}
  are known when the top of the stack is known.
If the top is~$γⱼ$, then for all~$j≠i~$ \[
  \textsf{jump}(γⱼ) =
  \cc{Rest.}\textsf{jump}(γⱼ)
\] whereas~$\textsf{jump}(γᵢ) = \cc{Rest}$.

The code also defines the \kk{static} variable \cc{build}, the stating point
of all fluent API call chains to be of type \cc{C$γ$1<E,¤,¤>}, i.e.,
  the stating configuration of the automaton is a stack whose top is~$γ₁$,
  its rest is empty (\cc{E}).
Any of the two jumps possible on this rest results with,~\cc{¤},
  an undefined stack.
Examples of accepting and rejecting call chains starting at \cc{A.build}
  can be seen in \cref{Figure:chain}.

\subsection{Transitions}
It remains to show the type encoding of~$δ$,
  the transition function.
Overall, there are a total of~$k·(ℓ+1)$
  entries in a transition table such as \cref{Table:A}.
Conceptually, these are encoded by selecting the correct return
  type of functions \cc{$σ$1()}, … ,\cc{$σ${}$k$()} and \cc{\$()} in each of the ‟Top of Stack” types.
Thanks to inheritance, we need to do so only in the cases that this return type is different from the
  default.

Overall, there are six kinds of entries in a transition table:
\begin{description}
  \item[\textsf{reject}]
  The default return type of \cc{\$()} in~\cc{C} is \cc{$ΣΣ$}, which
  is \emph{not} a subtype of~\cc{L}. Normally the result of a call chain that ends with \cc{\$()}
  cannot be assigned to a variable of type~\cc{L}. Moreover, since \cc{$ΣΣ$} is \kk{private},
  there is little that clients can do with this result.
  \item[\textsf{accept}]
  The only case in which fluent call chain ending with \cc{\$()} can return
  type~\cc{L} is when the type returned of the call just prior to~\cc{.\$()} covariantly
  changes the return type of~\cc{\$()} to~\cc{L}.†{This is not to be confused with dynamic binding;
  types of fluent API call chains are determined statically.}
  \par
  Recall that a jDPDA can only accept after its input is exhausted.
  In \cref{Table:A} we see that \textsf{accept} occurs when the top of the stack is~$γ₂$.
  We therefore add to the body of type \cc{C$γ$2} the line
  \begin{JAVA}
@Override L ¢\gobble$¢$();
  \end{JAVA}
  \item[$⊥$]
  When a prefix of the input is sufficient to conclude it must be rejected however it continues
  the transition function returns~$⊥$.
  In~$A$ this occurs when the top of the stack is~$γ₂$ and a~$σ₁$ character is read.
  To type encode~$δ(γ₁,σ₂) =⊥$, one must \emph{not} override \cc{$σ$2()} in type~\cc{C$γ$1};
  the inherited return type (l.17 \cref{Figure:A}) is the raw~\cc{C}.
  Subsequent calls in the chain will all receive and return a raw~\cc{C}
  (Recall that all \cc{$σ${}$i$()}~$i=1,…,ℓ$ are functions in~\cc{C} that return a raw~\cc{C}).
  Therefore, the final \cc{\$()} will reject.
  \par
  Two other situations in which a jDPDA rejects but not demonstrated in~$A$ are:
  a \textsf{jump} that encounters an empty stack, and reading a character from when the stack is empty.
  In our type encoding these are handled by the special
  types~\cc{E} and~\cc{¤} (ll.19--20 ibid), both extending~\cc{C} without
  overriding any of its methods. Again, remaining part of the call chain will stick to
  raw~\cc{C}s up until the final \cc{\$()} call rejects the input.
  \item[$\textsf{jump}(γᵢ)$]
  The design of the generic parameters makes the implementation of~$\textsf{jump}(γᵢ)$
  operations particularly simple.
  All that is required is to covariantly change the return type of the
  appropriate \cc{$σ${}$j$()} function to the appropriate \cc{JR$γ{}i$} or~\cc{Rest} parameter
  (recall that a jump occurs after popping the current element from the stack, so
  we refer to \cc{JR} type parameters rather than~\cc{J}'s).
  \par
  In \cref{Table:A} we find that~$δ(γ₂,σ₂) =\textsf{jump}(γ₁)$. Accordingly, the type of \cc{$σ$2()}
  in \cc{C$γ$2} (l.39) is \cc{JR$γ$1}.
  \item[$\textsf{push}(\zeta)$]
  Push operations are the most complex, since they involve a pop of the top stack element,
  and pushing any number, including zero, of new elements.
  The challenge is in constructing the correct $k+1$-parameter instantiation of~\cc{C},
  from the current parameters of the type.
  Each of these $k+1$ is also an instantiation of~\cc{C} which may require more such
  parameters.
  Even though the number of ingredients is small, the resulting type expressions
  tend to be excessively long and unreadable.
  \par
  The predicament is ameliorated a bit by the idea,
  demonstrated above with auxiliary type~\cc{Pʹ}
  (\cref{Figure:jump-stack-push}),
  of delegating the task of creating a complex type to an auxiliary
  generic type.
  The task of this sidekick is simplified if some of its generic
  parameters are sub-expressions that recur in the desired
  result.
  \par
  Cases in point
  are~$δ(γ₁,σ₁)=\textsf{push}(γ₁,γ₁)$, and~$δ(γ₁,σ₂)=\textsf{push}(γ₂,γ₂)$ of \cref{Table:A}.
  The corresponding sidekick types,
  (\cc{$γ$1$σ$1\_Push\_$γ$1$γ$1} and \cc{$γ$1$σ$2\_Push\_$γ$2$γ2$})
  can be found in lines 40--48 of \cref{Figure:A}.
  The first of these defines the correct return type
  of \cc{$σ$()}, the second of \cc{$σ$2}.
  Examine now the definition of type \cc{C$γ$1} in the figure,
  and in particular lines 23--27 which define the list of types it extends.
  Notice that it extends these two sidekicks, inheriting the covariant
  overrides of \cc{$σ$1()} and \cc{$σ$2()}.
  \par
  More generally, economy of expression may require that for each case
    of~$δ(γ,σ)=\textsf{push}(\zeta)$ in the transition table,
    one creates a sidekick type which
      overrides the appropriate \cc{$σ$()} function.
  The appropriate \cc{C$γ$} type then inherits the definition
    from the sidekick.
\end{description}

\begin{figure}[htbp]
  \caption{\label{Figure:A}Type encoding of jDPDA~$A$ (as defined in \cref{Table:A})}
  \javaInput[left=-2ex,minipage,listing style=numbered,width=\textwidth]{proof.full.listing}
\end{figure}
