\documentclass[nonatbib,preprint,numbers]{sigplanconf}

\authorinfo{Mr.\ U. N. Owen \& Co.}{}{}

\title
{%
  %%\begin{flushright}
  %  \textcolor{blue!20!black}
  %  {\footnotesize\mdseries\slshape
  %    \renewcommand\baselinestretch{0.3}
  %    Software design is language design\\
  %    (and vice versa).\vspace{-4ex}\tiny\\
  %    \renewcommand\baselinestretch{0.3}
  %    ---a programmers' proverb
  %  }
  %\end{flushright}
  An Algorithm For Generating Fluent APIs for  Java 
}

\usepackage{\jobname}
%\overfullrule=1mm
%\directlua{dofile("DetectUnderfull.lua")}

\begin{document}
\maketitle

\begin{abstract}
  This paper is a theoretical study of practical problem:
  the automatic generation of Java Fluent APIs from their specification.
We explain why the problem's core lies with 
  the expressive power of Java generics.
Our main result is that automatic generation is possible whenever 
  the specification is an instance of the set of deterministic context-free languages,
  a set which contains most ``practical'' languages.
Other contributions include a collection of techniques and idioms o
  the limited meta-programming possible with Java generics, 
  and an empirical measurement demonstrating that the runtime of
  the ``javac'' compiler of Java the may be exponential in
  the program's length, even for programs composed of 
  a handful of lines and which do not rely on overly 
  complex use of generics.

\end{abstract}

\section{Introduction}
Ever after their inception\urlref{http://martinfowler.com/bliki/FluentInterface.html} \emph{fluent APIs}
  increasingly gain popularity~\cite{Bauer:2005,Freeman:Pryce:06,Larsen:2012} and research
  interest~\cite{Deursen:2000,Kabanov:2008}.
In many ways, fluent APIs are a kind of
  \emph{internal} \emph{\textbf Domain \textbf Specific \textbf Language}:
They make it possible to enrich a host programming language without changing it.
Advantages are many: base language tools (compiler, debugger, IDE, etc.) remain
  applicable, programmers are saved the trouble of learning a new syntax, etc.
However, these advantages come at the cost of expressive power;
  in the words of Fowler:
  ‟\emph{Internal DSLs are limited by the syntax and structure of your base language.}”†
  {M. Fowler, \emph{Language Workbenches: The Killer-App for Domain Specific Languages?},
    2005
    \newline
  \url{http://www.martinfowler.com/articles/languageWorkbench.html#InternalDsl}}.
Indeed, in languages such as \CC, fluent APIs
  often make extensive use of operator overloading (examine, e.g., \textsf{Ara-Rat}~\cite{Gil:Lenz:07}),
  but this capability is not available in \Java.

Despite this limitation, fluent API in \Java can be rich and expressive, as demonstrated
  in \cref{Figure:DSL} showing use cases of the DSL of Apache Camel~\cite{Ibsen:Anstey:10}
(open-source integration framework),
and that of jOOQ\urlref{http://www.jooq.org}, a framework for writing
  SQL in \Java, much like Linq~\cite{Meijer:Beckman:Bierman:06}.

\begin{figure}[H]
  \caption{\label{Figure:DSL} Two examples of \Java fluent API}
  \begin{tabular}{@{}c@{}c@{}}
    \parbox[c]{44ex}{\javaInput[left=0ex]{camel-apache.java.fragment}} &
    \hspace{-3ex} \parbox[c]{59ex}{\javaInput[left=0ex]{jOOQ.java.fragment}}⏎
    \textbf{(a)} Apache Camel & \textbf{(b)} jOOQ
  \end{tabular}
\end{figure}

Other examples of fluent APIs in \Java are abundant:
  jMock~\cite{Freeman:Pryce:06},
  Hamcrest\urlref{http://hamcrest.org/JavaHamcrest/},
  EasyMock\urlref{http://easymock.org/},
  jOOR\urlref{https://github.com/jOOQ/jOOR},
  jRTF\urlref{https://github.com/ullenboom/jrtf}
  and many more.

\subsection{A Type Perspective on Fluent APIs}
\Cref{Figure:DSL}(B) suggests that jOOQ imitates SQL,
but is it possible at all to produce a fluent API
for the entire SQL language,
or XPath, HTML, regular expressions, BNFs, EBNFs, etc.?
Of course, with no operator overloading it is impossible
to fully emulate tokens; method names though make a good substitute for tokens, as done
in ‟\lstinline{.when(header(foo).isEqualTo("bar")).}” (\cref{Figure:DSL}).
The questions that motivate that this research are:
\begin{quote}
  \begin{itemize}
    \item Given a specification of a DSL, determine whether there exists
        a fluent API can be made for this specification?
    \item In the cases that such a fluent API is possible,
      can it be produced automatically?
    \item Is it feasible to produce a \emph{compiler-compiler} such as Bison~\cite{Bison} tool
        to convert a language specification into a fluent API?
\end{itemize}
\end{quote}

Inspired by the theory of formal languages and automata,
  this study explores what can be done, and what can not be done, with fluent API in \Java.

Consider some fluent API (or DSL) specification, permitting only certain call
chains and disallowing all others.
Now, think of the formal language that defines the set of these permissible chains.
The main contribution of this paper is a proof (with its implicit algorithm) that
there is always \Java type definition that \emph{realizes} this fluent definition, provided that this
language is \emph{deterministic context-free}, where
\begin{itemize}
  \item In saying that a type definition \emph{realizes} a specification of fluent
    API, we mean that call chains that conform with the API definition compile
    correctly, and, conversely, call chains that are forbidden by the API
    definition do not type-check, resulting in an appropriate compiler error.
  \item Roughly speaking, deterministic context free languages are those
    context free languages that can be recognized by an LR parser†{The ‟L"
    means reading the input left to right; the ‟R" stands for rightmost derivation}~\cite{Aho:86}.
    \par
    An important property of this family is that none of its members is ambiguous.
    Also, it is generally believed that most practical programming languages
    are deterministic context-free.
\end{itemize}

A problem related to that of recognizing a formal language,
is that of parsing, i.e., creating, for input which is within the language,
  a parse tree according to the language's grammar,
In the fluent APIs domain, the distinction between recognition and parsing is
  the distinction between compile time and runtime.
Before a program is run, the compiler checks whether the fluent API call is legal,
  and code completion tools will only suggest legal extensions of a current call chain.

In contrast, the parse tree is created at runtime.
Some fluent API definitions create the parse-tree
  iteratively, where each method invocations in the call chain adding
  more components to this tree.
However, it is always possible to generate this tree in ‟batch” mode:
This is done by maintaining a \emph{call list} which
  starts empty and grows at runtime by having each method invoked add to it
  a record storing the method's name and values of its parameters.
The list is completed at the end of the call list, at which point it is fed to an appropriate parser that
  converts it into a parse tree (or even an AST).

\subsection{Contribution}
The answers we provide for the three questions above is:
\begin{quote}
  \begin{enumerate}
  \item If the DSL specification is that of deterministic context-free
    language, then a fluent API exists for the language, but we do not know
    whether such a fluent API exists for more general languages.
  \par
  Recall that there are universal cubic time parsing
  algorithms~\cite{add:refs:to:this:cubics} which can parse (and recognize) any
  context free language. What we do not know is whether algorithms of this sort
  can be encoded within the framework of the \Java type system.
  \item
  There exists an algorithm to generate a fluent API that realize any
  deterministic context-free languages.  Moreover, this fluent API can create
  at runtime, a parse tree for the given language.  This parse tree can then be
  supplied as input to the library that implements the language's semantic.
  \item
  Unfortunately, a general purpose compiler-compiler
  is not yet feasible with the current algorithm.
  \begin{itemize}
    \item One difficulty is that the algorithm is complicated and relies on
      modules implementing some theoretical results, which, to the best of our
      knowledge have never been implemented.
    \item Another difficulty is that a certain design decision in the
      implementation of the standard \texttt{javac} compiler may choke on the
      Java code produced by the algorithm.
  \end{itemize}
  \end{enumerate}
\end{quote}

Other concrete contributions made by this work include
\begin{itemize}
  \item the understanding that the definition of fluent APIs is analogous to
      the definition of a formal language.
  \item a lower bound (deterministic pushdown automata).
    on the theoretical ‟computational complexity” of the \Java type system.
  \item an algorithm for producing a fluent API for deterministic context free languages.
  \item a collection of generic programming techniques, developed towards this algorithm.
  \item a demonstration that the runtime of Oracle's \texttt{javac} compiler may be exponential in the program size.
\end{itemize}


\paragraph{Outline.} \Cref{section:example} familiarizes the reader with
techniques of converting a formal language specification into \Java types that
realizes the fluent API defined by the language. The example in this section is
used in \cref{section:fajita} to demonstrate \Fajita.  The reminder of LL
parsing theory in \cref{section:intuition} is to explain the challenges to be
met by main algorithm for the compilation of an LL language into into the \Java
type-checking model.  The main algorithm is described in
\cref{section:algorithm}.  \Cref{section:zz} concludes.

\section{Type States and a Fluent API Example}
\label{section:example}
%! TEX root = 00.tex
The large body of research on the general topic of \emph{type-states} (see
e.g., these review articles~\cite{Aldrich:Sunshine:2009,Bierhoff:Aldrich:2005})
Informally, an object that belongs to a certain type (\kk{class} in the object
oriented lingo), has type-states, if not all methods defined in this object's
class are applicable to the object in all states it may be in.

File object is the classical example: It can be in one of two states: ‟open” or
‟closed”. Invoking a \cc{read()} method on the object is only permitted when
the file is in an ‟open” state. In addition, method \cc{open()} (respectively
\cc{close()}) can only be applied if the object is in the ‟closed”
(respectively, ‟open”) state.

A recent study~\cite{Beckman:2011} estimates that about~$7.2%$ of \Java
classes define protocols definable in terms of type-states.
This non-negligible prevalence raise two challenges:
\begin{enumerate}
  \item \emph{\textbf{Identification.}} Frequently, type-state receive little
    or no mention at all in the documentation. The challenge is in identifying
    the implicit type state in existing code.
    \par
    Specifically, given an implementation of a class (or more generally of a
    software framework), \emph{determine} which sequences of method calls are
    valid and which violate the type state requirement presumed by the
    implementation. \item \emph{\textbf{Maintenance and Enforcement.}} Having
    identified the type-states, the challenge is in automatically flagging out
    illegal sequence of calls that does not conform with the type-state.
    \par
    Part of this challenge is maintenance of these automatic flagging
    mechanisms as the type-state specification of the API evolves.
\end{enumerate}

\subsection{A Type State Example}

An object of type \cc{Seat}†
{%
  example inspired by earlier work of Richard Harter on the
  topic~\cite{Harter:05}.
}
is created in the \cc{down} state, but it can then be \cc{raise}d to the
\cc{up} state, and then be \cc{lower}ed to the \cc{down} state.
Such an object be used by two kinds of users, \cc{male}s and \cc{female}s, for
two distinct purposes: \cc{urinate} and \cc{defecate}.  


A fluent API enforcement of type states should signal the sequences of method
calls made in \cref{figure:toilette:illegal} as type errors. 

\begin{figure}[H]
  \begin{JAVA}
new Seat().male().raise().urinate();
new Seat().female().urinate();
  \end{JAVA}
  \caption{Legal sequences of calls in the toilette seat example}
  \label{figure:toilette:legal}
\end{figure}

At the same time, this type-state enforcement a fluent API should recognize the
sequences of \cref{figure:toilette:legal} as being 
  type correct.

\begin{figure}[H]
  \begin{JAVA}
new Seat().female().raise();
new Seat().male().raise().defecate();
new Seat().male().male();
new Seat().male().raise().urinate().female().urinate();\end{JAVA}
  \caption{Illegal sequences of calls in the toilette seat example}
  \label{figure:toilette:illegal}
\end{figure}

The protocol of a \cc{Seat} 

Method calls in 
It should be clear that the type checking engine of the compiler can
be employed to distinguish between legal and illegal sequences.
It should also be clear that fabricating the \kk{class}es, \kk{interface}s
and the \kk{extends} and \kk{implements} relationships between these, is
far from being trivial.

\subsection{Type State}
The toilette seat problem may be amusing to some, but it is not contrived in
any way.
  To illustrate, consider the toilette seat example.
In this example,
  there are a total of six methods that might be invoked.
\begin{quote}
  \begin{tabular}{lll}
    \cc{male()} & \cc{raise()} & \cc{urinate()}⏎
    \cc{female()} & \cc{lower()} & \cc{defecate()}⏎
  \end{tabular}
\end{quote}
A fluent API design specifies the order in which such calls can be made.

The \emph{first} novelty in this research is that the fluent API definition is
  through a CFG, written as a BNF.
\cref{figure:BNF} is such a specification for the toilette seat problem.

\begin{figure}[H]
  \begin{Grammar}
    \begin{aligned}
      \<Visitors> & ::= \<Down-Visitors> \hfill⏎
      \<Down-Visitors> & ::= \<Down-Visitor> \~\<Down-Visitors> \hfill⏎
      {} & \| \<Raising-Visitor> \~\<Up-Visitors> \hfill⏎
      {} & \| ε \hfill⏎
      \<Up-Visitors> & ::= \<Up-Visitor> \~\<Up-Visitors> \hfill⏎
      {} & \| \<Lowering-Visitor> \~\<Down-Visitors> \hfill⏎
      {} & \| ε \hfill⏎
      \<Up-Visitor> & ::= \cc{male()} \~\cc{urinate()} \hfill⏎
      \<Down-Visitor> & ::= \cc{female()} \~\<Action> \hfill⏎
                          & \| \cc{male()} \cc{defecate()} \hfill⏎
      \<Raising-Visitor> & ::= \cc{male()} \~\cc{raise()} \~\cc{urinate()} \hfill⏎
      \<Lowering-Visitor> & ::= \cc{female()} \~\cc{lower()} \~\<Action> \hfill⏎
                          & \| \cc{male()} \~\cc{lower()} \cc{defecate()} \hfill⏎
      \<Activity> & ::= \cc{urinate()} \hfill⏎
                          & \| \cc{defecate()} \hfill⏎
    \end{aligned}
  \end{Grammar}
  \caption{A BNF grammar for the toilette seat problem}
  \label{figure:BNF}
\end{figure}

\Self takes this grammar specification as input, and in response
  generates the corresponding
  \Java type hierarchy.

\subsection{Verbs and Nouns}
A second novelty of \Self is that the specification of a BNF such as in
  \cref{figure:BNF} can be also made with a \Java fluent API\@.
To do so, it is first necessary to
  define the set of \emph{grammar terminals}
  \begin{code}{JAVA}
enum ToiletteTerminals implements Terminal {¢¢
  male, female,
  urinate, defecate,
  lower, raise;
}
\end{code}
As common in fluent APIs we shall refer to these
as \emph{verbs}†{Admittedly, the words ‟male” and ‟female” are nouns.
  An excuse might be that the words are used as nouns to mean ‟male-visit” and ‟female-visit”.}.
Verbs are translated by \Self into methods.

We are also required to define the set of \emph{grammar variables}
\begin{code}{Java}
enum ToiletteVariables implements Variable {¢¢
  Visitors, Down_Visitors, Up_Visitors,
  Up_Visitor, Down_Visitor,
  Lowering_Visitor, Raising_Visitor,
  Activity
};
\end{code}
  We shall use the term ‟\emph{noun}” as synonymous to ‟variable”.

\subsection{Words and Grammar}
The terms ‟symbol” and ‟word” refer to an entity which is either
  a verb or a noun.

Once the verbs and the nouns are set, the grammar can be defined,
  using a fluent API generated by \Self itself as shown
  in \cref{figure:fluent}.

\begin{figure}[H]
  \begin{JAVA}[style=numbered]
new BNF()
  ¢¢.with(ToiletteTerminals.class)
  ¢¢.with(ToiletteSymbols.class)
  ¢¢.start(Visitors)
  ¢¢.derive(Visitors).to(Down_Visitors)
  ¢¢.derive(Down_Visitors)
    ¢¢.to(Down_Visitor).and(Down_Visitors)
    ¢¢.or(Raising_Visitor).and(Up_Visitors)
    ¢¢.orNone()
  ¢¢.derive(Up_Visitors)
    ¢¢.to(Up_Visitor).and(Up_Visitors)
    ¢¢.or(Lowering_Visitor).and(Down_Visitors)
    ¢¢.orNone()
  ¢¢.derive(Up_Visitor).to(male).and(urinate)
  ¢¢.derive(Down_Visitor)
    ¢¢.to(female).and(Action)
    ¢¢.or(male).and(defecate)
  ¢¢.derive(Raising_Visitor).to(male).and(raise).and(urinate)
  ¢¢.derive(Lowering_Visitor)
    ¢¢.to(female).and(lower).and(Action)
    ¢¢.or(male).and(lower).and(defecate)
  ¢¢.derive(Activity)
    ¢¢.to(urinate)
    ¢¢.or(defecate)
  ¢¢.go();
  \end{JAVA}
  \caption{A BNF grammar for the toilette seat problem}
  \label{figure:fluent}
\end{figure}

The call to function \cc{go()} (last line in \cref{figure:fluent}) instructs
  \Self to generate the code for the fluent API specified by the
  subsequent part of the expression.
Roughly speaking, nouns are translated to classes while verbs are translated to methods which
  take no parameters.
Two exceptions apply:
\begin{enumerate}
  \item
    Library classes such as \cc{String} and \cc{Integer}, just as user-defined
    classes such as \cc{Invoice} may be used as nouns.
    \Self generate class definitions only for classes whose name is declared
    in an \kk{enum} which is passed to \cc{with} verb in the BNF declaration.
  \item Verbs may take noun parameters, as explained below.
\end{enumerate}


\section{Introducing Fajita}
\label{section:fajita}
%! TEX root = 00.tex
There were a total of six methods that might be invoked in the example of the
previous section:
\begin{quote}
  \begin{tabular}{lll}
    \cc{male()}   & \cc{raise()} & \cc{urinate()}⏎
    \cc{female()} & \cc{lower()} & \cc{defecate()}⏎
  \end{tabular}
\end{quote}
\Cref{figure:BNF} is a BNF specification of the order in which they might

\begin{figure}[H]
  \begin{Grammar}
    \begin{aligned}
      \<Visitors>         & \Derives \<Down-Visitors> \hfill⏎
      \<Down-Visitors>    & \Derives \<Down-Visitor> \~\<Down-Visitors> \hfill⏎
      {}                  & \| \<Raising-Visitor> \~\<Up-Visitors> \hfill⏎
      {}                  & \| ε \hfill⏎
      \<Up-Visitors>      & \Derives \<Up-Visitor> \~\<Up-Visitors> \hfill⏎
      {}                  & \| \<Lowering-Visitor> \~\<Down-Visitors> \hfill⏎
      {}                  & \| ε \hfill⏎
      \<Up-Visitor>       & \Derives \cc{male()} \~\cc{urinate()} \hfill⏎
      \<Down-Visitor>     & \Derives \cc{female()} \~\<Action> \hfill⏎
                          & \| \cc{male()} \cc{defecate()} \hfill⏎
      \<Raising-Visitor>  & \Derives \cc{male()} \~\cc{raise()} \~\cc{urinate()} \hfill⏎
      \<Lowering-Visitor> & \Derives \cc{female()} \~\cc{lower()} \~\<Action> \hfill⏎
                          & \| \cc{male()} \~\cc{lower()} \cc{defecate()} \hfill⏎
      \<Activity>         & \Derives \cc{urinate()} \hfill⏎
                          & \| \cc{defecate()} \hfill⏎
    \end{aligned}
  \end{Grammar}
  \caption{A BNF grammar for the toilette seat problem}
  \label{figure:BNF}
\end{figure}

\Fajita takes this grammar specification as input, and in response
generates the corresponding \Java type hierarchy. 
(Our proof-of-concept
implementation does not yet deal with the construction of an AST from a concrete
method call chain.) 
actual chain; the

The \Fajita specification is made with a \Java fluent API\@.
\cref{figure:BNF} can be also 
To do so, it is first necessary to
define the set of \emph{grammar terminals}
\begin{quote}
  \javaInput[minipage,width=43ex,left=-2ex]{../Fragments/toilette-terminals.fragment}
\end{quote}

We are also required to define the set of \emph{grammar variables}
\begin{quote}
  \javaInput[minipage,width=43ex,left=-2ex]{../Fragments/toilette-variables.fragment}
\end{quote}

Once the set of terminals and nonterminals are fixed, the grammar can be
defined, in a fluent API generated by \Fajita itself as shown in
\cref{figure:fluent}.

\begin{figure}
  \javaInput[minipage,width=\linewidth,left=-2ex]{../Fragments/toilette-generation.fragment}
  \caption{A BNF grammar for the types state example}
  \label{figure:fluent}
\end{figure}

The call to function \cc{go()} (last line in \cref{figure:fluent}) instructs
\Fajita to generate the code for the fluent API specified by the
subsequent part of the expression.

Nonterminals are translated to classes while terms are translated to methods which
take no parameters.  

Library classes such as \cc{String} and \cc{Integer}, just as user-defined
classes such as \cc{Invoice} may be used as well.  \Fajita generate class
definitions only for classes whose name is declared in an \kk{enum} which is
passed to \cc{with} verb in the BNF declaration.  Recall also that methods that
take a parameters (\cref{figure:sql-bnf-java}) can be used as tokens as well.

Even though \Fajita is not at production level, we plan on submitting it to 
artifact evaluation. The current BNF specification of the tool is given in
\cref{figure:BNF:BNF} (which incidentally can be written in \Fajita fluent 
API iteself)


\begin{figure}
  \begin{Grammar}
    \begin{aligned}
      \<BNF>                & \Derives \<Header>\~\<Body>\~\<Footer> \hfill⏎
      \<Header>             & \Derives \<Variables> \~\<Terminals> \hfill⏎
      {}                    & \| \<Terminals> \~\<Variables> \hfill⏎
      \<Variables>          & \Derives \cc{with(Class<? \kk{extends} Variable>)}\hfill⏎
      \<Terminals>          & \Derives \cc{with(Class<? \kk{extends} Terminal>)}\hfill⏎
      \<Body>               & \Derives \<Start> \~\<Rules> \hfill⏎
      \<Start>              & \Derives \cc{start(\<Variable>)} \hfill⏎
      \<Rules>              & \Derives \<Rule> \~\<Rules>\hfill⏎
      {}                    & \| \<Rule> \hfill⏎
      \<Rule>               & \Derives \cc{derives(\<Variable>)} \<Conjunctions>\hfill⏎
      \<Conjunctions>       & \Derives \<First-Conjunction>\~\<Conjunctions>\hfill⏎
      \<First-Conjunction>  & \Derives \cc{to(\<Symbol>)}\~\<Symbols>\hfill⏎
      {}                    & \| \cc{toNone()}\hfill⏎
      \<Conjunctions> & \Derives \<Conjunction>\~\<Conjunctions>\hfill⏎
      {}                    & \| ε\hfill⏎
      \<Conjunction>  & \Derives \cc{or(\<Symbol>)}\~\<Symbols>\hfill⏎
      {}                    & \| \cc{orNone()} \hfill⏎
      \<Symbols>    & \Derives ε \hfill⏎
      {}                    & \| \cc{and(\<Symbol>)}\~\<Symbols> \hfill⏎
      \<Symbol>             & \Derives \cc{Variable} \hfill⏎
      {}                    & \| \<Verb>\hfill⏎
      {}                    & \| \<Verb>\~\cc{,} \<Noun> \hfill⏎
      \<Noun>               & \Derives \<Variable> \hfill⏎
      {}                    & \| \<Existing-Class> \hfill⏎
      \<Variable>           & \Derives \cc{Variable} \hfill⏎
      \<Footer>             & \Derives \cc{go()}\hfill⏎
    \end{aligned}
  \end{Grammar}
  \caption{A BNF grammar for defining BNF grammars}
  \label{figure:BNF:BNF}
\end{figure}


\section{LL Parsing and Realtime Constraints}
\label{section:intuition}
%! TEX root = 00.tex
Formal presentation of the algorithm that compiles an LL(1) grammar into an implementation of a
  language recognizer with \Java generics
  is delayed to~\cref{section:algorithm}.
This section gives an intuitive perspective on this algorithm.

We will first recall the essentials of the classical LL(1) parsing algorithm (\cref{section:essentials}).
Then, we will explain the limitations of the computational model
  offered by \Java generics (\cref{section:limitations}).

The discussion proceeds to the observation
  that underlies our emulation of the parsing algorithm
  within these limitations.

Building on all these, \Cref{section:generation} can
make the intuitive introduction to the main algorithm.
To this end, we revise here the classical
algorithm~\cite{Lewis:66} for converting an LL grammar
(in writing LL, we, here and henceforth really mean
LL(1)), and explain how it is modified to generate a
  recognizer executable on the computational model of
\Java generics.

\subsection{Essentials of LL parsing}
\label{section:essentials}
The traditional \emph{\textbf{LL} \textbf Parser} (\LLp)
  is a DPDA allowed to peek at the next input symbol.
Thus,~$δ$, its transition function, takes two parameters: the current
  state of the automaton, and a peek into the terminal next to be read.
The actual operation of the automaton is by executing in loop the step
  depicted in \cref{algorithm:ll-parser}.

\renewcommand\algorithmicdo{\textbf{\emph{}}}
\renewcommand\algorithmicthen{\textbf{\emph{}}}
\begin{algorithm}
  \caption{\label{algorithm:ll-parser}
  The iterative step of an \LLp}
  \begin{algorithmic}[1]
      \LET{$X$}{\Function pop()}\COMMENT{what's next to parse?}
      \IF[anticipate to match terminal~$X$]{$X∈Σ$}
        \IF[read terminal is not anticipated~$X$]{$X≠\Function next()$}
          \STATE{\textsc{Reject}}\COMMENT{automaton halts in error}
        \ELSE[terminal just read was the anticipated~$X$]
          \STATE{\textsc{Continue}}\COMMENT{restart, popping a new~$X$, etc.}
        \FI
      \ELSIF[anticipating end-of-input]{$X=\$$}
        \IF[not the anticipated end-of-input]{$\$≠\Function next()$}
          \STATE{\textsc{Reject}}\COMMENT{automaton halts in error}
        \ELSE[matched the anticipated end-of-input]
          \STATE{\textsc{Accept}}\COMMENT{all input successfully consumed}
        \FI
      \ELSE[anticipated~$X$ must be a nonterminal]
        \LET{$R$}{$\Function δ(X, \Function peek())$}\COMMENT{determine rule to parse~$X$}
        \IF[no rule found]{$R=⊥$}
          \STATE{\textsc{Reject}}\COMMENT{automaton halts in error}
        \ELSE[A rule was found]
          \LET{$(Z ::= Y₁,…,Yₖ)$}{R}\COMMENT{break into left/right}
          \STATE{\textbf{assert}~$Z=X$}\COMMENT{$δ$ constructed to return valid~$R$ only}
          \STATE{$\Function print(R)$}\COMMENT{rule~$R$ has just been applied}
          \STATE{\textbf{For}~$i=k,…,1$,~$\Function push(Yᵢ)$}\COMMENT{push in reverse order}
          \STATE{\textsc{Continue}}\COMMENT{restart, popping a new~$X$, etc.}
        \FI % No rule found
      \FI % Main
\end{algorithmic}
  \vspace{0.3ex}
  \hrule
  \vspace{0.3ex}

  \ReplaceInThesis{\scriptsize}{}
  \begin{enumerate}
      \item
  Input is a string of terminals drawn from alphabet~$Σ$, ending
    with a special symbol~$\$∉Σ$.
      \item
  Stack symbols are drawn from~$Σ∪❴\$❵∪Ξ$, where~$Ξ$ is
  the set of nonterminals of the grammar from which the automaton was generated.
\item
  Functions~$\Function pop(·)$ and~$\Function push(·)$
    operate on the pushdown stack; function~$\Function next()$ returns
  and consumes the next terminal from the input string;
    function~$\Function peek()$ returns this terminal without consuming it.
\item
  The automaton starts with the stack with the start symbol~$S∈Ξ$ pushed
    into its stack.
  \end{enumerate}
\end{algorithm}

The DPDA maintains a stack of ‟anticipated” symbols, which may
  be of three kinds: a terminal drawn from the input alphabet~$Σ$,
  an end-of-input symbol~$\$$, or one of~$Ξ$, the set of
  nonterminals of the underlying grammar.

If the top of the stack is an input symbol or the special,
 end-of-input symbol~$\$$, then it must match the next terminal
 in the input string, the matched terminal is then consumed from
 the input string.
If there is no match, the parser rejects.
The parser accepts if the input is exhausted with
  no rejections (i.e.,~$\$$ was matched).

The more interesting case is that~$X$, the popped symbol
  is a nonterminal: the DPDA peeks into the next terminal in the input
  string (without consuming it).
Based on this terminal, and~$X$, the transition function~$δ$
  determines~$R$-the derivation rule applied to derive~$X$.
The algorithm rejects if~$δ$ can offer no such rule.
Otherwise, it pushes into the stack, in reverse order, the symbols
  found in the right hand side of~$R$.

\subsection{LL(1) Parsing with \Java Generics?}
\label{section:limitations}
Can \cref{algorithm:ll-parser} be executed on the machinery
  available with \Java generics?
As it turns out, most operations conducted by the algorithm
  are easy.
The implementation of function~$δ$ can
  be found in the toolbox in \cite{Gil:Levy:2016}.
Similarly, any fixed sequence of push and pop
  operations on the stack can be conducted within a \Java
  transition function:
  the ‟\textbf{For}" at the algorithm can be unrolled.

Superficially, the algorithm appears to be doing a constant amount
  of work for each input symbol.
A little scrutiny falsifies such a conclusion.

In the case~$R=X::=Y$,~$Y$ 
  being a nonterminal, the algorithm will conduct
  a~$\Function pop(·)$ to remove~$X$ and a~$\Function push(·)$
  operation for~$Y$ before consuming a terminal from the input.
Further,~$δ$ may return next the rule~$Y::=Z$,
  and then the rule~$Z::=Z'$, etc.
Let~$k'$ be the number of such substitutions
  occurring while reading a single input token.

\begin{Definition}[$k'$ - subtitution factor]
  \label{substitution-factor}
  Let~$A$ be a nonterminal at the top of the stack
    and~$t$ be the next input symbol.
  If~$t \in \Function First(A)$ then~$k'$ is the number of 
  consecutive substitutions the parser will perform 
  (replacing the nonterminal at the top the stack with one of it's rules)
  until~$t$ will be consumed from the input.
\end{Definition}

Also, in the case~$R=X::=ε$ the DPDA does not push
  anything into the stack.
Further, in its next iteration, the DPDA conducts another
pop,
\[
  X'←\Function pop(),
\]
instruction and proceeds to consider this new~$X'$.
If it so happens, it could also be the case
  that the right hand side of rule~$R'$
  \[
    R' = δ(X', \Function peek())
  \]
  is once again empty,
  and then another~$\Function pop()$
    instruction occurs
\[
  X”←\Function pop(),
\]
  etc.
Let~$k^*$ be the number
  of such instructions in a certain such mishap.

\begin{Definition}[$k^*$ - pop factor]
  \label{pop-factor}
  Let~$X$ be a nonterminal at the top of the stack
    and~$t$ be the next input symbol.
  Then~$k^*$ is the number of consecutive substitutions
  the parser will perform \emph{of only~$ε$ rules}
  until~$t$ will be consumed from the input.
\end{Definition}

The cases~$k' > 0$ and~$k^* > 0$ are not rarities.
For example, let us concentrate on the grammar
  depicted in \cref{figure:running}.
This grammar, inspired by the prototypical
  specification of \Pascal \urlref{http://www.fit.vutbr.cz/study/courses/APR/public/ebnf.html},
  shall serve as running example.

\newsavebox{\Alphabet}
\begin{lrbox}{\Alphabet}
  \begin{tabularx}{0.40\linewidth}{l}
    \cc{program}, \cc{begin}, \cc{end},⏎
    \cc{label}, \cc{const}, \cc{id},⏎
    \cc{procedure}, \cc{;}, \cc{()}
  \end{tabularx}
\end{lrbox}

\begin{figure}
  \caption[An LL(1) grammar (fragment of the \Pascal's grammar)]{\label{figure:running}
    An LL(1) grammar over the alphabet
    \[
      Σ = \left❴\usebox\Alphabet\right❵.
    \]
    (inspired by the original \Pascal grammar; to serve as
    our running example)
  }
  \begin{Grammar}
    \begin{aligned}
      \<Program> & \Derives \cc{program} \cc{id} \<Parameters>\~\cc{;} \<Definitions> \<Body>\hfill⏎
      \<Body> & \Derives \cc{begin} \cc{end}\hfill⏎
      \<Definitions> & \Derives \<Labels>\~\<Constants>\~\<Nested>\hfill⏎
      \<Labels> & \Derives ε \| \cc{label} \<Label>\~\<MoreLabels> \hfill⏎
      \<Constants> & \Derives ε \| \cc{const} \<Constant>\~\<MoreConstants> \hfill⏎
      \<Label> & \Derives\cc{;} \hfill⏎
      \<Constant> & \Derives\cc{;} \hfill⏎
      \<MoreLabels> & \Derives ε \| \<Label>\~\<MoreLabels>\hfill⏎
      \<MoreConstants> & \Derives ε \| \<Constant>\~\<MoreConstants>\hfill⏎
      \<Nested> & \Derives ε \| \<Procedure>\~\<Nested> \hfill⏎
      \<Procedure> & \Derives \cc{procedure}\cc{id}\<Parameters>\~\cc{;}\<Definitions>⏎
                    & \<Body> \hfill⏎
      \<Parameters> & \Derives ε \| \cc{()} \hfill⏎
    \end{aligned}
  \end{Grammar}
\end{figure}

The grammar preserves the ‟theme” of nested definitions of \Pascal, while
trimming these down as much as possible. \cref{table:derived-strings}
presents some legal sequences derived by this grammar.

\begin{table}[ht]
  \caption{\label{table:derived-strings}
      Legal words in the language defined by \cref{figure:running}}
  \rowcolors{2}{}{olive!20}
  \centering
  \begin{tabular}{m{58ex}}
    \toprule
      \cc{program id ; begin end}⏎
      \cc{program id () ; label ; begin end}⏎
      \cc{program id () ; label ; ; ; ; const ; begin end}⏎
      \cc{program id ; procedure id ; procedure id ;} \newline
          \cc{begin end begin end begin end}⏎
    \bottomrule
  \end{tabular}
\end{table}
In order to demonstrate the problems with~$k'$ and~$k^*$ we first
  supply the~$\Function First(·)$, and~$\Function Follow(·)$
  sets of our example grammar.

The~$\Function First(·)$ set of symbol~$X$ is the set of all terminals
that might appear at the beginning of a string derived from~$X$ (algorithm for
computing~$\Function First(·)$ is presented in \cref{algorithm:first}),
thus, for example, if~$X$ is a terminal then its first set is the set
containing only~$X$.
The~$\Function First(·)$ set is extended for strings too,~$\Function First(α)$
contains all terminals that can begin a string derived from~$α$ (Appropriate
algorithm can be found at~\cref{algorithm:first-string}).

The~$\Function Follow(·)$ set of nonterminal~$A$ is the set of all
  terminals that might appear immediately after a string that was
  derived from~$A$ (Appropriate algorithm can be found
  at~\cref{algorithm:follow}).

\begin{algorithm}
  \caption{\label{algorithm:first}
  An algorithm for computing~$\Function First(X)$ for each grammar symbol~$X$
    in the input grammar~$G =⟨Σ,Ξ,P⟩$}
    \begin{algorithmic}
    \FOR[initialize~$\Function First(·)$ for all nonterminals]{$A∈Ξ$}
      \STATE{$\Function First(A)=∅$}
    \ENDFOR
    \FOR[initialize~$\Function First(·)$ for all terminals]{$t∈Σ$}
      \STATE{$\Function First(t)=❴t❵$}
    \ENDFOR
    \WHILE{No changes in any~$\Function First(·)$ set}
      \FOR[for every production]{$X::=Y₀…Yₖ∈P$}
        \STATE{$\Function First(X) \; ∪\! = \Function First(Y₀)$}
        \COMMENT{add~$\Function First(Y₀)$ to~$\Function First(X)$}
        \FOR[for each symbol in the RHS]{$i∈❴0…k❵$}
          \IF[if the prefix is nullable]{$\Function Nullable(Y₀…Y_{i-1})$}
          \STATE{$\Function First(X) \; ∪\! = \Function First(Yᵢ)$}
            \COMMENT{add~$\Function First(Yᵢ)$ to~$\Function First(X)$}
          \ENDIF
        \ENDFOR % Symbols in the right-hand-side of a production
      \ENDFOR % Productions
    \ENDWHILE
  \end{algorithmic}
\end{algorithm}

\begin{algorithm}
  \caption{\label{algorithm:first-string}
  An algorithm for computing~$\Function First(α)$ for some string of symbols~$α$.
  This algorithm relies on results from~\cref{algorithm:first}}
  \begin{algorithmic}
    \LET{$Y₀…Yₖ$}{$α$}\COMMENT{break~$α$ into its symbols}
    \STATE{$\Function First(α) \; ∪\! = \Function First(Y₀)$}
    \COMMENT{initialize~$\Function First(α)$ with~$\Function First(Y₀)$}
    \FOR[for each symbol in the string]{$i∈❴1…k❵$}
      \IF[if the prefix is nullable]{$\Function Nullable(Y₀…Y_{i-1})$}
        \STATE{$\Function First(α) \; ∪ \! = \Function First(Yᵢ)$}
        \COMMENT{add~$\Function First(Yᵢ)$ to~$\Function First(α)$}
      \ENDIF
    \ENDFOR
  \end{algorithmic}
\end{algorithm}

\begin{algorithm}
  \caption{\label{algorithm:follow}
  An algorithm for computing~$\Function Follow(A)$ for each nonterminal~$X$
    in the input grammar~$G =⟨Σ,Ξ,P⟩$ when~$\$∉Σ$
    and~$S∈Σ$ is the start symbol of~$G$}
  \begin{algorithmic}
    \STATE{$\Function Follow(S)=❴ \$ ❵$}
    \COMMENT{initialize the start symbol}
    \WHILE{No changes in any~$\Function Follow(·)$ set}
      \FOR[for each grammar rule]{$A::=Y₀…Yₖ∈P$}
        \FOR[for each symbol in the RHS]{$i∈❴0…k❵$}
          \IF[compute only for nonterminals]{$Yᵢ∉Ξ$}
            \CONTINUE
          \ENDIF
          \STATE{$\Function Follow(Yᵢ) \; ∪ \! = \Function First(Y_{i+1}…Yₖ)$}
          \IF[if the suffix is nullable]{$\Function Nullable(Y_{i+1}…Yₖ)$}
            \STATE{$\Function Follow(Yᵢ) \, ∪ \! = \Function First(A)$}
            \COMMENT{add~$\Function First(A)$}
          \ENDIF
        \ENDFOR % Symbol in the RHS
      \ENDFOR
    \ENDWHILE
  \end{algorithmic}
\end{algorithm}

For example, the nonterminal \<Definitions> can be derived to a string
  beginning with \cc{label} (if there are defined labels), or \cc{const}
  (if there are no labels defined), or \cc{procedure} (if there are no
  labels or constants defined).
$\Function Follow(\<Definitions>)$ on the other hand, contains only
\cc{begin} since~\<Definitions> is always followed by~\<Body>,
  and~$\Function First(\<Body>)$ contains only \cc{begin}.

The~$\Function First(·)$ and~$\Function Follow(·)$ sets for
  our running example (\cref{figure:running}) are listed in~\cref{table:running-first-follow}.
\begin{table}[ht]
  \caption{\label{table:running-first-follow}
    Sets~$\Function First(·)$ and~$\Function Follow(·)$ for nonterminals of
    the grammar in~\cref{figure:running}}
    \rowcolors{2}{}{olive!20}
    \centering
    \ReplaceInThesis{
      \begin{tabular}{m{16ex}m{14ex}m{19ex}}
    }{
      \begin{tabular}{m{0.21\linewidth}m{0.32\linewidth}m{0.37\linewidth}}
    }
      \toprule
    Nonterminal & $\Function First(·)$ & $\Function Follow(·)$⏎
      \midrule
    \<Program> & \cc{program} & $∅$⏎
    \<Labels> & \cc{label} & \cc{const}, \cc{procedure}, \ReplaceInThesis{\newline}{} \cc{begin}⏎
    \<Constants> & \cc{constant} & \cc{procedure}, \cc{begin}⏎
    \<Label> & \cc{;} & \cc{;}, \cc{const}, \ReplaceInThesis{\newline}{} \cc{procedure}, \ReplaceInThesis{\newline}{} \cc{begin}⏎
    \<MoreLabels> & \cc{;} & \cc{const}, \cc{procedure}, \ReplaceInThesis{\newline}{} \cc{begin}⏎
    \<Constant> & \cc{;} & \cc{;}, \cc{procedure}, \ReplaceInThesis{\newline}{} \cc{begin}⏎
    \<MoreConstants> & \cc{;} & \cc{procedure}, \cc{begin}⏎
    \<Nested> & \cc{procedure} & \cc{begin}⏎
    \<Procedure> & \cc{procedure} & \cc{procedure}, \cc{begin}⏎
    \<Definitions> & \cc{label}, \cc{const}, \cc{procedure}& \cc{begin}⏎
    \<Body> & \cc{begin} & \cc{procedure}, \cc{begin}⏎
    \<Parameters> & \cc{()} & \cc{;}⏎
      \bottomrule
  \end{tabular}
\end{table}

In our example, nonterminal \<Procedure> has only one derivation
  rule, which begins with the terminal \cc{procedure}.
Thus, all derivations of \<Procedure> must begin with terminal
  \cc{procedure} and the~$\Function First(·)$ set of \<Procedure> is
  contains only~\cc{procedure}, as can be seen
  in~\cref{table:running-first-follow}.
The~$\Function Follow(·)$ set of \<Procedure> contains two
  terminals.
The first is \cc{procedure}, because the nonterminal \<Procedure>
  is followed by \<Nested> in \<Nested>'s rule,
  and \<Nested> can begin only with \cc{procedure}.
The second is \cc{begin}, because \<Nested> is nullable, and
  \cc{begin} appears in the~$\Function Follow(·)$ set of \<Nested>.

\subsubsection{Examples for~\texorpdfstring{$k'$}{k'} for the \Pascal grammar fragment in~\texorpdfstring{\cref{figure:running}}{}}
Consider a state of the parser, in which~$\<Definitions>$ is on
  the top of the top of the stack, and the next token on the input string
  is \cc{label}.
Until that token is consumed, the parser will:
  \begin{enumerate}
    \item pop~$\<Definitions>$ from the stack, and push the
      nonterminals~$\<Labels>$,~$\<Constants>$, and~$\<Nested>$
      in reversed order.
    \item pop~$\<Labels>$ and push the symbols~$\cc{label}$,~$\<Label>$,
      and~$\<MoreLabels>$ in reversed order.
    \item match the top of the stack~\cc{label} with the input
      symbol~\cc{label} and consume it.
  \end{enumerate}
Since we first substituted~$\<Definitions>$, then~$\<Labels>$
  and only then consumed the input symbol,~$k'=2$.

The problem is that every such operation is an~$ε$-move of the
  parser's DPDA, and as was shown in~\cite{Gil:Levy:2016} it causes
  problems when we want to employ \Java's compiler to ‟run” our automaton.

Another case is when encountering~$\<Nested>$ on the top of the stack
  and \cc{procedure} in the input string.
$k'=2$ again, where at first the non-$ε$ rule of~$\<Nested>$ will
  be pushed, then the non-$ε$ rule of~$\<Procedure>$ will be
  pushed, and only then will the terminal~\cc{procedure} will match,
  and be consumed from the input string.

\subsubsection{Examples of \texorpdfstring{$k^*$}{k*} for the \Pascal grammar fragment in \texorpdfstring{\cref{figure:running}}{}}
Consider a state in which the parser is in, where the
  only rule of~$\<Program>$ is being processed,~$\<Definitions>$
  was just replaced on the stack by it's only rule, and~$\<Body>$ is below.
Assume also, that the first terminal in the input string is \cc{begin}.

Until the next input token will be consumed, the following will happen:
  \begin{enumerate}
    \item consulting with the transition
      function~$δ$,~$\<Labels>$'s~$ε$-rule will
      be chosen,
      thus,~$\<Labels>$ will be popped from the stack.
    \item the same will happen with~$\<Constants>$
      and~$\<Nested>$.
    \item only after seeing nonterminal~$\<Body>$ shall the right rule will be
      pushed, and the token matched.
  \end{enumerate}
In this case,~$k^*=3$ because we had to pop three nonterminals and
  ‟replace” them with their matching rules, that are all~$ε$-rules
  before we got to a nonterminal that will derive to something other
  then~$ε$.

In the last example~$k^*$ was determined by the size of~$\<Definitions>$, but that
  doesn't have to be the case all the time.
For example, if in our grammar, in~$\<Program>$'s rule,
  nonterminal~$\<Parameters>$ would follow~$\<Definitions>$,
  then in the last example, after popping~$\<Labels>$,~$\<Constants>$
  and~$\<Nested>$, nonterminal~$\<Parameters>$ would also be popped
  for similar reasons, and~$k^*$ would have been~$4$ in this case.

Again, the problem relies with the multiple pops that occur without reading the input.

\subsection{LL(1) parser generator}
\label{section:generation}
The LL(1) parser is based on a prediction table.
For a given nonterminal as the top of the stack, and an input symbol,
  the prediction table provides the next rule to be parsed.
The parser generator fills prediction table, leaving the
  rest to the parsing algorithm in~\cref{algorithm:ll-parser}
The prediction table construction is depicted in~\cref{algorithm:generation}

\begin{algorithm}
  \caption{\label{algorithm:generation}
  An algorithm for filling the prediction table~$\Function predict(A,b)$ for some
    grammar~$G =⟨Σ,Ξ,P⟩$ and an end-of-input symbol~$\$∉Σ$,
    and where~$A∈Ξ$ and~$b∈Σ∪❴\$❵$.}
    \begin{algorithmic}
    \FOR[for each grammar rule]{$r∈P$}
      \LET{$A ::=α$}{$r$}\COMMENT{break~$r$ into its RHS & LHS}
      \FOR[for each terminal in~$\Function First(α)$]{$t∈\Function First(α)$}
        \IF[is there a conflict?]{$\Function predict(A,t)≠⊥$}
          \STATE{\textsc{Error}}\COMMENT{grammar is not LL(1)}
        \FI
        \STATE{$\Function predict(A,t) = r$} \COMMENT{set prediction to rule~$r$}
      \ENDFOR
      \IF[$α$ might derive to~$ε$]{$\Function Nullable(α)$}
        \FOR[for each terminal in~$\Function Follow(A)$]{$t∈\Function Follow(A)$}
          \IF[is there a conflict?]{$\Function predict(A,t)≠⊥$}
            \STATE{\textsc{Error}}\COMMENT{grammar is not LL(1)}
          \FI
          \STATE{$\Function predict(A,t) = r$} \COMMENT{set prediction to rule~$r$}
        \ENDFOR
      \FI
    \ENDFOR
  \end{algorithmic}
\end{algorithm}




\section{Generating a Realtime Parser \\ from an LL Grammar}
\label{section:algorithm}
%! TEX root = 00.tex

\subsection{Solving~$k'$ using closure}
\subsection{Solving~$k^*$}

\subsection{Main Algorithm}
The idea of the main is simple

\endinput

In~\cref{section:intuition} we presented the practical issues of
implementing the LL(1) parser ‟as is”, using two indexes:
the substitution factor~$k'$, and the pop factor~$k^*$.

This section will cover the solutions to these issues and
introduce the algorithm for the recognizer generation.

\subsection{Substitution Factor~$k'$}
The problem caused by the substitution factor if~$k'>0$
is that the recognizer needs to perform several substitution
of the top of the stack with the next rule to parse.

What we would like to have, is a single substitution that composes
all consecutive substitutions at one step.

An important observation is that the only parameter
that changes during runtime is the top of the stack.
Thus, if we solve this issue for every nonterminal,
the sequence of operations could be computed statically.

The computation process is rather simple:
given a nonterminal~$A∈Ξ$ (the top of the stack)
†{in case there's a terminal at the top of the stack~$k'=0$
and there are no substitutions}
and the next input symbol~$t∈Σ∪❴\$❵$, the algorithm iteratively
builds the RHS of the rule which is the ‟closure” of the consecutive rules.

The algorithm applies the following inductive step:
use~$r = \Function predict(A,t)$ to predict the next rule as the LL(1)
parser would ; add the RHS of~$r$ to the beginning of the result.
If the first symbol of the result is~$t$, finish.
Otherwise, it is a nonterminal. Let set~$A$ to be that nonterminal, and
apply the inductive step again.
The full algorithm is given in~\cref{algorithm:llclosure}.

\cref{algorithm:llclosure} presents
\begin{algorithm}[p]
  \caption{\label{algorithm:llclosure}
    An algorithm for computing the rule~$\Function closure(r,b)$ for some rule~$r∈P$ and
  terminal~$b∈Σ$, when~$Ξ$ is the nonterminals set.}
  \begin{algorithmic}
    \STATE{$(A::=Y₀…Yₖ)←r$} \COMMENT{break~$r$ into its LHS & RHS}
    \IF[$r$ cannot be derived to begin with~$b$]{$b∉\Function First(Y₀…Yₖ)$}
    \STATE{\textsc{Error}}
    \FI
    \STATE{$\text{LHS}=A$} \COMMENT{the output rule shares the LHS with the input}
    \STATE{$\text{RHS}=Y₀…Yₖ$} \COMMENT{initially, the RHS is the original rule}
    \LET{$h$}{$\Function PopHead(\text{RHS})$} \COMMENT{unroll the first symbol in RHS}
    \WHILE[$h$ is always a nonterminal in the loop]{$h∉Σ$}
    \FOR[for each~$h$-rule]{$h::=Y₀'…Yₖ'←get(P)$}
    \IF[if this rule matches]{$b∈\Function First(Y₀'…Yₖ')$}
    \STATE{$\text{RHS}=Y₀'…Yₖ' + \text{RHS}$} \COMMENT{add it to closure}
    \BREAK \COMMENT{since were in LL(1), can't be other}
    \ENDIF
    \ENDFOR
    \STATE{$h←\Function PopHead(\text{RHS})$} \COMMENT{unroll the first symbol in RHS}
    \ENDWHILE
    \RETURN{$\text{LHS}::=\text{RHS}$}
  \end{algorithmic}
\end{algorithm}


\section{Conclusions}
\label{section:zz}
As should be obvious from \cref{Figure:fluent}, \SELF will be implemented
  in a bootstrapping fashion.
The specification of a BNF, is made using a fluent API.
The BNF for writing BNFs is given in \cref{Figure:BNF:BNF}

\begin{figure}[htbp]
  \scriptsize
  \begin{equation*}
    \def\<#1>{\/⟨\/\text{\textit{#1}}\/⟩\/~}
    \def\|{~|~}
    \let\oldCc=\cc
    \let\oldKk=\kk
    \def\cc#1{{\footnotesize\oldCc{#1}}~}
    \def\cc#1{{\footnotesize\olKk{#1}}~}
    \small
    \begin{aligned}
      \<BNF>              & ::=  \<Notation> \<Body> \<Footer> \hfill⏎
      \<Notation>         & ::=  \<Symbols> \<Terminals> \hfill⏎
      {}                  & \|  \<Terminals> \<Symbols> \hfill⏎
      \<Terminals>        & ::=  \cc{with(Symbols.class)}
      \<Symbols>          & ::=  \cc{with(Terminals.class)}
      \<Body>             & ::= \<Start> \<Rules> \hfill⏎
      \<Start>            & ::=  \cc{with(Class<? \kk{extends} Symbol)} 
      \<Rules>            & ::= \<First-Rule> \<More-Rules> \hfill⏎
      \<More-Rules>       & ::= \<Additional-Rule> \<More-Rules> \hfill⏎
      {}                  & \| ε \hfill⏎
      {}                  & \| \<Lowering-Visitor> \<Down-Visitors> \hfill⏎
      {}                  & \| ε \hfill⏎
      \<Up-Visitor>       & ::= \cc{male()} \cc{urinate()} \hfill⏎
      \<Down-Visitor>     & ::= \cc{female()} \<Action> \hfill⏎
                          & \| \cc{male()} \cc{defecate()} \hfill⏎
      \<Raising-Visitor>  & ::= \cc{male()} \cc{raise()} \cc{urinate()} \hfill⏎
      \<Lowering-Visitor> & ::= \cc{female()} \cc{lower()} \<Action> \hfill⏎
                          & \| \cc{male()} \cc{lower()} \cc{defecate()} \hfill⏎
      \<Activity>         & ::= \cc{urinate()} \hfill⏎
                          & \| \cc{defecate()} \hfill⏎
    \end{aligned}
  \end{equation*}
  \caption{A BNF grammar for the toilette seat problem}
  \label{Figure:BNF:BNF}
\end{figure}


\bibliographystyle{abbrv}\small
\bibliography{author-names,other-shorthands-abbreviated,%
  publishers-abbreviated,%
  conferences-abbreviated,%
  journals-abbreviated,journals-full,%
  yogi-tr,yogi-book,yogi-practice,yogi-journal,yogi-theory,yogi-confs,%
  GPCE,OOPSLA,PLDI,USENIX,ECOOP,%
  00}
\end{document}
