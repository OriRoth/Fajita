\documentclass[nonatbib,preprint,numbers]{sigplanconf}
\authorinfo{Mr.\ U. N. Owen}{}{}
\title{%
\begin{flushright}
  \scriptsize\bfseries
  Software design is language design. \\
    (And vice versa.) \\
  \footnotesize\mdseries\itshape
   a programmers' proverb
\end{flushright}
  \Huge \Fajita \\ 
  \huge \itshape \textbf Fluent \textbf API for \textsc{\textbf Java} \\
  \LARGE (\textbf Inspired by the \textbf Theory of \textbf Automata)
}

\usepackage{\jobname}

\begin{document}


\maketitle

\begin{abstract}
  This paper is a theoretical study of practical problem:
  the automatic generation of Java Fluent APIs from their specification.
We explain why the problem's core lies with 
  the expressive power of Java generics.
Our main result is that automatic generation is possible whenever 
  the specification is an instance of the set of deterministic context-free languages,
  a set which contains most ``practical'' languages.
Other contributions include a collection of techniques and idioms o
  the limited meta-programming possible with Java generics, 
  and an empirical measurement demonstrating that the runtime of
  the ``javac'' compiler of Java the may be exponential in
  the program's length, even for programs composed of 
  a handful of lines and which do not rely on overly 
  complex use of generics.

\end{abstract}

Must include in a place that would not be missed.
\begin{itemize}
  \item \Prolog Unification.  We investigated wildcards, intersection types and
        (the rather limited) type inference of Java, but failed to employ these to
        increase the expressive power of the computation model.
        \item~$aⁿbⁿcⁿ$
\end{itemize}

\section{Introduction}
Ever after their inception\urlref{http://martinfowler.com/bliki/FluentInterface.html} \emph{fluent APIs}
  increasingly gain popularity~\cite{Bauer:2005,Freeman:Pryce:06,Larsen:2012} and research
  interest~\cite{Deursen:2000,Kabanov:2008}.
In many ways, fluent APIs are a kind of
  \emph{internal} \emph{\textbf Domain \textbf Specific \textbf Language}:
They make it possible to enrich a host programming language without changing it.
Advantages are many: base language tools (compiler, debugger, IDE, etc.) remain
  applicable, programmers are saved the trouble of learning a new syntax, etc.
However, these advantages come at the cost of expressive power;
  in the words of Fowler:
  ‟\emph{Internal DSLs are limited by the syntax and structure of your base language.}”†
  {M. Fowler, \emph{Language Workbenches: The Killer-App for Domain Specific Languages?},
    2005
    \newline
  \url{http://www.martinfowler.com/articles/languageWorkbench.html#InternalDsl}}.
Indeed, in languages such as \CC, fluent APIs
  often make extensive use of operator overloading (examine, e.g., \textsf{Ara-Rat}~\cite{Gil:Lenz:07}),
  but this capability is not available in \Java.

Despite this limitation, fluent API in \Java can be rich and expressive, as demonstrated
  in \cref{Figure:DSL} showing use cases of the DSL of Apache Camel~\cite{Ibsen:Anstey:10}
(open-source integration framework),
and that of jOOQ\urlref{http://www.jooq.org}, a framework for writing
  SQL in \Java, much like Linq~\cite{Meijer:Beckman:Bierman:06}.

\begin{figure}[H]
  \caption{\label{Figure:DSL} Two examples of \Java fluent API}
  \begin{tabular}{@{}c@{}c@{}}
    \parbox[c]{44ex}{\javaInput[left=0ex]{camel-apache.java.fragment}} &
    \hspace{-3ex} \parbox[c]{59ex}{\javaInput[left=0ex]{jOOQ.java.fragment}}⏎
    \textbf{(a)} Apache Camel & \textbf{(b)} jOOQ
  \end{tabular}
\end{figure}

Other examples of fluent APIs in \Java are abundant:
  jMock~\cite{Freeman:Pryce:06},
  Hamcrest\urlref{http://hamcrest.org/JavaHamcrest/},
  EasyMock\urlref{http://easymock.org/},
  jOOR\urlref{https://github.com/jOOQ/jOOR},
  jRTF\urlref{https://github.com/ullenboom/jrtf}
  and many more.

\subsection{A Type Perspective on Fluent APIs}
\Cref{Figure:DSL}(B) suggests that jOOQ imitates SQL,
but is it possible at all to produce a fluent API
for the entire SQL language,
or XPath, HTML, regular expressions, BNFs, EBNFs, etc.?
Of course, with no operator overloading it is impossible
to fully emulate tokens; method names though make a good substitute for tokens, as done
in ‟\lstinline{.when(header(foo).isEqualTo("bar")).}” (\cref{Figure:DSL}).
The questions that motivate that this research are:
\begin{quote}
  \begin{itemize}
    \item Given a specification of a DSL, determine whether there exists
        a fluent API can be made for this specification?
    \item In the cases that such a fluent API is possible,
      can it be produced automatically?
    \item Is it feasible to produce a \emph{compiler-compiler} such as Bison~\cite{Bison} tool
        to convert a language specification into a fluent API?
\end{itemize}
\end{quote}

Inspired by the theory of formal languages and automata,
  this study explores what can be done, and what can not be done, with fluent API in \Java.

Consider some fluent API (or DSL) specification, permitting only certain call
chains and disallowing all others.
Now, think of the formal language that defines the set of these permissible chains.
The main contribution of this paper is a proof (with its implicit algorithm) that
there is always \Java type definition that \emph{realizes} this fluent definition, provided that this
language is \emph{deterministic context-free}, where
\begin{itemize}
  \item In saying that a type definition \emph{realizes} a specification of fluent
    API, we mean that call chains that conform with the API definition compile
    correctly, and, conversely, call chains that are forbidden by the API
    definition do not type-check, resulting in an appropriate compiler error.
  \item Roughly speaking, deterministic context free languages are those
    context free languages that can be recognized by an LR parser†{The ‟L"
    means reading the input left to right; the ‟R" stands for rightmost derivation}~\cite{Aho:86}.
    \par
    An important property of this family is that none of its members is ambiguous.
    Also, it is generally believed that most practical programming languages
    are deterministic context-free.
\end{itemize}

A problem related to that of recognizing a formal language,
is that of parsing, i.e., creating, for input which is within the language,
  a parse tree according to the language's grammar,
In the fluent APIs domain, the distinction between recognition and parsing is
  the distinction between compile time and runtime.
Before a program is run, the compiler checks whether the fluent API call is legal,
  and code completion tools will only suggest legal extensions of a current call chain.

In contrast, the parse tree is created at runtime.
Some fluent API definitions create the parse-tree
  iteratively, where each method invocations in the call chain adding
  more components to this tree.
However, it is always possible to generate this tree in ‟batch” mode:
This is done by maintaining a \emph{call list} which
  starts empty and grows at runtime by having each method invoked add to it
  a record storing the method's name and values of its parameters.
The list is completed at the end of the call list, at which point it is fed to an appropriate parser that
  converts it into a parse tree (or even an AST).

\subsection{Contribution}
The answers we provide for the three questions above is:
\begin{quote}
  \begin{enumerate}
  \item If the DSL specification is that of deterministic context-free
    language, then a fluent API exists for the language, but we do not know
    whether such a fluent API exists for more general languages.
  \par
  Recall that there are universal cubic time parsing
  algorithms~\cite{add:refs:to:this:cubics} which can parse (and recognize) any
  context free language. What we do not know is whether algorithms of this sort
  can be encoded within the framework of the \Java type system.
  \item
  There exists an algorithm to generate a fluent API that realize any
  deterministic context-free languages.  Moreover, this fluent API can create
  at runtime, a parse tree for the given language.  This parse tree can then be
  supplied as input to the library that implements the language's semantic.
  \item
  Unfortunately, a general purpose compiler-compiler
  is not yet feasible with the current algorithm.
  \begin{itemize}
    \item One difficulty is that the algorithm is complicated and relies on
      modules implementing some theoretical results, which, to the best of our
      knowledge have never been implemented.
    \item Another difficulty is that a certain design decision in the
      implementation of the standard \texttt{javac} compiler may choke on the
      Java code produced by the algorithm.
  \end{itemize}
  \end{enumerate}
\end{quote}

Other concrete contributions made by this work include
\begin{itemize}
  \item the understanding that the definition of fluent APIs is analogous to
      the definition of a formal language.
  \item a lower bound (deterministic pushdown automata).
    on the theoretical ‟computational complexity” of the \Java type system.
  \item an algorithm for producing a fluent API for deterministic context free languages.
  \item a collection of generic programming techniques, developed towards this algorithm.
  \item a demonstration that the runtime of Oracle's \texttt{javac} compiler may be exponential in the program size.
\end{itemize}


\paragraph{Outline.} 
section:the toillete seat example
section: generalization:
  how to generate toilette with DFA.
  generalization to pushdown.
  examples were pushdowns become handy (probably stack, scoping in html, in hint only: nested SQL queries).
section: theoretical background
reminder of ECOOP.



\Cref{Section:proposal} 
  explains how \Fajita may serve the designer of a fluent API. 
Then, \Cref{Section:theoretical-background} discusses the 
  the core computational theory problem that \Fajita 
  needs to solve: recognizing and parsing of formal languages
  within the framework of the limited abilities of \Java
  generics.
The language recognition algorithm that \Fajita
  implements is the subject of \cref{Section:recognizer}.
The main ideas behind the bootstrapping definition of \Self 
  are revealed in \Cref{Section:bootstrapping}. 
\Cref{Section:zz} concludes. 


\section{Intuition}
\label{Section:recognizer}
\renewcommand{\LET}[2]{\STATE{\textbf{Let} \ensuremath{\text{#1}←\text{#2}}}}
\newcommand\INPUT\REQUIRE
\newcommand\OUTPUT\ENSURE
\def\function#1(#2){\ensuremath{\textrm{\textup{\textsf{#1}}}(#2)}}
\def\table#1[#2]{\ensuremath{\textsf{#1}[#2]}}

\begin{algorithm}[p]
  \caption{\label{Algorithm:LLClosure}
  Function~$\function Closure(A,b)$: generates a closure of action from the original LL algorithm}
  \begin{algorithmic}
    \INPUT{a nonterminal~$A$}
    \INPUT{a terminal~$b$}
    \OUTPUT{the closure of consecutive actions}
    \LET{$l$}{$[]$}
    \LET{$X$}{$A$}
    \IF{$∀X→α .  b∉\function First(α)$}
      \RETURN{\textrm{reject}}
    \ENDIF
    \WHILE{$\textrm{True}$}
      \IF{$X∈Ξ$}
        \STATE{let~$Y₁ Y₂… Yₖ$ be~$α$ s.t.~$∃X→α∈G∧b∈\function First(α)$}
        \STATE{$\function append(l,Yₖ,Y_{k-1},…,Y₂)$}
        \STATE{$X=Y₁$}
      \ELSE[$X$ must be~$b$]
        \STATE{\textbf{break}}
      \ENDIF
    \ENDWHILE
    \RETURN{$l$}
  \end{algorithmic}
\end{algorithm}

\begin{Definition}[JETEM]
  JETEM is a variant of pushdown automata that maintains two
    stacks:
    \begin{description}
      \item[Call stack]
      \item[Epsilon stack]
    \end{description}
    At each step of the automaton, it reads a single character,
      and determines based on the character and the current top of the call stack, what to do:
    \begin{description}
      \item[Epsilon] if character is not in first, but is in follow
      \item[Call] if character is in first but not in follow
      \item[Reject] if character is not in first nor in follow
      \item[Accept] when input exhausted. 
    \end{description}


  \begin{enumerate}
      \item 

\end{Definition}


\section{\Fajita as a development tool}
\label{Section:proposal}
The thesis propounded by this research is that API design, and especially fluent API design
  can and should be made in terms of language design.
Software missionaries and preachers such as Fowler~\cite{Fowler:2005} have long claimed
  that API design resembles the design of a \textbf Domain \textbf Specific \textbf Language
  (henceforth \emph{DSL}, see, e.g.,~\cite{VanDeursen:Klint:2000,Hudak:1997,Fowler:2010} for review articles).
   In the words of Fowler ‟The difference between API design and DSL design is then rather small”~\cite{Fowler:2005})

The objective of this research is
  to take the unification of the notions of DSL and (fluent) API
  design one step further in automating the creation of fluent API out
  of a DSL specification.

The basic idea is that the programmer specifies a fluent API,
  and, then, this specification is then automatically translated
  to an implementation of a fluent API that conforms
  with this specification.
This translation generates the intricate type hierarchy
  and methods of types in it in such a way
  that only sequence of calls that conform
  to the specification would
  compile correctly (concretely, type-check).

  To illustrate, consider the toilette seat example.
In this example,
  there are a total of six methods that might be invoked.
\begin{quote}
  \begin{tabular}{lll}
    \cc{male()} & \cc{raise()} & \cc{urinate()}⏎
    \cc{female()} & \cc{lower()} & \cc{defecate()}⏎
  \end{tabular}
\end{quote}
A fluent API design specifies the order in which such calls can be made.

The \emph{first} novelty in this research is that the fluent API definition is
  through a CFG, written as a BNF.
\cref{Figure:BNF} is such a specification for the toilette seat problem.

\begin{figure}[htbp]
  \scriptsize
  \begin{equation*}
    \def\<#1>{⟨\text{\textcolor{black}{\mdseries\rmfamily\/\textit{#1}}\/}⟩}
    \let\oldCc=\cc
    \let\oldKk=\kk
    \def\~{\text{~}}
    \def\|{\~|\~\~\~\~}
    \def\cc#1{{\footnotesize\oldCc{#1}}~}
    \def\kk#1{{\footnotesize\oldKk{#1}}}
    \scriptsize
    \begin{aligned}
      \<Visitors> & ::= \<Down-Visitors> \hfill⏎
      \<Down-Visitors> & ::= \<Down-Visitor> \~\<Down-Visitors> \hfill⏎
      {} & \| \<Raising-Visitor> \~\<Up-Visitors> \hfill⏎
      {} & \| ε \hfill⏎
      \<Up-Visitors> & ::= \<Up-Visitor> \~\<Up-Visitors> \hfill⏎
      {} & \| \<Lowering-Visitor> \~\<Down-Visitors> \hfill⏎
      {} & \| ε \hfill⏎
      \<Up-Visitor> & ::= \cc{male()} \~\cc{urinate()} \hfill⏎
      \<Down-Visitor> & ::= \cc{female()} \~\<Action> \hfill⏎
                          & \| \cc{male()} \cc{defecate()} \hfill⏎
      \<Raising-Visitor> & ::= \cc{male()} \~\cc{raise()} \~\cc{urinate()} \hfill⏎
      \<Lowering-Visitor> & ::= \cc{female()} \~\cc{lower()} \~\<Action> \hfill⏎
                          & \| \cc{male()} \~\cc{lower()} \cc{defecate()} \hfill⏎
      \<Activity> & ::= \cc{urinate()} \hfill⏎
                          & \| \cc{defecate()} \hfill⏎
    \end{aligned}
  \end{equation*}
  \caption{A BNF grammar for the toilette seat problem}
  \label{Figure:BNF}
\end{figure}

\SELF takes this grammar specification as input, and in response
  generates the corresponding
  \Java type hierarchy.

A second novelty of \SELF is that the specification of a BNF such as the provided in
  \cref{Figure:BNF} can be made in using a \Java fluent API.
To do so, it is first necessary to
  define the set of \emph{grammar terminals}
  \begin{code}{Java}
enum ToiletteTerminals implements Terminal {
  male, female,
  urinate, defecate,
  lower, raise;
}
\end{code}
As common in fluent APIs we shall refer to these
as \emph{verbs}†{Admittedly, the words ‟male” and ‟female” are nouns; 
  in our context howefver they are used to mean ‟male-visit” and ‟femail-visit”.}
Verbs are translated by \SELF into methods.

We also requrired to define the set of \emph{grammar variables}
  \begin{code}{Java}
enum ToiletteVariables implements Variable {
  Visitors, Down_Visitors, Up_Visitors,
  Up_Visitor, Down_Visitor,
  Lowering_Visitor, Raising_Visitor,
  Actitivity
};
  \end{code}
We shall use the term ‟nouns” as synonymous to variable.
The terms ‟symbol” and ‟word” refer to an entity which is either
  a verb or a nound.

Once the verbs and the nouns are set, the grammar can be defined,
  using a fluent API generated by \SELF itself as shown
  in \cref{Figure:fluent}.

\begin{figure}[htbp]
  \scriptsize
  \begin{code}{Java}
new BNF()
  .with(ToiletteTerminals.class)
  .with(ToiletteSymbols.class)
  .start(Visitors)
  .derive(Visitors)
    .to(Down_Visitors)
  .derive(Down_Visitors)
    .to(Down_Visitor).and(Down_Visitors)
    .or(Raising_Visitor).and(Up_Visitors)
    .orNone()
  .derive(Up_Visitors)
    .to(Up_Visitor).and(Up_Visitors)
    .or(Lowering_Visitor).and(Down_Visitors)
    .orNone()
  .derive(Up_Visitor)
    .to(male).and(urinate)
  .derive(Down_Visitor)
    .to(female).and(Action)
    .or(male).and(defecate)
  .derive(Raising_Visitor)
    .to(male).and(raise).and(urinate)
  .derive(Lowering_Visitor)
    .to(female).and(lower).and(Action)
    .or(male).and(lower).and(defecate)
  .derive(Activity)
    .to(urinate)
    .or(defecate)
  .go();
  \end{code}
  \caption{A BNF grammar for the toilette seat problem}
  \label{Figure:fluent}
\end{figure}

The final call \cc{go} in \cref{Figure:fluent} instructs
  \SELF to generate the code for the fluent API specified by the
  subsequet part of the expresion.
Rougly speaking, nouns are transalted to classes while verbs are translated to methods which
  take no parameters.
Two exceptions apply:
\begin{enumerate}
  \item \SELF can use classes such as \cc{String} and \cc{Integer}
  \item Verbs may take noun parameters, as explained below.
\end{enumerate}


\section{Theoretical background}
\label{Section:theoretical-background}
\input background

\section{Bootstrapping \Fajita}
\label{Section:bootstrapping}
\input bootstrapping

\section{Conclusion and Future Work}
\label{Section:zz}
As should be obvious from \cref{Figure:fluent}, \SELF will be implemented
  in a bootstrapping fashion.
The specification of a BNF, is made using a fluent API.
The BNF for writing BNFs is given in \cref{Figure:BNF:BNF}

\begin{figure}[htbp]
  \scriptsize
  \begin{equation*}
    \def\<#1>{\/⟨\/\text{\textit{#1}}\/⟩\/~}
    \def\|{~|~}
    \let\oldCc=\cc
    \let\oldKk=\kk
    \def\cc#1{{\footnotesize\oldCc{#1}}~}
    \def\cc#1{{\footnotesize\olKk{#1}}~}
    \small
    \begin{aligned}
      \<BNF>              & ::=  \<Notation> \<Body> \<Footer> \hfill⏎
      \<Notation>         & ::=  \<Symbols> \<Terminals> \hfill⏎
      {}                  & \|  \<Terminals> \<Symbols> \hfill⏎
      \<Terminals>        & ::=  \cc{with(Symbols.class)}
      \<Symbols>          & ::=  \cc{with(Terminals.class)}
      \<Body>             & ::= \<Start> \<Rules> \hfill⏎
      \<Start>            & ::=  \cc{with(Class<? \kk{extends} Symbol)} 
      \<Rules>            & ::= \<First-Rule> \<More-Rules> \hfill⏎
      \<More-Rules>       & ::= \<Additional-Rule> \<More-Rules> \hfill⏎
      {}                  & \| ε \hfill⏎
      {}                  & \| \<Lowering-Visitor> \<Down-Visitors> \hfill⏎
      {}                  & \| ε \hfill⏎
      \<Up-Visitor>       & ::= \cc{male()} \cc{urinate()} \hfill⏎
      \<Down-Visitor>     & ::= \cc{female()} \<Action> \hfill⏎
                          & \| \cc{male()} \cc{defecate()} \hfill⏎
      \<Raising-Visitor>  & ::= \cc{male()} \cc{raise()} \cc{urinate()} \hfill⏎
      \<Lowering-Visitor> & ::= \cc{female()} \cc{lower()} \<Action> \hfill⏎
                          & \| \cc{male()} \cc{lower()} \cc{defecate()} \hfill⏎
      \<Activity>         & ::= \cc{urinate()} \hfill⏎
                          & \| \cc{defecate()} \hfill⏎
    \end{aligned}
  \end{equation*}
  \caption{A BNF grammar for the toilette seat problem}
  \label{Figure:BNF:BNF}
\end{figure}


% \textbf{Acknowledgment.}
% Inspiring correspondence with Gilad Bracha is gratefully acknowledged.

\bibliographystyle{abbrv}\small
\bibliography{author-names,other-shorthands-abbreviated,%
 publishers-abbreviated,%
 conferences-abbreviated,%
 journals-abbreviated,journals-full,%
 yogi-book,yogi-practice,yogi-journal,yogi-tr,%
 GPCE,OOPSLA,PLDI,USENIX,ECOOP,%
 00,yogi-confs}

\clearpage
\appendix
\section{The JLR Recognizer}
\input algorithm

\end{document}

Processing programming languages
\begin{description}
 \item[Lexical analysis] - the first step of the process in which the character strings generated by the
 programmer are aggregated to the abstract tokens defined by the language designer.
 \item[Syntactical analysis (parsing) ] - the second step, in which the processed strings of tokens
 conform to the rules of a formal grammar defined by the language's BNF (or EBNF).
 \item[Semantical analysis] - the next step, usually performed in unison with the previous step,
 in which the legal token sequences are given their semantic meaning.
\end{description}
Specifically, the proposal is that API design of follows the footsteps of
Accordingly, the designer of a fluent API has to follow these three conceptual
steps.
First is the identification of the \emph{vocabulary}, i.e.,
the set of method calls including type arguments that may take part in the
fluent API\@.
In this fluent API example
\begin{JAVA}
allowing (any(Object.class))
 ¢¢.method("get.*")
 ¢¢.withNoArguments();
\end{JAVA}
then, there are three method calls, and the vocabulary has three items in it.
\begin{itemize}
 \item~$ℓ₁ = \cc{any(Class<?>)}$
 \item~$ℓ₂ = \cc{allowing($ℓ₁$)}$
 \item~$ℓ₃ = \cc{method(String)}$
 \item~$ℓ₄ = \cc{withNoArguments()}$
\end{itemize}
