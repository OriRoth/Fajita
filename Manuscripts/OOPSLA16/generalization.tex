%! TEX root = 00.tex
Section: generalization:
  how to generate toilette with DFA.
  Generalization to pushdown.
  Examples were pushdown automata become essential are: probably stack, scoping in html,
    in hint only: nested SQL queries.
\begin{Definition}[JETEM]
  JETEM is a variant of pushdown automata that maintains two
    stacks:
    \begin{description}
      \item[Call stack]
      \item[Epsilon stack]
    \end{description}
    At each step of the automaton, it reads a single character,
      and determines based on the character and the current top of the call stack, what to do:
    \begin{description}
      \item[Epsilon] if character is not in first, but is in follow
      \item[Call] if character is in first but not in follow
      \item[Reject] if character is not in first nor in follow
      \item[Accept] when input exhausted.
    \end{description}
\end{Definition}

\begin{figure}[H]
  \begin{Grammar}
    \begin{aligned}
      \<BNF> & ::= \<Header>\~\<Body>\~\<Footer> \hfill⏎
      \<Header> & ::= \<Variables> \~\<Terminals> \hfill⏎
      {} & \| \<Terminals> \~\<Variables> \hfill⏎
      \<Variables> & ::= \cc{with(Class<? \kk{extends} Variable>)}\hfill⏎
      \<Terminals> & ::= \cc{with(Class<? \kk{extends} Terminal>)}\hfill⏎
      \<Body> & ::= \<Start> \~\<Rules> \hfill⏎
      \<Start> & ::= \cc{start(\<Variable>)} \hfill⏎
      \<Rules> & ::= \<Rule> \~\<Rules>\hfill⏎
      {} & \| \<Rule> \hfill⏎
      \<Rule> & ::= \cc{derives(\<Variable>)} \<Conjunctions>\hfill⏎
      \<Conjunctions> & ::= \<First-Conjunction>\~\<Extra-Conjunctions>\hfill⏎
      \<First-Conjunction> & ::= \cc{to(\<Symbol>)}\~\<Symbol-Sequence>\hfill⏎
      {} & \| \cc{toNone()}\hfill⏎
      \<Extra-Conjunctions> & ::= \<Extra-Conjunction>\~\<Extra-Conjunctions>\hfill⏎
      {} & \| ε\hfill⏎
      \<Extra-Conjunction> & ::= \cc{or(\<Symbol>)}\~\<Symbol-Sequence>\hfill⏎
      {} & \| \cc{orNone()} \hfill⏎
      \<Symbol-Sequence> & ::= ε \hfill⏎
      {} & \| \cc{and(\<Symbol>)}\~\<Symbol-Sequence> \hfill⏎
      \<Symbol> & ::= \cc{Variable} \hfill⏎
      {} & \| \<Verb>\hfill⏎
      {} & \| \<Verb>~\cc{,} \<Noun> \hfill⏎
      \<Noun> & ::= \<Variable> \hfill⏎
      {} & \| \<Existing-Class> \hfill⏎
      \<Variable> & ::= \cc{Variable} \hfill⏎
      \<Footer> & ::= \cc{go()}\hfill⏎
    \end{aligned}
  \end{Grammar}
  \caption{A BNF grammar for defining BNF grammars}
  \label{figure:BNF:BNF}
\end{figure}
