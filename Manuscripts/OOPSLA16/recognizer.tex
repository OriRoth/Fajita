\renewcommand{\LET}[2]{\STATE{\textbf{Let} \ensuremath{\text{#1}←\text{#2}}}}
\newcommand\INPUT\REQUIRE
\newcommand\OUTPUT\ENSURE
\def\function#1(#2){\ensuremath{\textrm{\textup{\textsf{#1}}}(#2)}}
\def\table#1[#2]{\ensuremath{\textsf{#1}[#2]}}

\begin{algorithm}[p]
  \caption{\label{Algorithm:LLClosure}
  Function~$\function Closure(A,b)$: generates a closure of action from the original LL algorithm}
  \begin{algorithmic}
    \INPUT{a nonterminal~$A$}
    \INPUT{a terminal~$b$}
    \OUTPUT{the closure of consecutive actions}
    \LET{$l$}{$[]$}
    \LET{$X$}{$A$}
    \IF{$∀X→α .  b∉\function First(α)$}
      \RETURN{\textrm{reject}}
    \ENDIF
    \WHILE{$\textrm{True}$}
      \IF{$X∈Ξ$}
        \STATE{let~$Y₁ Y₂… Yₖ$ be~$α$ s.t.~$∃X→α∈G∧b∈\function First(α)$}
        \STATE{$\function append(l,Yₖ,Y_{k-1},…,Y₂)$}
        \STATE{$X=Y₁$}
      \ELSE[$X$ must be~$b$]
        \STATE{\textbf{break}}
      \ENDIF
    \ENDWHILE
    \RETURN{$l$}
  \end{algorithmic}
\end{algorithm}

\begin{Definition}[JETEM]
  JETEM is a variant of pushdown automata that maintains two
    stacks:
    \begin{description}
      \item[Call stack]
      \item[Epsilon stack]
    \end{description}
    At each step of the automaton, it reads a single character,
      and determines based on the character and the current top of the call stack, what to do:
    \begin{description}
      \item[Epsilon] if character is not in first, but is in follow
      \item[Call] if character is in first but not in follow
      \item[Reject] if character is not in first nor in follow
      \item[Accept] when input exhausted. 
    \end{description}


  \begin{enumerate}
      \item 

\end{Definition}
