%! TEX root = 00.tex
\begin{enumerate}
  \item Gutterman
  \item Reference to emulation of DFA. We can only do
        a constant number of moves in each iteration.
  \item~$aⁿbⁿcⁿ$
  \item \Prolog Unification.
  \item We investigated wildcards, intersection types and
        (the rather limited) type inference of 
        Java, but failed to employ these to increase the expressive power of the computation model.
  \item Ecoop results.
  \item Lina reduction
\end{enumerate}

Lina's reduction size :Assume a grammar~$G=⟨Σ,Ξ, P⟩$.
    \begin{enumerate}
      \item On the first step, we transform the grammar to a DPDA~$⟨Q,Γ⟩$ (LL(1) - linear grow, LR(1) - exponential blowup)
      \item For an arbitrary~$k$ the size of stack symbols turns to~$Γᵏ$.
      \item The number of states grow to~$Q⨉Γᵏ$.
      \item The number of stack symbols grow to~$Γᵏ⨉(Q²⨉Γ^{2k})^{Q⨉Γᵏ}$
    \end{enumerate}

% Taken from proof.tex
On a first sight, the proof of \cref{theorem:Gil-Levy} could follow the techniques
  sketched in \cref{section:toolkit} to type encode a DPDA (\cref{definition:DPDA}).
The partial transition function~$δ$ may be type encoded as in \cref{figure:simple-binary},
and the stack data structure of a DPDA can be encoded as in \cref{figure:stack-encoding}.

The techniques however fail with~$ε$-transitions,
  which allow the automaton to move between an unbounded number of
  configurations and maneuver the stack in a non-trivial manner,
  without making any progress on the input.
The fault in the scheme lies with compile time computation being carried out
  by the~$\textsf{java}(σ)()$ functions, each converting
  their receiver type to the type of the receiver of the next call in the chain.
We are not aware of a \Java type encoding which makes
  it possible to convert an input type into an output type, where
  the output is computed from the input by an unbounded number of steps.
  †{With the presumption that the \Java compiler halts for all inputs (a presumption that does
    not hold for e.g., \CC, and was never proved for \Java), the claim that there is no \Java 
    type encoding for all DPDAs can be proved:
    Employing~$ε$-transitions, it is easy to construct an automaton~$A^∞$ that
    never halts on any input.
    A type encoding of~$A^∞$ creates programs that send the compiler in an infinite loop.
  }

The literature speaks of finite-delay DPDAs, in which the number
  of consecutive~$ε$-transitions is uniformly bounded and even of
  realtime DPDAs in which this bound is 0, i.e., no~$ε$-transitions.
Our proof relies on a special kind of realtime automata,
  described by Courcelle~\cite{Courcelle:77}.

\begin{Definition}[Simple-Jump Single-State Realtime Deterministic Pushdown Automaton]
  \label{definition:JDPDA}
  \slshape
  A \textit{simple-jump, single-state, realtime deterministic pushdown automaton}
  (jDPDA, for short) is a triplet~$⟨Γ,γ₁,δ⟩$
  where~$Γ$ is a set of stack elements,~$γ₁∈Γ$ is the initial stack element,
  and~$δ$ is the \emph{partial transition function},~$δ:Γ⨉Σ↛Γ^*∪j(Γ)$,
  \[
    j(Γ) = ❴ \textit{instruction \textup{\textsf{jump}}}(γ) \; | \;γ∈Γ❵.
  \]
  A configuration of a jDPDA is some~$c∈Γ^*$ representing the stack contents.
  Initially, the stack holds~$γ₁$ only.
  For technical reasons, assume that the input terminates with~$\$ \not∈Σ$, a special end-of-file character.
  \begin{itemize}
    \item At each step a jDPDA examines~$γ$,
    the element at the top of the stack,
    and~$σ∈Σ$, the next input character,
    and executes the following:
          \begin{quote}
            \begin{enumerate}
              \item consume~$σ$
              \item if~$δ(γ,σ)=ζ$,~$ζ∈Γ^*$, the automaton pops~$γ$, and pushes~$ζ$ into the stack.
              \item if~$δ(γ,σ)=\textsf{jump}(γ')$,~$γ'∈Γ$, then the automaton repetitively
                    pops stack elements up-to and including the first occurrence of~$γ'$.
            \end{enumerate}
          \end{quote}
    \item If the next character is~$\$$, the automaton may reject or accept (but nothing else),
          depending on the value of~$γ$.
  \end{itemize}
  In addition, the automaton rejects if~$δ(γ,σ) =⊥$ (i.e., undefined), or if it encounters
  an empty stack (either at the beginning of a step or on course of a \textsf{jump operation}).
\end{Definition}

\begin{table}[H]
  \caption{\label{table:A} The transition function of a jDPDA~$A$,~$Σ=❴σ₁,σ₂,σ₃❵$,~$Γ=❴γ₁,γ₂❵$ where~$γ₁$ is the initial element}
  \begin{tabular}{c|ll}
  & \hfill~$γ₁$ & \hfill~$γ₂$⏎
    \midrule
$σ₁$ & $\textsf{push}(γ₁,γ₁,γ₂)$ & $\textsf{push}(γ₂,γ₂)$⏎
$σ₂$ & \hfill$⊥$ & $\textsf{push}(ε)$⏎
$σ₃$ & \hfill$⊥$ & $\textsf{jump}(γ₁)$⏎
$\$$ & \hfill$\textsf{accept}$ & $\textsf{reject}$⏎
  \end{tabular}
\end{table}

As it turns out, every DCFG language is recognized by some jDPDA, and
conversely, every language accepted by a jDPDA is a DCFG
language~\cite{Courcelle:77}.  The proof of \cref{theorem:Gil-Levy} is
therefore reduced to type-encoding of a given jDPDA\@.  Towards this end, we
employ the type-encoding techniques developed above, and, in particular, the
jump-stack data structure (\cref{figure:jump}).

\subsection{Main Types}
Generation of a type encoding for a jDPDA starts with two empty types for sets~$L$,~$Σ^*$,
  where~$L$ represents the languages accepted by the jDPDA and~$Σ^*$ represents all words:
\begin{quote}
  \javaInput[minipage,width=53ex,left=-2ex]{proof.headers.listing}
\end{quote}
(The full type encoding is in \cref{figure:A} below; to streamline the reading, we bring
  excerpts as necessary.)

A configuration is encoded by a generic type~\cc{C}.
Essentially,~\cc{C} is a representation of the stack,
  but~$k+1$ type parameters are required:
\begin{itemize}
  \item \cc{Rest}, a type encoding of the stack after a pop (or \textsf{jump} with the top element), and,
  \item $k$ types, named \cc{JR$γ$1}, … ,\cc{JR$γ${}$k$}, encoding the type of \cc{Rest}
    after~$\textsf{jump}(γ₁)$,…~$\textsf{jump}(γₖ)$.
\end{itemize}

Note that these~$k+1$ parameters are sufficient for describing a configuration,
  i.e., if the top is~$γⱼ$, then for all~$j≠i~$
\[
  \textsf{jump}(γⱼ) = \cc{Rest.}\textsf{jump}(γⱼ)
\]
In the special case of~$\textsf{jump}(γᵢ)$ the returned type is still~$\cc{Rest}$,
  this is due to the fact that before a \textsf{jump} operation,
  we do not pop an element from the stack.

All instantiations of~\cc{C} must make sure that actual parameters are properly constrained,
  to ensure that they are (the type version of) pointers into the actual stack,
  not a trivial task, as will be seen shortly.

In the running example,~\cc{C} is defined as:
\begin{quote}
  \javaInput[minipage,width=61ex,left=-2ex]{proof.configuration.listing}
\end{quote}
This excerpt shows also classes~\cc{E} and~\cc{¤} which encode (as in \cref{figure:jump})
  the empty and the error configurations.

Type~\cc{C} defines~$ℓ+1$ functions (4 in the example), one for each possible input character,
  and one for the end-of-file character defined as \$.
Since~\cc{C} encodes an abstract configuration, return types of functions in it
  are the appropriate defaults which intentionally fail to emulate the automaton's execution.
  The return type of \cc{\$()} is \cc{ΣΣ} (rejection);
  the transition functions \cc{$σ$1()}, … \cc{$σ${}$ℓ$()}, return the raw type~\cc{C}.

\subsection{Top-of-Stack Types}

Types \cc{C$γ$1}, … ,\cc{C$γ${}$k$}, specializing~\cc{C},
  encode stacks whose top element is~$γ₁$, … ,$γₖ$.
In~$A$ there are two of these:
\begin{quote}
  \javaInput[minipage,width=42ex,left=-4ex]{proof.many.listing}
\end{quote}

In~$A$, types \cc{C$γ$1} and \cc{C$γ$2} take three parameters;
in general ‟Top of Stack" types take the aforementioned~$k+1$ parameters.

\begin{figure*}
  \caption{\label{figure:chain} Accepting and non-accepting call chains with the
  type encoding of jDPDA~$A$ (as defined in \cref{table:A}).
  All lines in \cc{accepts()} type-check, while all lines in \cc{rejects()} do not type-check.}
  \javaInput[minipage,left=-2ex,width=45ex]{proof.cases.listing}
\end{figure*}

The method signatures of these types are generated using the mentioned parameters.
The generating of methods will be discussed next.

The code defines the \kk{static} variable \cc{build}, the starting point
of all fluent API call chains, to be of type \cc{C$γ$1<E,¤,¤>}, i.e.,
  the starting configuration of the automaton is a stack whose top is~$γ₁$,
  and its \cc{Rest} parameter is empty (\cc{E}).
Any of the two jumps possible on this rest results with,~\cc{¤},
  an undefined stack.
Examples of accepting and rejecting call chains starting at \cc{A.build}
  can be seen in \cref{figure:chain}.

\subsection{Transitions}
It remains to show the type encoding of~$δ$,
  the transition function.
Overall, there are a total of~$k·(ℓ+1)$
  entries in a transition table such as \cref{table:A}.
Conceptually, these are encoded by selecting the correct return
  type of functions \cc{$σ$1()}, … ,\cc{$σ${}$k$()} and \cc{\$()} in each
  of the~$k$ ‟Top of Stack” types.
Thanks to inheritance, we need to do so only in the cases that this
  return type is different from the default.

Overall, there are six kinds of entries in a transition table:
\begin{description}

  \item[\textsf{reject}]
  The default return type of \cc{\$()} in~\cc{C} is \cc{$ΣΣ$}, which
  is \emph{not} a subtype of~\cc{L}. Normally the result of a call chain that ends with \cc{\$()}
  cannot be assigned to a variable of type~\cc{L}. Moreover, since \cc{$ΣΣ$} is \kk{private},
  there is little that clients can do with this result.

  \item[\textsf{accept}]
  The only case in which fluent call chain ending with \cc{\$()} can return
    type~\cc{L} is when the type returned of the call just prior to~\cc{.\$()} covariantly
    changes the return type of~\cc{\$()} to~\cc{L}.†{This is not to be confused with dynamic binding;
    types of fluent API call chains are determined statically.}
  \par
  Recall that a jDPDA can only accept after its input is exhausted.
  In \cref{table:A} we see that \textsf{accept} occurs when the top of the stack is~$γ₁$.
  We therefore add to the body of type \cc{C$γ$1} the line
  \begin{JAVA}
@Override L ¢\gobble$¢$();
  \end{JAVA}

  \item[$⊥$]
  When a prefix of the input is sufficient to conclude it must be rejected however it continues,
    the transition function returns~$⊥$.
  In~$A$ this occurs when the top of the stack is~$γ₁$ and one of~$σ_2$ or~$σ_3$ is read.
  To type encode~$δ(γ₁,σ₂) =⊥$, one must \emph{not} override \cc{$σ$2()} in type~\cc{C$γ$1};
    the inherited return type (l.15 \cref{figure:A}) is the raw~\cc{C}.
  Subsequent calls in the chain will all receive and return a raw~\cc{C}
    (Recall that all \cc{$σ${}$i$()},~$i=1,…,ℓ$, are functions in~\cc{C} that return a raw~\cc{C}).
  Therefore, the final \cc{\$()} will reject.
  \par
  Two other situations in which a jDPDA rejects but not demonstrated in~$A$ are:
    a \textsf{jump} that encounters an empty stack, and reading a character from when the stack is empty.
  In our type encoding these are handled by the special
    types~\cc{E} and~\cc{¤} (ll.17--18 ibid), both extend~\cc{C} without
    overriding any of its methods. Again, remaining part of the call chain will stick to
    raw~\cc{C}s up until the final \cc{\$()} call rejects the input.

  \item[$\textsf{jump}(γᵢ)$]
  The design of the generic parameters makes the implementation of~$\textsf{jump}(γᵢ)$
    operations particularly simple.
  All that is required is to covariantly change the return type of the
    appropriate \cc{$σ${}$j$()} function to the appropriate \cc{JR$γ{}i$} or~\cc{Rest} parameter
    (recall that a jump occurs after popping the current element from the stack, so % or is it now TOMER XXXXXXXXXXXXXXXXXX
    we refer to \cc{JR} type parameters rather than~\cc{J}'s).
  \par
  In \cref{table:A} we find that~$δ(γ₂,σ₃) =\textsf{jump}(γ₁)$. 
  Accordingly, the type of \cc{$σ$2()} in \cc{C$γ$2} (l.34) is \cc{JR$γ$1}.

  \item[$\textsf{push}(ζ)$]
  Push operations are the most complex, since they involve a pop of the top stack element,
    and pushing any number, including zero, of new elements.
  The challenge is in constructing the correct~$k+1$-parameter instantiation of~\cc{C},
    from the current parameters of the type.
  Each of these~$k+1$ is also an instantiation of~\cc{C} which may require more such
    parameters.
  Even though the number of ingredients is small, the resulting type expressions
    tend to be excessively long and unreadable.
  \par
  The predicament is ameliorated a bit by the idea,
    demonstrated above with auxiliary type~\cc{Pʹ}
    (\cref{figure:jump-stack-push}),
    of delegating the task of creating a complex type to an auxiliary
    generic type.
  The task of this sidekick is simplified if some of its generic
  parameters are sub-expressions that recur in the desired
  result.
  \par
  Cases in point
    are~$δ(γ₁,σ₁)=\textsf{push}(γ₁,γ₁,γ₂)$, and~$δ(γ₂,σ₁)=\textsf{push}(γ₂,γ₂)$ of \cref{table:A}.
  The corresponding sidekick types,
    (\cc{$γ$1$σ$1\_Push\_$γ$1$γ$1$γ$2} and \cc{$γ$2$σ$1\_Push\_$γ$2$γ$2})
    can be found in lines 36--43 of \cref{figure:A}.
  The first of these define the correct return type
    of \cc{$σ$1()} in case \cc{$γ$1} is the top element,
    the second of \cc{$σ$2}, in case \cc{$γ$2} is the top element.
  Examine now the definition of types \cc{C$γ$1},\cc{C$γ$2} in the figure,
    and in particular lines 21--23 and 29--31 which define the list of types they extend.
  Notice that each extends one of the sidekicks, inheriting the covariant
    overrides of \cc{$σ$1()}.
  \par
  More generally, economy of expression may require that for each case
    of~$δ(γ,σ)=\textsf{push}(ζ)$ in the transition table,
    one creates a sidekick type which overrides the appropriate \cc{$σ$()}
    function.
  The appropriate \cc{C$γ$} type then inherits the definition
    from the sidekick.
\end{description}

\begin{figure*}[htbp]
  \caption{\label{figure:A}Type encoding of jDPDA~$A$ (as defined in \cref{table:A})}
  \javaInput[minipage,listing style=numbered,width=1.08\columnwidth]{proof.full.listing}
\end{figure*}

\paragraph*{Conclusion} The proof of~\Cref{theorem:Gil-Levy} is an algorithm, taking as input some jDPDA,
  and returning as output a set of \Java type definitions.
The returned types, allow a call chain~$\textsf{java}(\alpha)$,
  such that the type of the returned object represents the 
  configuration of the input automaton after reading~$\alpha$.
If the automaton rejects after~$\alpha$, then the returned type is the illegal~$\Sigma\Sigma$, 
  and if the automaton accpets, the type shall be~$L$.


% Taken from prefix.tex

\begin{Theorem}\label{theorem:Gil-Levy:2}
  Let~$A$ be a DPDA recognizing a language~$L⊆Σ^*$.
  Then, there exists a \Java type definition,~$J_A$ for types~\cc{L},~\cc{A},~\cc{C} and
    other types such that the \Java command
  \begin{equation}
    \label{equation:prefix:result}
    \cc{C c = A.build.$\textsf{java}(α)$;}
  \end{equation}
    type checks against~$J_A$ if an only if there exists~$β∈Σ^*$ such
    that~$αβ∈L$ and type \cc{C} is the configuration of~$A$ after reading~$α$.
  Furthermore, for any such~$β$,~\cref{theorem:Gil-Levy} applies such that the
  \Java command
  \begin{equation}
    \cc{L~$ℓ$ = A.build.$\textsf{java}(αβ)$}\cc{.\$();}
  \end{equation}
    always type-checks.
  Finally, the program~$J_A$ can be effectively generated from~$A$.
\end{Theorem}

Informally, a call chain type-checks if and only if it is a prefix
  of some legal sequence.
Alternatively, a call chain won't type-check if there is no
  continuation that leads to a legal string in~$L$.

The proof resembles~\cref{theorem:Gil-Levy}'s proof.
We provide a similar implementation for a jump-stack
  †{recall that the two formal constructs are of the same expressiveness},
  that will not compile under illegal prefixes.

The main difference between the two theorems is:
  in~\cref{theorem:Gil-Levy} we allowed illegal call chains to compile,
  but not return the required~\cc{L} type, while in~\cref{theorem:Gil-Levy:2}
  the illegal chain won't compile at all.

Since the code suggested by the proof highly resembles the previously
  suggested code, mainly the differences will be discussed.

We will use the same running example, defined by~\cref{table:A}.

\subsection{Main Types}
The main types here are a subset of the previously defined main types.

\begin{quote}
  \javaInput[minipage,width=45ex,left=-2ex]{prefix-proof.configuration.listing}
\end{quote}

First, type \cc{$ΣΣ$} is removed.
A call chain that doesn't represent a valid prefix won't compile,
  thus, there is no need for an error return type such as \cc{$\SigmaΣ$}.
Second, \kk{interface}~\cc{C} is removed.
Without it, the configuration types won't have the
  methods \cc{$σ$1()}, … ,\cc{$σ${}$k$()} and \cc{\$()} from the supertype.
These inherited methods, is what differentiates the previous proof from the current.
Classes \cc{¤} and \cc{E} are defined similarly, except now they don't extend any type.

\subsection{Top-of-Stack Types}
Types \cc{C$γ$1}, … ,\cc{C$γ${}$k$}, still represent stacks
  with \cc{$γ$1}, … ,\cc{$γ${}$k$} as their top element,
  this time, the methods are defined ad-hock, in each type
  (they are not added in this figure as they are added with the use of sidekicks).
In~$A$ there are two such types:

\begin{quote}
  \javaInput[minipage,width=51ex,left=-2ex]{prefix-proof.many.listing}
\end{quote}


Note, that the type parameters of the former types hasn't changed,
  since the model we are trying to implement, hasn't changed.
  These~$k+1$ parameters still suffice for our cause.

\begin{figure*}
  \caption{\label{figure:prefix-chain} Accepting and non-accepting call chains with the
  type encoding of jDPDA~$A$ (as defined in \cref{table:A}).
  All lines in \cc{accepts} type-check, and all lines in \cc{rejects}
  cause type errors}
  \javaInput[minipage,width=42ex,left=-2ex]{prefix-proof.cases.listing}
\end{figure*}

In~\cref{figure:prefix-chain}, call chains in the~\cc{accepts()} method
  correctly type-checks (i.e., in~$L$), while the chains in~\cc{rejects()}
  do not type-check (i.e., these prefixes have no continuation that can lead to a legal word in~$L$),
  where the last method invocation generates an
\begin{quote}
  ‟\textsf{method~…~is undefined for the type~…~}”
\end{quote}
  error message.

The main difference between~\cref{figure:prefix-chain} and~\cref{figure:chain} is that there is no need to
  use an auxiliary function \cc{isL()} as in~\cref{figure:prefix-chain} since now illegal
  prefixes do not type-check.

\subsection{Transitions}
Due to the changes we expressed, the transition table is encoded slightly different.

Encoding of the legal operations \textsf{accept},~$\textsf{jump}(γᵢ)$ and~$\textsf{push}(ζ)$
  remains as in~\cref{theorem:Gil-Levy}, since we want the same behavior for legal call chains.
The minor differences are in the illegal operations \textsf{reject} and~$⊥$:

\begin{description}
 \item[\textsf{reject}] Since we add the methods ad-hock to each type, the reject entry means
   that the corresponding type, \emph{won't} have a~\cc{\$()} method, i.e., type~\cc{C$γ$2}
   doesn't have a method~\cc{\$()}.
 \item[$⊥$] We encounter~$⊥$ on the transition function when some input character~\cc{$σ$}
   is not allowed for the top of the stack element~$γ$. In that case, the corresponding type \cc{C$γ$}
   \emph{must not} have a method for~\cc{$σ$}, this way, invoking the methods will result in type error.
   In \cref{table:A} a~$⊥$ may occur when the top of the stack is~$γ₁$ and the input character is~$σ₂$,
   thus, no method \cc{$σ$2} is introduced in type~\cc{C$γ$1}.
\end{description}

The use of sidekicks is still allowed and recommended to improve readability of code.

\begin{figure*}
  \caption{\label{figure:prefix-A}Type encoding of jDPDA~$A$ (as defined in \cref{table:A})
    that allow a partial call chain, if and only if, there exists a legal continuation, that
    leads to a word in~$L$ (the language of~$A$}
  \javaInput[minipage,listing style=numbered,width=\textwidth]{prefix-proof.full.listing}
\end{figure*}

\paragraph*{Conclusion}
In this section, a proof, similar to the one in~\cref{section:proof} is provided.
An algorithm was introduced, to not only emulate the running of some jDPDA~$A$,
  but also to ‟halt it” in the earliest time possible, i.e., only if there is
  no legal call chain from this point to result in a legal word in the language of~$A$.

% Taken from result.tex

Let~$\textsf{java}$ be a function that translates a terminal~$σ∈Σ$
into a call to a uniquely named function (with respect to~$σ$).
Let~$\textsf{java}(α)$, be the function
  that translates a string~$α∈Σ^*$ into a fluent API call chain.
  If~$α=σ₁⋯σₙ∈Σ^*$, then \[
  \textsf{java}(α)=\textsf{java}(σ₁)\cc{().}⋯\cc{.}\textsf{java}(σₙ)\cc{()}
\]
For example, when~$Σ=❴a,b,c❵$ let~$\textsf{java}(a)=\cc{a}$,~$\textsf{java}(b)=\cc{b}$, and,~$\textsf{java}(c)=\cc{c}$.
With these, \[
    \textsf{java}(caba) = \cc{c().a().b().a()}
  \]

\begin{Theorem}\label{theorem:Gil-Levy}
  Let~$A$ be a DPDA recognizing a language~$L⊆Σ^*$.
  Then, there exists a \Java type definition,~$J_A$ for types~\cc{L},~\cc{A} and
    other types such that the \Java command
  \begin{equation}
    \label{equation:result}
    \cc{L~$ℓ$ = A.build.$\textsf{java}(α)$}\cc{.\$();}
  \end{equation}
  type checks against~$J_A$ if an only if~$α∈L$.
  Furthermore, program~$J_A$ can be effectively generated from~$A$.
\end{Theorem}

\Cref{equation:result} reads: starting from the \kk{static} field \cc{build} of \kk{class}~\cc{A},
  apply the sequence of call chain~$\textsf{java}(α)$, terminate with a call to the
  ending character~\cc{\$()} and then assign to newly declared \Java variable~\cc{$ℓ$} of type~\cc{L}.

The proof of the theorem is by a scheme for encoding in \Java types
  the pushdown automaton~$A=A(L)$ that recognizes language~$L$.
Concretely, the scheme assigns a type~$τ(c)$
  to each possible configuration~$c$ of~$A$.
Also, the type of \cc{A.build} is~$τ(c₀)$, where~$c₀$ is the initial configuration of~$A$,

Further, in each such type the scheme places
  a function~$σ()$ for every~$σ∈Σ$.
Suppose that~$A$ takes a transition from configuration~$cᵢ$ to configuration~$cⱼ$
  in response to an input character~$σₖ$.
Then, the return type of function \cc{$σₖ$()} in type~$τ(cᵢ)$ is type~$τ(cⱼ)$.

With this encoding the call chain in \cref{equation:result}
  mimics the computation of~$A$, starting at~$c₀$ and ending with
  rejection or acceptance.
The full proof is in \cref{section:proof}.

Since the depth of the stack is unbounded, the number of configurations of $A$ is unbounded,
  and the scheme must generate an infinite number of types.
Genericity makes this possible, since a generic type is
  actually device for creating an unbounded number of types.

There are several, mostly minor, differences between the structure of the \Java code
in \cref{equation:result}
and the examples of fluent API we saw above, e.g., in \cref{figure:DSL}:
\begin{description}
  \item[Prefix, i.e., the starting \cc{A.build} variable.]
  All variables and functions of \Java are defined within a class.
  Therefore, a call chain must start with an object (\cc{A.build} in \cref{equation:result})
  or, in case of \cc{static} methods, with the name of a class.
  In fluent API frameworks this prefix is typically eliminated
  with appropriate \cc{import} statements.
  \par
  If so desired, the same can be done by our type encoding scheme: define all
  methods in type~$τ(c₀)$ as \cc{static} and \cc{import static} these.
  \item[Suffix, i.e., the terminal \cc{.\$()} call.]
  In order to know whether~$α∈L$ the automaton recognizing~$L$ must
  know when~$α$ is terminated.
  \par
  With a bit of engineering, this suffix can also be eliminated.
  One way of doing so is by defining type~\cc{L} as an \kk{interface}, and by making all types~$τ(c)$,~$c$ is
  an accepting configuration, as subtype of~\cc{L}.
  \item[Parameterized methods.]
  Fluent API frameworks support call chains with phrases such as:
  \begin{itemize}
    \item ‟\lstinline{.when(header(foo).isEqualTo("bar")).}”,
    \item ‟\lstinline{.and(BOOK.PUBLISHED.gt(date("2008-01-01"))).}”, and,
    \item ‟\lstinline{.allowing(any(Object.class)).}”.
  \end{itemize}
  while our encoding scheme assumes methods with no parameters.  
  \par
    Methods with parameters contribute to the user
      experience and readability of fluent APIs but their ``computational expressive power"' is the same.
      In fact, extending
      \cref{theorem:Gil-Levy} to support these requires these conceptually simple steps 
      \begin{enumerate}
        \item Define the structure of parameters to methods with appropriate fluent API, which may or
          may not be, the same as the fluent API of the outer chain, or the fluent API of parameters to
          other methods. Apply the theorem to each of these fluent APIs.
        \item
          If there are several overloaded versions of a method, consider each such version as a distinct
          character in the alphabet~$Σ$ and in the type encoding of the automaton.
        \item
          Add code to the implementation of each method code to store the 
          store the value of its argument(s) in a record placed at the end of the fluent-call-list. 
      \end{enumerate}
\end{description}
