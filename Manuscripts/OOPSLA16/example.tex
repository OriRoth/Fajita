%! TEX root = 00.tex
Fluent APIs is related to the topic of type-states.
There is a large body of research on \emph{type-states} (see
e.g., these review articles~\cite{Aldrich:Sunshine:2009,Bierhoff:Aldrich:2005})
Informally, an object that belongs to a certain type (\kk{class} in the object
oriented lingo), has type-states, if not all methods defined in this object's
class are applicable to the object in all states it may be in.

File object is the classical example: It can be in one of two states: ‟open” or
‟closed”. Invoking a \cc{read()} method on the object is only permitted when
the file is in an ‟open” state. In addition, method \cc{open()} (respectively
\cc{close()}) can only be applied if the object is in the ‟closed”
(respectively, ‟open”) state.

A recent study~\cite{Beckman:11} estimates that about~$7.2%$ of \Java
classes define protocols definable in terms of type-states.
This non-negligible prevalence raise two challenges:
\begin{enumerate}
  \item \emph{\textbf{Identification.}} Frequently, type-state receive little
    or no mention at all in the documentation. The challenge is in identifying
    the implicit type state in existing code.
    \par
    Specifically, given an implementation of a class (or more generally of a
    software framework), \emph{determine} which sequences of method calls are
    valid and which violate the type state requirement presumed by the
    implementation. \item \emph{\textbf{Maintenance and Enforcement.}} Having
    identified the type-states, the challenge is in automatically flagging out
    illegal sequence of calls that does not conform with the type-state.
    \par
    Part of this challenge is maintenance of these automatic flagging
    mechanisms as the type-state specification of the API evolves.
\end{enumerate}

\subsection{A Type State Example}

An object of type \cc{Seat}†
{%
  example inspired by earlier work of Richard Harter on the
  topic~\cite{Harter:05}.
}
is created in the \cc{down} state, but it can then be \cc{raise}d to the
\cc{up} state, and then be \cc{lower}ed to the \cc{down} state.
Such an object can be used by two kinds of users, \cc{male}s and \cc{female}s, for
two distinct purposes: \cc{urinate} and \cc{defecate}.

Thus, there are a total of six methods that might be invoked
on an instance of \cc{Seat}:
\begin{quote}
  \begin{tabular}{lll}
    \cc{male()} & \cc{raise()} & \cc{urinate()}⏎
    \cc{female()} & \cc{lower()} & \cc{defecate()}⏎
  \end{tabular}
\end{quote}
A fluent API design specifies the order in which such calls can be made.
For example, a fluent API should recognize the sequences of
  \cref{figure:toilette:legal} as being type correct.
\begin{figure}[H]
  \caption{\label{figure:toilette:legal}
    Legal sequences of calls in the toilette seat example}
  \javaInput[minipage,width=\linewidth,left=-6ex]{toilette.legal.listing}
\end{figure}

At the same time, illegal sequences as made in
\cref{figure:toilette:illegal} should be signaled as type errors.

\begin{figure}[H]
 \caption{\label{figure:toilette:illegal}
   Illegal sequences of calls in the toilette seat example}
  \javaInput[minipage,width=\linewidth,left=-6ex]{toilette.illegal.listing}
\end{figure}

To generate a fluent API that meets these requirements, one must observe that
the protocol of a \cc{Seat} is defined completely by
  the DFA depicted in \cref{figure:type-state-automaton}.

\begin{figure}[H]
  \caption{\label{figure:type-state-automaton}%
    A deterministic finite automaton realizing the type states of
     class \cc{Seat}.
  }
  \input ../Figures/toilete-dfa.tikz
\end{figure}

Having constructed the automaton, generating the classes and methods is immediate
by following the \emph{encoding convention}:
\begin{enumerate}
  \item A state~$q$ in the automaton is encoded as an \kk{interface}.%
        †{%
          The name of this interface is arbitrary; all that it is required is that
          names assigned to distinct states are distinct.
          For readability sake, there is often an attempt to choose a name
          which makes a close approximation of the name of~$q$, within
          the limits of the language's syntax and Unicode vocabulary.
        }
  \item If there is an arc labeled~$ℓ$ leading from~$qᵢ$ to~$qⱼ$,
        then \kk{interface}~$qᵢ$ defines a
        method whose return type is \kk{interface}~$qⱼ$.%
        †{%
          Again, the name of this method is often selected as an approximation of the name of~$ℓ$.
        }
  \item The initial state~$q₀$ is made special in being the only type from
        which a fluent API call chain can start.%
        †{%
          This is achieved by generating a class~$c$ with \cc{public}
          constructor that \kk{implements} the \kk{interface} encoding type~$q₀$.
          If no other interfaces do not have a similar~$c$, then clients can only
          start a fluent API chain by creating an instance of~$c$, which is
          effectively the state~$q₀$.
        }
\end{enumerate}

The result of employing these rules to our toilette example is depicted at~\cref{figure:toilette-types} 

\begin{figure}
  \caption{\label{figure:toilette-types}
    A fluent API realizing the toilette seat object defined as a DFA
    in~\cref{figure:type-state-automaton}} 
  \javaInput[minipage,width=\linewidth,left=-4ex]{toilette.types.listing}
\end{figure}

Using the implementation in~\cref{figure:toilette-types} we achieved
our goals regarding the legal usage examples in~\cref{figure:toilette:legal}
  and the illegal usage examples in~\cref{figure:toilette:illegal}.
So far.⏎⏎

It should be clear that the type checking engine of the compiler can
be employed to distinguish between legal and illegal sequences.




% From that point on, the text is copied from the research proposal
\endinput




The \emph{first} novelty in this research is that the fluent API definition is
  through a CFG, written as a BNF.
\cref{figure:BNF} is such a specification for the toilette seat problem.

\begin{figure}[H]
  \begin{Grammar}
    \begin{aligned}
      \<Visitors> & ::= \<Down-Visitors> \hfill⏎
      \<Down-Visitors> & ::= \<Down-Visitor> \~\<Down-Visitors> \hfill⏎
      {} & \| \<Raising-Visitor> \~\<Up-Visitors> \hfill⏎
      {} & \| ε \hfill⏎
      \<Up-Visitors> & ::= \<Up-Visitor> \~\<Up-Visitors> \hfill⏎
      {} & \| \<Lowering-Visitor> \~\<Down-Visitors> \hfill⏎
      {} & \| ε \hfill⏎
      \<Up-Visitor> & ::= \cc{male()} \~\cc{urinate()} \hfill⏎
      \<Down-Visitor> & ::= \cc{female()} \~\<Action> \hfill⏎
                          & \| \cc{male()} \cc{defecate()} \hfill⏎
      \<Raising-Visitor> & ::= \cc{male()} \~\cc{raise()} \~\cc{urinate()} \hfill⏎
      \<Lowering-Visitor> & ::= \cc{female()} \~\cc{lower()} \~\<Action> \hfill⏎
                          & \| \cc{male()} \~\cc{lower()} \cc{defecate()} \hfill⏎
      \<Activity> & ::= \cc{urinate()} \hfill⏎
                          & \| \cc{defecate()} \hfill⏎
    \end{aligned}
  \end{Grammar}
  \caption{A BNF grammar for the toilette seat problem}
  \label{figure:BNF}
\end{figure}

\Fajita takes this grammar specification as input, and in response
  generates the corresponding
  \Java type hierarchy.

\subsection{Verbs and Nouns}
A second novelty of \Fajita is that the specification of a BNF such as in
  \cref{figure:BNF} can be also made with a \Java fluent API\@.
To do so, it is first necessary to
  define the set of \emph{grammar terminals}
  \begin{code}{JAVA}
enum ToiletteTerminals implements Terminal {¢¢
  male, female,
  urinate, defecate,
  lower, raise;
}
\end{code}
As common in fluent APIs we shall refer to these
as \emph{verbs}†{Admittedly, the words ‟male” and ‟female” are nouns.
  An excuse might be that the words are used as nouns to mean ‟male-visit” and ‟female-visit”.}.
Verbs are translated by \Fajita into methods.

We are also required to define the set of \emph{grammar variables}
\begin{code}{Java}
enum ToiletteVariables implements Variable {¢¢
  Visitors, Down_Visitors, Up_Visitors,
  Up_Visitor, Down_Visitor,
  Lowering_Visitor, Raising_Visitor,
  Activity
};
\end{code}
  We shall use the term ‟\emph{noun}” as synonymous to ‟variable”.

\subsection{Words and Grammar}
The terms ‟symbol” and ‟word” refer to an entity which is either
  a verb or a noun.

Once the verbs and the nouns are set, the grammar can be defined,
  using a fluent API generated by \Fajita itself as shown
  in \cref{figure:fluent}.

\begin{figure}[H]
  \begin{JAVA}[style=numbered]
new BNF()
  ¢¢.with(ToiletteTerminals.class)
  ¢¢.with(ToiletteSymbols.class)
  ¢¢.start(Visitors)
  ¢¢.derive(Visitors).to(Down_Visitors)
  ¢¢.derive(Down_Visitors)
    ¢¢.to(Down_Visitor).and(Down_Visitors)
    ¢¢.or(Raising_Visitor).and(Up_Visitors)
    ¢¢.orNone()
  ¢¢.derive(Up_Visitors)
    ¢¢.to(Up_Visitor).and(Up_Visitors)
    ¢¢.or(Lowering_Visitor).and(Down_Visitors)
    ¢¢.orNone()
  ¢¢.derive(Up_Visitor).to(male).and(urinate)
  ¢¢.derive(Down_Visitor)
    ¢¢.to(female).and(Action)
    ¢¢.or(male).and(defecate)
  ¢¢.derive(Raising_Visitor).to(male).and(raise).and(urinate)
  ¢¢.derive(Lowering_Visitor)
    ¢¢.to(female).and(lower).and(Action)
    ¢¢.or(male).and(lower).and(defecate)
  ¢¢.derive(Activity)
    ¢¢.to(urinate)
    ¢¢.or(defecate)
  ¢¢.go();
  \end{JAVA}
  \caption{A BNF grammar for the toilette seat problem}
  \label{figure:fluent}
\end{figure}

The call to function \cc{go()} (last line in \cref{figure:fluent}) instructs
  \Fajita to generate the code for the fluent API specified by the
  subsequent part of the expression.
Roughly speaking, nouns are translated to classes while verbs are translated to methods which
  take no parameters.
Two exceptions apply:
\begin{enumerate}
  \item
    Library classes such as \cc{String} and \cc{Integer}, just as user-defined
    classes such as \cc{Invoice} may be used as nouns.
    \Fajita generate class definitions only for classes whose name is declared
    in an \kk{enum} which is passed to \cc{with} verb in the BNF declaration.
  \item Verbs may take noun parameters, as explained below.
\end{enumerate}
