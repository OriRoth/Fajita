%! TEX root = 00.tex

Advancing toward the algorithm, we now present a somewhat simplified
  version of the LL(1) parsing algorithm, that will help us encode it
  using \Java Generics, but will not weaken the computational model.

The simplified LL parser (SLL) is an LL parser with a ‟higher”
  perspective.
While the LL's stack symbols are the terminals and
  nonterminals of the grammar, the SLL's symbols are the grammar's
  rules.

As the LL parser, the SLL is consisted of a stack automaton
  and a prediction table†{The same table as the LL prediction table}.
The SLL makes it's decision based on the next input symbol and
  the top of the stack.
It may also decide whether to consume the input token (a feature that
  will later be revoked).

The SLL's stack symbols are ‟items” - grammar rules with an index,
  indicating what is the prefix of the rule that has already been
  parsed and what is the suffix that still needs to be parsed.
The SLL's stack symbols are similar to the famous LR items,
  both in notation and meaning.
For example, if the SLL currently parses the rule~$\<A> ::=αβ$,
  where~$α$ was already parsed, then the symbol at the top of 
  the stack will be denoted by~$\<A> ::=α·β$.

  The SLL begins by pushing the grammar start symbol's rule†{
    Of course, the grammar's start symbol might have more than one rule,
    but since the grammar is LL, each rule has a
    distinct~$\Function First(·)$ set, thus by peeking into the
    first input symbol we can choose a start rule.}
  with a dot before the first symbol, and perform in loop the
  step depicted
  in~\cref{algorithm:sll-parser}.

\begin{algorithm}[p]
  \caption{\label{algorithm:sll-parser}
  SLL's parsing algorithm step}
  \begin{algorithmic}
    \LET{$X ::=α·Yβ$}{\Function pop()}\COMMENT{what's the anticipated input?}
    \IF[if rule is already parsed]{$|Yβ|=0$}
      \IF{$X$ is a start symbol}
        \IF[not the anticipated end-of-input]{$\$≠\Function next()$}
          \STATE{\textsc{Reject}}\COMMENT{SLL halts in error}
        \ELSE
          \STATE{\textsc{Accept}}\COMMENT{all input successfully consumed}
        \FI
      \FI
      \CONTINUE\COMMENT{restart, popping a new rule etc.}
    \FI
    \STATE{$\Function push(X::=αY·β)$}\COMMENT{push following item}
    \IF[$Y$ is a terminal]{$Y∈Σ$}
      \IF[read terminal is not anticipated~$Y$]{$Y≠\Function next()$}
        \STATE{\textsc{Reject}}\COMMENT{SLL halts in error}
      \ELSE[read terminal is the anticipated~$Y$]
        \CONTINUE\COMMENT{restart, popping a new~$X::=…$, etc.}
      \FI
    \ELSE[$Y$ must be a nonterminal]
      \LET{$R$}{$\Function δ(Y, \Function peek())$}\COMMENT{which rule to reduce?}
      \IF[no rule found]{$R=⊥$}
        \STATE{\textsc{Reject}}\COMMENT{SLL halts in error}
      \ELSE[A rule was found]
        \LET{$(Z ::= Y₁,…,Yₖ)$}{R}\COMMENT{break~$R$ into left/right}
        \STATE{$\Function print(R)$}\COMMENT{rule~$R$ has just been applied}
        \STATE{$\Function push(Z ::=·Y₁,…,Yₖ)$}\COMMENT{push the next rule}
        \CONTINUE\COMMENT{restart, popping a new rule etc.}
      \FI % No rule found
    \FI % Y is a terminal or a nonterminal
  \end{algorithmic}
\end{algorithm}

The SLL maintains a stack of derivation rules that are being parsed
  at the given moment, each rule has a ‟$·$” somewhere in its RHS,
  that splits the rule to the prefix that was already parsed (can
  be empty) and the suffix that is anticiped to be parsed (can be
  empty).

The parsing step of the SLL always begins with popping a rule
  from the stack.
  Upon popping the rule
  \[
    \<A> ::= Y₀…Y_{i-1}·Yᵢ…Yₖ
  \]
 the SLL operates as follows:
\begin{enumerate}
  \item
    If~$\<A>$ is a start symbol, and~$i-1=k$ (i.e., it was fully parsed)
      then the SLL expects the end-of-input symbol~$\$$ at the input
      string ; matching it will cause the SLL to accept, and failing
      to match it will cause rejection.
  \item
    If~$i-1=k$ but~$\<A>$ is not the start symbol, the rule is 
      simply popped (as it was fully parsed).
  \item
    If~$Yᵢ∈Σ$ then like the LL parser, the SLL tries to match
      it with the next terminal from the input string.
    Matching will advance the rule by one step, i.e.,
      push~$\<A> ::= Y₀…Yᵢ·Y_{i+1}…Yₖ$, while failing to match
      causes the SLL to reject.
  \item
    The last case is that~$\<Y\ensuremath{ᵢ}>∈Ξ$.
    In this case, as in LL parsing, we look for the right
      derivation of~$\<Y\ensuremath{ᵢ}>$, using the prediction table.
      If the table returned a rule~$\<Y\ensuremath{ᵢ}> ::= Y₀'…Yₘ'$,
      the SLL will push~$\<Y\ensuremath{ᵢ}> ::=·Y₀'…Yₘ'$.
    Failing to find such rule causes rejection.
\end{enumerate}
