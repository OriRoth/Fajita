\title{%
  \Huge \SELF \\ 
  \huge \itshape \textbf Fluent \textbf API for \textsc{\textbf Java} \\
  \LARGE (\textbf Inspired by the \textbf Theory of \textbf Automata)
} 
\documentclass[10pt,twocolumn]{article}

\usepackage{\jobname}

\author{Tome Levy\\
	Department of Computer Science\\
	Technion---Israel Institute of Technology\\
	\texttt{\small \href{mailto:stlevy@campus.technion.ac.il}{stlevy@campus.technion.ac.il}}}
\date{
  Research Proposal\\
\small Advised by Prof.\ Yossi Gil
}

\begin{document}
\maketitle
  
\begin{abstract}
	This paper is a theoretical study of practical problem:
  the automatic generation of Java Fluent APIs from their specification.
We explain why the problem's core lies with 
  the expressive power of Java generics.
Our main result is that automatic generation is possible whenever 
  the specification is an instance of the set of deterministic context-free languages,
  a set which contains most ``practical'' languages.
Other contributions include a collection of techniques and idioms o
  the limited meta-programming possible with Java generics, 
  and an empirical measurement demonstrating that the runtime of
  the ``javac'' compiler of Java the may be exponential in
  the program's length, even for programs composed of 
  a handful of lines and which do not rely on overly 
  complex use of generics.
 
\end{abstract}

\section{Introduction}
Ever after their inception\urlref{http://martinfowler.com/bliki/FluentInterface.html} \emph{fluent APIs}
  increasingly gain popularity~\cite{Bauer:2005,Freeman:Pryce:06,Larsen:2012} and research
  interest~\cite{Deursen:2000,Kabanov:2008}.
In many ways, fluent APIs are a kind of
  \emph{internal} \emph{\textbf Domain \textbf Specific \textbf Language}:
They make it possible to enrich a host programming language without changing it.
Advantages are many: base language tools (compiler, debugger, IDE, etc.) remain
  applicable, programmers are saved the trouble of learning a new syntax, etc.
However, these advantages come at the cost of expressive power;
  in the words of Fowler:
  ‟\emph{Internal DSLs are limited by the syntax and structure of your base language.}”†
  {M. Fowler, \emph{Language Workbenches: The Killer-App for Domain Specific Languages?},
    2005
    \newline
  \url{http://www.martinfowler.com/articles/languageWorkbench.html#InternalDsl}}.
Indeed, in languages such as \CC, fluent APIs
  often make extensive use of operator overloading (examine, e.g., \textsf{Ara-Rat}~\cite{Gil:Lenz:07}),
  but this capability is not available in \Java.

Despite this limitation, fluent API in \Java can be rich and expressive, as demonstrated
  in \cref{Figure:DSL} showing use cases of the DSL of Apache Camel~\cite{Ibsen:Anstey:10}
(open-source integration framework),
and that of jOOQ\urlref{http://www.jooq.org}, a framework for writing
  SQL in \Java, much like Linq~\cite{Meijer:Beckman:Bierman:06}.

\begin{figure}[H]
  \caption{\label{Figure:DSL} Two examples of \Java fluent API}
  \begin{tabular}{@{}c@{}c@{}}
    \parbox[c]{44ex}{\javaInput[left=0ex]{camel-apache.java.fragment}} &
    \hspace{-3ex} \parbox[c]{59ex}{\javaInput[left=0ex]{jOOQ.java.fragment}}⏎
    \textbf{(a)} Apache Camel & \textbf{(b)} jOOQ
  \end{tabular}
\end{figure}

Other examples of fluent APIs in \Java are abundant:
  jMock~\cite{Freeman:Pryce:06},
  Hamcrest\urlref{http://hamcrest.org/JavaHamcrest/},
  EasyMock\urlref{http://easymock.org/},
  jOOR\urlref{https://github.com/jOOQ/jOOR},
  jRTF\urlref{https://github.com/ullenboom/jrtf}
  and many more.

\subsection{A Type Perspective on Fluent APIs}
\Cref{Figure:DSL}(B) suggests that jOOQ imitates SQL,
but is it possible at all to produce a fluent API
for the entire SQL language,
or XPath, HTML, regular expressions, BNFs, EBNFs, etc.?
Of course, with no operator overloading it is impossible
to fully emulate tokens; method names though make a good substitute for tokens, as done
in ‟\lstinline{.when(header(foo).isEqualTo("bar")).}” (\cref{Figure:DSL}).
The questions that motivate that this research are:
\begin{quote}
  \begin{itemize}
    \item Given a specification of a DSL, determine whether there exists
        a fluent API can be made for this specification?
    \item In the cases that such a fluent API is possible,
      can it be produced automatically?
    \item Is it feasible to produce a \emph{compiler-compiler} such as Bison~\cite{Bison} tool
        to convert a language specification into a fluent API?
\end{itemize}
\end{quote}

Inspired by the theory of formal languages and automata,
  this study explores what can be done, and what can not be done, with fluent API in \Java.

Consider some fluent API (or DSL) specification, permitting only certain call
chains and disallowing all others.
Now, think of the formal language that defines the set of these permissible chains.
The main contribution of this paper is a proof (with its implicit algorithm) that
there is always \Java type definition that \emph{realizes} this fluent definition, provided that this
language is \emph{deterministic context-free}, where
\begin{itemize}
  \item In saying that a type definition \emph{realizes} a specification of fluent
    API, we mean that call chains that conform with the API definition compile
    correctly, and, conversely, call chains that are forbidden by the API
    definition do not type-check, resulting in an appropriate compiler error.
  \item Roughly speaking, deterministic context free languages are those
    context free languages that can be recognized by an LR parser†{The ‟L"
    means reading the input left to right; the ‟R" stands for rightmost derivation}~\cite{Aho:86}.
    \par
    An important property of this family is that none of its members is ambiguous.
    Also, it is generally believed that most practical programming languages
    are deterministic context-free.
\end{itemize}

A problem related to that of recognizing a formal language,
is that of parsing, i.e., creating, for input which is within the language,
  a parse tree according to the language's grammar,
In the fluent APIs domain, the distinction between recognition and parsing is
  the distinction between compile time and runtime.
Before a program is run, the compiler checks whether the fluent API call is legal,
  and code completion tools will only suggest legal extensions of a current call chain.

In contrast, the parse tree is created at runtime.
Some fluent API definitions create the parse-tree
  iteratively, where each method invocations in the call chain adding
  more components to this tree.
However, it is always possible to generate this tree in ‟batch” mode:
This is done by maintaining a \emph{call list} which
  starts empty and grows at runtime by having each method invoked add to it
  a record storing the method's name and values of its parameters.
The list is completed at the end of the call list, at which point it is fed to an appropriate parser that
  converts it into a parse tree (or even an AST).

\subsection{Contribution}
The answers we provide for the three questions above is:
\begin{quote}
  \begin{enumerate}
  \item If the DSL specification is that of deterministic context-free
    language, then a fluent API exists for the language, but we do not know
    whether such a fluent API exists for more general languages.
  \par
  Recall that there are universal cubic time parsing
  algorithms~\cite{add:refs:to:this:cubics} which can parse (and recognize) any
  context free language. What we do not know is whether algorithms of this sort
  can be encoded within the framework of the \Java type system.
  \item
  There exists an algorithm to generate a fluent API that realize any
  deterministic context-free languages.  Moreover, this fluent API can create
  at runtime, a parse tree for the given language.  This parse tree can then be
  supplied as input to the library that implements the language's semantic.
  \item
  Unfortunately, a general purpose compiler-compiler
  is not yet feasible with the current algorithm.
  \begin{itemize}
    \item One difficulty is that the algorithm is complicated and relies on
      modules implementing some theoretical results, which, to the best of our
      knowledge have never been implemented.
    \item Another difficulty is that a certain design decision in the
      implementation of the standard \texttt{javac} compiler may choke on the
      Java code produced by the algorithm.
  \end{itemize}
  \end{enumerate}
\end{quote}

Other concrete contributions made by this work include
\begin{itemize}
  \item the understanding that the definition of fluent APIs is analogous to
      the definition of a formal language.
  \item a lower bound (deterministic pushdown automata).
    on the theoretical ‟computational complexity” of the \Java type system.
  \item an algorithm for producing a fluent API for deterministic context free languages.
  \item a collection of generic programming techniques, developed towards this algorithm.
  \item a demonstration that the runtime of Oracle's \texttt{javac} compiler may be exponential in the program size.
\end{itemize}


\paragraph{Outline.} 
The remainder of this document is organized as follows.
\Cref{Section:terminology}, intended mostly to the general, non software engineering, 
  computer science audience is a brief reminder of pertinent terminology.
This terminology is then used in \cref{Section:proposal} for 
  an introductory exposition of \SELF.
The main ideas behind the bootstrapping definition of \SELF 
  are revealed in \Cref{Section:bootstrapping}. 
\Cref{Section:zz} concludes. 

\section{Terminology}
\label{Section:terminology}
\subsection{Method Chaining}
The term \emph{method chaining} is illustrated in neat examples, such
as in the \nth{3} to \nth{6} line of the following \Java function:
\begin{Code}{JAVA}{Method chaining}
String time(int hours, int minutes, int seconds) {¢¢
  StringBuilder sb = new StringBuilder();
  return sb
    ¢¢.append(hours).append(':')
    ¢¢.append(minutes).append(':')
    ¢¢.append(seconds)
    .toString();
}
\end{Code}
Method chaining is to say that the same object, \cc{sb} of type \cc{StringBuilder} in the above,
   is the receiver of a chain of methods.
This is achieved by making each variant of the method \cc{append} return the receiver, denoted by the
  implicit argument \kk{this}.
With this convention, the tedious code
\begin{JAVA}
sb.append(hours);
sb.append(':');
sb.append(minutes);
sb.append(':');
sb.append(seconds);
return sb.toString():\end{JAVA}
becomes a bit shorter and less repetitive.

\subsection{Fluent API}
Now, definition found in the web of the term \emph{fluent API†{API =
    \textbf Application \textbf Program \textbf Interface
}} are a bit illusive.
Invariably, such definitions liken fluent API to method chaining;
  they also tend to emphasize that fluent API is ‟more” than method
  chaining, and ask the reader to extrapolate from one term to the other.

Indeed, fluent API looks much like method chaining; indeed method chaining is a kind
  of fluent API{}.
However, the main distinction between the two terms is the identity of the receiver.
In method chaining, all methods are invoked on the same object, whereas in fluent API
  the receiver of each method in the chain may be arbitrary.
Perhaps surprisingly, this difference makes fluent API significantly more expressive.
Consider, for example, the following code fragment (drawn from JMock~\cite{Freeman:Pryce:06})
\begin{Code}{Java}{Fluent API}
allowing (any(Object.class))
  ¢¢.method("get.*")
  ¢¢.withNoArguments();
\end{Code}
Let the return type of function \cc{allowing} be denoted by~$τ₁$ and let the return type of function \cc{method} be denoted by~$τ₂$.
Then, the fact that~$τ₁≠τ₂$ means that the set of methods that can be placed after the dot
in the partial call chain
\begin{code}{Java}
 allowing(any(Object.class)).
\end{code}
is distinct from the set of methods that can be placed after the dot in the partial call chain
\begin{code}{Java}
allowing(any(Object.class)).method("get.*").
\end{code}
This distinction make it possible to design expressive and rich fluent APIs, in which a sequence ‟chained” calls is not only readable, but also
robust, in the sense that the sequence is type correct, only when it the sequence makes sense semantically.

\subsection{The Type Safe Toilette Seat}

An object of type toilette seat is created in the \cc{down} state, but it can
then be \cc{raise}d to the \cc{up} state, and then be \cc{lower}ed to the
\cc{down} state.†{%
  This example is inspired by earlier work of
  Richard Harter on the topic~\cite{Harter:05}.
}
Such an object be used by two kinds of users, \cc{male}s and \cc{female}s, for two distinct purposes: \cc{urinate} and \cc{defecate}.
Now a good fluent API design is one by which the sequences of method calls in
\cref{Figure:toilette:legal} are type correct.

\begin{figure}[htbp]
  \begin{JAVA}
new Seat().male().raise().urinate();
new Seat().female().urinate();\end{JAVA}
  \caption{Legal sequences of calls in the toilette seat example}
  \label{Figure:toilette:legal}
\end{figure}

Conversely, sequences of method calls made in \cref{Figure:toilette:illegal} 
  should be illegal in the desired implementation of the fluent API.

\begin{figure}[htbp]
  \begin{JAVA}
new Seat().female().raise();
new Seat().male().raise().defecate();
new Seat().male().male();
new Seat().male().raise().urinate().female().urinate();\end{JAVA}
  \caption{Illegal sequences of calls in the toilette seat example}
  \label{Figure:toilette:illegal}
\end{figure}
It should be clear that the type checking engine of the compiler can
be employed to distinguish between legal and illegal sequences.
It should also be clear that fabricating the \kk{class}es, \kk{interface}s
and the \kk{extends} and \kk{implements} relationships between these, is
far from being trivial.

\subsection{Type State}
The toilette seat problem may be amusing to some, but it is not contrived in
any way.  In fact, there is huge body of research on the general topics of
\emph{type-states}. (See e.g., review articles such
as~\cite{Aldrich:Sunshine:2009,Bierhoff:Aldrich:2005}) Informally, an object
that belongs to a certain type (\kk{class} in the object oriented lingo), has
type-states, if not all methods defined in this object's class are applicable
to the object in all states it may be in.

A classical example of type-states is a file object: which can be in one of two
states: ‟open” or ‟closed”. Invoking a \cc{read()} method on the object is only
permitted when the file is an ‟open” state.  In addition, method \cc{open()}
(respectively \cc{close()}) can only be applied if the object is in the
‟closed” (respectively, ‟open”) state.

Objects with type states such as toilette seats and files are not rarities.
In fact, a recent study~\cite{Search:For:Aldrich} estimates 
  that about 70% of \Java classes feature type states.
Type-state pose two main challenges to software engineering 
\begin{enumerate}
  \item \emph{\textbf{Identification.}} 
    In the typical case, type-state 
        receive little to no mention at all in the documentation.
    The identification problem is find the
    type state implicit in existing code: Given an implementation of a class
    (or more generally of a software framework), 
    \emph{determine} which sequences of method calls are valid and which violate the 
    type state hidden in the code.
  \item \emph{\textbf{Maintenance and Enforcement.}}
    Having identified the type-states, the challenge is in automatically flagging out 
      use of the API that is non-conforming with the type-state, furthermore, with the 
      evolution of an API, the challenge is in updating the type-state information, 
      and the type checking of code of clients. 
\end{enumerate}


\section{This Proposal}
\label{Section:proposal}
The thesis propounded by this research is that API design, and especially fluent API design
  can and should be made in terms of language design.
Software missionaries and preachers such as Fowler~\cite{Fowler:2005} have long claimed
  that API design resembles the design of a \textbf Domain \textbf Specific \textbf Language
  (henceforth \emph{DSL}, see, e.g.,~\cite{VanDeursen:Klint:2000,Hudak:1997,Fowler:2010} for review articles).
   In the words of Fowler ‟The difference between API design and DSL design is then rather small”~\cite{Fowler:2005})

The objective of this research is
  to take the unification of the notions of DSL and (fluent) API
  design one step further in automating the creation of fluent API out
  of a DSL specification.

The basic idea is that the programmer specifies a fluent API,
  and, then, this specification is then automatically translated
  to an implementation of a fluent API that conforms
  with this specification.
This translation generates the intricate type hierarchy
  and methods of types in it in such a way
  that only sequence of calls that conform
  to the specification would
  compile correctly (concretely, type-check).

  To illustrate, consider the toilette seat example.
In this example,
  there are a total of six methods that might be invoked.
\begin{quote}
  \begin{tabular}{lll}
    \cc{male()} & \cc{raise()} & \cc{urinate()}⏎
    \cc{female()} & \cc{lower()} & \cc{defecate()}⏎
  \end{tabular}
\end{quote}
A fluent API design specifies the order in which such calls can be made.

The \emph{first} novelty in this research is that the fluent API definition is
  through a CFG, written as a BNF.
\cref{Figure:BNF} is such a specification for the toilette seat problem.

\begin{figure}[htbp]
  \scriptsize
  \begin{equation*}
    \def\<#1>{⟨\text{\textcolor{black}{\mdseries\rmfamily\/\textit{#1}}\/}⟩}
    \let\oldCc=\cc
    \let\oldKk=\kk
    \def\~{\text{~}}
    \def\|{\~|\~\~\~\~}
    \def\cc#1{{\footnotesize\oldCc{#1}}~}
    \def\kk#1{{\footnotesize\oldKk{#1}}}
    \scriptsize
    \begin{aligned}
      \<Visitors> & ::= \<Down-Visitors> \hfill⏎
      \<Down-Visitors> & ::= \<Down-Visitor> \~\<Down-Visitors> \hfill⏎
      {} & \| \<Raising-Visitor> \~\<Up-Visitors> \hfill⏎
      {} & \| ε \hfill⏎
      \<Up-Visitors> & ::= \<Up-Visitor> \~\<Up-Visitors> \hfill⏎
      {} & \| \<Lowering-Visitor> \~\<Down-Visitors> \hfill⏎
      {} & \| ε \hfill⏎
      \<Up-Visitor> & ::= \cc{male()} \~\cc{urinate()} \hfill⏎
      \<Down-Visitor> & ::= \cc{female()} \~\<Action> \hfill⏎
                          & \| \cc{male()} \cc{defecate()} \hfill⏎
      \<Raising-Visitor> & ::= \cc{male()} \~\cc{raise()} \~\cc{urinate()} \hfill⏎
      \<Lowering-Visitor> & ::= \cc{female()} \~\cc{lower()} \~\<Action> \hfill⏎
                          & \| \cc{male()} \~\cc{lower()} \cc{defecate()} \hfill⏎
      \<Activity> & ::= \cc{urinate()} \hfill⏎
                          & \| \cc{defecate()} \hfill⏎
    \end{aligned}
  \end{equation*}
  \caption{A BNF grammar for the toilette seat problem}
  \label{Figure:BNF}
\end{figure}

\SELF takes this grammar specification as input, and in response
  generates the corresponding
  \Java type hierarchy.

A second novelty of \SELF is that the specification of a BNF such as the provided in
  \cref{Figure:BNF} can be made in using a \Java fluent API.
To do so, it is first necessary to
  define the set of \emph{grammar terminals}
  \begin{code}{Java}
enum ToiletteTerminals implements Terminal {
  male, female,
  urinate, defecate,
  lower, raise;
}
\end{code}
As common in fluent APIs we shall refer to these
as \emph{verbs}†{Admittedly, the words ‟male” and ‟female” are nouns; 
  in our context howefver they are used to mean ‟male-visit” and ‟femail-visit”.}
Verbs are translated by \SELF into methods.

We also requrired to define the set of \emph{grammar variables}
  \begin{code}{Java}
enum ToiletteVariables implements Variable {
  Visitors, Down_Visitors, Up_Visitors,
  Up_Visitor, Down_Visitor,
  Lowering_Visitor, Raising_Visitor,
  Actitivity
};
  \end{code}
We shall use the term ‟nouns” as synonymous to variable.
The terms ‟symbol” and ‟word” refer to an entity which is either
  a verb or a nound.

Once the verbs and the nouns are set, the grammar can be defined,
  using a fluent API generated by \SELF itself as shown
  in \cref{Figure:fluent}.

\begin{figure}[htbp]
  \scriptsize
  \begin{code}{Java}
new BNF()
  .with(ToiletteTerminals.class)
  .with(ToiletteSymbols.class)
  .start(Visitors)
  .derive(Visitors)
    .to(Down_Visitors)
  .derive(Down_Visitors)
    .to(Down_Visitor).and(Down_Visitors)
    .or(Raising_Visitor).and(Up_Visitors)
    .orNone()
  .derive(Up_Visitors)
    .to(Up_Visitor).and(Up_Visitors)
    .or(Lowering_Visitor).and(Down_Visitors)
    .orNone()
  .derive(Up_Visitor)
    .to(male).and(urinate)
  .derive(Down_Visitor)
    .to(female).and(Action)
    .or(male).and(defecate)
  .derive(Raising_Visitor)
    .to(male).and(raise).and(urinate)
  .derive(Lowering_Visitor)
    .to(female).and(lower).and(Action)
    .or(male).and(lower).and(defecate)
  .derive(Activity)
    .to(urinate)
    .or(defecate)
  .go();
  \end{code}
  \caption{A BNF grammar for the toilette seat problem}
  \label{Figure:fluent}
\end{figure}

The final call \cc{go} in \cref{Figure:fluent} instructs
  \SELF to generate the code for the fluent API specified by the
  subsequet part of the expresion.
Rougly speaking, nouns are transalted to classes while verbs are translated to methods which
  take no parameters.
Two exceptions apply:
\begin{enumerate}
  \item \SELF can use classes such as \cc{String} and \cc{Integer}
  \item Verbs may take noun parameters, as explained below.
\end{enumerate}


\section{Bootstrapping Definition}
\label{Section:boostrapping}
As should be obvious from \cref{Figure:fluent}, \Self will be implemented
  in a bootstrapping fashion.
The specification of a BNF, is made itself using a fluent API.

This section describes how this is achieved.

\subsection{Reflective BNF}
\cref{Figure:BNF:BNF} is a \emph{reflective BNF}:
It uses the notation introduced in \cref{Figure:BNF}
  to specify this same notation.

\begin{figure}[H]
  \begin{Grammar}
    \begin{aligned}
      \<BNF>                     & ::= \<Header>\~\<Body>\~\<Footer> \hfill⏎
      \<Header>                  & ::= \<Variables> \~\<Terminals> \hfill⏎
      {}                         & \| \<Terminals> \~\<Variables> \hfill⏎
      \<Variables>               & ::= \cc{with(Class<? \kk{extends} Variable>)}\hfill⏎
      \<Terminals>               & ::= \cc{with(Class<? \kk{extends} Terminal>)}\hfill⏎
      \<Body>                    & ::= \<Start> \~\<Rules> \hfill⏎
      \<Start>                   & ::= \cc{start(\<Variable>)} \hfill⏎
      \<Rules>                   & ::= \<Rule> \~\<Rules>\hfill⏎
      {}                         & \| \<Rule> \hfill⏎
      \<Rule>                    & ::= \cc{derives(\<Variable>)} \<Conjunctions>\hfill⏎
      \<Conjunctions>            & ::= \<First-Conjunction>\~\<Extra-Conjunctions>\hfill⏎
      \<First-Conjunction>       & ::= \cc{to(\<Symbol>)}\~\<Symbol-Sequence>\hfill⏎
      {}                         & \| \cc{toNone()}\hfill⏎
      \<Extra-Conjunctions> & ::= \<Extra-Conjunction>\~\<Extra-Conjunctions>\hfill⏎
      {}                         & \| ε\hfill⏎
      \<Extra-Conjunction>  & ::= \cc{or(\<Symbol>)}\~\<Symbol-Sequence>\hfill⏎
      {}                         & \| \cc{orNone()} \hfill⏎
      \<Symbol-Sequence>         & ::= ε \hfill⏎
      {}                         & \| \cc{and(\<Symbol>)}\~\<Symbol-Sequence> \hfill⏎
      \<Symbol>                  & ::= \cc{Variable} \hfill⏎
      {}                         & \| \<Verb>\hfill⏎
      {}                         & \| \<Verb> \cc{,} \<Noun> \hfill⏎
      \<Noun>                    & ::= \<Variable> \hfill⏎
      {}                         & \| \<Existing-Class> \hfill⏎
      \<Variable>                & ::= \cc{Variable} \hfill⏎
      \<Footer>                  & ::= \cc{go()}\hfill⏎
    \end{aligned}
  \end{Grammar}
  \caption{A BNF grammar for defining BNF grammars}
  \label{Figure:BNF:BNF}
\end{figure}
% this is not a formal BNF because <Symbol> and <Variable> are not defined in the Symbols set.
\begin{comment}
%%%%%%%%%%%%%%%%%%%%%%%%%%%
Note that this specification can only be approximate;
the figure uses verbs as replacement to indentation,
and special symbols such as~$|$,~$::-$ and~$ε$.
%%%%%%%%%%%%%%%%%%%%%%%%%%%
\end{comment}

From \cref{Figure:BNF:BNF} we learn
  that a BNF has three components: header, body and footer.
  \begin{enumerate}
    \item The sets of terminals and variables are defined in the header part.
    \item The body starts with a definition of the start symbol, followed by a list of derivation
  rules.
\item The footer is simply the verb \cc{go()} which instructs \Self
  to generate the JAVA that realizes the fluent API specified by the grammar.
  \end{enumerate}

A derivation rule starts with a variable, and is then followed by disjunctive alternatives.

The choice of verbs that may occur in, and between, these alternatives not incidental;
  fluency was in mind:
\begin{description}
  \item[\cc{to}] to introduce the first symbol in the first conjunction.
  \item[\cc{or}] to introduce the first symbol in each subsequent conjunction.
  \item[\cc{and}] to introduce all but the first symbol in each such conjunction.
  \item[\cc{toNone}] to declare that the first conjunction is empty.
  \item[\cc{orNone}] to declare any subsequent conjunction is empty.
\end{description}

\subsection{Reflective BNF of fluent API}

To translate \cref{Figure:BNF:BNF} into a fluent
API chain, the verbs and nouns must be defined.

Verb definitions are made in the JAVA excerpt in
\cref{Figure:Verbs}.

\begin{figure}[htb]
  \begin{JAVA}[style=JAVA]
enum BNFTerminals implements Terminal {¢¢
  toNone,orNone,go,// No parameters
  start,derive     // One parameter
  with,            // One parameter, overloaded
  or,and,to,       // One parameter (or more), overloaded, variadic
  ;
}\end{JAVA}
  \caption{The verbs of \Self}
  \label{Figure:Verbs}
\end{figure}
Each of the enumerands in the figure is destined to be a
  name of a method in a class to be generated by \Self.

Noun definitions are made in the JAVA excerpt in \cref{Figure:Nouns}.

\begin{figure}[htb]
  \begin{JAVA}[style=JAVA]
enum BNFVariables implements Variable {¢¢
  BNF, Header, Body, Footer,
  Terminals, Variables, Start,
  Rules,Rule,Conjunctions, Extra_Conjunctions,
  First_Conjunction, Extra_Conjunction, Symbol_Sequence,
  Symbol, Variable, Noun;
}\end{JAVA}
  \caption{The nouns of \Self}
  \label{Figure:Nouns}
\end{figure}
  \Self will eventually generate a JAVA with
  a class named after each the enumerands in the figure.

The
enumerations \cc{BNFTerminals} and
  \cc{BNFVariables}
  are now employed in \cref{Figure:BNF:fluent}.

\begin{figure}
  \begin{Code}{Java}
new BNF()
  ¢¢.with(BNFTerminals.class)
  ¢¢.with(BNFSymbols.class)
  ¢¢.start(BNF)
  ¢¢.derive(BNF)
    ¢¢.to(Header).and(Body).and(Footer)
  ¢¢.derive(Header)
    ¢¢.to(Variables).and(Terminals)
    ¢¢.or(Terminals).and(Variables)
  ¢¢.derive(Variables)
    ¢¢.to(with, Variable.class)
  ¢¢.derive(Terminals)
    ¢¢.to(with, Terminal.class)
  ¢¢.derive(Body)
    ¢¢.to(Start).and(Rules)
  ¢¢.derive(Start)
    ¢¢.to(start, Variable)
  ¢¢.derive(Rules)
    ¢¢.to(Rule).and(Rules)
    ¢¢.or(Rule)
  ¢¢.derive(Conjunctions)
    ¢¢.to(First_Conjunction).and(Extra_Conjunctions)
  ¢¢.derive(First_Conjunction)
    ¢¢.to(to,Symbol).and(Symbol_Sequence)
    ¢¢.or(toNone)
  ¢¢.derive(Extra_Conjunctions)
    ¢¢.to(Extra_Conjunction).and(Extra_Conjunctions)
    ¢¢.orNone()
  ¢¢.derive(Extra_Conjunction)
    ¢¢.to(or,Symbol).and(Symbol_Sequence)
    ¢¢.orNone()
  ¢¢.derive(Symbol_Sequence)
    ¢¢.toNone()
    ¢¢.or(and, Symbol).and(Symbol_Sequence)
  ¢¢.derive(Symbol)
    ¢¢.to(Verb)
    ¢¢.or(Verb,Noun)
    ¢¢.or(Variable)
  ¢¢.derive(Footer)
    ¢¢.to(go)
¢¢.go();
\end{Code}
  \caption{A BNF grammar for \Self API}
  \label{Figure:BNF:fluent}
\end{figure}

The JAVA excerpt in the figure is a rather long
  sequence of method calls.
This fluent API sequence is a reflective BNF
  of the \Self API;
  indeed, we may check that \cref{Figure:BNF:fluent} reiterates \cref{Figure:BNF:BNF}
  (with notational changes as appropriate).

\subsection{Parametrized Verbs}
The granularity of grammars of programming languages typically goes down to the \emph{lexical token} level,
  but no deeper.
Such tokens, the building blocks of grammars, come in two flavors:
\begin{itemize}
  \item \emph{Monomorphic tokens} are tokens such as punctuation marks and
    certain keywords such as ‟\kk{if}”, ‟\cc{static}” and ‟\cc{class}”.
    Such tokens carry no information other than their mere presence.
  \item \emph{Polymorphic tokens} are tokens which carry content beyond
    presence (or absence). The prime example of these are identifiers.
\end{itemize}

This distinction applies also to fluent APIs:
Methods, or verbs, are the tokens, and a fluent APIs sequence consists of
method calls that come in two kinds: those that do not take parameters (such as \cc{toNone()} in \cref{Figure:BNF:fluent}),
and those that do (such as the call \cc{derives($·$)} in the figure).

\Self supports verbs with, and without, noun parameters.
The following examples,`drawn from \cref{Figure:fluent} and \cref{Figure:BNF:fluent},
  demonstrate,
\begin{JAVA}
  ¢¢.to(male)
  ¢¢.to(with, Terminal.class)
  ¢¢.to(with, Variable.class)
  ¢¢.to(start, Variable)\end{JAVA}
(The above is made possible with minor \Java language trickery,
  involving overloading and use of variadic signatures,
  with respect to function name \cc{to}.
Same trickery was applied to verbs \cc{or}, and \cc{and}.)


\section{Conclusion}
\label{Section:zz}
As should be obvious from \cref{Figure:fluent}, \SELF will be implemented
  in a bootstrapping fashion.
The specification of a BNF, is made using a fluent API.
The BNF for writing BNFs is given in \cref{Figure:BNF:BNF}

\begin{figure}[htbp]
  \scriptsize
  \begin{equation*}
    \def\<#1>{\/⟨\/\text{\textit{#1}}\/⟩\/~}
    \def\|{~|~}
    \let\oldCc=\cc
    \let\oldKk=\kk
    \def\cc#1{{\footnotesize\oldCc{#1}}~}
    \def\cc#1{{\footnotesize\olKk{#1}}~}
    \small
    \begin{aligned}
      \<BNF>              & ::=  \<Notation> \<Body> \<Footer> \hfill⏎
      \<Notation>         & ::=  \<Symbols> \<Terminals> \hfill⏎
      {}                  & \|  \<Terminals> \<Symbols> \hfill⏎
      \<Terminals>        & ::=  \cc{with(Symbols.class)}
      \<Symbols>          & ::=  \cc{with(Terminals.class)}
      \<Body>             & ::= \<Start> \<Rules> \hfill⏎
      \<Start>            & ::=  \cc{with(Class<? \kk{extends} Symbol)} 
      \<Rules>            & ::= \<First-Rule> \<More-Rules> \hfill⏎
      \<More-Rules>       & ::= \<Additional-Rule> \<More-Rules> \hfill⏎
      {}                  & \| ε \hfill⏎
      {}                  & \| \<Lowering-Visitor> \<Down-Visitors> \hfill⏎
      {}                  & \| ε \hfill⏎
      \<Up-Visitor>       & ::= \cc{male()} \cc{urinate()} \hfill⏎
      \<Down-Visitor>     & ::= \cc{female()} \<Action> \hfill⏎
                          & \| \cc{male()} \cc{defecate()} \hfill⏎
      \<Raising-Visitor>  & ::= \cc{male()} \cc{raise()} \cc{urinate()} \hfill⏎
      \<Lowering-Visitor> & ::= \cc{female()} \cc{lower()} \<Action> \hfill⏎
                          & \| \cc{male()} \cc{lower()} \cc{defecate()} \hfill⏎
      \<Activity>         & ::= \cc{urinate()} \hfill⏎
                          & \| \cc{defecate()} \hfill⏎
    \end{aligned}
  \end{equation*}
  \caption{A BNF grammar for the toilette seat problem}
  \label{Figure:BNF:BNF}
\end{figure}


\bibliographystyle{abbrv}
%\bibliography{publishers,other_shorthands,institutions,author_names,journals_full,yogi-journal,yogi-book,00}
\bibliography{author-names,other-shorthands,publishers-abbreviated,yogi-book,00}
\end{document}

Processing programming languages
\begin{description}
  \item[Lexical analysis] - the first step of the process in which the character strings generated by the 
  programmer are aggregated to the abstract tokens defined by the language designer.
  \item[Syntactical analysis (parsing) ] - the second step, in which the processed strings of tokens 
  conform to the rules of a formal grammar defined by the language's BNF (or EBNF).
  \item[Semantical analysis] - the next step, usually performed in unison with the previous step, 
  in which the legal token sequences are given their semantic meaning.
\end{description}
Specifically, the proposal is that API design of follows the footsteps of
Accordingly, the designer of a fluent API has to follow these three conceptual
steps.
First is the identification of the \emph{vocabulary}, i.e.,
the set of method calls including type arguments that may take part in the
fluent API.
In this fluent API example
\begin{lcode}{Java}
allowing (any(Object.class))
  ¢¢.method("get.*")
  ¢¢.withNoArguments();
\end{lcode}
then, there are three method calls, and the vocabulary has three items in it.
\begin{itemize}
  \item~$ℓ₁ = \cc{any(Class<?>)}$
  \item~$ℓ₂ = \cc{allowing($ℓ₁$)}$
  \item~$ℓ₃ = \cc{method(String)}$
  \item~$ℓ₄ = \cc{withNoArguments()}$
\end{itemize}
