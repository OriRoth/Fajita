The thesis propounded by this research is that API design, and especially fluent API design
  can and should be made in terms of language design.
Software missionaries and preachers such as Fowler~\cite{Fowler:2005} have long claimed
  that API design is in fact the design of a \textbf Domain \textbf Specific \textbf Language
  (henceforth \emph{DSL}, see, e.g.,~\cite{VanDeursen:Klint:2000,Hudak:1997,Fowler:2010} for review articles).
    (In the words of Fowler~\cite{Fowler:I:think})

The objective of this research is
  to take the unification of the notions of DSL and (fluent) API
  design one step further in automating the creation of fluent API out
  of a DSL specification.
The basic idea is that the programmer issues a fluent API
  specification, and that specification is translated automatically
  to an implementation of fluent API that conforms with this specification.
In the toilette seat example, there are six methods that can be invoked by the client of
  class \cc{Seat}:
\begin{quote}	
  \begin{tabular}{lll}
    \cc{male()}   & \cc{raise()} & \cc{urinate()}⏎
    \cc{female()} & \cc{lower()} & \cc{defecate()}⏎
  \end{tabular}
\end{quote}
a fluent API design specifies the order in which such calls can be made.

The first novelty in this research is that this specification can be made
using the Backus-Naur-Form for the specification of context grammar free grammar.

In the toilette seat problem, the programmer may specify the grammar as in 
  \cref{Figure:BNF}, and the \SELF shall generate from this specification 
  the necessary (and rather intricate) inheritance hierarchy that realizes
  this language.
This hierarchy will be such that only sequence of method calls conforming 
  to the BNF of \cref{Figure:BNF} will be recognized by the language.

\begin{figure}[htbp]
  \begin{equation*}
    \def\<#1>{\/⟨\/\text{\textit{#1}}\/⟩\/{ }}
    \def\|{|{\ }{\ }{\ }{\ }\!}
    \let\oldCc=\cc
    \def\cc#1{{\footnotesize\oldCc{#1}}{\ }}
    \small
    \begin{aligned}
      \<Visitors>         & ::=  \<Down-Visitors>     \hfill⏎
      \<Down-Visitors>    & ::=  \<Down-Visitor>      \<Down-Visitors>  \hfill⏎
      {}                  & \|   \<Raising-Visitor>   \<Up-Visitors>    \hfill⏎
      {}                  & \|   ε                    \hfill⏎
      \<Up-Visitors>      & ::=  \<Up-Visitor>        \<Up-Visitors>    \hfill⏎
      {}                  & \|   \<Lowering-Visitor>  \<Down-Visitors>  \hfill⏎
      {}                  & \|   ε                    \hfill⏎
      \<Up-Visitor>       & ::=  \cc{male()}          \cc{urinate()}    \hfill⏎
      \<Down-Visitor>     & ::=  \cc{female()}        \<Action>         \hfill⏎
                          & \|                  \cc{male()}          \cc{defecate()}  \hfill⏎
      \<Raising-Visitor>  & ::=  \cc{male()}          \cc{raise()}      \cc{urinate()}  \hfill⏎
      \<Lowering-Visitor> & ::=  \cc{female()}        \cc{lower()}      \<Action>       \hfill⏎
                                & \|                  \cc{male()}          \cc{lower()}           \cc{defecate()}  \hfill⏎
      \<Activity>         & ::=  \cc{urinate()}       \hfill⏎
                          & \|                  \cc{defecate()}  \hfill⏎
    \end{aligned}
  \end{equation*}
  \caption{A BNF grammar for the toilette seat problem}
  \label{Figure:BNF}
\end{figure}

A second novelty of \SELF is that the specification of a BNF such as the provided in
  \cref{Figure:BNF} can be made in \Java itself. 
Moreover, it can be made using fluent API generated by \SELF itself.

\begin{figure}[htbp]
  \begin{lcode}{Java}
enum toilette implements Terminal {
    male, female, urinate, defecate, lower, raise;
};

enum symbols implements Variable {
  Visitors, Down_Visitors, Up_Visitors, 
  Up_Visitor, Down_Visitor, Lowering_Visitor, Raising_Visitor
  Actitivity
};
BNF toiletteBnf =
  new BNFBuilder(toilette.class, symbols.class)
  .start(Visitors)
  .derive(Visitors).to(Down_Visitors)
  .derive(Down_Visitors).to(Down_Visitor).and(Down_Visitors)
      .or().to(Raising_Visitor).and(Up_Visitors)
      .or().toEpsilon()
  .derive(Up_Visitors).to(Up_Visitor).and(Up_Visitors)
      .or().to(Lowering_Visitor).and(Down_Visitors)
      .or().toEpsilon()
  .derive(Up_Visitor).to(male).and(urinate)
  .derive(Down_Visitor).to(female).and(Action)
      .or().to(male).and(defecate)
  .derive(Raising_Visitor).to(male).and(raise).and(urinate)
  .derive(Lowering_Visitor).to(female).and(lower).and(Action)
      .or().to(male).and(lower).and(defecate)
  .derive(Activity).to(urinate)
      .or().to(defecate)
  .finish();
  \end{lcode}
  \caption{A BNF grammar for the toilette seat problem}
  \label{Figure:BNF}
\end{figure}
Processing programming languages
\begin{description}
  \item[Lexical analysis] - the first step of the process in which the character strings generated by the 
  programmer are aggregated to the abstract tokens defined by the language designer.
  \item[Syntactical analysis (parsing) ] - the second step, in which the processed strings of tokens 
  conform to the rules of a formal grammar defined by the language's BNF (or EBNF).
  \item[Semantical analysis] - the next step, usually performed in unison with the previous step, 
  in which the legal token sequences are given their semantic meaning.
\end{description}
Specifically, the proposal is that API design of follows the footsteps of
Accordingly, the designer of a fluent API has to follow these three conceptual
steps.
First is the identification of the \emph{vocabulary}, i.e.,
the set of method calls including type arguments that may take part in the
fluent API.
In this fluent API example
\begin{lcode}{Java}
allowing (any(Object.class))
  ¢¢.method("get.*")
  ¢¢.withNoArguments();
\end{lcode}
then, there are three method calls, and the vocabulary has three items in it.
\begin{itemize}
  \item~$ℓ₁ = \cc{any(Class<?>)}$
  \item~$ℓ₂ = \cc{allowing($ℓ₁$)}$
  \item~$ℓ₃ = \cc{method(String)}$
  \item~$ℓ₄ = \cc{withNoArguments()}$
\end{itemize}
