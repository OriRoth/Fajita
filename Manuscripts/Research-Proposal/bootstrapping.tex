As should be obvious from \cref{Figure:fluent}, \SELF will be implemented
  in a bootstrapping fashion.
The specification of a BNF, is made itself using a fluent API.

Consider now \cref{Figure:BNF:BNF}
  which uses the notation introduced in \cref{Figure:BNF}
  to specify this same notation.
This specification can only be approximate; 
  the figure uses verbs as replacement to indentation, 
  and special symbols such as $|$, $::-$ and $\epsilon$

\begin{figure}[H]
  \begin{Grammar}
    \begin{aligned}
      \<BNF>              & ::=              \<Header>\~           \<Body>\~                \<Footer>           \hfill⏎
      \<Header>           & ::=              \<Symbols>      \~      \<Terminals>           \hfill⏎
      {}                  & \|            \<Terminals> \~         \<Symbols>             \hfill⏎
      \<Terminals>        & ::=              \cc{with(Class<? \kk{extends} Terminal>)}\hfill⏎
      \<Symbols>          & ::=              \cc{with(Class<? \kk{extends} Variable>)}\hfill⏎
      \<Body>             & ::=              \<Start>   \~           \<Rules>               \hfill⏎
      \<Start>            & ::=              \cc{start($\<Variable>$)}  \hfill⏎
      \<Rules>            & ::=              \<First-Rule> \~        \<Additional-Rules>    \hfill⏎
      \<Additional-Rules> & ::=              \<Additional-Rules>\~   \<Additional-Rule>     \hfill⏎
      {}                  & \|             ε                     \hfill⏎
      \<First-Rule>       & ::=              \cc{to($\<Symbol>$)}\~  \<Sequence>            \hfill⏎
      {}                  & \|              \cc{toNone()}    \hfill⏎
      \<Additional-Rule>  & ::=              \cc{or($\<Symbol>$)}\~  \<Sequence>            \hfill⏎
      {}                  & \|              \cc{orNone()}    \hfill⏎
      \<Sequence>         & ::=                 ε                     \hfill⏎
      {\~}                & \| \cc{and($\<Symbol>)$}  \hfill⏎
      \<Variable>         & ::=  \cc{Terminal} \hfill⏎
      {\~}                & \|             \cc{Variable}\hfill⏎
      \<Footer>           & ::=  \cc{go()}\hfill⏎
    \end{aligned}
  \end{Grammar}
  \caption{A BNF grammar for BNF grammars}
  \label{Figure:BNF:BNF}
\end{figure}

From the figure we learn that a BNF has three components: header, body and footer.
In the header part, the terminals and variables are defined.
The body starts with a definition of the start symbol, followed by a list of derivation 
  rules, while the footer is simply the verb \cc{go()} which instructs \SELF
  to generate the code that realizes the fluent API specified by the grammar.

A derivation rule starts with a variable, and followed by disjunctive alternatives.
The choice of verbs is not incidental; fluency was in mind:
\begin{description}
  \item[\cc{to}] to introduce the first symbol in the first conjunction.
  \item[\cc{or}] to introduce the first symbol in each subsequent conjunction.
  \item[\cc{and}] to introduce all but the first symbol in each such conjunction.
  \item[\cc{toNone}] to declare that the first conjunction is empty.
  \item[\cc{orNone}] to declare any subsequent conjunction is empty.
\end{description}

To translate \cref{Figure:BNF:BNF} into a fluent 
  API chain, the verbs must be defined:
  \begin{JAVA}
enum BNFTerminals implements Terminal {
  start, with, or, and
  to, toNone, orNone;
}
\end{JAVA}

Secondly, the list of nouns must be defined.
\begin{JAVA}
enum BNFVariables implements Variable {
  BNF, Header, Body, Footer,
  Start, Terminals, Symbols,
  Start, Rules, Additional_Rules,
  First_Rule, Additional_Rule,
  Sequence, Symbol;
}
\end{JAVA}

\cref{Figure:BNF:fluent} employs \cc{enum} \cc{BNFTerminals} and 
  \kk{enum} \cc{BNFVariables} to issue a long fluent API
  call that generates the fluent API used in \SELF. 

\begin{figure}[htbp]
  \begin{JAVA}[style=numbered]
new BNF()
  .with(BNFTerminals.class)       
  .with(BNFSymbols.class)
  .start(BNF)

  .derive(BNF)
    .to(Header).and(Body).and(Footer)
  .derive(Header)
    .to(Symbols).and(Terminals)           
    .or(Terminals).and(Symbols)
  .derive(Terminals)
    .to(with, Terminal.class)
  .derive(Symbols)
    .to(with, Variable.class)
  .derive(Body)
    .to(Start).and(Rules)
  .derive(Start)
    .to(start, Variable)
  .derive(Rules)
    .to(First_Rule).and(Additional_Rules)    
  .derive(Additional_Rules)
    .to(Additional_Rules).and(Additional_Rule)    
    .orNone() 
  .derive(First_Rule)
    .to(to,Symbol).and(Sequence)
    .or(toNone)
  .derive(Additional_Rule)
    .or(or, Symbol).and(Sequence)
    .or(orNone)
  .derive(Sequence)
    .toNone()
    .or(and, Symbol).and(Sequence)
  .derive(Variable)
    .to(Terminal)
    .or(Variable)
  .derive(Footer).
     .to(go)

.go();
  \end{JAVA}
  \caption{A BNF grammar for the toilette seat problem}
  \label{Figure:BNF:fluent}
\end{figure}
