We coninue on our journey to encode DPDA in \Java.
In this section we encode the tricky ``jump'' action defined in~\cref{Definition:JDPDA}.

A \emph{jump-stack} is a stack data structure whose elements are drawn from a finite set~$Γ$,
  except that jump-stack supports~$\textsf{jump}(γ)$,~$γ∈Γ$ operations,
    which means
  ‟repetetively pop elements from the stack up to and including the first occurrence of~$γ$”.
Let $k=|\Gamma|$. 

\begin{wrapfigure}[16]{r}{42ex}
  \caption{Skeleton of type encoding for the jump-stack data structure}%
  \label{Figure:jump}%
  \lstset{style=numbered}
  \javaInput[minipage,left=-2ex]{jump-stack.listing}
\end{wrapfigure}

\Cref{Figure:jump} shows the skeleton of type-encoding of a jump-stack whose
elements are drawn from type~\cc{$Γ$}
(\cref{Figure:unary-function}), i.e., either~\cc{$γ$1} or~\cc{$γ$2}.

Just like \cc{Stack} (\cref{Figure:stack-encoding}(b)),
  the generic type \cc{JS} which encodes jump-stacks, takes
  a \cc{Rest} parameter which is the type of a jump-stack after popping.
In addition \cc{JS} takes $k$ type parameters, one for~$γ∈Γ$,
  which is the type encoding of the jump-stack after a~$\textsf{jump}(γ)$
  operation.
In the figure, there are two such parameters: \cc{J\_$γ$1}, and
  \cc{J\_$γ$2}.

Functions defined in \cc{JS} include not only the standard stack opertions: \cc{top},
\cc{pop()}, \cc{$γ1$()} and~\cc{$γ2$()} (encoding a push of~$γᵢ$,~$i=1,2$),
  but also functions \cc{jump\_$γ$1} and \cc{jump\_$γ$2},
  which encode~$\textsf{jump}(γᵢ)$
  thanks to the return type being~\cc{J\_$γ$i},~$i=1,2$.

The type hierarchy rooted at \cc{JS} is similar to that of
\cref{Figure:stack-encoding}(a):
  Two of the specializations are parameterless and are
  almost identical to their \cc{Stack}
  counterparts:
\cc{JS.E} encodes an empty jump-stack; \cc{JS.¤} encodes a jump-stack in error,
e.g., a after popping from \cc{JS.E}.



Type \cc{JS.P} (line 15 in the figure) makes  the third specialization of \cc{JS}, representing 
  a stack with one or more elements.
There are no overriden functions in \cc{JS.P}; it achieves
  it purpose by the parameters it takes and those it passes
  to the type it extends.

\begin{wrapfigure}[10]r{43ex}
  \caption{\label{Figure:jump-stack-push} Type \cc{JS.P} encoding a non-empty jump-stack}
  \javaInput[minipage,width=43ex,left=-2ex]{jump-stack-push.listing}
\end{wrapfigure}

Specifically, \cc{JS.P} takes 
the same \cc{Top} and \cc{Rest} paramters (ll.16--17) as type \cc{Stack.P}:
  as well as $k$ additional paramters:
  \cc{J\_$γ$1} and \cc{J\_$γ$2} (ll.18--18)
which are the types encoding the jump-stack
  after the executation~$\textsf{jump}(γᵢ)$,~$i=1,2$.
Type \cc{JP.P'} passes these four parameters 
to type \cc{Pʹ} which it extends (l.21)
The fifth parameter to \cc{Pʹ} (l.22) is the current incarnation of \cc{P}, i.e., 
  \cc{P<Top, Rest, J\_γ1, J\_γ2>}.

The auxliary type \cc{JS.Pʹ} itself is depicted in \cref{Figure:jump-stack}.
Extending type \cc{JS} and passing the correct \cc{Rest} parameter to it, 
\cc{JS.Pʹ} inherits a correct declaration of function \cc{pop()} (l.6~\cref{Figure:jump}) 
