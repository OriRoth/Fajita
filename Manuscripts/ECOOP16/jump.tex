We continue on our journey to encode DPDA in \Java.
In this section we encode the tricky ‟jump” action defined in~\cref{Definition:JDPDA}.

A \emph{jump-stack} is a stack data structure whose elements are drawn from a finite set~$Γ$,
  except that jump-stack supports~$\textsf{jump}(γ)$,~$γ∈Γ$ operations
  (which means ‟repetitively \emph{pop} elements from the stack up 
  to and including the first occurrence of~$γ$”).

\begin{wrapfigure}[18]{r}{42ex}
  \caption{Skeleton of type encoding for the jump-stack data structure}%
  \label{Figure:jump}%
  \lstset{style=numbered}
  \javaInput[minipage,left=-2ex]{jump-stack.listing}
\end{wrapfigure}

\Cref{Figure:jump} shows the skeleton of type-encoding, in parameterized type \cc{JS},  
  of a jump-stack whose elements are drawn from type~\cc{$Γ$}
(\cref{Figure:unary-function}), i.e., either~\cc{$γ$1} or~\cc{$γ$2}.

Just like \cc{Stack} (\cref{Figure:stack-encoding}(b)), \cc{JS} takes 
  a \cc{Rest} parameter encoding the type of a jump-stack after popping.
In addition \cc{JS} takes~$k=|Γ|$ type parameters, one for each~$γ∈Γ$,
  which is the type encoding of the jump-stack after a~$\textsf{jump}(γ)$
  operation.
In the figure, there are two such parameters: \cc{J\_$γ$1}, and
  \cc{J\_$γ$2}.

Functions defined in \cc{JS} include not only the standard stack operations: \cc{top},
\cc{pop()}, \cc{$γ1$()} and~\cc{$γ2$()} (encoding a push of~$γᵢ$,~$i=1,2$),
but also~$k$ functions encoding $\textsf{jump}(\gamma)$, $\gamma \in \Gamma$.
In our case, these are \cc{jump\_$γ$1} and \cc{jump\_$γ$2},
  which encode~$\textsf{jump}(γᵢ)$
  thanks to their return type being~\cc{J\_$γ${}$i$},~$i=1,2$.

The type hierarchy rooted at \cc{JS} is similar to that of
\cref{Figure:stack-encoding}(a):
Two of the specializations are parameter-less and are
  almost identical to their \cc{Stack}
  counterparts:
\cc{JS.E} encodes an empty jump-stack; \cc{JS.¤} encodes a jump-stack in error,
e.g., a after popping from \cc{JS.E}.
The body of these two types is omitted here.

Type \cc{JS.P} (line 15 in the figure) makes the third specialization of \cc{JS}, encodind
type \cc{JS.P} (line 15 in \cref{Figure:stack-encoding}) makes the third specialization of \cc{JS}, encoding
  a stack with one or more elements.
Just like in \cref{Figure:id}, there are no overridden functions in \cc{JS.P}; it achieves
  its purpose by the parameters it takes and those it passes
  to the type it extends.

\begin{wrapfigure}[12]r{30ex}
  \caption{\label{Figure:jump-stack-push} Auxiliary type~\cc{Pʹ} encoding succinctly a non-empty jump-stack}
  \javaInput[minipage,width=30ex,left=-2ex]{jump-stack-push.listing}
\end{wrapfigure}

Specifically, \cc{JS.P} takes
the same \cc{Top} and \cc{Rest} parameters (ll.16--17) as type \cc{Stack.P}:
  as well as~$k$ additional parameters:
  \cc{J\_$γ$1} and \cc{J\_$γ$2} (ll.18--19)
which are the types encoding the jump-stack
  after the execution~$\textsf{jump}(γᵢ)$,~$i=1,2$.
Type \cc{JP.P} passes these four parameters
to type \cc{Pʹ} which it extends (l.21).
The fifth parameter to \cc{Pʹ} (l.22) is the current incarnation of~\cc{P}, i.e.,
  \cc{P<Top, Rest, J\_γ1, J\_γ2>}.

The auxiliary (and \kk{private}) type \cc{Pʹ} itself is depicted in \cref{Figure:jump-stack-push}.
By extending type \cc{JS} and passing the correct \cc{Rest} (respectively, \cc{J\_$γ$1}, \cc{J\_$γ$2})
  parameter to it, \cc{Pʹ} inherits correct declaration of function \cc{pop()} (l.6)
  (respectively \cc{jump\_$γ$1} (l.10), \cc{jump\_$γ$2} (l.11)).

  More importantly, the \cc{Me} type parameter to \cc{Pʹ} represents type \cc{JP.P} 
    that extends \cc{Pʹ}.
Type \cc{Me} also captures the actual parameters \emph{included} to \cc{JP.P},
  which makes it possible to write the return type of \cc{$γ$1()} and \cc{$γ$2()} more succinctly.
Let, e.g.,~$τ=\cc{P<γ1, Me, Me, J\_γ2>}$ be the return type of \cc{$γ$1()}.
The first two parameters to $\tau$ say that pushing~\cc{$γ$1},
  results in a compound jump-stack, whose top element is \cc{$γ$1}, 
  and where the rest of the jump-stack is the curent type.
The third parameter to~$τ$ says that since \cc{$γ$1} was pushed the result
  of a~$\textsf{jump}(γ₁)$ is the type of the receiver.
The fourth parameter is \cc{J\_$γ$2} since a push of~$γ₁$ does not
  change the result of~$\textsf{jump}(γ₁)$.

\begin{wrapfigure}[12]r{42ex}
  \caption{\label{Figure:jump-stack-example} Use cases for the~\cc{JS} type hierarchy}
  \javaInput[minipage,width=42ex,left=-2ex]{jump-stack-example.listing}
\end{wrapfigure}

Extending type \cc{JS} and passing the correct \cc{Rest} parameter to it,
\cc{JS.Pʹ} inherits a correct declaration of function \cc{pop()} (l.6~\cref{Figure:jump})

Some use cases for the encoded jump stack data structure are in \cref{Figure:jump-stack-example}.
The type of variable \cc{\_1} encodes a stack into which \cc{$γ$2}, \cc{$γ$1}, \cc{$γ$1} were pushed
  (in this order).
Examining the type of \cc{\_2} we see that executing \cc{jump\_$γ$2} on
  \cc{\_1}, yields the empty stack in a single step.
The type of \cc{\_3} is that state of the same stack
  after executing~\cc{jump\_$γ$1};
  it is exactly the same as popping a single element from the stack.
