The above discussion, formalism and conjecture may seem at first sight too abstract
  and of little practical value.
Re-positioning perspective may shed a different light:~\cref{Theorem:Gutterman} is important not only because it tells us
  that the question of termination of a \CC compiler is as difficult
  the halting problem~\cite{Turing:1936}, but also because it
  justifies the high resource investment in
  template programming~\cite{Musser:Stepanov:1989,Dehnert:Stepanov:2000
  ,Backhouse:Jansson:1999, Austern:1998,Bracha:Odersky:1998,Garcia:Jarvi:2003}.

In the fashion,~\Cref{Theorem:Gil-Levy} has important
  implications to the emerging trend of employing fluent APIs
  to make complex software systems more accessible.
\Self open a window to employ the conjecture for
  automatic generation of very rich such APIs.
The theoretical part of this work dwells on the proof of~\cref{Theorem:Gil:Levy}.
Its engineering part is concerned is
  a software implementation that would make JAVA
  such as in~\cref{Figure:fluent} generate
  the required \Java JAVA that realizes the
  defined grammar, so that JAVA such as
  found in~\cref{Figure:toilette:legal} is type-correct,
  whereas JAVA found in~\cref{Figure:toilette:illegal} is not.	

An important side effect of the implementation is that IDEs with built-in \Java completion
  (found e.g., in Eclipse~†{\href{http://www.eclipse.org/}{Eclipse home page}} and IntelliJ~†{\href{https://www.jetbrains.com/idea/}{Intellij home page}})
  will assist the programmer in making a correct use of the API.
Combining a smart \Java completion feature with \Self generated interfaces,
  can greatly increase a programmer's productivity, while \Self itself can be used
  to cut development time for embedded DSL's dramatically.


Accordingly, the contribution of this work is double folded:
  gaining better understanding of the computational expressiveness of
  \Java generics and type hierarchy, and, a better tool
  for designing, experimenting with and perfecting fluent APIs.

\subsection{Further Research}
As A. Bekkmaerman noted in her thesis~\cite{Bekkerman:04}, a natural expantion of
  BNF based parser generators as Bison and \Self is an EBNF parser generator,
  meaning the input grammar will be provided in a EBNF formation instead of a
  less intuitive, more conflicted, BNF formation.
We believe this extention is viable, and can greatly improve the ease of
  use of \Self.
