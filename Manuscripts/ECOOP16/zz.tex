The above discussion, formalism and conjecture may seem at first sight too abstract
  and of little practical value.
Re-positioning perspective may shed a different light:~\cref{Theorem:Gutterman} is important not only because it tells us
  that the question of termination of a \CC compiler is as difficult
  the halting problem~\cite{Turing:1936}, but also because it
  justifies the high resource investment in
  template programming~\cite{Musser:Stepanov:1989,Dehnert:Stepanov:2000
  ,Backhouse:Jansson:1999, Austern:1998,Bracha:Odersky:Stoutamire:Wadler:98,X:Garcia:Jarvi:Lumsdaine:Siek:Willcock:03}.

In the fashion,~\Cref{Theorem:Gil-Levy} has important
  implications to the emerging trend of employing fluent APIs
  to make complex software systems more accessible.
The theoretical part of this work dwells on the proof of~\cref{Theorem:Gil:Levy}.
Its engineering part is concerned is
  a software implementation that would make \Java
  such as in~\cref{Figure:fluent} generate
  the required \Java that realizes the
  defined grammar, so that \Java such as
  found in~\cref{Figure:toilette:legal} is type-correct,
  whereas \Java found in~\cref{Figure:toilette:illegal} is not.

Accordingly, the contribution of this work is double folded:
  gaining better understanding of the computational expressiveness of
  \Java generics and type hierarchy, and, a better tool
  for designing, experimenting with and perfecting fluent APIs.

One may ask whether our theoretical result is the
  best possible:
It is possible to implement a fluent API for general
  (that is, nondeterministic) context free languages?
  We give reason why this possibility is unlikely.

\subsection{Further Research}
Note the upcoming \textsf{Fajita}?

The algorithm for generating the pushdown automaton is unfortunately impractical.
It is indeed polynomial time and space,
  yet a consumer of formidable resources.
