The \Java compiler, is the main computational tool we use
  in this manuscript.
In particular, the \Java Generics mechanism is what
  boosted our expressiveness from the trivial \emph{regular languages}
  set, to the practical, reasonable \emph{deterministic context free languages} set.

An interesting question that was raised during this research,
  is what the runtime complexity of the \Java compiler is.
The reason this question is interesting, is its implication
  on the computational expressiveness of type-encoding.

For example, if we recognized that the \Java parser spends
  linear time on its input, we could say that it's not
  likely that we can type-encode nondeterministic CFG.
The reason is the fact that the best known algorithms
  today for parsing general nondeterministic CFG,
  as mentioned in~\Cref{Section:preliminaries}, run in super quadratic time.
And it is highly unlikely that the type-checker of \Java incidentally
  found an algorithm that is practically perfect in big-O notation.

As we explored this venue, we discovered that the type-checker of
  \Java actually runs in exponential time.

Consider an encoded binary type tree. \kk{class}~\cc{N}
  (denotes node) defined in~\cref{Figure:compiler} defines such tree.

\begin{wrapfigure}[6]r{27ex}
  \caption{\label{Figure:compiler} Encoding of an binary type tree}
  \javaInput[minipage,width=27ex,left=-2ex]{compiler.listing}
\end{wrapfigure}

The class has two type parameters, \cc{RL} (denotes right leaf),
  and \cc{LL} (denotes left leaf), both values are insignificant
  as they act as the base case of a recursive definition.
The class has only one method, \cc{d()} (denotes the keyword double),
  that performs our recursive step : create a new node, that its two
  sons, are the instance of the receiver itself.
The actual return type of \cc{d()} is
  \cc{N$⟨\text{N}⟨\text{LL, RL}⟩\text{, N}⟨\text{LL, RL}⟩⟩$}.
It is clear that the tree represented after each~\cc{d()} invocation is
  of a more than doubled size.

Consider a sequence of \cc{d()} invocations on a \kk{new} \cc{N()} object,
  for which the \Java compiler tries to compute a type.
If the input of the compiler is the former sequence of invocations,
  and we will define its length as~$n$,
  than surely, the type length of the result is~$O(2ⁿ)$.

\begin{wrapfigure}r{43ex}%
  \caption{\label{Figure:compile-empiric} Compile time (sec) \emph{vs.} sequential method invocation length}%
  \gnuplotloadfile[terminal=pdf,terminaloptions={crop size 2.5in,1.5in color enhanced font ",8" linewidth 1}]{../Figures/kill.gnuplot}
\end{wrapfigure}%

An empirical result of the former example is exhibited in \cref{Figure:compile-empiric}.
The \Java compiler therefore runs at an exponential complexity at worst case,
  not only that, it is not confined to~$O(2ⁿ)$, it is clear that
  writing a similar type-encoding for a ternary type tree would result in
  a lower-bound for the \Java compiler of~$O(3ⁿ)$, and for a k-nary tree
  we would get~$O(kⁿ)$.

Thus, we get that the compile time of a \Java is in the E complexity class :~$O(2^O(n))$.

The theoretical question of type-encoding a stronger machine (i.e., Turing Machine)
  is outside the scope of this paper.
