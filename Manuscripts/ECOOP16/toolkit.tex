\newcommand\javaInput[2][]{\tcbinputlisting{codebox=Java,listing options={style=Java},listing file=#2,#1}}

Let~$g:Γ→Γ$ be a function,
  where set~$Γ$ is finite.
We argue that~$g$ can
  be represented using the compile-time mechanism of \Java.
  \cref{Figure:unary-function} encodes such a partial function for~$Γ=❴γ₁,γ₂❵$, where~$g(γ₁)=γ₂$
  and~$g(γ₂)=⊥$, i.e.,~$g(γ₂)$ is undefined.

\begin{figure}[hbt]
  \caption{\label{Figure:unary-function}%
    Type encoding of the partial function~$g:Γ↛Γ$,
    defined by~$Γ=❴γ₁,γ₂❵$,~$g(γ₁)=γ₂$ and~$g(γ₂)=⊥$.
  }
  \begin{tabular}{@{}c@{}c@{}c@{}}
    \hspace{-7ex}
    \parbox[c]{0.26\linewidth}{%
      \input ../Figures/unary-function-classification.tikz
    }%
    &
    \hspace{-1ex}
    \parbox[c]{0.64\linewidth}{%
      \javaInput{gamma.listing}
    }%
    &
    \hspace{-18ex}
    \parbox[c]{0.84\linewidth}{%
      \javaInput{gamma-example.listing}
    }%
⏎
\textbf{(a)} type hierarchy & \textbf{(b)} implementation & \hspace{-62ex} \textbf{(c)} use cases
  \end{tabular}
\end{figure}

The type hierarchy depicted in~\cref{Figure:unary-function}(a) shows five classes:
Abstract class~\cc{Γ} represents the set~$Γ$, final classes~\cc{γ1},~\cc{γ2}
  that extend~\cc{$Γ$}, represent the actual members of the set~$Γ$.
The remaining two classes are private final class~\cc{¤} that stands for an error value,
  and abstract class~\cc{$Γ'$} that denotes the augmented set~$Γ∪❴\text{¤}❵$.
Accordingly, both classes~\cc{¤} and~\cc{$Γ$} extend~\cc{$Γ'$}.†{The use
  short names, e.g.,~\cc{$Γ$} instead of \cc{$Γ'.Γ$},
    is made possible by to an appropriate \kk{import} statement.
    For brevity, all \kk{import} statements are omitted.}

The full implementation of these classes is provided in~\cref{Figure:unary-function}(b)†{Remember that \Java admits Unicode characters in identifier names}.
This actual code excerpt should be placed as a nested class of some appropriate host class. Import statements are omitted, here and henceforth for brevity.

The use cases in~\cref{Figure:unary-function}(c) explain better
  what we mean in saying that function~$g$ is encoded in the type system:
  An instance of class~\cc{$γ$1} returns a value of type~\cc{$γ$2} upon
  method call~\cc{g()}, while
  an instance of class~\cc{$γ$2} returns a value of our~\kk{private}
  error type~\cc{$Γ'$.¤} upon the same call.

Three recurring idioms employed in~\cref{Figure:unary-function}(b) are:
\begin{enumerate}
  \item An~\kk{abstract} class encodes a set.
    Abstract classes that extend it encode
      subsets, while~\kk{final} classes encode set members.
  \item The interest of frugal management of name-spaces is served
    by the agreement that if a class~\cc{$X$}~\kk{extends} another class~\cc{$Y$}, then~\cc{$X$} is also defined
    as a~\kk{static} member class of~$Y$.
  \item Body of functions is limited to a single~\kk{return}~\kk{null}\cc{;} command.
      This is to stress that at runtime, the code does not carry out any useful or interesting computation,
      and the class structure is solely for providing compile-time type checks.
†{%
A consequence of these idioms is that the augmented class~\cc{$Γ'$} is visible to clients.
It can be made~\cc{private}. Just move class~\cc{$Γ$} to outside of~\cc{$Γ'$}, defying the second idiom.
}
\end{enumerate}

We saw that unary functions can be encoded with inheritance and overriding.
With the absence of multi-methods, encoding higher arity functions is a greater challenge.

Consider the partial binary function~$f: R⨉S↛Γ$, defined by
\begin{equation}
  \label{Equation:simple-binary}
  \begin{array}{ccc}
    R=❴r₁,r₂❵ & f(r₁,s₁)=γ₁ & f(r₂,s₁)=γ₁⏎
    S=❴s₁,s₂❵ & f(r₁,s₂)=γ₂ & f(r₂, s₂)=⊥
  \end{array}.
\end{equation}
A \Java type encoding of this definition of function~$f$
  is in~\cref{Figure:simple-binary}(a); use cases
    are in~\cref{Figure:simple-binary}(b).

\begin{figure}[hbt]
  \caption{\label{Figure:simple-binary}%
    Type encoding of partial binary function~$f: R⨉S↛Γ$,
    where~$R=❴r₁,r₂❵$,~$S=❴s₁,s₂❵$, and~$f$
  defined by~$f(r₁,s₁)=γ₁$,~$f(r₁,s₂)=γ₂$,~$f(r₂,s₁)=γ₁$, and~$f(r₂, s₂)=⊥$.}
    \begin{tabular}{cc}
      \hspace{-2.5ex}
      \parbox[c]{0.57\linewidth}{\javaInput{binary-function.listing}}
        &
      \hspace{-16ex}
      \parbox[c]{53ex}{\javaInput[minipage,width=53ex]{binary-function-example.listing}}
⏎
\hspace{-7ex}\parbox{0.57\linewidth}{\textbf{(a)} implementation (except for classes~\cc{$Γ$},~\cc{$Γ'$},~\cc{$γ$1}, and~\cc{$γ$2} given above
in \cref{Figure:unary-function}).}
& \hspace{-5ex}\textbf{(b)} use cases⏎
    \end{tabular}
  \end{figure}

We see that to compute~$f(r₁,s₁)$ at compile time we write~\cc{f.r1().s1()}
and that the fluent API call chain~\cc{f.r2().s2().g()} results in
  a compile time error since~$f(r₂, s₂)=⊥$.

Class~\cc{f} in the implementation sub-figure serves as
  the starting point of the little fluent API defined here.
The return type of~\kk{static} member functions~\cc{r1()} and~\cc{r2()}
  is the respective sub-class of class~\cc{R}:
The return type of function~\cc{r1()} is class~\cc{R.r1};
  the return type of function~\cc{r2()} is class~\cc{R.r2}.

Instead of representing set~$S$ as a class,
  its members are realized as methods~\cc{s1()} and~\cc{s2()} in class~\cc{R}.
These functions are defined as~\kk{abstract} with return type~\cc{$Γ$'}
  in~\cc{R}.
Both functions are overridden in classes~\cc{r1} and~\cc{r2},
   with the appropriate co-variant change of their return type,

It should be clear now that the encoding scheme presented
  in \Cref{Figure:simple-binary} can be generalized to functions
  with any number of arguments, provided that the domain and range sets are finite.
The encoding of sets of unbounded size require means for creating an unbounded
 number of types.
Genericity can be employed to serve this end, as we shall see next.

\Cref{Figure:stack-use-cases} shows some use cases of a type encoding of
  a stack of unbounded depth, yet can only store members of the set~$Γ$.
With type encoding these are precisely classes~\cc{$γ$1}
  and \cc{$γ$2} defined in \cref{Figure:unary-function}.

\begin{figure}[htb]
  \caption{\label{Figure:stack-use-cases}%
    Use cases of a compile-time stack data structures.
  }
  \javaInput[minipage]{stack-use-cases.listing}
\end{figure}

The figure demonstrates a stack that starts with five items in it.
These are popped in order. Just before popping the last item, its
  value is examined.
Trying then to pop from an empty stack, or to examine its top, ends with
  a compile time error.

The expression⏎
  \mbox{\qquad\qquad} \cc{Stack.empty.$γ$1().$γ$1().$γ$2().$γ$1().$γ$1()}⏎
represents the sequence of pushing the value~$γ₁$ into an
empty stack, followed by~$γ₁$,~$γ₂$,~$γ₁$, and, finally,~$γ₁$.
This expression's type is that of~\cc{\_1}:⏎
\mbox{\qquad\qquad} \cc{P<$γ$1,P<$γ$1,P<$γ$¢2,P<$γ$1,P<$γ$1,E>>>>>}.⏎
We see in this type a recurring building block: generic type~\cc{P}, \emph{short for ‟Push”}.
There are two parameters to~\cc{P}:
  \begin{enumerate}
    \item the \emph{top} of the stack, always a subtype of~\cc{$Γ$},
    \item the \emph{rest} of the stack, which can be of two kinds:
          \begin{enumerate}
            \item another instantiation of~\cc{P} (in most cases),
            \item non-generic type~\cc{E}, \emph{short for ‟Empty”}, that encodes the empty
                  set (only at the deepest~\cc{P}, rest is empty).
          \end{enumerate}
  \end{enumerate}
Incidentally, \kk{static} field \cc{Stack.bottom} is of type~\cc{E}.

\Cref{Figure:stack-encoding}(a) gives the type inheritance hierarchy of the \Java
implementation†{%
  unless otherwise stated, in saying implementation we mean actual
  code extract
}
implementation in~\cref{Figure:stack-encoding}(b).

\begin{figure}[htb]
  \caption{\label{Figure:stack-encoding} Type encoding of an unbounded
  stack data containing members of~$Γ$ (as defined in \cref{Figure:unary-function}).}
    \begin{tabular}{cc}
      \hspace{-12ex}
      \parbox[c]{0.3\linewidth}{%
        \input ../Figures/stack-classification.tikz
      }
      &
      \hspace{-4ex}
      \parbox[c]{75ex}{\javaInput[minipage,width=75ex]{stack.listing}}⏎
      \hspace{-12ex} \textbf{(a)} type hierarchy & \textbf{(b)} implementation
    \end{tabular}
\end{figure}

We see that the second parameter of generic type~\cc{P} must extend class \cc{Stack},
  and that both types~\cc{P} and~\cc{E} extend \cc{Stack}.
Other points to noice are:
\begin{itemize}
  \item The type at the top of the stack is precisely the return type of \cc{top()};
        it is overridden in~\cc{P} so that its return type is the first argument of~\cc{P}.
        The return type of \cc{top()} in~\cc{B} is the error value {$Γ'$.¤}.
  \item Pushing into the stack is encoded as functions~\cc{$γ$1()} and~\cc{$γ$2()};
        the two are overriden with appropriate covariant change of the return type in~\cc{P} and~\cc{E}.
        \par
        It is of some notice that the notion of covariance recurs:
          Overriding permits covariant change of the return type of a function,
          Similar covariant change is permitted in extending generic type extending another, e.g.,
          \begin{JAVA}
class Heap<M extends Mammal> {¢¢
  // ¢…¢
}
class School<W extends Whale> extends Heap<W> {¢¢
  // ¢…¢
}
          \end{JAVA}
  \item Since empty stack cannot be poposed, function \cc{pop()}, is overriden in~\cc{E} to return
        the type \cc{Stack.¤}. This type is indeed a kind of a stack, except that all four stack
        functions: \cc{top()}, \cc{push()},~\cc{$γ$1()}, and,~\cc{$γ$2()}, return the appropriate error type.
\end{itemize}
