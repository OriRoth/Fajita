This paper is a theoretical study of a practical problem:
  the automatic generation of \NonCitingUse{Java} Fluent APIs from their specification.
We explain why the problem's core lies with 
  the expressive power of \NonCitingUse{Java} generics.
Our main result is that automatic generation is possible whenever 
  the specification is an instance of the set of deterministic context-free languages,
  a set which contains most ``practical'' languages.
Other contributions include a collection of techniques and idioms of
  the limited meta-programming possible with \NonCitingUse{Java} generics, 
  and an empirical measurement demonstrating that the runtime of
  the ``javac'' compiler of \NonCitingUse{Java} the may be exponential in
  the program's length, even for programs composed of 
  a handful of lines and which do not rely on overly 
  complex use of generics.
