%% LyX 2.1.3 created this file.  For more info, see http://www.lyx.org/.
%% Do not edit unless you really know what you are doing.
\documentclass[english]{article}
\usepackage[T1]{fontenc}
\usepackage{geometry}
\geometry{verbose,tmargin=2cm,bmargin=2cm,lmargin=1.5cm,rmargin=1.5cm}
\usepackage{float}

\makeatletter

%%%%%%%%%%%%%%%%%%%%%%%%%%%%%% LyX specific LaTeX commands.

\usepackage{babel}
\begin{document}
\title{Instantiating state's type argument}
\maketitle


\subsubsection*{Building the automaton's dependency graph}
\begin{description}
\item [{Input:}] LR(1) parser.
\item [{Output:}] A dependency graph defining the dependencies between
instantiations of state classes
\item [{Notations:}]~\end{description}
\begin{itemize}
\item $Q^{B}$ - a state that we are look at the instantion from his context.
\item Let $\sigma$ be a transition from $Q^{B}$ then $Q_{\sigma}^{B}$
is the instantiation of $action\left(Q_{B},\sigma\right)$ in context
of $Q^{B}$
\item Let $\sigma_{2}$be a transition from $Q_{\sigma_{1}}$ then $Q_{\sigma_{1}\sigma_{2}}$
is the instantiation of $action\left(Q_{\sigma_{1}}^{B},\sigma_{2}\right)$
in context of $Q^{B}$
\item the symbols a,b,c,d will be terminals (not including \$)
\item the symbols A,B,C,D will be nonterminals(not including S').
\item the symbols X,Y,Z,W will be either terminals or nonterminals.
\item $\alpha,\beta,\gamma$ will be strings of grammar symbols\end{itemize}
\begin{description}
\item [{Method:}]~

\begin{enumerate}
\item $V=\emptyset$,$E=\emptyset$
\item For each $Q^{B}\in Parser$

\begin{enumerate}
\item $\forall\left[A\rightarrow\epsilon\cdot,b\right]\in Q^{B}$
\begin{enumerate}
\item $V=V\bigcup\left\{ Q_{Ab}^{B},Q_{b}^{B}\right\} $
\item $E=E\bigcup\left\{ \left(Q_{Ab}^{B},Q_{b}^{B}\right)\right\} $
\end{enumerate}
\item $\forall\left[A\rightarrow\cdot X\alpha,b\right]\in Q^{B}$ 

\begin{enumerate}
\item $Q_{Ab}^{B}=Find\left(Q^{B},A,b\right)$
\item If $Q_{Ab}^{B}\neq null$ :

\begin{enumerate}
\item $V=V\bigcup\left\{ Q_{Ab}^{B},Q_{X}^{B}\right\} $
\item $E=E\bigcup\left\{ \left(Q_{Ab}^{B},Q_{X}^{B}\right)\right\} $
\end{enumerate}
\end{enumerate}
\item $\forall\left[A\rightarrow\cdot Cd,b\right]\in Q^{B}$ 

\begin{enumerate}
\item $Q_{Ab}^{B}=Find\left(Q^{B},A,b\right)$
\item If $Q_{Ab}^{B}\neq null$ : 

\begin{enumerate}
\item $Q_{Cd}^{B}=Find\left(Q^{B},C,d\right)$
\item $V=V\bigcup\left\{ Q_{Ab}^{B},Q_{Cd}^{B}\right\} $
\item $E=E\bigcup\left\{ \left(Q_{Ab}^{B},Q_{Cd}^{B}\right)\right\} $
\end{enumerate}
\end{enumerate}
\item $\forall Q_{Ab}^{B}\in V$

\begin{enumerate}
\item $V=V\bigcup\left\{ Q_{A}^{B},Q_{b}^{A}\right\} $
\item $E=E\bigcup\left\{ \left(Q_{A}^{B},Q_{Ab}^{B}\right)\right\} $
\item $Q_{b}^{A}=Find\left(Q^{A},b\right)$
\item $E=E\bigcup\left\{ \left(Q_{b}^{A},Q_{Ab}^{B}\right)\right\} $
\end{enumerate}
\end{enumerate}
\end{enumerate}

/{*}{*}


Returns the state that will consume b, or null, if it cannot be determined.


{*}{*}/
\begin{description}
\item [{Function}] $Find\left(Q^{B},A,b\right)$ returns ContexedState\end{description}
\begin{enumerate}
\item If $action\left(Q_{A}^{B},b\right)=s_{i}$

\begin{enumerate}
\item return $Q_{i}$
\end{enumerate}
\item else //$Q_{A}^{B}$ reduces by rule $C\rightarrow\alpha\cdot,b$

\begin{enumerate}
\item If $\left|\alpha\right|>1$

\begin{enumerate}
\item return null //``Inherited TA''
\end{enumerate}
\item else if $\left|\alpha\right|=1$ ($\alpha=A$ )

\begin{enumerate}
\item return $Find\left(Q^{B},C,b\right)$
\end{enumerate}
\item else //$\alpha=\epsilon$

\begin{enumerate}
\item return $Find\left(Q^{A},C,b\right)$
\end{enumerate}
\end{enumerate}
\end{enumerate}
\begin{description}
\item [{Function}] $Find\left(Q^{B},b\right)$ returns ContexedState\end{description}
\begin{enumerate}
\item If $action\left(Q^{B},b\right)=s_{i}$

\begin{enumerate}
\item return $Q_{i}$
\end{enumerate}
\item else //$Q^{B}$ reduces by rule $C\rightarrow\alpha\cdot,b$

\begin{enumerate}
\item If $\left|\alpha\right|>0$

\begin{enumerate}
\item return null //``Inherited TA''
\end{enumerate}
\item else //$\alpha=\epsilon$

\begin{enumerate}
\item return $Find\left(Q^{B},C,b\right)$
\end{enumerate}
\end{enumerate}
\end{enumerate}
\item [{Complexity:}]~

\begin{itemize}
\item $V=O\left(\right)$
\item $E=O\left(\right)$
\end{itemize}
\end{description}

\subsubsection*{Instantiating states}
\begin{description}
\item [{Input:}] A dependency graph $G=\left(V,E\right)$ for state $Q_{B}$.
\item [{Output:}] An instantiated class for s.
\item [{Method:}]~

\begin{enumerate}
\item Handle dependency cycles.
\item Run a topoligical sort on $G$ ( we now know $G$ is a $DAG$)
\item for each $S$ in topological order:

\begin{enumerate}
\item If $S=Q_{X}^{B}$

\begin{enumerate}
\item $FillSingleTransition\left(Q_{X}^{B}\right)$
\end{enumerate}
\item Else if $S=Q_{XY}^{B}$

\begin{enumerate}
\item $FillDoubleTransition\left(Q_{XY}^{B}\right)$
\end{enumerate}
\end{enumerate}
\end{enumerate}
\begin{description}
\item [{Function}] $FillSingleTransition\left(Q_{X}^{B}\right):$

\begin{enumerate}
\item For each triplet $\left(A,i,a\right)\in Q_{X}^{B}$ 

\begin{enumerate}
\item If $\left(A,i-1,a\right)\in Q^{B}$ then //inherited TA(Type Argument)

\begin{enumerate}
\item $Q_{X}^{B}\left[\left(A,i,a\right)\right]:=Q^{B}\left[\left(A,i-1,a\right)\right]$
\end{enumerate}
\item Else // synthesized TA

\begin{enumerate}
\item $Q_{X}^{B}\left[\left(A,i,a\right)\right]:=Q_{Aa}^{B}$
\end{enumerate}
\end{enumerate}
\end{enumerate}
\item [{Function}] $FillDoubleTransition\left(Q_{XY}^{B}\right)$:

\begin{enumerate}
\item For each triplet $\left(A,i,a\right)\in Q_{XY}^{B}$ 

\begin{enumerate}
\item If $\left(A,i-1,a\right)\in Q^{B}$ then //inherited TA(Type Argument)

\begin{enumerate}
\item $Q_{XY}^{B}\left[\left(A,i,a\right)\right]:=Q_{X}^{B}\left[\left(A,i-1,a\right)\right]$
\end{enumerate}
\item Else // synthesized from $Q_{X}^{B}$ 

\begin{enumerate}
\item $Q_{XY}^{B}\left[\left(A,i,a\right)\right]:=Q_{Y}^{X}$
\end{enumerate}
\end{enumerate}
\end{enumerate}
\end{description}
\item [{Complexity:}]~


Instantiation algorithm time complexity: $O\left(\right)$\end{description}


\end{document}
