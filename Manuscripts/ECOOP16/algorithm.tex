An LR($1$) parser (henceforth, LR parser) for a grammar~$G=(Ξ,ξ,R)$ (Nonterminals set, initial start element,production rules set respectively)
  has three components:
\begin{description}
  \item[An automaton] whose states are~$Q=❴q₀,…,qₙ❵$
  \item[A stack] which may contain any member of~$Q$.
  \item[Specification of state transition] with the aid of two tables:
  \begin{description}
    \item[Goto table] which defines a partial function~$δ:Q⨉Ξ↛Q$ of transitions
    between the states of the automaton.
    \item[Action table] which
    defines a partial function\[η:Q⨉Σ↛ ❴ \textsf{Shift}(q) \,|\, q∈Q❵ ∪ ❴\textsf{Reduce}(r) \,| \, r∈G❵.\]
  \end{description}
\end{description}
An \emph{LR item} is a rule~$r∈R$ and a dot index in the right-hand side of the rule,
  i.e.,~$A→α·β$ is a LR item if~$A∈Ξ$,~$A→αβ∈R$,
  the dot index in this example is~$|α|$.
Every state~$q∈Q$ is a set of~\emph{LR items},
  where every such item is in the~$Kernel_{q}$ set,
  or in the~$NonKernel_{q}$ set.
Every state~$q$ is the unification of~$Kernel_{q}$ and~$NonKernel_{q}$.
The parser begins by pushing~$q₀$ (the initial state) into the empty stack,
and then repetitively executes the following:
Examine~$q∈Q$, the state at the top of the stack.
If~$q=qₙ$ (the accepting state) and the input string is wholly consumed, then the parser stops and the input is accepted.

Let~$σ∈Σ$ be the next input symbol.
If~$δ(q,σ)=⊥$ (undefined), the parser stops in rejecting the input.

If~$δ(q,σ) = \textsf{Shift}(q')$, the parser pushes state~$q'$
into the stack and consumes~$σ$.
If however,~$δ(q,σ) = \textsf{Reduce}(r)$,~$r=ξ→β$,
then the parser does not consume~$σ$.
Instead it makes~$|β|$ pop operations.
Let~$q'$ be the state at the top of the stack after these pops, then
the parser pushes a state,~$q”$,
defined by applying the transition function to~$ξ$, the left-hand side of the reduced production and~$q'$,
i.e.,~$q”=η(q',ξ)$.

The specifics of parser generation and specifically
  calculating the sets~$Kernel_{q}$ and~$NonKernel_{q}$, lies beyond our interest here.
We do mention however that the same LR parser can serve parsing table drawn using weaker LR parser generators,
including SLR and LALR\@.

LR parsers are actually DPDAs\@. As we mentioned before in \cref{Section:proof},~$ε$-moves
  fails the emulation of a DPDA, thus, in order to type-encode a LR parser, we first need
  to convert it to the encode-able jump-DPDA\@.

A \emph{JLR parser}, or \emph{Jump LR Parser} is a LR parser, that
  is based on a variation of the jump-DPDA\@.
The objective of the JLR parser is refined in~\cref{Theorem:JLR}.

\begin{Theorem}
  \label{Theorem:JLR}
  Let P be a LR parser, then there exists a jump-DPDA J,
  such that for every string~$α∈Σ^*$
  P accepts~$α$ if and only if J accepts~$α$
\end{Theorem}

For a LR parser~$⟨Q,δ_{LR}, η_{LR}⟩$ defined over a grammar~$⟨Ξ,ξ,R⟩$
A JLR is like the jump-DPDA~$A=⟨Γ,q₀,δ⟩$ defined in~\cref{Definition:JDPDA},
  where~$Γ= Q∪(Σ_\$↛Q)$ are the stack elements,~$q₀∈Q$
  is the initial state, and~$delta:Γ⨉Σ↛Γ^*∪j(Σ_\$)$ is defined as follows:
  \begin{itemize}
   \item~$δ(q,σ)= j(σ)$ if~$η_{LR}(q,σ)=Reduce(A→α)∧α≠ε$

   % Automization based on the second example (Balanced parenthesis)

   % Green rule , only on kernel items
   \item~$δ(q,σ)= q \: p~$ if~$η_{LR}(q,σ)=Shift(p)$ and we advanced only on~$\textsf{Kernel}_q$ items
   \item~$δ(q,σ)= q \: p~$ if~$η_{LR}(q,σ)=Reduce(A→ε)$
     and~$δ_{LR}(q,A)=Goto(q')$ and~$η_{LR}(q',σ)=Shift(p)$
     and we advanced only on~$\textsf{Kernel}_q$ items

  % Red rule , only on non-kernel items

   \item~$δ(q,σ)= q \: f \: p~$ if~$η_{LR}(q,σ)=Shift(p)$ and we advanced only on~$I⊆\textsf{NonKernel}_q$
     items, then
    \[
      f = ❴σ' \mapsto q' \; | \;∃i∈I . lookahead(i)=σ' \,∧\, η_{LR}(q,σ')=Shift(q')❵
    \]

   \item~$δ(q,σ)= q \: f \: p~$ if~$η_{LR}(q,σ)=Reduce(A→ε)$
     and~$δ_{LR}(q,A)=Goto(q')$ and~$η_{LR}(q',σ)=Shift(p)$
     and we advanced only on~$I⊆\textsf{NonKernel}_q$
     items, then
     \[
      f = ❴σ' \mapsto q' \; | \;∃i∈I . lookahead(i)=σ' \,∧\, η_{LR}(q,σ')=Shift(q')❵
     \]

  \end{itemize}
\subsection{Informally}

For a LR parser~$⟨Q,δ_{LR}, η_{LR}⟩$ defined over a grammar~$⟨Ξ,ξ,R⟩$
A JLR is like the jump-DPDA~$A=⟨Γ,q₀,δ⟩$ defined in~\cref{Definition:JDPDA},

\begin{itemize}
  \item When we \textsf{reduce()} a jump operation is performed.
    The jump will be for the corresponding look-ahead symbol.
  \item On a shift op:
  if we shift on a~$item∈\textsf{Kernel}$
    then everything as usual.
  Else,~$item∈\textsf{Closure(Kernel)}$
      if the look-ahead of this item will be consumed by a kernel item:
        a new Jump symbol will be inserted on the stack.
      Else:
        the look-ahead will cause a reduction of one of the kernel items,
        in that case, the previous Jump symbol is still relevant.
  \item[Even computation Invariant]
    Let~$L$ be an LR parser,~$J$ is the corresponding JLR, and word~$w =σ₁…σₖ \$∈Σ^*\$~$ (we denote~$wᵢ$ to be~$σᵢ…σₖ \$$).
    if when~$L$ parses~$w$, it reaches a configuration~$(wᵢ,γζ)$,
    then when~$J$ parses~$w$, it reaches the configuration~$(wᵢ,γζ')$ where~$ζ= \textsf{clean}(ζ')$
    and \textsf{clean} is a function that removes all~$c \notin Q$ from the input string.
  \item[Jump invariant]
    when jumping on a lookahead symbol~$a∈Σ_\$~$ during the computation of~$J$,
    all consecutive reductions performed on~$L$ from the parallel configuration, are ‟done” by the \textsf{jump} operation.
  \item[Parsing] can always be added to this model, because when we get to a item~$A→α·B,a$, and we shift on
    one of the items~$B→·β,a$, we know that every legal parsing will result in (thats actually not accurate, what happens
    if we shift on another item~$A→·γ,a/b$?)
\end{itemize}
\paragraph{Problem 1} if we shift on more than one items, with different lookaheads,
  and put multiple terminals on the stack, who should we put first? the ones that are
  not first are the problem; if we jump to them, we still got other terminals on the
  top of the stack, out invariant says we always want the top of the stack to be a
  state element.

\paragraph{Solution 1} if save on the stack a single element for the
  ‟multiple shifting items” which will be of type:~$\mathcal{P}(Σ)$
  and we transform the jump operation on~$σ∈Σ$to be ‟Remove elements
  from the stack until a~$χ∈\mathcal{P}(Σ)$,~$σ∈χ$ is popped”.

\paragraph{Problem 2} When we~$\textsf{Jump}(σ)$ , we consume~$σ$, and remove
  from stack as expected, the problem is, the corresponding lookaheads symbol
  should be ‟consumed” and the shift on~$σ$ need to happen again.

\paragraph{Solution 2} Instead of pushing subsets of~$Σ$ we push partial
  mappings~$Σ_{\$}↛Q$ (where~$Γ=Q∪(Σ_{\$}↛Q)$) jumping is still defined
  on symbols~$σ∈Σ_{\$}$, and when the jump is performed we pop from the stack
  until we popped some~$f∈(Σ↛Q)$ for which~$f(σ)≠⊥$, and then push~$f(σ)$ back onto the stack.

\paragraph{Problem 3} When we have a state \textsf{Base} that has
  two items~$A₁ \rightarrow α₁·β₁,a$ and~$A₂ \rightarrow α₂·β₂,a$ we can't decide which function to put,
  the mapping of~$a$ might be for the consecutive
  transitions~$\textsf{Base}.goto(A₁).shift(a)$ or to~$\textsf{Base}.goto(A₂).shift(a)$.

\paragraph{Failed Solution 3} We cannot solve the problem by changing the function elements from~$\Sigma_\$ \rightarrow \neg Q$
  to~$\Sigma_\$ \times \Xi \rightarrow \neg Q$. The reason resides on the identity of the input nonterminal.
  When we build the JLR's transition function, and want to add a function element to the stack, we choose for some~$\sigma \in \Sigma$
  a transition to be performed after a goto, to a certain state~$q \in Q $, problem is, there might be more than one such transition
  for that~$\sigma$, i.e., when for some~$q,q',q'' \in Q$, and~$A,B \in \Xi$ ,~$q.goto(A).shift(a)=q'$,~$q.goto(B).shift(a)=q''$.

\paragraph{Problem 4} We add a function element on the stack for lookahead~$a∈Σ$
  when when we have an item that will consume~$a$ after some reduction,
  i.e., an item of type~$ A→·B aα, c$ for some~$A,B∈\Xi$ and~$c∈σ$,
  and we don't add an item when we have a rule that will reduce
  upon seeing such lookahead~$a$, i.e.,~$A \rightarrow α·B , a$
  ( not quite sure that these examples are as general as possible. should verify ).
  What happens if both appear in the same state?

\paragraph{Solution 4} I believe that in \emph{every} such situation,
  we would see a Shift/Reduce conflict, at some state reachable from the discussed state.

\paragraph{Misc.}
We also start with a double element on the stack,~$\$q0$
