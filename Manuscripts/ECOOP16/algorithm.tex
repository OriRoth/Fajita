LR parser performs as usual except for:

\begin{itemize}
	\item When we \textsf{reduce()} a jump operation is performed.
	  The jump will be for the corresponding lookahead symbol.
	  
	\item On a shift op:
	if we shift on a~$item \in \textsf{Kernel}$
	  then everything as usual.
	Else,~$item \in \textsf{Closure(Kernel)}$
	    if the lookahead of this item will be consumed by a kernel item:
	      a new Jump symbol will be inserted on the stack.
	    Else:
	      the lookahead will cause a reduction of one of the kernel items,
	      in that case, the previous Jump symbol is still relevant.
	\item[Jump Invariant] For each ID~$(q,w,\gamma \zeta)$, if the corresponding lookahead $\ell$ will supply
	  ~$\textsf{Jump}(\ell)= \text{the state after all consecutive reductions}$
\end{itemize}
Definition of an LR parser is somewhat tricky as it is defined with terms of 
  CFGs and PDAs altogether.
\begin{Definition}[JLR parser- Jump LR(1) Parser]
	Let P be a LR(1) parser, then there exists a jump-DPDA J,
	such that for every string~$\alpha \in \Sigma ^ *$
	P accepts~$\alpha$ if and only if J accepts~$\alpha$ 
\end{Definition}



