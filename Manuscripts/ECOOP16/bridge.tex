We recognize that the \Java parser spends sub quadratic time
  on its input.
Moreover, we know that the known algorithms today for parsing general 
  non-deterministic CFG,
Therefore,
  the type checker implicitly implemented with the fluent API implementation,
  cannot recognize languages whose time of recognition is quadratic or above that.

The reason resides on the assumption that
Explain why we cannot do general non-deterministic languages. This require quadratic time.
Java type checking works in less than quadratic time. So, if it was possible to
do general context free languages, it would be possible to
  break the record~\cite{cubic time algorithms}.
It is highly unlikely that such a record is broken incidentally.

The same consideration
For the same reason, it is highly unlikely to be able to do even deterministic
context free languages~\cite{http://www.sciencedirect.com/science/article/pii/S0019995868910875}.

   Mention that there might be few languages between union of LR(k) and deterministic context free.

The reason is that Generate works in reasonable time \Self generates classes which encode, in types, with a bottom-up, left-right time parser.

We show that a particular non-deterministic context free language,
cannot be of a fluent API\@.
Specifically, consider~$L$, the language of all~$αα^r$, where~$\omega∈Σ^*$,~$Σ=❴0,1❵$,
and~$α^r$ being the sequence~$α$ in reversed order.
Language~$L$ is context free since it is defined by the BNF.
\begin{Grammar}
  \begin{aligned}
    \<S> & ::= S | 0S0 | 1s1 |ε
  \end{aligned}
\end{Grammar}
It is also non-deterministic~\cite{Hopcroft:2001}.

We argue that there is no type encoding that supports fluent API whose
  which permits the set of type correct call chains defined by~$L$.
If there was one, then, we could have done something dramatic?
Is it really known that recognizing~$L$ requires quadratic
  time.
To understand
 the right one.

Explain why we cannot do the lambda calculus.
