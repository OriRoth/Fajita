\begin{theorem}\label{Theorem:Gil-Levy:2}
  Let~$A$ be a DPDA recognizing a language~$L⊆Σ^*$.
  Then, there exists a \Java type definition,~$J_A$ for types~\cc{L},~\cc{A} and
    other types such that the \Java command
  \begin{equation}
    \label{Equation:result}
    \cc{A.build~$\textsf{java}(α)$;}
  \end{equation}
  type checks against~$J_A$ if an only if there exists~$β∈Σ^*$ such
  that~$αβ∈L$.
  Furthermore, program~$J_A$ can be effectively generated from~$A$.
\end{theorem}

Informally, a call chain type-checks if and only if it is a prefix
  of some legal sequence.
Alternatively, a call chain won't type-check if there is no
  continuation that leads to a legal string in~$L$.

The proof resembles~\cref{Theorem:Gil-Levy}'s proof.
We provide a similar implementation for a jump-stack (see\cref{Definition:JDPDA}),
  that will not compile under illegal prefixes.

The main difference between the two theorems is:
  in~\cref{Theorem:Gil-Levy} we allowed illegal call chains to compile,
  but not return the required~\cc{L} type, while in~\cref{Theorem:Gil-Levy:2}
  the illegal chain won't compile at all.
  
We will use the same running example, defined in~\cref{Table:A}.

Since the code suggested by the proof highly resembles the previously
  suggested code, we will describe the differences.
  
\subsection{Main Types}
The main types here are a subset of the previously defined main types.

\begin{quote}
  \javaInput[left=-2ex,minipage,width=54ex]{prefix-proof.configuration.listing}
\end{quote}

First, \kk{class} \cc{$\Sigma\Sigma$} is removed. 
A call chain that doesn't represent a valid prefix won't compile, 
  thus, there is no need for an error return type such as \cc{$\Sigma\Sigma$}.
Second, \kk{interface} \cc{C} is removed. 
Without it, the configuration types won't have the 
  methods \cc{$σ$1()}, … ,\cc{$σ${}$k$()} and \cc{\$()} from the supertype.
These inherited methods, is what differentiates the previous proof from the current.

\subsection{Top-of-Stack Types}
Types \cc{C$γ$1}, … ,\cc{C$γ${}$k$}, still represent stacks
  with \cc{$γ$1}, … ,\cc{$γ${}$k$} as their top,
  this time, the methods are defined ad-hock, in each type.
In A there are two such types:

\begin{quote}
  \javaInput[left=-2ex,minipage,width=45ex]{prefix-proof.many.listing}
\end{quote}

\subsection{Transitions}
In the proof of~\cref{Theorem:Gil-Levy} we defined class~\cc{C} to have 
  default return values for each of the methods \cc{$σ$1()}, … ,\cc{$σ${}$k$()} and \cc{\$()},
  that cannot be continued in a fashion that makes sense (i.e., cannot return a type~\cc{L}).
  
The different handling with the entries of the transition table is as following:
\begin{description}
 \item 
\end{description}
