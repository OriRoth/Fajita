\begin{Definition}[Deterministinc Pushdown Automata]
  A \emph{deterministic pushdown autoamton} (PDA)~$M$ is an 8-tuple
  \[
    M =⟨Q,Σ,Γ, q₀,⊥, A,δ,η⟩
  \]
  where~$Q$ is the a finite set of
  \emph{the states of~$M$},~$Σ$ is a finite
  \emph{set of input symbols},~$Γ$ is a finite
  \emph{set of stack symbols},~$q₀∈Q$ is the initial state,~$⊥∈Γ$
  is a \emph{special symobl designating the bottom of the stack}
  and~$A⊆Q$ is the \emph{set of accepting states} while~$δ$ and~$η$ are
  the \emph{partial functions of state transition}
  \[
    \begin{array}{crlc}
      δ: & Q⨉Σ⨉Γ & → & Q⨉Γ^*⏎
      η: & Q⨉Γ & → & Q⨉Γ^*,⏎
    \end{array}
  \]
\end{Definition}

\begin{Definition}[Simplified Pushdown Automata]
  A \emph{simplified pushdown autoamton} (SPDA)~$M$ of order~$k$ is a 7-tuple
  \[
    M =⟨Q,Σ,Γ,q₀,⊥, A,δ⟩
  \]
  where, as before,~$Q$ is the a finite set of
  \emph{the states of~$M$},~$Σ$ is a finite
  \emph{set of input symbols},~$Γ$ is a finite
  \emph{set of stack symbols},~$q₀∈Q$ is the initial state,~$⊥∈Γ$
  is a \emph{special symobl designating the bottom of the stack}
  and~$A⊆Q$ is the \emph{set of accepting states}.
  The signature of function~$δ$, the \emph{generalized partial transition function}
  of an SPDA of order~$k$, is:
  \[
      δ: Q⨉Σ⨉Γ^k → Q⨉Γ^*⏎
  \]
\end{Definition}

\begin{Theorem}
  \label{Theorem:SPDA}
  For every SPDA~$M$ there exists a set~$J_M$ of \Java type declarations, such that
    I do not know what to so say here yet.
\end{Theorem}

The remainder of this section is dedicated to the proof of \cref{Theorem:SPDA}.
In particular, we describe how~$J_M$ is constructed from the
  specification of~$M$.

First is the stack.
\begin{JAVA}
class S<T extends S> {
}
\end{JAVA}
For each state~$ξ∈Q$, we generate a \Java class~$ξ$,
\begin{JAVA}
class ¢$ξ$¢ <T extends S> extends S<T>{
  // ¢…¢
}
\end{JAVA}
In addition, we define a special class~$⊥$ that designates the bottom of the stack.
\begin{JAVA}
class ¢$⊥$¢ extends S<¢$⊥$¢> {
  // ¢…¢
}
\end{JAVA}
The stack contents~$⊥ξ₁ξ₁ξ₂ξ₁$,~$ξ₁,ξ₂∈Q$
is represented by the type
\begin{JAVA}
  ¢$ξ₁$¢ < ¢$ξ₂$¢ < ¢$ξ₁$¢ < ¢$⊥$ > > >
\end{JAVA}

\subsubsection{Building the automaton's dependency graph}

\begin{description}
  \item [Input:] LR(1) parser.
  \item [Output:] A dependency graph defining the dependencies between
  instantiations of state classes
  \item [Notations:]
\end{description}
\begin{itemize}
  \item~$Q^B$ a state that we are look at the instantiation from his context.
  \item Let~$σ$ be a transition from~$Q^B$ then~$Q_σ^B$
        is the instantiation of~$action(Q_B,σ)$ in context
        of~$Q^B$
  \item Let~$σ₂$be a transition from~$Q_{σ₁}$ then~$Q_{σ₁σ₂}$
        is the instantiation of~$action(Q_{σ₁}^B,σ₂)$
        in context of~$Q^B$
  \item the symbols a,b,c,d will be terminals (not including \$)
  \item the symbols A,B,C,D will be nonterminals(not including S').
  \item the symbols X,Y,Z,W will be either terminals or nonterminals.
        \item~$α,β,γ$ will be strings of grammar symbols
\end{itemize}
\begin{description}
  \item [Method:]
  \begin{enumerate}
    \item~$V=∅$,$E=∅$
    \item For each~$Q^B∈Parser$
          \begin{enumerate}
            \item~$∀[A\rightarrowε·,b]∈Q^B$
            \begin{enumerate}
              \item~$V=V\bigcup❴ Q_{Ab}^B,Q_{b}^B❵~$
              \item~$E=E\bigcup❴ (Q_{Ab}^B,Q_{b}^B)❵~$
            \end{enumerate}
            \item~$∀[A→·Xα,b]∈Q^B$
            \begin{enumerate}
              \item~$Q_{Ab}^B=Find(Q^B,A,b)$
              \item If~$Q_{Ab}^B≠null$ :
                    \begin{enumerate}
                      \item~$V=V\bigcup❴ Q_{Ab}^B,Q_{X}^B❵~$
                      \item~$E=E\bigcup❴ (Q_{Ab}^B,Q_{X}^B)❵~$
                    \end{enumerate}
            \end{enumerate}
            \item~$∀[A→·Cd,b]∈Q^B$
            \begin{enumerate}
              \item~$Q_{Ab}^B=Find(Q^B,A,b)$
              \item If~$Q_{Ab}^B≠null$ :
                    \begin{enumerate}
                      \item~$Q_{Cd}^B=Find(Q^B,C,d)$
                      \item~$V=V\bigcup❴ Q_{Ab}^B,Q_{Cd}^B❵~$
                      \item~$E=E\bigcup❴ (Q_{Ab}^B,Q_{Cd}^B)❵~$
                    \end{enumerate}
            \end{enumerate}
            \item~$∀Q_{Ab}^B∈V$
            \begin{enumerate}
              \item~$V=V\bigcup❴ Q_{A}^B,Q_{b}^{A}❵~$
              \item~$E=E\bigcup❴ (Q_{A}^B,Q_{Ab}^B)❵~$
              \item~$Q_{b}^{A}=Find(Q^{A},b)$
              \item~$E=E\bigcup❴ (Q_{b}^{A},Q_{Ab}^B)❵~$
            \end{enumerate}
          \end{enumerate}
  \end{enumerate}
  /{*}{*}
  Returns the state that will consume b, or null, if it cannot be determined.
  {*}{*}/
  \begin{description}
    \item [Function]~$Find(Q^B,A,b)$ returns ContexedState
  \end{description}
  \begin{enumerate}
    \item If~$action(Q_{A}^B,b)=sᵢ$
          \begin{enumerate}
            \item return~$Qᵢ$
          \end{enumerate}
    \item else //$Q_{A}^B$ reduces by rule~$C\rightarrowα·,b$
          \begin{enumerate}
            \item If~$|α|>1$
                  \begin{enumerate}
                    \item return null //‟Inherited TA”
                  \end{enumerate}
            \item else if~$|α|=1$ ($α=A$ )
                  \begin{enumerate}
                    \item return~$Find(Q^B,C,b)$
                  \end{enumerate}
            \item else //$α=ε$
                  \begin{enumerate}
                    \item return~$Find(Q^{A},C,b)$
                  \end{enumerate}
          \end{enumerate}
  \end{enumerate}
  \begin{description}
    \item [Function]~$Find(Q^B,b)$ returns ContexedState
  \end{description}
  \begin{enumerate}
    \item If~$action(Q^B,b)=sᵢ$
          \begin{enumerate}
            \item return~$Qᵢ$
          \end{enumerate}
    \item else //$Q^B$ reduces by rule~$C\rightarrowα·,b$
          \begin{enumerate}
            \item If~$|α|>0$
                  \begin{enumerate}
                    \item return null //‟Inherited TA”
                  \end{enumerate}
            \item else //$α=ε$
                  \begin{enumerate}
                    \item return~$Find(Q^B,C,b)$
                  \end{enumerate}
          \end{enumerate}
  \end{enumerate}
  \item [Complexity:]
  \begin{itemize}
    \item~$V=O()$
    \item~$E=O()$
  \end{itemize}
\end{description}
\subsubsection*{Instantiating states}
\begin{description}
  \item [Input:] A dependency graph~$G=(V,E)$ for state~$Q_B$.
  \item [Output:] An instantiated class for s.
  \item [Method:]
  \begin{enumerate}
    \item Handle dependency cycles.
    \item Run a topoligical sort on~$G$ ( we now know~$G$ is a~$DAG$)
    \item for each~$S$ in topological order:
          \begin{enumerate}
            \item If~$S=Q_{X}^B$
                  \begin{enumerate}
                    \item~$FillSingleTransition(Q_{X}^B)$
                  \end{enumerate}
            \item Else if~$S=Q_{XY}^B$
                  \begin{enumerate}
                    \item~$FillDoubleTransition(Q_{XY}^B)$
                  \end{enumerate}
          \end{enumerate}
  \end{enumerate}
  \begin{description}
    \item [Function]~$FillSingleTransition(Q_{X}^B):$
    \begin{enumerate}
      \item For each triplet~$(A,i,a)∈Q_{X}^B$
            \begin{enumerate}
              \item If~$(A,i-1,a)∈Q^B$ then //inherited TA(Type Argument)
                    \begin{enumerate}
                      \item~$Q_{X}^B[(A,i,a)]:=Q^B[(A,i-1,a)]$
                    \end{enumerate}
              \item Else // synthesized TA
                    \begin{enumerate}
                      \item~$Q_{X}^B[(A,i,a)]:=Q_{Aa}^B$
                    \end{enumerate}
            \end{enumerate}
    \end{enumerate}
    /{*}{*}
    merges~$Q^{X}_{Y},Q^B_{X}$ into an extended instantiation of~$Q_{XY}^B$
    {*}{*}/
    \item [Function]~$FillDoubleTransition(Q_{XY}^B)$:
    \begin{enumerate}
      \item For each triplet~$(A,i,a)∈Q_{XY}^B$
            \begin{enumerate}
              \item If~$((A,i-1,a)∈Q^B_{X})∧((A,i-2,a)\notin Q^B)$ //Inherited TA or Synthesized from~$Q^B$
                    \begin{enumerate}
                      \item~$Q_{XY}^B[(A,i,a)]:=Q_{X}^B[(A,i-1,a)]$
                    \end{enumerate}
              \item Else // Synthesized from~$Q_{X}^B$
                    \begin{enumerate}
                      \item~$Q_{XY}^B[(A,i,a)]:=Q^{X}_{Y}[(A,i-1,a)]$
                    \end{enumerate}
            \end{enumerate}
    \end{enumerate}
  \end{description}
  \item [Complexity:]
  Instantiation algorithm time complexity:~$O()$
\end{description}
