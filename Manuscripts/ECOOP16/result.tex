Let~$\textsf{java}$ be a function that translates a terminal~$σ∈Σ$
into a call to a uniquely named function (with respect to~$σ$).
Let~$\textsf{java}(α)$, be the function
  that translates a string~$α∈Σ^*$ into a fluent API call chain.
  If~$α=σ₁⋯σₙ∈Σ^*$, then \[
  \textsf{java}(α)=\cc{.}\textsf{java}(σ₁)\cc{().}⋯\cc{.}\textsf{java}(σₙ)\cc{()}.
\]
For example, when~$Σ=❴a,b,c❵$ let~$\textsf{java}(a)=\cc{a}$,~$\textsf{java}(b)=\cc{b}$, and,~$\textsf{java}(c)=\cc{c}$.
With these, \[
    \textsf{java}(caba) = \cc{.c().a().b().a()}
  \]

\begin{theorem}\label{Theorem:Gil-Levy}
  Let~$D$ be a DPDA recognizing a language~$L⊆Σ^*$.
  Then, there exists a \Java type definition,~$J_D$ for types~\cc{L},~\cc{D} and
    other types such that the \Java command
  \begin{equation}
    \label{Equation:result}
    \cc{L~$ℓ$ = M.build~$\textsf{java}(α)$}\cc{.\$();}
  \end{equation}
  type checks against~$J_M$ if an only if~$α∈L$.
  Furthermore, program~$J_M$ can be effectively generated from~$M$.
\end{theorem}

\Cref{Equation:result} reads: starting from the \kk{static} field \cc{build} of \kk{class}~\cc{M},
  apply the sequence of call chain~$\textsf{java}(α)$, terminate with a call to the
  ending character~\cc{\$()} and then assign to newly declared \Java variable~\cc{$ℓ$} of type~\cc{L}.

The proof of the theorem is by a scheme for encoding in \Java types
  the pushdown automaton~$A=A(L)$ that recognizes language~$L$.
Concretely, the scheme assigns a type~$τ(c)$.
  to each possible configuration~$c$ of~$A$.
Also, the type of \cc{M.build} is~$τ(c₀)$, where~$c₀$ initial configuration of~$A$,

Further, in each such type the scheme places
  a function~$σ()$ for every~$σ∈Σ$.
Suppose that~$A$ takes a transition from configuration~$cᵢ$ to configuration~$cⱼ$
  in response to an input chracter~$σₖ$.
Then, the return type of function \cc{$σₖ$()} in type~$τ(cᵢ)$ is type~$τ(cⱼ)$.

With this encoding the call chain in \cref{Equation:result}
  mimics the computation of~$A$, starting at~$c+0$ and ending with
  rejection or acceptance.
More details of the proof are in \cref{Section:proof}.

There are several, mostly minor, differences between the structure of the \Java code
in \cref{Equation:result}
and the examples of fluent API we saw above, e.g., in \cref{Figure:DSL}:
\begin{description}
  \item[Prefix, i.e., the stating \cc{M.build} variable.]
  All variables and functions of \Java are defined within a class.
  Therefore, a call chain must start with an object (\cc{M.build} in \cref{Equation:result})
  or, in case of \cc{static} methods, with the name of a class.
  In fluent API frameworks this prefix is typically eliminated
  with appropriate \cc{import} statements.
  \begin{quote}
  If so desired, the same can be done by our type encoding scheme: define all
  methods in type~$τ(c₀)$ as \cc{staic} and \cc{import static} of of these.
\end{quote}
  \item[Suffix, i.e., the terminal \cc{.\$()} call.]
  In order to know whether~$α∈L$ the automaton recognizing~$L$ must
  know when~$α$ is terminated.
  \begin{quote}
  With a bit of engineering, this suffix can also be eliminated.
  One way of doing so is by defining type~\cc{L} as an \kk{interface}, and by making all types~$τ(c)$,~$c$ is
  an accepting configuration, as subypte of~\cc{L}.
  \end{quote}
  \item[Parametrized methods.]
  Fluent APIs frameworks support call chains with including phrases such as
    \lstset{language=Java,style=code}, 
  \begin{itemize}
    \item \lstinline{.when(header(foo).isEqualTo("bar")).}, and, 
    \item \lstinline{.and(BOOK.PUBLISHED.gt(date("2008-01-01"))).}.
    \item \lstinline{.allowing(any(Object.class)).}
  \end{itemize}
  while our encoding scheme assumes methods which do not take any parameters.  
  \begin{quote}
    Even though  methods with parameters contribute to the user
      experience and readability of fluent APIS, extending 
      \cref{Theorem:Gil-Levy} to support requires three simple steps: 
      \begin{itemize}
        \item Define the structure of parameters to methods with appropriate fluent API, which may or
          may not be the same as the API of the outer chain, or the API of parameters to
          other methods. Apply the threorem to each of these fluent APIs.
        \item 
      \end{itemize}
  \end{quote}
\end{description}
