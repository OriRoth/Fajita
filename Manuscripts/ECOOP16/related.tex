Modern programming languages acquire high-level constructs
  at a staggering rate.
The imminent adoption of closures in \Java and \CC,
  the generators of \CSharp, and ‟concepts” in
  \CC are just a few examples.

A theoretical motivation for this work
  is the exploration of the computational
  expressiveness of such features.
For example, it is known (see e.g.,~\cite{Gutterman:2003}) that
  \kk{template}s in \CC are Turing complete in the following precise sense:

\begin{Theorem}
  \label{Theorem:Gutterman}
  For every Turing machine,~$m$, there exists a \CC program,~$Cₘ$ such that
    compilation of~$Cₘ$ of terminates if and only if
      Turing-machine~$m$ halts.
  Furthermore, program~$Cₘ$ can be effectively generated from~$m$.
\end{Theorem}

Intuitively, the proof relies on the fact that \kk{template}s
  feature recursive invocation and conditionals (in the form of
  ‟\emph{template specialization}”).

The above discussion, formalism and conjecture may seem at first sight too abstract
  and of little practical value.
Re-positioning perspective may shed a different light:~\cref{Theorem:Gutterman} is important not only because it tells us
  that the question of termination of a \CC compiler is as difficult
  the halting problem~\cite{Turing:1936}, but also because it
  justifies the high resource investment in
  template programming~\cite{Musser:Stepanov:1989,Dehnert:Stepanov:2000
  ,Backhouse:Jansson:1999, Austern:1998,Bracha:Odersky:Stoutamire:Wadler:98,X:Garcia:Jarvi:Lumsdaine:Siek:Willcock:03}.
  
In the same fashion, it is mundane to make the judgment that
  \Java's generics are not Turing-complete: all recursive calls
  in these are unconditional.
In a sense, this article shall give a lower bound on the
  expressive power of \Java generics in terms of the Chomsky hierarchy~\cite{Chomsky:1963}.
This objective is more precisely expressed in the following conjecture.

Boost is a cool c++ templating library!\cite{Abrahams:Gurtovoy:04} I think. 
This how you can calculate the derivative of a function with \CC compiler! \cite{Gil:Gutterman:98}

Mention funny tricks with annotations to Java. There is \cite{Papi:08} from 
  Washington State university. He fought for more support for annotations 
  and built a system for implementing non
Expression templates is a \CC technique for passing expressions as function arguments. \cite{Veldhuizen:95}

Mention work by \cite{Bracha} on non-standard type systems.  

There has already been a similar \Java implementation for regular expressions
  \urlref{https://github.com/verhas/fluflu}.

Eric Bodden wrote an article about fluent APIs, static and dynamic analysis, and type-state~\cite{Bodden:14}

