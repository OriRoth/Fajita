\title{%
  \Self \protect \thanks {%
    \textbf
    Fluent \textbf API for \textsc{\textbf Java}
    (\textbf Inspired by the \textbf Theory of \textbf Automata)
  }
  \newline
  \color{red}{%
    \rmfamily\scshape Thou Mortal, Be Warned. \newline
    Thou Shallt Not Remove \newline
    This Commandment \newline
    While There Are Signs of Haste \newline
    in This Document!!!!\newline
  }
}

\documentclass[a4paper,USenglish]{lipics}
\usepackage{\jobname}

%\author{Tome Levy⏎
% Department of Computer Science⏎
% Technion---Israel Institute of Technology⏎
% \texttt{\small \href{mailto:stlevy@campus.technion.ac.il}{stlevy@campus.technion.ac.il}}}

\author{Anonymized for the submission}

\begin{document}

\maketitle
\begin{abstract}
  This paper is a theoretical study of practical problem:
  the automatic generation of Java Fluent APIs from their specification.
We explain why the problem's core lies with 
  the expressive power of Java generics.
Our main result is that automatic generation is possible whenever 
  the specification is an instance of the set of deterministic context-free languages,
  a set which contains most ``practical'' languages.
Other contributions include a collection of techniques and idioms o
  the limited meta-programming possible with Java generics, 
  and an empirical measurement demonstrating that the runtime of
  the ``javac'' compiler of Java the may be exponential in
  the program's length, even for programs composed of 
  a handful of lines and which do not rely on overly 
  complex use of generics.

\end{abstract}

\section{Introduction}
Ever after their inception\urlref{http://martinfowler.com/bliki/FluentInterface.html} \emph{fluent APIs}
  increasingly gain popularity~\cite{Bauer:2005,Freeman:Pryce:06,Larsen:2012} and research
  interest~\cite{Deursen:2000,Kabanov:2008}.
In many ways, fluent APIs are a kind of
  \emph{internal} \emph{\textbf Domain \textbf Specific \textbf Language}:
They make it possible to enrich a host programming language without changing it.
Advantages are many: base language tools (compiler, debugger, IDE, etc.) remain
  applicable, programmers are saved the trouble of learning a new syntax, etc.
However, these advantages come at the cost of expressive power;
  in the words of Fowler:
  ‟\emph{Internal DSLs are limited by the syntax and structure of your base language.}”†
  {M. Fowler, \emph{Language Workbenches: The Killer-App for Domain Specific Languages?},
    2005
    \newline
  \url{http://www.martinfowler.com/articles/languageWorkbench.html#InternalDsl}}.
Indeed, in languages such as \CC, fluent APIs
  often make extensive use of operator overloading (examine, e.g., \textsf{Ara-Rat}~\cite{Gil:Lenz:07}),
  but this capability is not available in \Java.

Despite this limitation, fluent API in \Java can be rich and expressive, as demonstrated
  in \cref{Figure:DSL} showing use cases of the DSL of Apache Camel~\cite{Ibsen:Anstey:10}
(open-source integration framework),
and that of jOOQ\urlref{http://www.jooq.org}, a framework for writing
  SQL in \Java, much like Linq~\cite{Meijer:Beckman:Bierman:06}.

\begin{figure}[H]
  \caption{\label{Figure:DSL} Two examples of \Java fluent API}
  \begin{tabular}{@{}c@{}c@{}}
    \parbox[c]{44ex}{\javaInput[left=0ex]{camel-apache.java.fragment}} &
    \hspace{-3ex} \parbox[c]{59ex}{\javaInput[left=0ex]{jOOQ.java.fragment}}⏎
    \textbf{(a)} Apache Camel & \textbf{(b)} jOOQ
  \end{tabular}
\end{figure}

Other examples of fluent APIs in \Java are abundant:
  jMock~\cite{Freeman:Pryce:06},
  Hamcrest\urlref{http://hamcrest.org/JavaHamcrest/},
  EasyMock\urlref{http://easymock.org/},
  jOOR\urlref{https://github.com/jOOQ/jOOR},
  jRTF\urlref{https://github.com/ullenboom/jrtf}
  and many more.

\subsection{A Type Perspective on Fluent APIs}
\Cref{Figure:DSL}(B) suggests that jOOQ imitates SQL,
but is it possible at all to produce a fluent API
for the entire SQL language,
or XPath, HTML, regular expressions, BNFs, EBNFs, etc.?
Of course, with no operator overloading it is impossible
to fully emulate tokens; method names though make a good substitute for tokens, as done
in ‟\lstinline{.when(header(foo).isEqualTo("bar")).}” (\cref{Figure:DSL}).
The questions that motivate that this research are:
\begin{quote}
  \begin{itemize}
    \item Given a specification of a DSL, determine whether there exists
        a fluent API can be made for this specification?
    \item In the cases that such a fluent API is possible,
      can it be produced automatically?
    \item Is it feasible to produce a \emph{compiler-compiler} such as Bison~\cite{Bison} tool
        to convert a language specification into a fluent API?
\end{itemize}
\end{quote}

Inspired by the theory of formal languages and automata,
  this study explores what can be done, and what can not be done, with fluent API in \Java.

Consider some fluent API (or DSL) specification, permitting only certain call
chains and disallowing all others.
Now, think of the formal language that defines the set of these permissible chains.
The main contribution of this paper is a proof (with its implicit algorithm) that
there is always \Java type definition that \emph{realizes} this fluent definition, provided that this
language is \emph{deterministic context-free}, where
\begin{itemize}
  \item In saying that a type definition \emph{realizes} a specification of fluent
    API, we mean that call chains that conform with the API definition compile
    correctly, and, conversely, call chains that are forbidden by the API
    definition do not type-check, resulting in an appropriate compiler error.
  \item Roughly speaking, deterministic context free languages are those
    context free languages that can be recognized by an LR parser†{The ‟L"
    means reading the input left to right; the ‟R" stands for rightmost derivation}~\cite{Aho:86}.
    \par
    An important property of this family is that none of its members is ambiguous.
    Also, it is generally believed that most practical programming languages
    are deterministic context-free.
\end{itemize}

A problem related to that of recognizing a formal language,
is that of parsing, i.e., creating, for input which is within the language,
  a parse tree according to the language's grammar,
In the fluent APIs domain, the distinction between recognition and parsing is
  the distinction between compile time and runtime.
Before a program is run, the compiler checks whether the fluent API call is legal,
  and code completion tools will only suggest legal extensions of a current call chain.

In contrast, the parse tree is created at runtime.
Some fluent API definitions create the parse-tree
  iteratively, where each method invocations in the call chain adding
  more components to this tree.
However, it is always possible to generate this tree in ‟batch” mode:
This is done by maintaining a \emph{call list} which
  starts empty and grows at runtime by having each method invoked add to it
  a record storing the method's name and values of its parameters.
The list is completed at the end of the call list, at which point it is fed to an appropriate parser that
  converts it into a parse tree (or even an AST).

\subsection{Contribution}
The answers we provide for the three questions above is:
\begin{quote}
  \begin{enumerate}
  \item If the DSL specification is that of deterministic context-free
    language, then a fluent API exists for the language, but we do not know
    whether such a fluent API exists for more general languages.
  \par
  Recall that there are universal cubic time parsing
  algorithms~\cite{add:refs:to:this:cubics} which can parse (and recognize) any
  context free language. What we do not know is whether algorithms of this sort
  can be encoded within the framework of the \Java type system.
  \item
  There exists an algorithm to generate a fluent API that realize any
  deterministic context-free languages.  Moreover, this fluent API can create
  at runtime, a parse tree for the given language.  This parse tree can then be
  supplied as input to the library that implements the language's semantic.
  \item
  Unfortunately, a general purpose compiler-compiler
  is not yet feasible with the current algorithm.
  \begin{itemize}
    \item One difficulty is that the algorithm is complicated and relies on
      modules implementing some theoretical results, which, to the best of our
      knowledge have never been implemented.
    \item Another difficulty is that a certain design decision in the
      implementation of the standard \texttt{javac} compiler may choke on the
      Java code produced by the algorithm.
  \end{itemize}
  \end{enumerate}
\end{quote}

Other concrete contributions made by this work include
\begin{itemize}
  \item the understanding that the definition of fluent APIs is analogous to
      the definition of a formal language.
  \item a lower bound (deterministic pushdown automata).
    on the theoretical ‟computational complexity” of the \Java type system.
  \item an algorithm for producing a fluent API for deterministic context free languages.
  \item a collection of generic programming techniques, developed towards this algorithm.
  \item a demonstration that the runtime of Oracle's \texttt{javac} compiler may be exponential in the program size.
\end{itemize}


\textbf{Outline.}
The first main result is presented in~\cref{Section:proof}.
Towards this,~\cref{Section:preliminaries}, the preliminaries, 
  recalls the central notions we rely on: DSL, fluent API,
  context free languages, pushdown automata, etc. 
At the end of this section, the accumulated vocabulary is used to state this
  result more formally.
This terminology is used again in \Cref{Section:related} to offer 
  a perspective on related work.
\Cref{Section:toolkit} then builds a small toolkit of idioms and techniques
  for programming generics in \Java.
This toolkit is used in the subsequent~\cref{Section:proof} for
  proving our main theoretical result.
On~\Cref{Section:compiler} we will discuss our main computational tool, 
  e.g., the \Java compiler, and its expressiveness.

\Cref{Section:fajita} introduces \Self.
The algorithm that drives \Self, of generating the complex
  type encoding behind the parser generator
  is described in~\cref{Section:algorithm}.
The main ideas behind the bootstrapping definition of \Self
  are revealed in~\Cref{Section:bootstrapping}.
\Cref{Section:AST} shows how \Self can be used not only
  for recognizing the input language,
  but also for creating the appropriate parse tree, and even the \emph{\textbf Abstract \textbf Syntax \textbf Tree} (AST):
  The fluent API code generated by \Self builds,
    when run, the AST of any permissible call chain.
    And, of course, this AST is compliant with the BNF definition
      of the fluent API.


\section{Reminders and Preliminaries}
\label{Section:preliminaries}
\subsection{Method Chaining \emph{vs.} Fluent API}
The pattern ‟invoke function on variable \cc{sb}” occurs
six times in the code in \cref{Figure:chaining}(a).

\begin{figure}[H]
  \caption{\label{Figure:chaining}%
    Recurring invocations of the pattern ‟invoke function on the same
      receiver”, before, and after method chaining.
  }
  \begin{tabular}{cc}
  \begin{lcode}[minipage,width=44ex,box align=center]{Java}
String time(int hours, int minutes, int seconds) {¢¢
  StringBuilder sb = new StringBuilder();
  sb.append(hours);
  sb.append(':');
  sb.append(minutes);
  sb.append(':');
  sb.append(seconds);
  return sb.toString();
}\end{lcode}
\hfill
&
  \begin{lcode}[minipage,width=44ex,box align=center]{Java}
String time(int hours, int minutes, int seconds) {¢¢
    return new StringBuilder()
      ¢¢.append(hours).append(':')
      ¢¢.append(minutes).append(':')
      ¢¢.append(seconds)
      ¢¢.toString();
}\end{lcode}
\\
\textbf{(a)} before & \textbf{(b)} after 
\end{tabular}
\end{figure}

Some languages, e.g., \Smalltalk offer syntactic sugar, called \emph{method cascading}, for this pattern.
Method chaining is a ‟programmer made” syntactic sugar for this pattern:
  If a method~$f$ returns its receiver, i.e., \kk{this},
  then, instead of the series of two commands \mbox{\cc{o.$f$(); o.$g$();}}, clients can write
  only one command \mbox{\cc{o.$f()$.$g$();}}.
  \cref{Figure:chaining}(b) is the method chaining
  (also, shorter and arguably clearer) version of
  \cref{Figure:non-chaining}(a).
It is made possible by the designer of class \cc{StreamBuilder} making sure that all overloaded variants of
  of \cc{append} return their receiver.


Now, definition found in the web of the term \emph{fluent API†{API =
    \textbf Application \textbf Program \textbf Interface
}} are a bit illusive.
Invariably, such definitions liken fluent API to method chaining;
  they also tend to emphasize that fluent API is ‟more” than method
  chaining, and ask the reader to extrapolate from one term to the other.

The distinction between the two terms is the identity of the receiver.
In method chaining, all methods are invoked on the same object, whereas in fluent API
  the receiver of each method in the chain may be arbitrary.
Perhaps surprisingly, this difference makes fluent API more expressive.
Consider, for example, the following JAVA fragment (drawn from JMock~\cite{Freeman:Pryce:06})
\begin{JAVA}
allowing (any(Object.class))
  ¢¢.method("get.*")
  ¢¢.withNoArguments();
\end{JAVA}
Let the return type of function \cc{allowing} be denoted by~$τ₁$ and let the
  return type of function \cc{method} be denoted by~$τ₂$.
Then, the fact that~$τ₁≠τ₂$ means that the set of methods that can be placed after the dot
  in the partial call chain
\begin{JAVA}
allowing(any(Object.class)).
\end{JAVA}
is distinct from the set of methods that can be placed after the dot in the partial call chain
\begin{JAVA}
allowing(any(Object.class)).method("get.*").
\end{JAVA}
This distinction makes it possible to design expressive and rich fluent APIs, in which a
  sequence ‟chained” calls is not only readable, but also robust, in the sense that the
  sequence is type correct, only when it the sequence makes sense semantically.

\subsection{Fluent API and the Type Safe Toilette Seat}

An object of type toilette seat is created in the \cc{down} state, but it can
then be \cc{raise}d to the \cc{up} state, and then be \cc{lower}ed to the
\cc{down} state.†{%
  This example is inspired by earlier work of
  Richard Harter on the topic~\cite{Harter:05}.
}
Such an object be used by two kinds of users, \cc{male}s and \cc{female}s, for two distinct purposes:
  \cc{urinate} and \cc{defecate}.
Now a good fluent API design is one by which the sequences of method calls in
  \cref{Figure:toilette:legal} are type correct.

\begin{figure}[htbp]
  \begin{JAVA}
new Seat().male().raise().urinate();
new Seat().female().urinate();\end{JAVA}
  \caption{Legal sequences of calls in the toilette seat example}
  \label{Figure:toilette:legal}
\end{figure}

Conversely, sequences of method calls made in \cref{Figure:toilette:illegal}
  should be illegal in the desired implementation of the fluent API.

\begin{figure}[htbp]
  \begin{JAVA}
new Seat().female().raise();
new Seat().male().raise().defecate();
new Seat().male().male();
new Seat().male().raise().urinate().female().urinate();\end{JAVA}
  \caption{Illegal sequences of calls in the toilette seat example}
  \label{Figure:toilette:illegal}
\end{figure}
It should be clear that the type checking engine of the compiler can
be employed to distinguish between legal and illegal sequences.
It should also be clear that fabricating the \kk{class}es, \kk{interface}s
and the \kk{extends} and \kk{implements} relationships between these, is
far from being trivial.

\subsection{What is fluent API?}

\subsection{Context Free Languages and Pushdown Automata: Reminder and Terminology}
Each of the notions discussed here is probably common knowledge
 (see e.g.,~\cite{Hopcroft:book:2001,must be others} for more precise definitions).
The purpose here is to set a unifying common vocabulary. 

Let~$Σ$ be a finite alphabet of \emph{terminals} (often called input symbols).
A \emph{language} over $Σ$
  is a subset of~$Σ^*$ (the set of all strings, including the empty string,
  whose characters are drawn from~$Σ$).
Keep~$Σ$ implicit henceforth.

A \emph{\textbf Nondeterministic \textbf Pushdown \textbf Automaton} (NPDA) is a device for language recognition, 
  made of a non-deterministic finite automaton 
  and a stack of unbounded depth of states.
An NPDA begins by pushing into the initial state into the empty stack.
In each step the NPDA examines the next input symbol and the state at the
  top of the stack.
It then non-deterministically between three options: to proceed to the next input symbol,  
  to pop a state from a stack, or to push a state into it. 

The language recognized by an NPDA is the set of strings the it accepts,
  either by reaching an empty stack or by encountering an empty state. 

A \emph{\textbf Context-\textbf Free \textbf Grammar}(CFG) is a formal description of a language.
A CFG~$G$ has three components:~$Ξ$,
a set of \emph{variables} (also called nonterminals), a unique \emph{start variable}~$ξ∈Ξ$, and
  a finite set of (production) \emph{rules}.
A rule~$r∈G$ describes the derivation of a variable~$ξ∈Ξ$ into
  a string of \emph{symbols}, where symbols are either terminals or variables.
Accordingly,~$r$ is written as~$r=ξ→β$, where~$β∈\left(Σ∪Ξ\right)^*$.
This description is often called BNF.
The \emph{language} of a CFG is the set of strings of terminals (and terminals only)
  that can be derived from the start symbol, following any sequence of applications of the rules.
CFG languages include regular languages, and are contained in the set
  of ‟context-sensitive” languages.

The expressive power of NPDAs and BNFs is the same: 
  For every language recognized by a BNF, there is an NPDA that recognizes it. 
Conversely, there is a BNF definition for any language recognized by some NPDA. 

NPDAs run in exponential deterministic time. 
A more sane, but weaker, alternative is offered by LR(1) parsers.  
An LR(1) parser for a grammar~$G$ has three components:
\begin{description}
  \item[An automaton] whose states are~$Q=❴q₀,…,qₙ❵$
  \item[A stack] which may contain any member of~$Q$.
  \item[Specification of state transition] with the aid of two tables:
        \begin{description}
          \item[Goto table] which defines a partial function~$δ:Q⨉Ξ↛Q$ of transitions
          between the states of the automaton.
          \item[Action table] which
            defines a partial function\[η:Q⨉Σ↛ ❴ \textsf{Shift}(q) \,|\, q∈Q❵ ∪ ❴\textsf{Reduce}(r) \,| \, r∈G❵.\]
        \end{description}
\end{description}
The parser begins by pushing~$q₀$ (the initial state) into the stack,
and then repetitively executes the following:
Examine~$q∈Q$, the state at the top of the stack.
If~$q=qₙ$ (the accepting state), then the parser stops in accepting the input.
Let~$σ∈Σ$ be the next input symbol.
If~$η(q,σ)=⊥$, the parser stops in rejecting the input.
If~$η(q',σ) = \textsf{Shift}(q')$, the parser pushes state~$q'$ into the stack.
If however,~$a(q,σ) = \textsf{Reduce}(r)$,
where~$r=ξ→β$,
the parser pops~$|β|$ stack symbols from the stack.
Let~$q'$ be the state at the top of the stack after these pops.
The parser then pops~$q'$, the new state it reached,
  and pushes instead a state,~$q”$, 
  defined by applying the transition function on~$ξ$, the variable just discovered and~$q'$ , i.e.,~$q”=δ(q',ξ)$.

More general, LL($k$)-parsers, $k>1$, can be defined. These make their
  decisions based on the next $k$ input symbols, rather than just the first of these.
These are rarely used, because they offer essentially the same expressive power as 
  LL(1) parsers, at a greater toll on resources. 

To characterize the expressive power of LR(1) parsers, we need the notion 
of ``\emph{\textbf Deterministic \textbf Pushdown \textbf Automaton}'' (DPDA). 
DPDA are similar to NPDA, except that they are forced 
  to make deterministic choices.
More formally, 
\begin{Definition}[Deterministinc Pushdown Automaton]
  \label{Definition:DPDA}
  A \emph{deterministic pushdown automaton} (DPDA),~$M$, is a 7-tuple
  \[
    M =⟨Q,Γ, q₀,⊥, A,δ,η⟩
  \]
  where~$Q$ is a finite set of
  \emph{the states of~$M$},~$Γ$ is a finite
  \emph{set of stack symbols},~$q₀∈Q$ is the initial state,~$⊥∈Γ$
  is a \emph{special symbol designating the bottom of the stack}
  and~$A⊆Q$ is the \emph{set of accepting states} while~$δ$ and~$η$ are
  the \emph{partial functions of state transition}
  \[
    \begin{array}{crlc}
      δ: & Q⨉Σ⨉Γ &↛& Q⨉Γ^*⏎
      η: & Q⨉Γ &↛& Q⨉Γ^*,⏎
    \end{array}
  \]
\end{Definition}

As mentioned above, NPDA languages are the same as CFG languages.
It is convenient to speak of also of DCFG languages, the \emph{deterministic context-free grammar} languages, 
  which are those context-free languages recognizable by DPDA.
  
DCFG languages are strictly contain regular languages, and are strictly contained
  in CFG languages.
DPDA are however easier to parse. Recognition of a DPDA language 
  can be done in linear time and one pass.
  In contrast, the best algorithms for recognizing NPDA languages run in super-quadratic time~\cite{CYK,I forget the nnames}. 

Where do LR(1) languages stand with respect to DCFG context free languages? 
LR(1) languages are indeed equivalent to DCFG languages, but 
the answer hinges on the distinction 
  between LR(1) languages and LR(1) grammars. 
To actually produce an LR(1) parser for a given language, 
  one needs to find an``\emph{LR(1) grammar}'', for the language.
Such a grammar is amenable to 
  the automatic production of an LR(1) parser.
However the time for obtaining this LR(1) grammar and the processing it, 
  though polynomial, may be prohibitive. 


\begin{wrapfigure}r{0.5\linewidth}
  \caption{ \label{Figure:expressiveness}
  Hierarchy of CFGs and pushdown automata}
  \input ../Figures/expressiveness-diagram.tikz
\end{wrapfigure}
In order to see the full picture of the expressiveness of the discussed classes, we will only
  only mention the LL($*$) class and it's expressiveness. The LL($*$)~\cite{Parr:2011} class does not contain,
  and neither contained by any of the LR($k$) grammar classes, in addition, LL($*$) can also
  parse some context-sensitive grammars.
\Cref{Figure:expressiveness} is a visualization of the discussed grammar's hierarchy.




\section{Related Work}
\label{Section:related}
Modern programming languages acquire high-level constructs
  at a staggering rate.
The imminent adoption of closures in \Java and \CC,
  the generators of \CSharp, and ‟concepts” in
  \CC are just a few examples.

A theoretical motivation for this work
  is the exploration of the computational
  expressiveness of such features.
For example, it is known (see e.g.,~\cite{Gutterman:2003}) that
  \kk{template}s in \CC are Turing complete in the following precise sense:

\begin{Theorem}
  \label{Theorem:Gutterman}
  For every Turing machine,~$m$, there exists a \CC program,~$Cₘ$ such that
    compilation of~$Cₘ$ of terminates if and only if
      Turing-machine~$m$ halts.
  Furthermore, program~$Cₘ$ can be effectively generated from~$m$.
\end{Theorem}

Intuitively, the proof relies on the fact that \kk{template}s
  feature recursive invocation and conditionals (in the form of
  ‟\emph{template specialization}”).

There has already been a similar \Java implementation for regular
  languages\urlref{https://github.com/verhas/fluflu}.
  
In the same fashion, it is mundane to make the judgment that
  \Java's generics are not Turing-complete: all recursive calls
  in these are unconditional.
In a sense, this article shall give a lower bound on the
  expressive power of \Java generics in terms of the Chomsky hierarchy~\cite{Chomsky:1963}.
This objective is more precisely expressed in the following conjecture.

Boost is a cool c++ templating library!\cite{Abrahams:Gurtovoy:04} I think. 
This how you can calculate the derivative of a function with \CC compiler! \cite{Gil:Gutterman:98}

Mention funny tricks with annotations to Java. There is \cite{Papi:08} from 
  Washington State university. He fought for more support for annotations 
  and built a system for implementing non
Expression templates is a \CC technique for passing expressions as function arguments. \cite{Veldhuizen:95}

Mention work by \cite{Bracha} on non-standard type systems.  

Eric Bodden wrote an article about fluent APIs, static and dynamic analysis, and type-state~\cite{Bodden:14}

\subsection{Type State}
There is large body of research on \emph{type-states} (See e.g., review articles such
  as~\cite{Aldrich:Sunshine:2009,Bierhoff:Aldrich:2005}).
Informally, an object that belongs to a certain type, has
type-states, if not all methods defined in this object's class are applicable
to the object in all states it may be in.

A classical example of type-states is a file object which can be in one of two
states: ‟open” or ‟closed”. Invoking a \cc{read()} method on the object is only
permitted when the file is in an ‟open” state. In addition, method \cc{open()}
(respectively \cc{close()}) can only be applied if the object is in the
‟closed” (respectively, ‟open”) state.

Objects with type states such as files are not rarities.
In fact, a recent study~\cite{Beckman:2011} estimates
  that about 7.2% of \Java classes define protocols, that can be interpreted as type-state.
Type-state pose two main challenges to software engineering
\begin{enumerate}
  \item \emph{\textbf{Identification.}}
    In the typical case, type-state
        receive little to no mention at all in the documentation.
    The identification problem is to find the implicit
    type state in existing \Java: Given an implementation of a class
    (or more generally of a software framework),
    \emph{determine} which sequences of method calls are valid and which violate the
    type state hidden in the \Java.
  \item \emph{\textbf{Maintenance and Enforcement.}}
    Having identified the type-states, the challenge is in automatically flagging out
      illegal sequence of calls that does not conform
      with the type-state, furthermore, with the
      evolution of an API, the challenge is in updating the type-state information,
      and the type checking of \Java of clients.
\end{enumerate}

\begin{wrapfigure}[9]r{35.05ex}
 \begin{tabular}[align=center]{m{7ex} | m{9ex} @{}| m{9ex}}
 & \cc{open()} & \cc{close()}⏎ \hline
 ‟closed”\ & \color{blue}{\emph{become ‟open”}} & \color{red}{\emph{runtime error}}⏎\hline
 ‟open” & \color{red}{\emph{runtime error}} & \color{blue}{\emph{become ‟closed”}}⏎
 \end{tabular}
\end{wrapfigure}

\begin{wrapfigure}[9]r{35.05ex}
\caption{\label{Figure:box}Fluent API of a box object, defined by a DFA}
  \input ../Figures/open-close-example.tikz
\end{wrapfigure}

To make the proof concrete, consider this example of fluent API definition:
An instance of class \cc{Box}
may receive two method invocations: \cc{open()} and \cc{close()},
and can be in either ‟open” or ‟closed” state,
Initially the instance is ‟closed”.
Its behavior henceforth is defined by \cref{Figure:box}.

To realize this definition, we need a type definition by which \cc{\kk{new} Box().open().close()}, more generally
  blue, or accepting states in the figure, type-check.
Conversely, with this type definition, compile time type error should occur in \cc{\kk{new} Box.close()},
  and, more generally, in the red state.

Some skill is required to make this type definition: proper design of class \cc{Box}, perhaps with
  some auxiliary classes extending it, an appropriate method definition here and there, etc.

The proof makes a general recipe for handling examples of this sort:
\begin{itemize}
  \item First, consider the language defined by the fluent API\@.
        In the box example, this language is defined by the regular expression
        \[
          L = \big(\cc{.open().close()}\big)^*\big(\cc{.open()}\:|\:ε\big).
        \]
  \item Second, check whether this language is deterministic context-free.
        If it is, the fluent API can be realized, and,
        there is an algorithm to produce the respective type definition.
        In the box example, since language~$L$ is specified by a regular expression,
        it is trivially deterministic context-free.
        \par
        It follows from the proof that there exists a type definition
        which realizes the box example.
        Moreover, there is a way
        to automatically produce this type definition.
\end{itemize}

The proof is a construction of a \Java type encoding of
  the \emph{deterministic pushdown automaton} that recognizes
  a given \emph{deterministic context free language}.
With the generated types and methods, the compilation process of
  any chain of fluent API calls, actually runs the pushdown automaton against the
  specific input string that the chain represents.
When used appropriately, if this run of the automaton ends with an accepting state†{The acceptance of a PDA can also be defined by an empty stack, we will use the accepting state type of PDAs},
  then the fluent API chain type checks correctly.
If however this run ends with a failure, i.e., non-accepting state,
  compile time error will occur.


\section{The Jump-Stack Data-Structure}
\label{Section:jump}
A \emph{jump-stack} is a stack data structure whose elements are drawn from a finite set~$Γ$,
  except that jump-stack supports~$\textsf{jump}(γ)$,~$γ∈Γ$ operations,
    which means
  ‟repetetively pop elements from the stack up to and including the first occurrence of~$γ$”.
Let $k=|\Gamma|$. 

\begin{wrapfigure}[16]{r}{42ex}
  \caption{Skeleton of type encoding for the jump-stack data structure}%
  \label{Figure:jump}%
  \lstset{style=numbered}
  \javaInput[minipage,left=-2ex]{jump-stack.listing}
\end{wrapfigure}

\Cref{Figure:jump} shows the skeleton of type-encoding of a jump-stack whose
elements are drawn from type~\cc{$Γ$}
(\cref{Figure:unary-function}), i.e., either~\cc{$γ$1} or~\cc{$γ$2}.

Just like \cc{Stack} (\cref{Figure:stack-encoding}(b)),
  the generic type \cc{JS} which encodes jump-stacks, takes
  a \cc{Rest} parameter which is the type of a jump-stack after popping.
In addition \cc{JS} takes $k$ type parameters, one for~$γ∈Γ$,
  which is the type encoding of the jump-stack after a~$\textsf{jump}(γ)$
  operation.
In the figure, there are two such parameters: \cc{J\_$γ$1}, and
  \cc{J\_$γ$2}.

Functions defined in \cc{JS} include not only the standard stack opertions: \cc{top},
\cc{pop()}, \cc{$γ1$()} and~\cc{$γ2$()} (encoding a push of~$γᵢ$,~$i=1,2$),
  but also functions \cc{jump\_$γ$1} and \cc{jump\_$γ$2},
  which encode~$\textsf{jump}(γᵢ)$
  thanks to the return type being~\cc{J\_$γ$i},~$i=1,2$.

The type hierarchy rooted at \cc{JS} is similar to that of
\cref{Figure:stack-encoding}(a):
  Two of the specializations are parameterless and are
  almost identical to their \cc{Stack}
  counterparts:
\cc{JS.E} encodes an empty jump-stack; \cc{JS.¤} encodes a jump-stack in error,
e.g., a after popping from \cc{JS.E}.


Type \cc{JS.P} (line 15 in the figure) makes  the third specialization of \cc{JS}, representing 
  a stack with one or more elements.
There are no overriden functions in \cc{JS.P}; it achieves
  it purpose by the parameters it takes and those it passes
  to the type it extends.

\begin{wrapfigure}[10]r{43ex}
  \caption{\label{Figure:jump-stack-push} Type \cc{JS.P} encoding a non-empty jump-stack}
  \javaInput[minipage,width=43ex,left=-2ex]{jump-stack-push.listing}
\end{wrapfigure}

Specifically, \cc{JS.P} takes 
the same \cc{Top} and \cc{Rest} paramters (ll.16--17) as type \cc{Stack.P}:
  as well as $k$ additional paramters:
  \cc{J\_$γ$1} and \cc{J\_$γ$2} (ll.18--18)
which are the types encoding the jump-stack
  after the executation~$\textsf{jump}(γᵢ)$,~$i=1,2$.
Type \cc{JP.P'} passes these four parameters 
to type \cc{Pʹ} which it extends (l.21)
The fifth parameter to \cc{Pʹ} (l.22) is the current incarnation of \cc{P}, i.e., 
  \cc{P<Top, Rest, J\_γ1, J\_γ2>}.

The auxliary type \cc{JS.Pʹ} itself is depicted in \cref{Figure:jump-stack}.
Extending type \cc{JS} and passing the correct \cc{Rest} parameter to it, 
\cc{JS.Pʹ} inherits a correct declaration of function \cc{pop()} (l.6~\cref{Figure:jump}) 



\section{Proof of \Cref{Theorem:Gil-Levy}}
\label{Section:proof}
We now turn to the proof of \cref{Theorem:Gil-Levy}, which will be showing a type encoding for the DPDA that recognizes a certain DCFG language.
The proof is by reduction to type encoding of SDPDAs, which are simpler version of deterministic pushdown automata.

\subsection{Reduction}
The difficulty in type-encoding of general DPDAs is that of~$ε$-moves:
A DPDA is allowed to make a transition into a different state,
  pop one or more stack elements, and then, pop, push or even both pop and push, any number of stack elements,
  all without consuming a single input symbol.

In contrast, our type encoding relies on the encoding of an input string~$α∈Σ^*$, one symbol at a time, as a sequence of method calls.
A step of computation occurs in the domain of \Java types only at method calls. There are no-provisions for encoding~$ε$-moves, however deterministic
  they are.
The following definition is of a deterministic pushdown automata without such moves.

\begin{Definition}
  \label{Definition:SDPDA}
  \slshape
  A \emph{\textbf Simplified \textbf Deterministic \textbf Pushdown \textbf Automaton} (SDPDA) of order~$k$ is
    a quintuple~$⟨Γ,Q,q₀,A,δ⟩$,
  where~$Γ$,~$Q$,~$q₀∈Q$,~$A⊆Q$ are precisely as in \cref{Definition:DPDA}.
  The signature of function~$δ$, the \emph{generalized partial transition function}
  of an SDPDA of order~$k$, is however~$δ: Q⨉\left(Γ∪❴\vdash❵\right)ᵏ⨉Σ↛Q⨉Γ^*$.
  \par
  An SDPDA begins as a DPDA\@. At each step it examines~$σ∈Σ$,
    the next input symbol,~$q∈Q$, the current state,
    and~$γ₁,⋯,γₖ∈Γ$, the~$k$ top most stack elements,
  \par
  If the stack is empty, or~$q∈A$, the automaton stops in accepting.
  If the transition function is undefined, i.e.,~$δ(q,γ₁,…,γₖ,σ)=⊥$ the automaton
    stops in rejection.
  Otherwise, and since there are no~$ε$-moves, there are~$q'∈Q$ and~$ζ∈Γ^*$
    such that~$δ(q,γ₁,…,γₖ,σ)=(q',ζ)$.
  The automaton then pops~$γ₁$ through~$γₖ$, pushes the sequence~$ζ$, and
    moves to state~$q'$.
\end{Definition}

We say that
a \emph{configuration}~$(q,γ)$,~$q∈Q$,~$γ∈Γ$,
  is an \emph{$ε$-configuration} if~$δ(q,γ,ε)=(q',ζ)$ for
    some~$q'∈Q$ and a sequence of stack elements~$ζ$,
Recall that by definition of DPDAs,~$δ(q,γ,σ)=⊥$ for
  all~$σ∈Σ$, i.e., there is precisely one transition
  leading out of an~$ε$-configuration.

Our objective is to remove all~$ε$-configurations,
  replacing these with the non-$ε$ configuration that
  they (eventually) lead to.
This is easy to do when~$|ζ|≥1$, in which
  case the top of the stack is known to be~$γ'$,~$γ'$ being the first
    element of~$ζ$; the configuration at the end of transition
    would be then be~$(q',γ')$, which in its turn may, or may not be,
    an~$ε$-configuration.

However, if~$ζ=ε$, we do not know which element resides at the top
  of the stack.
The transition from the current configuration can thus lead to \emph{multiple} configurations,
  depending on the actual element left at the top of the stack after popping~$γ$.
The definition of a configuration thus must be changed to include the \emph{two} top
  most stack elements.
But, even this would fail if the next configuration is also an~$ε$-configuration that pushes no elements.

To be able to compute the closure of~$ε$-configuration, we shall
  therefore include with the configuration the~$k$ top most stack elements,
  where~$k$ is defined to be large enough so that
  the top of the stack must never be guessed while going
  through~$ε$-configurations.

To compute~$k$ we need to compute the largest possible overdraft
  from the stack along a series of transitions through~$ε$-configuration.
To do so, define the following directed weighted graph:
Nodes in this graph are the~$|Q|·|Γ|$ distinct configurations
  of the form~$(q,γ)$,~$q∈Q$,~$γ∈Γ$.
Let a node~$(q,γ)$ be an~$ε$-configuration for which~$δ(q,γ,ε)=(q',ζ)$.
Then, there is an edge from~$(q,γ)$ to all~$|Γ|$ nodes of the form~$(q',γ')$,~$γ'∈Γ$, i.e.,
  to all possible ‟guesses” of the stack element beneath~$γ$.
Each such edge takes the weight~$1-|ζ|$ to denote the fact
  that the total ‟charge” to the stack is~$1-|ζ|$:
    one stack element (specifically~$γ$) is popped,
    while~$|ζ|$ (which could also be zero), are pushed.
The value of~$k$ is simply the heaviest path in this graph.

In the former graph, cycles might occur.
Sadly, we cannot solve those cycles, since arbitrary number of symbols might be removed from the
  stack, and therefore, the required~$ε$-closure cannot be computed.
As shown by someone~\cite{i:need:to::find:it}, DPDAs with no~$ε$-moves are less expressive than a normal DPDAs,
  and therefore, not all DPDA can be computed.

Now, we shall describe the transition function of an SDPDA in the terms of a DPDA (assuming the former graph is DAG).
Let~$M=⟨Q,Γ,q₀,A,δ⟩$ be a DPDA, our goal is to describe~$Sₘ=⟨Qₘ,Γ,q_{0m},Aₘ,δₘ⟩$
  of order k, the corresponding SDPDA\@.
Notice that the stack elements are the same in both.
The other components of~$Sₘ$ are constructed as follows:
\begin{itemize}
 \item~$Qₘ$ is the set~$Q⨉\left(Γ∪❴\vdash❵\right)ᵏ$
 \item~$q_{0m}$ is the state~$(q₀,\vdash^{k-1}γ)$ where~$γ$ is the stack start element of~$Γ$.
 \item~$Aₘ$ is the set~$❴(q,ζ)∈Qₘ | q∈A❵$
 \item for~$δ(q,γ,σ) = (q',ζ)$
 \begin{itemize}
  \item if~$σ≠ε$ than for all~$(q,ζ'γ')$ such that~$ζ'γ∈Γᵏ$
    we define~$δₘ(q,ζ'γ,σ)=(q',ζ'ζ)$
  \item if~$σ=ε$ then~$δₘ(q,ζ,ε)=(q',ζ')$ when~$(q',ζ')$
    is the~$ε$-closure on~$(q,ζ)$.
    The~$ε$-closure of~$(q,ζ)$ is the series of consecutive~$ε$-transitions from~$δ$
    that end with a single ‟input consuming” transition.
    This computation is possible due to the definition of~$k$, that assures us, that during this ‟static”
    computation on~$δ$, we will at all times know the top of the stack, and therefore, this computation
    is viable.
 \end{itemize}

\end{itemize}

\subsection{Type Encoding}
\begin{Theorem}
  \label{Theorem:SDPDA}
  For every SDPDA~$a$ there exists a set~$J_a$ of \Java type definitions, such that
  the command \[
    \cc{A_= M.build~$\textsf{java}(α)$.\$()};
  \]
  compiles against~$Jₘ$ if an only if~$α$ is the language recognized by~$M$.
\end{Theorem}

The remainder of this section is dedicated to the proof of \cref{Theorem:SDPDA}.
In particular, we describe how~$J_M$ is constructed from the
  specification of~$M$.

We will use the following notation for
The stack contents~$⊥ξ₁ξ₁ξ₂ξ₁$,

\begin{figure}[H]
\begin{JAVA}
class Stack<Head extends ¢$Γ$¢, Rest extends Stack<?,?> > {¢¢
  Head head;
  Rest rest;
  Stack(Head head, Rest rest) {¢¢ this.head = head; this.rest = rest;}
  Rest pop() {¢¢ return rest; };
}
\end{JAVA}
For each state~$ξ∈Q$, we generate a \Java class~$ξ$,
\begin{JAVA}
class ¢$ξ$¢ <T extends S> extends S<T> {¢¢
  ¢$ξ$¢(T t) {¢¢ super(t); }
  // ¢…¢
}
\end{JAVA}
\end{figure}
In addition, we define a special class~$\vdash$ to designate the empty stack.
\begin{JAVA}
class ¢$\vdash$¢ extends Stack<¢$Γ$¢, ¢$\vdash$¢> {¢¢
  ¢$\vdash$ ¢() {¢¢ super(null); }
  ¢$\vdash$ ¢pop() {¢¢ throw new RunTimeException(); }
}
\end{JAVA}
The stack contents~$⊥ξ₁ξ₁ξ₂ξ₁$,
where~$ξ₁,ξ₂∈Q$ are stack elements,
is represented by the following type
\begin{JAVA}
  ¢$ξ₁$¢ < ¢$ξ₂$¢ < ¢$ξ₁$¢ < ¢$⊥$ > > >
\end{JAVA}
This is a general concept for implementing an unbounded stack with \Java's type system,
that will be extended in the future.


\subsection{Encoding states of pushdown automaton with \Java generics}
The generalized transition function \cref{Equation:generalized:transition}
  takes~$k+2$ arguments: a state~$q∈Q$, an input symbol~$σ∈Σ$
    and~$k$ stack elements drawn from~$Γ$.
To capture the full behavior of the automaton, the transition function
  must be able to pass along, in one way or another, the full contents of the stack.

Since the scheme described above can only be applied for binary functions,
  we shall pack together the state~$q$, the stack, and the~$k$ top most
  elements of it into a single type.
The next input symbol,~$σ∈Σ$, the remaining argument of function~$δ$,
  is encoded as a method.

The type part of the encoding is obtained by instantiating a generic type as follows:
Let~\cc{Q} be the abstract class that represents~$Q$, and let~\cc{q} be the concrete class that
  implements~\cc{Q} for an automaton state~$q∈Q$.
Then, to accommodate the extra~$k$ parameters of~$δ$, we add~$k$ generic parameters
  to class~$Q$ and to every class~\cc{q} that implements it.
Yet another such parameter is added for representing stack contents.

Consider the case~$k=2$.
\begin{figure}
  \begin{JAVA}
abstract class Q<S, ¢$γ₁$¢, ¢$γ₂$¢
  \end{JAVA}
\end{figure}

\begin{figure}
  \caption{\label{Figure:SDPDA:hierarchy}%
    Type hierarchy of the type encoding of a simple SDPDA.
  }
  \begin{adjustbox}{}
    \input ../Figures/automaton.tikz
  \end{adjustbox}
\end{figure}

\begin{figure}
  \caption{\label{Figure:SDPDA:example}%
    A type encoding of a simple pushdown automaton.
  }
  \javaInput[minipage,width=\linewidth]{spda.listing}
\end{figure}
As might be expected, this type is obtained
The
The

To encode this function within


\section{\Java Compiler expressiveness}
\label{Section:compiler}
The \Java compiler, is the main computational tool we use
  in this manuscript.
In particular, the \Java Generics mechanism is what
  boosted our expressiveness from the trivial \emph{regular languages}
  set, to the practical, reasonable \emph{deterministic context free languages} set.

An interesting question that was raised during this research,
  is what the runtime complexity of the \Java compiler is.
The reason this question is interesting, is its implication
  on the computational expressiveness of type-encoding.

For example, if we recognized that the \Java parser spends
  linear time on its input, we could say that it's not
  likely that we can type-encode nondeterministic CFG.
The reason is the fact that the best known algorithms
  today for parsing general nondeterministic CFG,
  as mentioned in~\Cref{Section:preliminaries}, run in super quadratic time.
And it is highly unlikely that the type-checker of \Java incidentally
  found an algorithm that is practically perfect in big-O notation.

As we explored this venue, we discovered that the type-checker of
  \Java actually runs in exponential time.

Consider an encoded binary type tree. \kk{class}~\cc{N}
  (denotes node) defined in~\cref{Figure:compiler} defines such tree.

\begin{wrapfigure}[6]r{27ex}
  \caption{\label{Figure:compiler} Encoding of an binary type tree}
  \javaInput[minipage,width=27ex,left=-2ex]{compiler.listing}
\end{wrapfigure}

The class has two type parameters, \cc{RL} (denotes right leaf),
  and \cc{LL} (denotes left leaf), both values are insignificant
  as they act as the base case of a recursive definition.
The class has only one method, \cc{d()} (denotes the keyword double),
  that performs our recursive step : create a new node, that its two
  sons, are the instance of the receiver itself.
The actual return type of \cc{d()} is
  \cc{N$⟨\text{N}⟨\text{LL, RL}⟩\text{, N}⟨\text{LL, RL}⟩⟩$}.
It is clear that the tree represented after each~\cc{d()} invocation is
  of a more than doubled size.

Consider a sequence of \cc{d()} invocations on a \kk{new} \cc{N()} object,
  for which the \Java compiler tries to compute a type.
If the input of the compiler is the former sequence of invocations,
  and we will define its length as~$n$,
  than surely, the type length of the result is~$O(2ⁿ)$.

\begin{wrapfigure}r{43ex}%
  \caption{\label{Figure:compile-empiric} Compile time (sec) \emph{vs.} sequential method invocation length}%
  \gnuplotloadfile[terminal=pdf,terminaloptions={crop size 2.5in,1.5in color enhanced font ",8" linewidth 1}]{../Figures/kill.gnuplot}
\end{wrapfigure}%

An empirical result of the former example is exhibited in \cref{Figure:compile-empiric}.
The \Java compiler therefore runs at an exponential complexity at worst case,
  not only that, it is not confined to~$O(2ⁿ)$, it is clear that
  writing a similar type-encoding for a ternary type tree would result in
  a lower-bound for the \Java compiler of~$O(3ⁿ)$, and for a k-nary tree
  we would get~$O(kⁿ)$.

Thus, we get that the compile time of a \Java is in the E complexity class :~$O(2^O(n))$.

The theoretical question of type-encoding a stronger machine (i.e., Turing Machine)
  is outside the scope of this paper.


\section{Conclusion}
\label{Section:zz}
As should be obvious from \cref{Figure:fluent}, \SELF will be implemented
  in a bootstrapping fashion.
The specification of a BNF, is made using a fluent API.
The BNF for writing BNFs is given in \cref{Figure:BNF:BNF}

\begin{figure}[htbp]
  \scriptsize
  \begin{equation*}
    \def\<#1>{\/⟨\/\text{\textit{#1}}\/⟩\/~}
    \def\|{~|~}
    \let\oldCc=\cc
    \let\oldKk=\kk
    \def\cc#1{{\footnotesize\oldCc{#1}}~}
    \def\cc#1{{\footnotesize\olKk{#1}}~}
    \small
    \begin{aligned}
      \<BNF>              & ::=  \<Notation> \<Body> \<Footer> \hfill⏎
      \<Notation>         & ::=  \<Symbols> \<Terminals> \hfill⏎
      {}                  & \|  \<Terminals> \<Symbols> \hfill⏎
      \<Terminals>        & ::=  \cc{with(Symbols.class)}
      \<Symbols>          & ::=  \cc{with(Terminals.class)}
      \<Body>             & ::= \<Start> \<Rules> \hfill⏎
      \<Start>            & ::=  \cc{with(Class<? \kk{extends} Symbol)} 
      \<Rules>            & ::= \<First-Rule> \<More-Rules> \hfill⏎
      \<More-Rules>       & ::= \<Additional-Rule> \<More-Rules> \hfill⏎
      {}                  & \| ε \hfill⏎
      {}                  & \| \<Lowering-Visitor> \<Down-Visitors> \hfill⏎
      {}                  & \| ε \hfill⏎
      \<Up-Visitor>       & ::= \cc{male()} \cc{urinate()} \hfill⏎
      \<Down-Visitor>     & ::= \cc{female()} \<Action> \hfill⏎
                          & \| \cc{male()} \cc{defecate()} \hfill⏎
      \<Raising-Visitor>  & ::= \cc{male()} \cc{raise()} \cc{urinate()} \hfill⏎
      \<Lowering-Visitor> & ::= \cc{female()} \cc{lower()} \<Action> \hfill⏎
                          & \| \cc{male()} \cc{lower()} \cc{defecate()} \hfill⏎
      \<Activity>         & ::= \cc{urinate()} \hfill⏎
                          & \| \cc{defecate()} \hfill⏎
    \end{aligned}
  \end{equation*}
  \caption{A BNF grammar for the toilette seat problem}
  \label{Figure:BNF:BNF}
\end{figure}


\bibliographystyle{abbrv}\small
\bibliography{author-names,other-shorthands,journals-full,publishers-abbreviated,journals-abbreviated,yogi-book,00}
\end{document}

Processing programming languages
\begin{description}
  \item[Lexical analysis] - the first step of the process in which the character strings generated by the
  programmer are aggregated to the abstract tokens defined by the language designer.
  \item[Syntactical analysis (parsing) ] - the second step, in which the processed strings of tokens
  conform to the rules of a formal grammar defined by the language's BNF (or EBNF).
  \item[Semantical analysis] - the next step, usually performed in unison with the previous step,
  in which the legal token sequences are given their semantic meaning.
\end{description}
Specifically, the proposal is that API design of follows the footsteps of
Accordingly, the designer of a fluent API has to follow these three conceptual
steps.
First is the identification of the \emph{vocabulary}, i.e.,
the set of method calls including type arguments that may take part in the
fluent API\@.
In this fluent API example
\begin{JAVA}
allowing (any(Object.class))
  ¢¢.method("get.*")
  ¢¢.withNoArguments();
\end{JAVA}
then, there are three method calls, and the vocabulary has three items in it.
\begin{itemize}
  \item~$ℓ₁ = \cc{any(Class<?>)}$
  \item~$ℓ₂ = \cc{allowing($ℓ₁$)}$
  \item~$ℓ₃ = \cc{method(String)}$
  \item~$ℓ₄ = \cc{withNoArguments()}$
\end{itemize}
