\title{Formal Language Recognition with the Java Type Checker}
\documentclass[a4paper,USenglish]{lipics-v2016}

\usepackage{\jobname}
\bibliographystyle{plainurl}

\author[1]{Yossi Gil}
\author[1]{Tomer Levy}
\affil[1]{%
 Department of Computer Science, The
 Technion---Israel Institute of Technology, Haifa, Israel.
}

\def\ReplaceInThesis#1#2{#1}

\begin{document}


\maketitle
\hfill
\parbox{40ex}{%
  \begin{flushright}
    \scriptsize\itshape ``\NonCitingUse{Java} generics are 100\% pure syntactic sugar,
    and do not support meta-programming''\footnotemark
  \end{flushright}
}
\newline

\footnotetext{Found on stackoverflow:
  \tiny
      \url{http://programmers.stackexchange.com/questions/95777/generic-programming-how-often-is-it-used-in-industry}
    }

\begin{abstract}
  This paper is a theoretical study of practical problem:
  the automatic generation of Java Fluent APIs from their specification.
We explain why the problem's core lies with 
  the expressive power of Java generics.
Our main result is that automatic generation is possible whenever 
  the specification is an instance of the set of deterministic context-free languages,
  a set which contains most ``practical'' languages.
Other contributions include a collection of techniques and idioms o
  the limited meta-programming possible with Java generics, 
  and an empirical measurement demonstrating that the runtime of
  the ``javac'' compiler of Java the may be exponential in
  the program's length, even for programs composed of 
  a handful of lines and which do not rely on overly 
  complex use of generics.

\end{abstract}

\section{Introduction}
Ever after their inception\urlref{http://martinfowler.com/bliki/FluentInterface.html} \emph{fluent APIs}
  increasingly gain popularity~\cite{Bauer:2005,Freeman:Pryce:06,Larsen:2012} and research
  interest~\cite{Deursen:2000,Kabanov:2008}.
In many ways, fluent APIs are a kind of
  \emph{internal} \emph{\textbf Domain \textbf Specific \textbf Language}:
They make it possible to enrich a host programming language without changing it.
Advantages are many: base language tools (compiler, debugger, IDE, etc.) remain
  applicable, programmers are saved the trouble of learning a new syntax, etc.
However, these advantages come at the cost of expressive power;
  in the words of Fowler:
  ‟\emph{Internal DSLs are limited by the syntax and structure of your base language.}”†
  {M. Fowler, \emph{Language Workbenches: The Killer-App for Domain Specific Languages?},
    2005
    \newline
  \url{http://www.martinfowler.com/articles/languageWorkbench.html#InternalDsl}}.
Indeed, in languages such as \CC, fluent APIs
  often make extensive use of operator overloading (examine, e.g., \textsf{Ara-Rat}~\cite{Gil:Lenz:07}),
  but this capability is not available in \Java.

Despite this limitation, fluent API in \Java can be rich and expressive, as demonstrated
  in \cref{Figure:DSL} showing use cases of the DSL of Apache Camel~\cite{Ibsen:Anstey:10}
(open-source integration framework),
and that of jOOQ\urlref{http://www.jooq.org}, a framework for writing
  SQL in \Java, much like Linq~\cite{Meijer:Beckman:Bierman:06}.

\begin{figure}[H]
  \caption{\label{Figure:DSL} Two examples of \Java fluent API}
  \begin{tabular}{@{}c@{}c@{}}
    \parbox[c]{44ex}{\javaInput[left=0ex]{camel-apache.java.fragment}} &
    \hspace{-3ex} \parbox[c]{59ex}{\javaInput[left=0ex]{jOOQ.java.fragment}}⏎
    \textbf{(a)} Apache Camel & \textbf{(b)} jOOQ
  \end{tabular}
\end{figure}

Other examples of fluent APIs in \Java are abundant:
  jMock~\cite{Freeman:Pryce:06},
  Hamcrest\urlref{http://hamcrest.org/JavaHamcrest/},
  EasyMock\urlref{http://easymock.org/},
  jOOR\urlref{https://github.com/jOOQ/jOOR},
  jRTF\urlref{https://github.com/ullenboom/jrtf}
  and many more.

\subsection{A Type Perspective on Fluent APIs}
\Cref{Figure:DSL}(B) suggests that jOOQ imitates SQL,
but is it possible at all to produce a fluent API
for the entire SQL language,
or XPath, HTML, regular expressions, BNFs, EBNFs, etc.?
Of course, with no operator overloading it is impossible
to fully emulate tokens; method names though make a good substitute for tokens, as done
in ‟\lstinline{.when(header(foo).isEqualTo("bar")).}” (\cref{Figure:DSL}).
The questions that motivate that this research are:
\begin{quote}
  \begin{itemize}
    \item Given a specification of a DSL, determine whether there exists
        a fluent API can be made for this specification?
    \item In the cases that such a fluent API is possible,
      can it be produced automatically?
    \item Is it feasible to produce a \emph{compiler-compiler} such as Bison~\cite{Bison} tool
        to convert a language specification into a fluent API?
\end{itemize}
\end{quote}

Inspired by the theory of formal languages and automata,
  this study explores what can be done, and what can not be done, with fluent API in \Java.

Consider some fluent API (or DSL) specification, permitting only certain call
chains and disallowing all others.
Now, think of the formal language that defines the set of these permissible chains.
The main contribution of this paper is a proof (with its implicit algorithm) that
there is always \Java type definition that \emph{realizes} this fluent definition, provided that this
language is \emph{deterministic context-free}, where
\begin{itemize}
  \item In saying that a type definition \emph{realizes} a specification of fluent
    API, we mean that call chains that conform with the API definition compile
    correctly, and, conversely, call chains that are forbidden by the API
    definition do not type-check, resulting in an appropriate compiler error.
  \item Roughly speaking, deterministic context free languages are those
    context free languages that can be recognized by an LR parser†{The ‟L"
    means reading the input left to right; the ‟R" stands for rightmost derivation}~\cite{Aho:86}.
    \par
    An important property of this family is that none of its members is ambiguous.
    Also, it is generally believed that most practical programming languages
    are deterministic context-free.
\end{itemize}

A problem related to that of recognizing a formal language,
is that of parsing, i.e., creating, for input which is within the language,
  a parse tree according to the language's grammar,
In the fluent APIs domain, the distinction between recognition and parsing is
  the distinction between compile time and runtime.
Before a program is run, the compiler checks whether the fluent API call is legal,
  and code completion tools will only suggest legal extensions of a current call chain.

In contrast, the parse tree is created at runtime.
Some fluent API definitions create the parse-tree
  iteratively, where each method invocations in the call chain adding
  more components to this tree.
However, it is always possible to generate this tree in ‟batch” mode:
This is done by maintaining a \emph{call list} which
  starts empty and grows at runtime by having each method invoked add to it
  a record storing the method's name and values of its parameters.
The list is completed at the end of the call list, at which point it is fed to an appropriate parser that
  converts it into a parse tree (or even an AST).

\subsection{Contribution}
The answers we provide for the three questions above is:
\begin{quote}
  \begin{enumerate}
  \item If the DSL specification is that of deterministic context-free
    language, then a fluent API exists for the language, but we do not know
    whether such a fluent API exists for more general languages.
  \par
  Recall that there are universal cubic time parsing
  algorithms~\cite{add:refs:to:this:cubics} which can parse (and recognize) any
  context free language. What we do not know is whether algorithms of this sort
  can be encoded within the framework of the \Java type system.
  \item
  There exists an algorithm to generate a fluent API that realize any
  deterministic context-free languages.  Moreover, this fluent API can create
  at runtime, a parse tree for the given language.  This parse tree can then be
  supplied as input to the library that implements the language's semantic.
  \item
  Unfortunately, a general purpose compiler-compiler
  is not yet feasible with the current algorithm.
  \begin{itemize}
    \item One difficulty is that the algorithm is complicated and relies on
      modules implementing some theoretical results, which, to the best of our
      knowledge have never been implemented.
    \item Another difficulty is that a certain design decision in the
      implementation of the standard \texttt{javac} compiler may choke on the
      Java code produced by the algorithm.
  \end{itemize}
  \end{enumerate}
\end{quote}

Other concrete contributions made by this work include
\begin{itemize}
  \item the understanding that the definition of fluent APIs is analogous to
      the definition of a formal language.
  \item a lower bound (deterministic pushdown automata).
    on the theoretical ‟computational complexity” of the \Java type system.
  \item an algorithm for producing a fluent API for deterministic context free languages.
  \item a collection of generic programming techniques, developed towards this algorithm.
  \item a demonstration that the runtime of Oracle's \texttt{javac} compiler may be exponential in the program size.
\end{itemize}


\textbf{Outline.} \small
\Cref{section:fluent} is a brief reminder of method chaining,
  and fluent APIs, accompanied a discussion of how this work is related to type states.
It is followed by a similar reminder of context-free languages, pushdown automata,
  and such in \cref{section:pushdown}.
Based on the vocabulary established this far,
  the main result is stated in~\cref{section:result}.

Towards the proof in \cref{section:proof}, \cref{section:toolkit}
  shows idioms and techniques for encoding computation with  
  the \Java type-checker.
\Cref{section:jump} makes use of these for encoding
  ``jump-stack'', a non-trivial data-structure,
  which is used, with suitable modifications, in the proof.

In \cref{section:applicability}, we discuss the challenges in
  translating the proof into a compiler-compiler for fluent APIs.
In particular, this section demonstrates our claim (that may be
  surprising to some) that the standard \Java compiler may spend
  an exponential time on compiling rather simple programs.
\cref{section:zz} concludes with directions for further research.
\normalsize

\section{Method Chaining, Fluent APIs, and, Type States}
\label{section:fluent}
\input fluent

\section{Context-Free Languages and Pushdown Automata: Reminder and Terminology}
\label{section:pushdown}
\input pushdown

\section{Statement of the Main Result}
\label{section:result}
\input result

\section{Techniques of Type Encoding}
\label{section:toolkit}
\input toolkit

\section{The Jump-Stack Data-Structure}
\label{section:jump}
\input jump

\section{Proof of \Cref{Theorem:Gil-Levy}}
\label{section:proof}
We now turn to the proof of \cref{Theorem:Gil-Levy}, which will be showing a type encoding for the DPDA that recognizes a certain DCFG language.
The proof is by reduction to type encoding of SDPDAs, which are simpler version of deterministic pushdown automata.

\subsection{Reduction}
The difficulty in type-encoding of general DPDAs is that of~$ε$-moves:
A DPDA is allowed to make a transition into a different state,
  pop one or more stack elements, and then, pop, push or even both pop and push, any number of stack elements,
  all without consuming a single input symbol.

In contrast, our type encoding relies on the encoding of an input string~$α∈Σ^*$, one symbol at a time, as a sequence of method calls.
A step of computation occurs in the domain of \Java types only at method calls. There are no-provisions for encoding~$ε$-moves, however deterministic
  they are.
The following definition is of a deterministic pushdown automata without such moves.

\begin{Definition}
  \label{Definition:SDPDA}
  \slshape
  A \emph{\textbf Simplified \textbf Deterministic \textbf Pushdown \textbf Automaton} (SDPDA) of order~$k$ is
    a quintuple~$⟨Γ,Q,q₀,A,δ⟩$,
  where~$Γ$,~$Q$,~$q₀∈Q$,~$A⊆Q$ are precisely as in \cref{Definition:DPDA}.
  The signature of function~$δ$, the \emph{generalized partial transition function}
  of an SDPDA of order~$k$, is however~$δ: Q⨉\left(Γ∪❴\vdash❵\right)ᵏ⨉Σ↛Q⨉Γ^*$.
  \par
  An SDPDA begins as a DPDA\@. At each step it examines~$σ∈Σ$,
    the next input symbol,~$q∈Q$, the current state,
    and~$γ₁,⋯,γₖ∈Γ$, the~$k$ top most stack elements,
  \par
  If the stack is empty, or~$q∈A$, the automaton stops in accepting.
  If the transition function is undefined, i.e.,~$δ(q,γ₁,…,γₖ,σ)=⊥$ the automaton
    stops in rejection.
  Otherwise, and since there are no~$ε$-moves, there are~$q'∈Q$ and~$ζ∈Γ^*$
    such that~$δ(q,γ₁,…,γₖ,σ)=(q',ζ)$.
  The automaton then pops~$γ₁$ through~$γₖ$, pushes the sequence~$ζ$, and
    moves to state~$q'$.
\end{Definition}

We say that
a \emph{configuration}~$(q,γ)$,~$q∈Q$,~$γ∈Γ$,
  is an \emph{$ε$-configuration} if~$δ(q,γ,ε)=(q',ζ)$ for
    some~$q'∈Q$ and a sequence of stack elements~$ζ$,
Recall that by definition of DPDAs,~$δ(q,γ,σ)=⊥$ for
  all~$σ∈Σ$, i.e., there is precisely one transition
  leading out of an~$ε$-configuration.

Our objective is to remove all~$ε$-configurations,
  replacing these with the non-$ε$ configuration that
  they (eventually) lead to.
This is easy to do when~$|ζ|≥1$, in which
  case the top of the stack is known to be~$γ'$,~$γ'$ being the first
    element of~$ζ$; the configuration at the end of transition
    would be then be~$(q',γ')$, which in its turn may, or may not be,
    an~$ε$-configuration.

However, if~$ζ=ε$, we do not know which element resides at the top
  of the stack.
The transition from the current configuration can thus lead to \emph{multiple} configurations,
  depending on the actual element left at the top of the stack after popping~$γ$.
The definition of a configuration thus must be changed to include the \emph{two} top
  most stack elements.
But, even this would fail if the next configuration is also an~$ε$-configuration that pushes no elements.

To be able to compute the closure of~$ε$-configuration, we shall
  therefore include with the configuration the~$k$ top most stack elements,
  where~$k$ is defined to be large enough so that
  the top of the stack must never be guessed while going
  through~$ε$-configurations.

To compute~$k$ we need to compute the largest possible overdraft
  from the stack along a series of transitions through~$ε$-configuration.
To do so, define the following directed weighted graph:
Nodes in this graph are the~$|Q|·|Γ|$ distinct configurations
  of the form~$(q,γ)$,~$q∈Q$,~$γ∈Γ$.
Let a node~$(q,γ)$ be an~$ε$-configuration for which~$δ(q,γ,ε)=(q',ζ)$.
Then, there is an edge from~$(q,γ)$ to all~$|Γ|$ nodes of the form~$(q',γ')$,~$γ'∈Γ$, i.e.,
  to all possible ‟guesses” of the stack element beneath~$γ$.
Each such edge takes the weight~$1-|ζ|$ to denote the fact
  that the total ‟charge” to the stack is~$1-|ζ|$:
    one stack element (specifically~$γ$) is popped,
    while~$|ζ|$ (which could also be zero), are pushed.
The value of~$k$ is simply the heaviest path in this graph.

In the former graph, cycles might occur.
Sadly, we cannot solve those cycles, since arbitrary number of symbols might be removed from the
  stack, and therefore, the required~$ε$-closure cannot be computed.
As shown by someone~\cite{i:need:to::find:it}, DPDAs with no~$ε$-moves are less expressive than a normal DPDAs,
  and therefore, not all DPDA can be computed.

Now, we shall describe the transition function of an SDPDA in the terms of a DPDA (assuming the former graph is DAG).
Let~$M=⟨Q,Γ,q₀,A,δ⟩$ be a DPDA, our goal is to describe~$Sₘ=⟨Qₘ,Γ,q_{0m},Aₘ,δₘ⟩$
  of order k, the corresponding SDPDA\@.
Notice that the stack elements are the same in both.
The other components of~$Sₘ$ are constructed as follows:
\begin{itemize}
 \item~$Qₘ$ is the set~$Q⨉\left(Γ∪❴\vdash❵\right)ᵏ$
 \item~$q_{0m}$ is the state~$(q₀,\vdash^{k-1}γ)$ where~$γ$ is the stack start element of~$Γ$.
 \item~$Aₘ$ is the set~$❴(q,ζ)∈Qₘ | q∈A❵$
 \item for~$δ(q,γ,σ) = (q',ζ)$
 \begin{itemize}
  \item if~$σ≠ε$ than for all~$(q,ζ'γ')$ such that~$ζ'γ∈Γᵏ$
    we define~$δₘ(q,ζ'γ,σ)=(q',ζ'ζ)$
  \item if~$σ=ε$ then~$δₘ(q,ζ,ε)=(q',ζ')$ when~$(q',ζ')$
    is the~$ε$-closure on~$(q,ζ)$.
    The~$ε$-closure of~$(q,ζ)$ is the series of consecutive~$ε$-transitions from~$δ$
    that end with a single ‟input consuming” transition.
    This computation is possible due to the definition of~$k$, that assures us, that during this ‟static”
    computation on~$δ$, we will at all times know the top of the stack, and therefore, this computation
    is viable.
 \end{itemize}

\end{itemize}

\subsection{Type Encoding}
\begin{Theorem}
  \label{Theorem:SDPDA}
  For every SDPDA~$a$ there exists a set~$J_a$ of \Java type definitions, such that
  the command \[
    \cc{A_= M.build~$\textsf{java}(α)$.\$()};
  \]
  compiles against~$Jₘ$ if an only if~$α$ is the language recognized by~$M$.
\end{Theorem}

The remainder of this section is dedicated to the proof of \cref{Theorem:SDPDA}.
In particular, we describe how~$J_M$ is constructed from the
  specification of~$M$.

We will use the following notation for
The stack contents~$⊥ξ₁ξ₁ξ₂ξ₁$,

\begin{figure}[H]
\begin{JAVA}
class Stack<Head extends ¢$Γ$¢, Rest extends Stack<?,?> > {¢¢
  Head head;
  Rest rest;
  Stack(Head head, Rest rest) {¢¢ this.head = head; this.rest = rest;}
  Rest pop() {¢¢ return rest; };
}
\end{JAVA}
For each state~$ξ∈Q$, we generate a \Java class~$ξ$,
\begin{JAVA}
class ¢$ξ$¢ <T extends S> extends S<T> {¢¢
  ¢$ξ$¢(T t) {¢¢ super(t); }
  // ¢…¢
}
\end{JAVA}
\end{figure}
In addition, we define a special class~$\vdash$ to designate the empty stack.
\begin{JAVA}
class ¢$\vdash$¢ extends Stack<¢$Γ$¢, ¢$\vdash$¢> {¢¢
  ¢$\vdash$ ¢() {¢¢ super(null); }
  ¢$\vdash$ ¢pop() {¢¢ throw new RunTimeException(); }
}
\end{JAVA}
The stack contents~$⊥ξ₁ξ₁ξ₂ξ₁$,
where~$ξ₁,ξ₂∈Q$ are stack elements,
is represented by the following type
\begin{JAVA}
  ¢$ξ₁$¢ < ¢$ξ₂$¢ < ¢$ξ₁$¢ < ¢$⊥$ > > >
\end{JAVA}
This is a general concept for implementing an unbounded stack with \Java's type system,
that will be extended in the future.


\subsection{Encoding states of pushdown automaton with \Java generics}
The generalized transition function \cref{Equation:generalized:transition}
  takes~$k+2$ arguments: a state~$q∈Q$, an input symbol~$σ∈Σ$
    and~$k$ stack elements drawn from~$Γ$.
To capture the full behavior of the automaton, the transition function
  must be able to pass along, in one way or another, the full contents of the stack.

Since the scheme described above can only be applied for binary functions,
  we shall pack together the state~$q$, the stack, and the~$k$ top most
  elements of it into a single type.
The next input symbol,~$σ∈Σ$, the remaining argument of function~$δ$,
  is encoded as a method.

The type part of the encoding is obtained by instantiating a generic type as follows:
Let~\cc{Q} be the abstract class that represents~$Q$, and let~\cc{q} be the concrete class that
  implements~\cc{Q} for an automaton state~$q∈Q$.
Then, to accommodate the extra~$k$ parameters of~$δ$, we add~$k$ generic parameters
  to class~$Q$ and to every class~\cc{q} that implements it.
Yet another such parameter is added for representing stack contents.

Consider the case~$k=2$.
\begin{figure}
  \begin{JAVA}
abstract class Q<S, ¢$γ₁$¢, ¢$γ₂$¢
  \end{JAVA}
\end{figure}

\begin{figure}
  \caption{\label{Figure:SDPDA:hierarchy}%
    Type hierarchy of the type encoding of a simple SDPDA.
  }
  \begin{adjustbox}{}
    \input ../Figures/automaton.tikz
  \end{adjustbox}
\end{figure}

\begin{figure}
  \caption{\label{Figure:SDPDA:example}%
    A type encoding of a simple pushdown automaton.
  }
  \javaInput[minipage,width=\linewidth]{spda.listing}
\end{figure}
As might be expected, this type is obtained
The
The

To encode this function within


\section{The Prefix Theorem}
\label{section:prefix}
\begin{theorem}\label{Theorem:Gil-Levy:2}
  Let~$A$ be a DPDA recognizing a language~$L⊆Σ^*$.
  Then, there exists a \Java type definition,~$J_A$ for types~\cc{L},~\cc{A} and
    other types such that the \Java command
  \begin{equation}
    \label{Equation:result}
    \cc{A.build~$\textsf{java}(α)$;}
  \end{equation}
  type checks against~$J_A$ if an only if there exists~$β∈Σ^*$ such
  that~$αβ∈L$.
  Furthermore, program~$J_A$ can be effectively generated from~$A$.
\end{theorem}

Informally, a call chain type-checks if and only if it is a prefix
  of some legal sequence.
Alternatively, a call chain won't type-check if there is no
  continuation that leads to a legal string in~$L$.

The proof resembles~\cref{Theorem:Gil-Levy}'s proof.
We provide a similar implementation for a jump-stack (see\cref{Definition:JDPDA}),
  that will not compile under illegal prefixes.

The main difference between the two theorems is:
  in~\cref{Theorem:Gil-Levy} we allowed illegal call chains to compile,
  but not return the required~\cc{L} type, while in~\cref{Theorem:Gil-Levy:2}
  the illegal chain won't compile at all.
  
We will use the same running example, defined in~\cref{Table:A}.

Since the code suggested by the proof highly resembles the previously
  suggested code, we will describe the differences.
  
\subsection{Main Types}
The main types here are a subset of the previously defined main types.

\begin{quote}
  \javaInput[left=-2ex,minipage,width=54ex]{prefix-proof.configuration.listing}
\end{quote}

First, \kk{class} \cc{$\Sigma\Sigma$} is removed. 
A call chain that doesn't represent a valid prefix won't compile, 
  thus, there is no need for an error return type such as \cc{$\Sigma\Sigma$}.
Second, \kk{interface} \cc{C} is removed. 
Without it, the configuration types won't have the 
  methods \cc{$σ$1()}, … ,\cc{$σ${}$k$()} and \cc{\$()} from the supertype.
These inherited methods, is what differentiates the previous proof from the current.

\subsection{Top-of-Stack Types}
Types \cc{C$γ$1}, … ,\cc{C$γ${}$k$}, still represent stacks
  with \cc{$γ$1}, … ,\cc{$γ${}$k$} as their top,
  this time, the methods are defined ad-hock, in each type.
In A there are two such types:

\begin{quote}
  \javaInput[left=-2ex,minipage,width=45ex]{prefix-proof.many.listing}
\end{quote}

\subsection{Transitions}
In the proof of~\cref{Theorem:Gil-Levy} we defined class~\cc{C} to have 
  default return values for each of the methods \cc{$σ$1()}, … ,\cc{$σ${}$k$()} and \cc{\$()},
  that cannot be continued in a fashion that makes sense (i.e., cannot return a type~\cc{L}).
  
The different handling with the entries of the transition table is as following:
\begin{description}
 \item 
\end{description}


\section{Notes on Practical Applicability}
\label{section:applicability}
\Cref{Theorem:Gil-Levy} and its proof above provide
  a concrete algorithm for converting an EBNF specification of a fluent API into
its realization:
\begin{quote}
  \begin{enumerate}
    \item Convert the specification into a plain BNF form
    \urlref{http://lampwww.epfl.ch/teaching/archive/compilation-ssc/2000/part4/parsing/node3.html}.
    \item Convert this BNF into a definition of a DPDA. If conversion fails,
      then the given specification is not deterministic context-free.
    \item Convert this DPDA into a jDPDA. (Conversion is guaranteed to succeed.)
    \item Apply the proof to generate appropriate \Java type definitions, making sure to
        augment methods with code to maintain the fluent-call-list.
        Parsing the fluent-call-list can done either in each method,
        or lazily, when the product of the fluent API call chain is to
         be used.
  \end{enumerate}
\end{quote}
Although possible, a practical tool that uses the proof directly 
  is a challenge. 
Part of the problem is the complexity of the 
  algorithms used, some of which, e.g., the DPDA and jDPDA equivalance have never been 
  implemented.
Yet another issue that clients of compiler-compiler have grown to expect 
  facililities such as means for resolving ambiguities, manipulation 
  of attributes, etc.
Also, for a fluent API to be elegant and useful, 
  it should support method with parameters whose parameters are also defined by a  fluent API:
these two APIs may mutually recursive and even the same. 
Support of these features through four or so algorithmic abstractions 
  may turn out to be a decent engineering task.

\begin{wrapfigure}[6]r{27ex}
  \caption{\label{Figure:compiler} Encoding of an binary type tree}
  \javaInput[minipage,width=27ex,left=-2ex]{compiler.listing}
\end{wrapfigure}
Yet another challenge is controling the compiler's  
  runtime.
Learning that linear time parsers and lexical analyzers are possible, 
  and being accustomed to seeing these in practice, one 
  may expect the compiler would run in linear, or at least polynomial time. 
As it turns out, this time is exponential in the worst case (at least for \texttt{javac}).
An encoding of an S-expression in type~\cc{Cons} (\cref{Figure:compiler}) 
  is a not terribly complex such worst case.


\begin{wrapfigure}r{43ex}%
  \begin{minipage}{43ex}
  \caption{\label{Figure:compile-empiric} Compilation time
    (sec†{measured on an Intel i5-2520M CPU @ 2.50GHz~$⨉$4, 3.7GB memory, Ubuntu 15.04 64-bit, \texttt{javac} 1.8.0\_66}%
    ) \emph{vs.}
      length of call chain.
}
  \gnuplotloadfile[terminal=pdf,terminaloptions={crop size 2.5in,1.5in color enhanced font ",8" linewidth 1}]{../Figures/kill.gnuplot}
\end{minipage}
\end{wrapfigure}%
Type \cc{Cons} takes two type parameters, \cc{Car} and \cc{Cdr} (denoting left and right branches).
Denote the return type of \cc{d()} by \[
  τ= \cc{Cons< Cons<Car, Cdr>, Cons<Car, Cdr> >}.
\]
Let~$σ$ denote the type of the \kk{this} implicit parameter to~\cc{d}.
Now, since~$τ= \cc{Cons<}σ,σ\cc{>}$, we have~$|τ|≥2|σ|$,
  where the size of a type is measured, e.g., in number of characters in its textual representation.
Therefore, in a chain of~$n$ calls to \cc{d()}
\begin{equation}
  \label{Equation:n}
  \cc{(Cons<?,?>(null)).}\overbrace{\cc{d().}⋯\cc{.d()}}^{\text{$n$ times}}\cc{;}
\end{equation}
the size of the resulting type is~$O(2ⁿ)$.


\Cref{Figure:compile-empiric} shows, on the doubly logarithmic plane, the runtime (on a Lenovo X220)
of the \texttt{javac} compiler (version 1.8.0\_66) in face of a \Java program
  assembled from \cref{Figure:compiler} and \cref{Equation:n} placed as the
  single command of \cc{main()}.
Exponetial growth is demonstrated by the righthand side of the plot,
  in which curve converges on a straight line.
(In fact, a variation of the construction may lead to even super-exponential growth rate of the size of types.)

We believe that this exponential growth is due to a design flaw in the compiler.
Had the compiler used a representation of types that allows sharing of,
  of expression types, compilation time would be linear. 

Still, with current compiler technology, the type encoding scheme demonstated in \cref{Figure:A}
 might not be scaleable.




\section{Conclusion and Future Work}
\label{section:zz}
As should be obvious from \cref{Figure:fluent}, \SELF will be implemented
  in a bootstrapping fashion.
The specification of a BNF, is made using a fluent API.
The BNF for writing BNFs is given in \cref{Figure:BNF:BNF}

\begin{figure}[htbp]
  \scriptsize
  \begin{equation*}
    \def\<#1>{\/⟨\/\text{\textit{#1}}\/⟩\/~}
    \def\|{~|~}
    \let\oldCc=\cc
    \let\oldKk=\kk
    \def\cc#1{{\footnotesize\oldCc{#1}}~}
    \def\cc#1{{\footnotesize\olKk{#1}}~}
    \small
    \begin{aligned}
      \<BNF>              & ::=  \<Notation> \<Body> \<Footer> \hfill⏎
      \<Notation>         & ::=  \<Symbols> \<Terminals> \hfill⏎
      {}                  & \|  \<Terminals> \<Symbols> \hfill⏎
      \<Terminals>        & ::=  \cc{with(Symbols.class)}
      \<Symbols>          & ::=  \cc{with(Terminals.class)}
      \<Body>             & ::= \<Start> \<Rules> \hfill⏎
      \<Start>            & ::=  \cc{with(Class<? \kk{extends} Symbol)} 
      \<Rules>            & ::= \<First-Rule> \<More-Rules> \hfill⏎
      \<More-Rules>       & ::= \<Additional-Rule> \<More-Rules> \hfill⏎
      {}                  & \| ε \hfill⏎
      {}                  & \| \<Lowering-Visitor> \<Down-Visitors> \hfill⏎
      {}                  & \| ε \hfill⏎
      \<Up-Visitor>       & ::= \cc{male()} \cc{urinate()} \hfill⏎
      \<Down-Visitor>     & ::= \cc{female()} \<Action> \hfill⏎
                          & \| \cc{male()} \cc{defecate()} \hfill⏎
      \<Raising-Visitor>  & ::= \cc{male()} \cc{raise()} \cc{urinate()} \hfill⏎
      \<Lowering-Visitor> & ::= \cc{female()} \cc{lower()} \<Action> \hfill⏎
                          & \| \cc{male()} \cc{lower()} \cc{defecate()} \hfill⏎
      \<Activity>         & ::= \cc{urinate()} \hfill⏎
                          & \| \cc{defecate()} \hfill⏎
    \end{aligned}
  \end{equation*}
  \caption{A BNF grammar for the toilette seat problem}
  \label{Figure:BNF:BNF}
\end{figure}



\small
\bibliography{author-names,other-shorthands-abbreviated,%
 publishers-abbreviated,%
 conferences-abbreviated,%
 journals-abbreviated,journals-full,%
 yogi-book,yogi-practice,yogi-journal,yogi-theory,yogi-tr,yogi-misc,%
 GPCE,OOPSLA,PLDI,USENIX,ECOOP,%
 yogi-confs}	


\end{document}


Processing programming languages
\begin{description}
 \item[Lexical analysis] - the first step of the process in which the character strings generated by the
 programmer are aggregated to the abstract tokens defined by the language designer.
 \item[Syntactical analysis (parsing) ] - the second step, in which the processed strings of tokens
 conform to the rules of a formal grammar defined by the language's BNF (or EBNF).
 \item[Semantical analysis] - the next step, usually performed in unison with the previous step,
 in which the legal token sequences are given their semantic meaning.
\end{description}
Specifically, the proposal is that API design of follows the footsteps of
Accordingly, the designer of a fluent API has to follow these three conceptual
steps.
First is the identification of the \emph{vocabulary}, i.e.,
the set of method calls including type arguments that may take part in the
fluent API\@.
In this fluent API example
\begin{JAVA}
allowing (any(Object.class))
 ¢¢.method("get.*")
 ¢¢.withNoArguments();
\end{JAVA}
then, there are three method calls, and the vocabulary has three items in it.
\begin{itemize}
 \item~$ℓ₁ = \cc{any(Class<?>)}$
 \item~$ℓ₂ = \cc{allowing($ℓ₁$)}$
 \item~$ℓ₃ = \cc{method(String)}$
 \item~$ℓ₄ = \cc{withNoArguments()}$
\end{itemize}
