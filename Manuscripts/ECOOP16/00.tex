\title{%
Formal Language Recognition with the Java Type Checker
%   \newline
%   \color{red}{%
%     \rmfamily\scshape Thou Mortal, Be Warned. \newline
%     Thou Shallt Not Remove \newline
%     This Commandment \newline
%     While There Are Signs of Haste \newline
%     in This Document!!!!\newline
%   }
}

\documentclass[a4paper,USenglish]{lipics}
\usepackage{\jobname}

%\author{Tome Levy⏎
% Department of Computer Science⏎
% Technion---Israel Institute of Technology⏎
% \texttt{\small \href{mailto:stlevy@campus.technion.ac.il}{stlevy@campus.technion.ac.il}}}

\author{Anonymized for the submission}
\begin{document}

\maketitle
\hfill
  \parbox{40ex}{
    \begin{flushright}
    \scriptsize\itshape ``\protect \Java   generics are 100\protect\% pure syntactic sugar,
    and do not support meta-programming''%
    \footnote{
      stackoverflow
      \tiny
       \url{http://programmers.stackexchange.com/questions/95777/generic-programming-how-often-is-it-used-in-industry}
    }
    \end{flushright}
  }
\newline

\begin{abstract}
  This paper is a theoretical study of practical problem:
  the automatic generation of Java Fluent APIs from their specification.
We explain why the problem's core lies with 
  the expressive power of Java generics.
Our main result is that automatic generation is possible whenever 
  the specification is an instance of the set of deterministic context-free languages,
  a set which contains most ``practical'' languages.
Other contributions include a collection of techniques and idioms o
  the limited meta-programming possible with Java generics, 
  and an empirical measurement demonstrating that the runtime of
  the ``javac'' compiler of Java the may be exponential in
  the program's length, even for programs composed of 
  a handful of lines and which do not rely on overly 
  complex use of generics.

\end{abstract}

\section{Introduction}
Ever after their inception\urlref{http://martinfowler.com/bliki/FluentInterface.html} \emph{fluent APIs}
  increasingly gain popularity~\cite{Bauer:2005,Freeman:Pryce:06,Larsen:2012} and research
  interest~\cite{Deursen:2000,Kabanov:2008}.
In many ways, fluent APIs are a kind of
  \emph{internal} \emph{\textbf Domain \textbf Specific \textbf Language}:
They make it possible to enrich a host programming language without changing it.
Advantages are many: base language tools (compiler, debugger, IDE, etc.) remain
  applicable, programmers are saved the trouble of learning a new syntax, etc.
However, these advantages come at the cost of expressive power;
  in the words of Fowler:
  ‟\emph{Internal DSLs are limited by the syntax and structure of your base language.}”†
  {M. Fowler, \emph{Language Workbenches: The Killer-App for Domain Specific Languages?},
    2005
    \newline
  \url{http://www.martinfowler.com/articles/languageWorkbench.html#InternalDsl}}.
Indeed, in languages such as \CC, fluent APIs
  often make extensive use of operator overloading (examine, e.g., \textsf{Ara-Rat}~\cite{Gil:Lenz:07}),
  but this capability is not available in \Java.

Despite this limitation, fluent API in \Java can be rich and expressive, as demonstrated
  in \cref{Figure:DSL} showing use cases of the DSL of Apache Camel~\cite{Ibsen:Anstey:10}
(open-source integration framework),
and that of jOOQ\urlref{http://www.jooq.org}, a framework for writing
  SQL in \Java, much like Linq~\cite{Meijer:Beckman:Bierman:06}.

\begin{figure}[H]
  \caption{\label{Figure:DSL} Two examples of \Java fluent API}
  \begin{tabular}{@{}c@{}c@{}}
    \parbox[c]{44ex}{\javaInput[left=0ex]{camel-apache.java.fragment}} &
    \hspace{-3ex} \parbox[c]{59ex}{\javaInput[left=0ex]{jOOQ.java.fragment}}⏎
    \textbf{(a)} Apache Camel & \textbf{(b)} jOOQ
  \end{tabular}
\end{figure}

Other examples of fluent APIs in \Java are abundant:
  jMock~\cite{Freeman:Pryce:06},
  Hamcrest\urlref{http://hamcrest.org/JavaHamcrest/},
  EasyMock\urlref{http://easymock.org/},
  jOOR\urlref{https://github.com/jOOQ/jOOR},
  jRTF\urlref{https://github.com/ullenboom/jrtf}
  and many more.

\subsection{A Type Perspective on Fluent APIs}
\Cref{Figure:DSL}(B) suggests that jOOQ imitates SQL,
but is it possible at all to produce a fluent API
for the entire SQL language,
or XPath, HTML, regular expressions, BNFs, EBNFs, etc.?
Of course, with no operator overloading it is impossible
to fully emulate tokens; method names though make a good substitute for tokens, as done
in ‟\lstinline{.when(header(foo).isEqualTo("bar")).}” (\cref{Figure:DSL}).
The questions that motivate that this research are:
\begin{quote}
  \begin{itemize}
    \item Given a specification of a DSL, determine whether there exists
        a fluent API can be made for this specification?
    \item In the cases that such a fluent API is possible,
      can it be produced automatically?
    \item Is it feasible to produce a \emph{compiler-compiler} such as Bison~\cite{Bison} tool
        to convert a language specification into a fluent API?
\end{itemize}
\end{quote}

Inspired by the theory of formal languages and automata,
  this study explores what can be done, and what can not be done, with fluent API in \Java.

Consider some fluent API (or DSL) specification, permitting only certain call
chains and disallowing all others.
Now, think of the formal language that defines the set of these permissible chains.
The main contribution of this paper is a proof (with its implicit algorithm) that
there is always \Java type definition that \emph{realizes} this fluent definition, provided that this
language is \emph{deterministic context-free}, where
\begin{itemize}
  \item In saying that a type definition \emph{realizes} a specification of fluent
    API, we mean that call chains that conform with the API definition compile
    correctly, and, conversely, call chains that are forbidden by the API
    definition do not type-check, resulting in an appropriate compiler error.
  \item Roughly speaking, deterministic context free languages are those
    context free languages that can be recognized by an LR parser†{The ‟L"
    means reading the input left to right; the ‟R" stands for rightmost derivation}~\cite{Aho:86}.
    \par
    An important property of this family is that none of its members is ambiguous.
    Also, it is generally believed that most practical programming languages
    are deterministic context-free.
\end{itemize}

A problem related to that of recognizing a formal language,
is that of parsing, i.e., creating, for input which is within the language,
  a parse tree according to the language's grammar,
In the fluent APIs domain, the distinction between recognition and parsing is
  the distinction between compile time and runtime.
Before a program is run, the compiler checks whether the fluent API call is legal,
  and code completion tools will only suggest legal extensions of a current call chain.

In contrast, the parse tree is created at runtime.
Some fluent API definitions create the parse-tree
  iteratively, where each method invocations in the call chain adding
  more components to this tree.
However, it is always possible to generate this tree in ‟batch” mode:
This is done by maintaining a \emph{call list} which
  starts empty and grows at runtime by having each method invoked add to it
  a record storing the method's name and values of its parameters.
The list is completed at the end of the call list, at which point it is fed to an appropriate parser that
  converts it into a parse tree (or even an AST).

\subsection{Contribution}
The answers we provide for the three questions above is:
\begin{quote}
  \begin{enumerate}
  \item If the DSL specification is that of deterministic context-free
    language, then a fluent API exists for the language, but we do not know
    whether such a fluent API exists for more general languages.
  \par
  Recall that there are universal cubic time parsing
  algorithms~\cite{add:refs:to:this:cubics} which can parse (and recognize) any
  context free language. What we do not know is whether algorithms of this sort
  can be encoded within the framework of the \Java type system.
  \item
  There exists an algorithm to generate a fluent API that realize any
  deterministic context-free languages.  Moreover, this fluent API can create
  at runtime, a parse tree for the given language.  This parse tree can then be
  supplied as input to the library that implements the language's semantic.
  \item
  Unfortunately, a general purpose compiler-compiler
  is not yet feasible with the current algorithm.
  \begin{itemize}
    \item One difficulty is that the algorithm is complicated and relies on
      modules implementing some theoretical results, which, to the best of our
      knowledge have never been implemented.
    \item Another difficulty is that a certain design decision in the
      implementation of the standard \texttt{javac} compiler may choke on the
      Java code produced by the algorithm.
  \end{itemize}
  \end{enumerate}
\end{quote}

Other concrete contributions made by this work include
\begin{itemize}
  \item the understanding that the definition of fluent APIs is analogous to
      the definition of a formal language.
  \item a lower bound (deterministic pushdown automata).
    on the theoretical ‟computational complexity” of the \Java type system.
  \item an algorithm for producing a fluent API for deterministic context free languages.
  \item a collection of generic programming techniques, developed towards this algorithm.
  \item a demonstration that the runtime of Oracle's \texttt{javac} compiler may be exponential in the program size.
\end{itemize}


\textbf{Outline.}
\Cref{Section:fluent} is a brief reminder of fluent APIs 
  and DSLs, and is followed by a similar reminder of 
  context-free languages, pushdown automata, and the such in \cref{Section:pushdown}.
Based on the vocabulary established in these, 
  the main result is stated in~\cref{Section:result}.
\Cref{Section:related} offers a perspective on related work.

Towards the proof in \cref{Section:proof}, \cref{Section:toolkit} 
  shows idioms and techniques for encoding computation with    
  the \Java type-checker.
\Cref{Section:jump} makes use of these for encoding 
  ``jump-stack'', a non-trivial data-structure,
  which is used, with suitable modifications in the proof. 

In \cref{Section:applicability}, we discuss the challenges in
  translating the proof into a compiler-compiler for fluent APIs.
In particular, this section demonstrates our claim (that may be
  surprising to some) that the standard \Java compiler may spend
  an exponential time on compiling rather simple programs.
\cref{Section:zz} concludes.

  
\section{Background I/II: Method Chaining \emph{vs.} Fluent API}
\label{Section:fluent}
The pattern ‟invoke function on variable \cc{sb}”, specifically with
  a function named \cc{append}, occurs six times in the code in \cref{Figure:chaining}(a), designed
  to format a clock reading, given as integers hours, minutes and
  seconds.

\begin{figure}[H]
  \caption{\label{Figure:chaining}%
    Recurring invocations of the pattern ‟invoke function on the same
      receiver”, before, and after method chaining.
  }%
    \begin{tabular}{@{}cc@{}}%
  \begin{lcode}[minipage,width=44ex,box align=center]{Java}
String time(int hours, int minutes, int seconds) {¢¢
  final StringBuilder sb = new StringBuilder();
  sb.append(hours);
  sb.append(':');
  sb.append(minutes);
  sb.append(':');
  sb.append(seconds);
  return sb.toString();
}\end{lcode}
\hfill
&
\hspace{1ex}
  \begin{lcode}[minipage,width=44ex,box align=center]{Java}
String time(int hours, int minutes, int seconds) {¢¢
    return new StringBuilder()
      ¢¢.append(hours).append(':')
      ¢¢.append(minutes).append(':')
      ¢¢.append(seconds)
      ¢¢.toString();
}\end{lcode}
⏎
\textbf{(a)} before & \textbf{(b)} after
\end{tabular}
\end{figure}

Some languages, e.g., \Smalltalk offer syntactic sugar, called \emph{cascading},
  for abbreviating this pattern.
\emph{Method chaining} is a ‟programmer made” syntactic sugar serving the same purpose:
  If a method~$f$ returns its receiver, i.e., \kk{this},
  then, instead of the series of two commands: \mbox{\cc{o.$f$(); o.$g$();}}, clients can write
  only one: \mbox{\cc{o.$f$().$g$();}}.
  \cref{Figure:chaining}(b) is the method chaining
  (also, shorter and arguably clearer) version of
  \cref{Figure:chaining}(a).
It is made possible thanks to the designer of class \cc{StringBuilder} ensuring that 
  all overloaded variants of
  \cc{append} return their receiver.

The distinction between \emph{fluent API} and method chaining is the identity of the receiver:
In method chaining, all methods are invoked on the same object, whereas in fluent API
the receiver of each method in the chain may be arbitrary.
Fluent APIs are more interesting for this reason.
Consider, e.g., the following \Java code fragment (drawn from JMock~\cite{Freeman:Pryce:06})
\[
  \cc{allowing(any(Object.\kk{class})).method("get.*").withNoArguments();}
\]
Let the return type of function \cc{allowing} (respectively \cc{method}) be denoted by~$τ₁$
(respectively~$τ₂$).
Then, the fact that~$τ₁≠τ₂$ means that the set of methods that can be placed after the dot
in the partial call chain~$\cc{allowing(any(Object.\kk{class})).}$
is not necessarily the same set of methods that can be placed after the 
dot in the partial call chain~$\cc{allowing(any(Object.\kk{class})).method("get.*").}$.
This distinction makes it possible to design expressive and rich fluent APIs, in which a
sequence of ‟chained” calls is not only readable, but also robust, in the sense that the
sequence is type correct only when it makes sense semantically.

There is large body of research on \emph{type-states} 
(See e.g., review articles such
  as~\cite{Aldrich:Sunshine:2009,Bierhoff:Aldrich:2005}).
Informally, an object that belongs to a certain type, has
type-states, if not all methods defined in this object's class are applicable
to the object in all states it may be in.
As it turns out, objects with type states are quite frequent: a recent study~\cite{Beckman:2011} estimates
  that about 7.2% of \Java classes define protocols, that can be interpreted as type-state.

In a sense, type states define the ``language'' of the protocol of an object. 
The protocol of the type-state \cc{Box} class defined in \cref{Figure:box} 
  admits the chain \cc{\kk{new} Box().open().close()} but not the 
  the chain \cc{\kk{new} Box().open().open()}.

\begin{figure}[H]
  \caption{\label{Figure:box}Fluent API of a box object, defined by a DFA and a table}
  \begin{tabular}{cc}
    \hspace{7ex}\parbox[c]{40ex}{%
      \begin{tabular}[align=center]{m{7ex} | m{9ex} @{}| m{9ex}}
        & \cc{open()} & \cc{close()}⏎ \hline
        ‟closed”\ & \color{blue}{\emph{become ‟open”}} & \color{red}{\emph{runtime error}}⏎\hline
        ‟open” & \color{red}{\emph{runtime error}} & \color{blue}{\emph{become ‟closed”}}⏎
      \end{tabular}
    } &
    \hspace{-1ex}\parbox[c]{40ex}{\usetikzlibrary{automata,positioning,topaths}
\tikzstyle{state-style}=[state,every node={draw=black},font=\scriptsize,text width=5ex,align=center,on grid,node distance=13ex]
\begin{tikzpicture}

\node[state-style,accepting] (closed) {closed};
\node[state-style,accepting] (opened) [right=2.9 of closed] {opened};
\node[state-style,text width=4ex] (error) [above right=of closed,red] {runtime error};


\path[->,distance = 2ex,above] 

				(closed) edge[below] node {\cc{open()}} (opened)
				(opened) edge[bend left,below] node {\cc{close()}} (closed)
				(closed) edge[bend left,above left] node {\cc{close()}} (error)
				(opened) edge[bend right,right] node {\cc{open()}} (error);
\draw[<-] (closed) -- node[below left] {start} ++(-5ex,4ex);
\end{tikzpicture}}
    ⏎⏎
    \hspace{0ex}\textbf{(a)} Definition by table & \hspace{-2ex}\textbf{(b)} Definition by DFA
  \end{tabular}
\end{figure}

As mentioned above, tools such as fluflu realize
  type-state based on their finite automaton description.
Our approach is a bit more expressive: examine the language $L$ defined by the type-state, 
  e.g., in the box example,  
        \[
          L = \big(\cc{.open().close()}\big)^*\big(\cc{.open()}\:|\:ε\big).
        \]
If $L$ is deterministic context-free, a fluent API can be made for it. 

To make the proof concrete, consider this example of fluent API definition:
An instance of class \cc{Box} may receive two 
  method invocations: \cc{open()} and \cc{close()}, and can be in either 
  ‟open” or ‟closed” state.
Initially the instance is ‟closed”.
Its behavior henceforth is defined by \cref{Figure:box}.

To realize this definition, we need a type definition by which \cc{\kk{new} Box().open().close()}, more generally
  blue, or accepting states in the figure, type-check.
Conversely, with this type definition, compile time type error should occur in \cc{\kk{new} Box.close()},
  and, more generally, in the red state.

Some skill is required to make this type definition: proper design of class \cc{Box}, perhaps with
  some auxiliary classes extending it, an appropriate method definition here and there, etc.
As mentioned above, the process of generating the Flunet API is as follows:

\begin{itemize}
  \item First, consider the language defined by the fluent API\@.
        In the box example, this language is defined by the regular expression
  \item Second, check whether this language is deterministic context-free.
        If it is, the fluent API can be realized, and,
        there is an algorithm to produce the respective type definition.
        In the box example, since language~$L$ is specified by a regular expression,
        it is trivially deterministic context-free.
\end{itemize}

The proof is a construction of a \Java type encoding of
  the \emph{deterministic pushdown automaton} that recognizes
  a given \emph{deterministic context-free language}.
With the generated types and methods, the compilation process of
  any chain of fluent API calls, actually runs the pushdown automaton against the
  specific input string that the chain represents.
When used appropriately, if this run of the automaton ends with an 
  accepting state†{The acceptance of a PDA can also be defined 
  by an empty stack, we will use the accepting state type of PDAs},
  then the fluent API chain type checks correctly.
If however this run ends in a failure, i.e., non-accepting state,
  compile time error will occur.

The Box example discussed above, is a classic example for type-states,
  since a box object can be in one of two states: ``opened'' or ``closed''.

Type-state pose two main challenges to software engineering:
\begin{enumerate}
  \item \emph{\textbf{Identification.}}
    In the typical case, type-state
        receive little to no mention at all in the documentation.
    The identification problem is to find the implicit
    type state in existing \Java: Given an implementation of a class
    (or more generally of a software framework),
    \emph{determine} which sequences of method calls are valid and which violate the
    type state hidden in the \Java.
  \item \emph{\textbf{Maintenance and Enforcement.}}
    Having identified the type-states, the challenge is in automatically flagging out
      illegal sequence of calls that does not conform
      with the type-state, furthermore, with the
      evolution of an API, the challenge is in updating the type-state information,
      and the type checking of \Java of clients.
\end{enumerate}


\section{Background II/II: Context-Free Languages and Pushdown Automata: Reminder and Terminology}
\label{Section:pushdown}
Notions discussed here are probably common knowledge
 (see e.g.,~\cite{Hopcroft:book:2001,Linz:2001} for a text book description,
 or~\cite{Autebert:97} for a scientific review).
The purpose here is to set a unifying common vocabulary.

Let~$Σ$ be a finite alphabet of \emph{terminals} (often called input characters or tokens).
A \emph{language} over~$Σ$
  is a subset of~$Σ^*$.
Keep~$Σ$ implicit henceforth.

A \emph{\textbf Nondeterministic \textbf Pushdown \textbf Automaton} (NPDA) is a device for language recognition,
  made of a nondeterministic finite automaton
  and a stack of unbounded depth of (stack) \emph{elements}.
An NPDA begins execution with a single copy of the initial element on the stack.
In each step, the NPDA
  examines the next input token,
  the state of the automaton,
  and the top of the stack.
It then pops the top element from the stack, and nondeterministically chooses which actions of
  its transition function to perform:
  Consuming the next input token,
  moving to a new state,
  or, pushing any number of elements to the stack.
Nondeterminism effectively means
  that any combination of these actions may be selected.

The language recognized by an NPDA is the set of strings that it accepts,
  either by reaching an accepting state or by encountering an empty stack.

A \emph{\textbf Context-\textbf Free \textbf Grammar}(CFG) is a formal description of a language.
A CFG~$G$ has three components:~$Ξ$ a set of \emph{variables} (also called nonterminals),
  a unique \emph{start variable}~$ξ∈Ξ$, and a finite set of (production) \emph{rules}.
A rule~$r∈G$ describes the derivation of a variable~$ξ∈Ξ$ into
  a string of \emph{symbols}, where symbols are either terminals or variables.
Accordingly, rule~$r∈G$ is written as~$r=ξ→β$, where~$β∈\left(Σ∪Ξ\right)^*$.
This description is often called BNF\@.
The \emph{language} of a CFG is the set of strings of terminals (and terminals only)
  that can be derived from the start symbol, following any sequence of applications of the rules.
CFG languages make a proper superset of regular languages, and a proper subset of 
  ‟context-sensitive” languages~\cite{Hopcroft:79}.

The expressive power of NPDAs and BNFs is the same:
  For every language defined by a BNF, there exists an NPDA that recognizes it.
Conversely, there is a BNF definition for any language recognized by some NPDA.

NPDAs run in exponential deterministic time.
 A more sane, but weaker, alternative is found in LR($1$) parsers,
  which are deterministic linear time and space.
Such parsers employ a stack and a finite automaton structure,
  to parse the input.
 More general, LR($k$) parsers,~$k>1$, can be defined. These make their
 decisions based on the next~$k$ input character, rather than just the first of these.
 General LR($k$) parsers are rarely used, since they offer essentially
 the same expressive power†{they recognize the same set of languages~\cite{Knuth:65}},
 at a greater toll on resources (e.g., size of the automaton).
In fact, the expressive power of LR($k$),~$k\ge1$ parsers, is that
 of ‟\emph{\textbf Deterministic \textbf Pushdown \textbf Automaton}” (DPDA),
  which are similar to NPDA, except that their conduct is deterministic.

\begin{Definition}[Deterministic Pushdown Automaton]
  \label{Definition:DPDA}
  \slshape
  A \emph{deterministic pushdown automaton} (DPDA) is a quintuple~$⟨Q,Γ,q₀,A,δ⟩$
  where~$Q$ is a finite set of \emph{states},~$Γ$ is a finite
  \emph{set of elements},~$q₀∈Q$ is the initial state,
  and~$A⊆Q$ is the \emph{set of accepting states} while~$δ$ is
  the \emph{partial state transition function}~$δ:Q⨉Γ⨉(Σ∪❴ε❵)↛Q⨉Γ^*$.
  \par
  A DPDA begins its work in state~$q₀$ with a single designated stack element residing on the stack.
  At each step, the automaton examines: the current state~$q∈Q$, 
  the element~$γ∈Γ$ at the top of the stack, and~$σ$, the next input token, 
  Based on the values of these, it decides how to proceed:
  \begin{enumerate}
    \item If~$q∈A$ and the input is exhausted, the automaton accepts the input and stops.
    \item Suppose that~$δ(q,γ,ε)≠⊥$ (in this case, the definition of a DPDA
          requires that~$δ(q,γ,a')=⊥$ for all~$σ'∈Σ$), and let~$δ(q,γ,ε)=(q',ζ)$.
          Then the automaton pops~$γ$ and pushes the string of stack
          elements~$ζ∈Γ^*$ into the stack.
    \item If~$δ(q,γ,a)=(q',ζ)$, then the same happens, but the automaton also
          irrevocably consumes the token~$σ$.
    \item If~$δ(q,γ,ε)=δ(q,γ,a)=⊥$ the automaton rejects the input and stops.
  \end{enumerate}
\end{Definition}

A \emph{configuration} is the pair of the current state and the stack contents.
Configurations represent the complete information on the 
    state of an automaton at any given point during its computation.
A \emph{transition} of a DPDA takes it from one configuration to another.
Transitions which do not consume an input character are called~\emph{$ε$-transitions}.

As mentioned above, NPDA languages are the same as CFG languages.
Equivalently, \emph{DCFG languages} (deterministic context-free grammar languages)
  are context-free languages that are recognizable by a DPDA.
The set of DCFG languages is still a proper superset of regular languages,
  but a proper subset of CFG languages.


\section{Statement of the Main Result} 
\label{Section:result}
Let~$\textsf{java}$ be a function that translates a terminal~$σ∈Σ$
into a call to a uniquely named function (with respect to~$σ$).
Let~$\textsf{java}(α)$, be the function
  that translates a string~$α∈Σ^*$ into a fluent API call chain.
  If~$α=σ₁⋯σₙ∈Σ^*$, then \[
  \textsf{java}(α)=\cc{.}\textsf{java}(σ₁)\cc{().}⋯\cc{.}\textsf{java}(σₙ)\cc{()}.
\]
For example, when~$Σ=❴a,b,c❵$ let~$\textsf{java}(a)=\cc{a}$,~$\textsf{java}(b)=\cc{b}$, and,~$\textsf{java}(c)=\cc{c}$.
With these, \[
    \textsf{java}(caba) = \cc{.c().a().b().a()}
  \]

\begin{theorem}\label{Theorem:Gil-Levy}
  Let~$D$ be a DPDA recognizing a language~$L⊆Σ^*$.
  Then, there exists a \Java type definition,~$J_D$ for types~\cc{L},~\cc{D} and
    other types such that the \Java command
  \begin{equation}
    \label{Equation:result}
    \cc{L~$ℓ$ = M.build~$\textsf{java}(α)$}\cc{.\$();}
  \end{equation}
  type checks against~$J_M$ if an only if~$α∈L$.
  Furthermore, program~$J_M$ can be effectively generated from~$M$.
\end{theorem}

\Cref{Equation:result} reads: starting from the \kk{static} field \cc{build} of \kk{class}~\cc{M},
  apply the sequence of call chain~$\textsf{java}(α)$, terminate with a call to the
  ending character~\cc{\$()} and then assign to newly declared \Java variable~\cc{$ℓ$} of type~\cc{L}.

The proof of the theorem is by a scheme for encoding in \Java types
  the pushdown automaton~$A=A(L)$ that recognizes language~$L$.
Concretely, the scheme assigns a type~$τ(c)$.
  to each possible configuration~$c$ of~$A$.
Also, the type of \cc{M.build} is~$τ(c₀)$, where~$c₀$ initial configuration of~$A$,

Further, in each such type the scheme places
  a function~$σ()$ for every~$σ∈Σ$.
Suppose that~$A$ takes a transition from configuration~$cᵢ$ to configuration~$cⱼ$
  in response to an input chracter~$σₖ$.
Then, the return type of function \cc{$σₖ$()} in type~$τ(cᵢ)$ is type~$τ(cⱼ)$.

With this encoding the call chain in \cref{Equation:result}
  mimics the computation of~$A$, starting at~$c+0$ and ending with
  rejection or acceptance.
More details of the proof are in \cref{Section:proof}.

There are several, mostly minor, differences between the structure of the \Java code
in \cref{Equation:result}
and the examples of fluent API we saw above, e.g., in \cref{Figure:DSL}:
\begin{description}
  \item[Prefix, i.e., the stating \cc{M.build} variable.]
  All variables and functions of \Java are defined within a class.
  Therefore, a call chain must start with an object (\cc{M.build} in \cref{Equation:result})
  or, in case of \cc{static} methods, with the name of a class.
  In fluent API frameworks this prefix is typically eliminated
  with appropriate \cc{import} statements.
  \begin{quote}
  If so desired, the same can be done by our type encoding scheme: define all
  methods in type~$τ(c₀)$ as \cc{staic} and \cc{import static} of of these.
\end{quote}
  \item[Suffix, i.e., the terminal \cc{.\$()} call.]
  In order to know whether~$α∈L$ the automaton recognizing~$L$ must
  know when~$α$ is terminated.
  \begin{quote}
  With a bit of engineering, this suffix can also be eliminated.
  One way of doing so is by defining type~\cc{L} as an \kk{interface}, and by making all types~$τ(c)$,~$c$ is
  an accepting configuration, as subypte of~\cc{L}.
  \end{quote}
  \item[Parametrized methods.]
  Fluent APIs frameworks support call chains with including phrases such as
  \begin{itemize}
    \lstset{language=Java,style=code}, 
    \item \lstinline{.when(header(foo).isEqualTo("bar")).}, and, 
    \item \lstinline{.and(BOOK.PUBLISHED.gt(date("2008-01-01"))).}.
    \item \lstinline{.allowing(any(Object.class)).}
  \end{itemize}
  while our encoding scheme assumes methods which do not take any parameters.  
  \begin{quote}
    Even though  methods with parameters contribute to the user
      experience and readability of fluent APIS, extending 
      \cref{Theorem:Gil-Levy} to support requires three simple steps: 
      \begin{itemize}
        \item Define the structure of parameters to methods with appropriate fluent API, which may or
          may not be the same of the API of the outer chain, or the API of parameters to
          other methods. Apply the threorem to each of these fluent APIs.
        \item 

      \end{itemize}
  \end{quote}
\end{description}


\section{Related Work}
\label{Section:related}
Modern programming languages acquire high-level constructs
  at a staggering rate.
The imminent adoption of closures in \Java and \CC,
  the generators of \CSharp, and ‟concepts” in
  \CC are just a few examples.

A theoretical motivation for this work
  is the exploration of the computational
  expressiveness of such features.
For example, it is known (see e.g.,~\cite{Gutterman:2003}) that
  \kk{template}s in \CC are Turing complete in the following precise sense:

\begin{Theorem}
  \label{Theorem:Gutterman}
  For every Turing machine,~$m$, there exists a \CC program,~$Cₘ$ such that
    compilation of~$Cₘ$ of terminates if and only if
      Turing-machine~$m$ halts.
  Furthermore, program~$Cₘ$ can be effectively generated from~$m$.
\end{Theorem}

Intuitively, the proof relies on the fact that \kk{template}s
  feature recursive invocation and conditionals (in the form of
  ‟\emph{template specialization}”).

There has already been a similar \Java implementation for regular
  languages\urlref{https://github.com/verhas/fluflu}.
  
In the same fashion, it is mundane to make the judgment that
  \Java's generics are not Turing-complete: all recursive calls
  in these are unconditional.
In a sense, this article shall give a lower bound on the
  expressive power of \Java generics in terms of the Chomsky hierarchy~\cite{Chomsky:1963}.
This objective is more precisely expressed in the following conjecture.

Boost is a cool c++ templating library!\cite{Abrahams:Gurtovoy:04} I think. 
This how you can calculate the derivative of a function with \CC compiler! \cite{Gil:Gutterman:98}

Mention funny tricks with annotations to Java. There is \cite{Papi:08} from 
  Washington State university. He fought for more support for annotations 
  and built a system for implementing non
Expression templates is a \CC technique for passing expressions as function arguments. \cite{Veldhuizen:95}

Mention work by \cite{Bracha} on non-standard type systems.  

Eric Bodden wrote an article about fluent APIs, static and dynamic analysis, and type-state~\cite{Bodden:14}

\subsection{Type State}
There is large body of research on \emph{type-states} (See e.g., review articles such
  as~\cite{Aldrich:Sunshine:2009,Bierhoff:Aldrich:2005}).
Informally, an object that belongs to a certain type, has
type-states, if not all methods defined in this object's class are applicable
to the object in all states it may be in.

A classical example of type-states is a file object which can be in one of two
states: ‟open” or ‟closed”. Invoking a \cc{read()} method on the object is only
permitted when the file is in an ‟open” state. In addition, method \cc{open()}
(respectively \cc{close()}) can only be applied if the object is in the
‟closed” (respectively, ‟open”) state.

Objects with type states such as files are not rarities.
In fact, a recent study~\cite{Beckman:2011} estimates
  that about 7.2% of \Java classes define protocols, that can be interpreted as type-state.
Type-state pose two main challenges to software engineering
\begin{enumerate}
  \item \emph{\textbf{Identification.}}
    In the typical case, type-state
        receive little to no mention at all in the documentation.
    The identification problem is to find the implicit
    type state in existing \Java: Given an implementation of a class
    (or more generally of a software framework),
    \emph{determine} which sequences of method calls are valid and which violate the
    type state hidden in the \Java.
  \item \emph{\textbf{Maintenance and Enforcement.}}
    Having identified the type-states, the challenge is in automatically flagging out
      illegal sequence of calls that does not conform
      with the type-state, furthermore, with the
      evolution of an API, the challenge is in updating the type-state information,
      and the type checking of \Java of clients.
\end{enumerate}

\begin{wrapfigure}[9]r{35.05ex}
 \begin{tabular}[align=center]{m{7ex} | m{9ex} @{}| m{9ex}}
 & \cc{open()} & \cc{close()}⏎ \hline
 ‟closed”\ & \color{blue}{\emph{become ‟open”}} & \color{red}{\emph{runtime error}}⏎\hline
 ‟open” & \color{red}{\emph{runtime error}} & \color{blue}{\emph{become ‟closed”}}⏎
 \end{tabular}
\end{wrapfigure}

\begin{wrapfigure}[9]r{35.05ex}
\caption{\label{Figure:box}Fluent API of a box object, defined by a DFA}
  \input ../Figures/open-close-example.tikz
\end{wrapfigure}

To make the proof concrete, consider this example of fluent API definition:
An instance of class \cc{Box}
may receive two method invocations: \cc{open()} and \cc{close()},
and can be in either ‟open” or ‟closed” state,
Initially the instance is ‟closed”.
Its behavior henceforth is defined by \cref{Figure:box}.

To realize this definition, we need a type definition by which \cc{\kk{new} Box().open().close()}, more generally
  blue, or accepting states in the figure, type-check.
Conversely, with this type definition, compile time type error should occur in \cc{\kk{new} Box.close()},
  and, more generally, in the red state.

Some skill is required to make this type definition: proper design of class \cc{Box}, perhaps with
  some auxiliary classes extending it, an appropriate method definition here and there, etc.

The proof makes a general recipe for handling examples of this sort:
\begin{itemize}
  \item First, consider the language defined by the fluent API\@.
        In the box example, this language is defined by the regular expression
        \[
          L = \big(\cc{.open().close()}\big)^*\big(\cc{.open()}\:|\:ε\big).
        \]
  \item Second, check whether this language is deterministic context-free.
        If it is, the fluent API can be realized, and,
        there is an algorithm to produce the respective type definition.
        In the box example, since language~$L$ is specified by a regular expression,
        it is trivially deterministic context-free.
        \par
        It follows from the proof that there exists a type definition
        which realizes the box example.
        Moreover, there is a way
        to automatically produce this type definition.
\end{itemize}

The proof is a construction of a \Java type encoding of
  the \emph{deterministic pushdown automaton} that recognizes
  a given \emph{deterministic context free language}.
With the generated types and methods, the compilation process of
  any chain of fluent API calls, actually runs the pushdown automaton against the
  specific input string that the chain represents.
When used appropriately, if this run of the automaton ends with an accepting state†{The acceptance of a PDA can also be defined by an empty stack, we will use the accepting state type of PDAs},
  then the fluent API chain type checks correctly.
If however this run ends with a failure, i.e., non-accepting state,
  compile time error will occur.


\section{Techniques of Type Encoding}
\label{Section:toolkit}
This section presents techniques of type encoding in \Java.
Some readers may prefer to skip through to the next section,
where these are employed in the proof of \Cref{Theorem:Gil-Levy}.

Let~$g:Γ↛Γ$ be a partial function,
  from the finite set~$Γ$ into itself.
We argue that~$g$ can
  be represented using the compile-time mechanism of \Java.
  \cref{Figure:unary-function} encodes such a partial function for~$Γ=❴γ₁,γ₂❵$, where~$g(γ₁)=γ₂$
  and~$g(γ₂)=⊥$, i.e.,~$g(γ₂)$ is undefined.

\begin{figure}[hbt]
  \caption{\label{Figure:unary-function}%
    Type encoding of the partial function~$g:Γ↛Γ$,
    defined by~$Γ=❴γ₁,γ₂❵$,~$g(γ₁)=γ₂$ and~$g(γ₂)=⊥$.
  }
  \begin{tabular}{@{}c@{}c@{}c@{}}
    \hspace{-7ex}
    \parbox[c]{0.26\linewidth}{%
      \input ../Figures/unary-function-classification.tikz
    }%
    &
    \hspace{-1ex}
    \parbox[c]{0.64\linewidth}{%
      \javaInput[left=0ex]{gamma.listing}
    }%
    &
    \hspace{-18ex}
    \parbox[c]{0.84\linewidth}{%
      \javaInput[left=0ex,toprule=3pt,leftrule=3pt,bottomrule=3pt]{gamma-example.listing}
    }%
⏎
\textbf{(a)} type hierarchy & \textbf{(b)} implementation & \hspace{-62ex} \textbf{(c)} use cases
  \end{tabular}
\end{figure}

The type hierarchy depicted in~\cref{Figure:unary-function}(a) shows five classes:
Abstract class~\cc{Γ} represents the set~$Γ$, final classes~\cc{γ1},~\cc{γ2}
  that extend~\cc{$Γ$}, represent the actual members of the set~$Γ$.
The remaining two classes are private final class~\cc{¤} that stands for an error value,
  and abstract class~\cc{$Γ'$} that denotes the augmented set~$Γ∪❴\text{¤}❵$.
Accordingly, both classes~\cc{¤} and~\cc{$Γ$} extend~\cc{$Γ'$}.†{The use
  short names, e.g.,~\cc{$Γ$} instead of \cc{$Γ'.Γ$},
    is made possible by to an appropriate \kk{import} statement.
    For brevity, all \kk{import} statements are omitted.}

The full implementation of these classes is provided in~\cref{Figure:unary-function}(b)†{Remember that \Java admits Unicode characters in identifier names}.
This actual code excerpt should be placed as a nested class of some appropriate host class. Import statements are omitted, here and henceforth for brevity.

The use cases in~\cref{Figure:unary-function}(c) explain better
  what we mean in saying that function~$g$ is encoded in the type system:
  An instance of class~\cc{$γ$1} returns a value of type~\cc{$γ$2} upon
  method call~\cc{g()}, while
  an instance of class~\cc{$γ$2} returns a value of our~\kk{private}
  error type~\cc{$Γ'$.¤} upon the same call.

Three recurring idioms employed in~\cref{Figure:unary-function}(b) are:
\begin{enumerate}
  \item An~\kk{abstract} class encodes a set.
    Abstract classes that extend it encode
      subsets, while~\kk{final} classes encode set members.
  \item The interest of frugal management of name-spaces is served
    by the agreement that if a class~\cc{$X$}~\kk{extends} another class~\cc{$Y$}, then~\cc{$X$} is also defined
    as a~\kk{static} member class of~$Y$.
  \item Body of functions is limited to a single~\kk{return}~\kk{null}\cc{;} command.
      This is to stress that at runtime, the code does not carry out any useful or interesting computation,
      and the class structure is solely for providing compile-time type checks.
†{%
A consequence of these idioms is that the augmented class~\cc{$Γ'$} is visible to clients.
It can be made~\cc{private}. Just move class~\cc{$Γ$} to outside of~\cc{$Γ'$}, defying the second idiom.
}
\end{enumerate}

Having seen how how inheritance and overriding make possible
  the encoding of unary functions, we turn now to encoding higher arity functions.
With the absence of multi-methods, other techniques must be used.

Consider the partial binary function~$f: R⨉S↛Γ$, defined by
\begin{equation}
  \label{Equation:simple-binary}
  \begin{array}{ccc}
    R=❴r₁,r₂❵ & f(r₁,s₁)=γ₁ & f(r₂,s₁)=γ₁⏎
    S=❴s₁,s₂❵ & f(r₁,s₂)=γ₂ & f(r₂, s₂)=⊥
  \end{array}.
\end{equation}
A \Java type encoding of this definition of function~$f$
  is in~\cref{Figure:simple-binary}(a); use cases
    are in~\cref{Figure:simple-binary}(b).

\begin{figure}[hbt]
  \caption{\label{Figure:simple-binary}%
    Type encoding of partial binary function~$f:R⨉S↛Γ$,
    where~$R=❴r₁,r₂❵$,~$S=❴s₁,s₂❵$, and~$f$
  is specified by~$f(r₁,s₁)=γ₁$,~$f(r₁,s₂)=γ₂$,~$f(r₂,s₁)=γ₁$, and~$f(r₂, s₂)=⊥$.}
    \begin{tabular}{cc}
    \hspace{-3.5ex}
      \parbox[c]{0.57\linewidth}{%
        \javaInput[left=0ex]{binary-function.listing}
      }
          &
      \hspace{-16ex}
      \parbox[c]{51ex}{\javaInput[minipage,leftrule=3pt,toprule=3pt,bottomrule=3pt,width=51ex]{binary-function-example.listing}}
⏎
\parbox{0.57\linewidth}
{\textbf{(a)} implementation (for classes~\cc{$Γ$},~\cc{$Γ'$},~\cc{$γ$1}, and~\cc{$γ$2},
which is in \cref{Figure:unary-function}).}
& \hspace{-5ex}\textbf{(b)} use cases⏎
    \end{tabular}
  \end{figure}

As the figure shows, to compute~$f(r₁,s₁)$ at compile time we write~\cc{f.r1().s1()}.
Also, the fluent API call chain~\cc{f.r2().s2().g()} results in a compile time error since~$f(r₂, s₂)=⊥$.

Class~\cc{f} in the implementation sub-figure serves as
  the starting point of the little fluent API defined here.
The return type of~\kk{static} member functions~\cc{r1()} and~\cc{r2()}
  is the respective sub-class of class~\cc{R}:
The return type of function~\cc{r1()} is class~\cc{R.r1};
  the return type of function~\cc{r2()} is class~\cc{R.r2}.

Instead of representing set~$S$ as a class,
  its members are realized as methods~\cc{s1()} and~\cc{s2()} in class~\cc{R}.
These functions are defined as~\kk{abstract} with return type~\cc{$Γ$'}
  in~\cc{R}.
Both functions are overridden in classes~\cc{r1} and~\cc{r2},
   with the appropriate co-variant change of their return type,

It should be clear now that the encoding scheme presented
  in \Cref{Figure:simple-binary} can be generalized to functions
  with any number of arguments, provided that the domain and range sets are finite.
The encoding of sets of unbounded size require means for creating an unbounded
 number of types.
Genericity can be employed to serve this end.

\begin{wrapfigure}[8]{r}{29ex}
  \caption{\label{Figure:id}%
  Covrariant return type of function \cc{id()}
  with \Java generics.
  }
  \javaInput[left=-2ex,minipage,width=29ex]{id.listing}
\end{wrapfigure}

Consider first \Cref{Figure:id} showing a a genericity based recipe for
  a function whose return type
  is the same as the receiver type.
  This recipe is applied in the figure to classes~\cc{A},~\cc{B}, and~\cc{C}.
  In each of these classes, the return type of \cc{id} is,
  without overriding, is (at least) the class itself.

\Cref{Figure:stack-use-cases} shows some use cases of a type encoding of
  a stack of unbounded depth, yet can only store members of the set~$Γ$.
With type encoding these are precisely classes~\cc{$γ$1}
  and \cc{$γ$2} defined in \cref{Figure:unary-function}.

\begin{figure}[!htp]
  \caption{\label{Figure:stack-use-cases}%
    Use cases of a compile-time stack data structure.
  }
  \javaInput[minipage]{stack-use-cases.listing}
\end{figure}

The figure demonstrates a stack that starts with five items in it.
These are popped in order. Just before popping the last item, its
  value is examined.
Trying then to pop from an empty stack, or to examine its top, ends with
  a compile time error.

The expression⏎
  \mbox{\qquad\qquad} \cc{Stack.empty.$γ$1().$γ$1().$γ$2().$γ$1().$γ$1()}⏎
represents the sequence of pushing the value~$γ₁$ into an
empty stack, followed by~$γ₁$,~$γ₂$,~$γ₁$, and, finally,~$γ₁$.
This expression's type is that of~\cc{\_1}:⏎
\mbox{\qquad\qquad} \cc{P<$γ$1,P<$γ$1,P<$γ$2,P<$γ$1,P<$γ$1,E>>>>>}.⏎
A recurring building block occurs in this type.
This is generic type~\cc{P}, \emph{short for ‟Push”}, which takes two parameters:
  \begin{enumerate}
    \item the \emph{top} of the stack, always a subtype of~\cc{$Γ$},
    \item the \emph{rest} of the stack, which can be of two kinds:
          \begin{enumerate}
            \item another instantiation of~\cc{P} (in most cases),
            \item non-generic type~\cc{E}, \emph{short for ‟Empty”}, which encodes the empty
                  set (only at the deepest~\cc{P}, rest is empty).
          \end{enumerate}
  \end{enumerate}
Incidentally, \kk{static} field \cc{Stack.bottom} is of type~\cc{E}.

\Cref{Figure:stack-encoding}(a) gives the type inheritance hierarchy of the \Java
implementation†{%
  unless otherwise stated, in saying implementation we mean actual
  code extract
}
is in~\cref{Figure:stack-encoding}(b).

\begin{figure}[H]
  \caption{Type encoding of an unbounded stack data structure.}
  \label{Figure:stack-encoding}
  \begin{tabular}{cc}
    \parbox[c]{0.3\linewidth}{%
      \input ../Figures/stack-classification.tikz
    } &
    \hspace{-3ex} \parbox[c]{63ex}{\javaInput[minipage]{stack.listing}}⏎
    \textbf{(a)} type hierarchy &
    \hspace{-3ex} \parbox[t]{63ex}{%
    \textbf{(b)} implementation (except
    for classes~\cc{$Γ$},~\cc{$Γ'$},~\cc{$γ$1}, and~\cc{$γ$2}, which is in \cref{Figure:unary-function}).}
  \end{tabular}
\end{figure}

The code in the figure shows that the ‟rest” parameter of~\cc{P} must extend class \cc{Stack},
  and that both types~\cc{P} and~\cc{E} extend \cc{Stack}.
Other points to notice are:
\begin{itemize}
  \item The type at the top of the stack is precisely the return type of \cc{top()};
        it is overridden in~\cc{P} so that its return type is the first argument of~\cc{P}.
        The return type of \cc{top()} in~\cc{B} is the error value {$Γ'$.¤}.
  \item Pushing into the stack is encoded as functions~\cc{$γ$1()} and~\cc{$γ$2()};
        the two are overridden with appropriate covariant change of the return type in~\cc{P} and~\cc{E}.

  \item Since empty stack cannot be popped, function \cc{pop()} is overridden in~\cc{E} to return
    the \emph{error} type \cc{Stack.¤}. This type is indeed a kind of a stack, except that each of the four stack
        functions: \cc{top()}, \cc{push()},~\cc{$γ$1()}, and,~\cc{$γ$2()}, return an appropriate error type.
\end{itemize}

\begin{wrapfigure}[5]r{38ex}
  \caption{\label{Figure:generic} Covariance with generics}
  \javaInput[minipage]{mammal.listing}
\end{wrapfigure}
As a matter of curiosity: covariance recurs here.
Overriding permits covariant change of the return type of a function,
As can be seen in \cref{Figure:generic}, similar covariant change might happen in extending a generic type:
The type of the parameter (\kk{extends} \cc{Whales}) to a generic (\cc{Heap}) may
be specialized with the inheritance (class \cc{School} extending \cc{Heap}).

This recursive generic type technique can used to encode S-expressions: In the spirit of
  \cref{Figure:stack-encoding}, the idea is to make use of a \cc{Cons} generic type
  with co-variant \cc{car()} and \cc{cdr()} methods.

\begin{figure}[htb]%
  \caption{Peeping into the stack}%
  \label{Figure:peep}%
  \javaInput[minipage,listing style=numbering]{peep.listing}
\end{figure}


\section{The Jump-Stack Data-Structure}
\label{Section:jump}
A \emph{jump-stack} is a stack data structure whose elements are drawn from a finite set~$Γ$,
  except that jump-stack supports~$\textsf{jump}(γ)$,~$γ∈Γ$ operations,
    which means
  ‟repetetively pop elements from the stack up to and including the first occurrence of~$γ$”.
Let $k=|\Gamma|$. 

\begin{wrapfigure}[16]{r}{42ex}
  \caption{Skeleton of type encoding for the jump-stack data structure}%
  \label{Figure:jump}%
  \lstset{style=numbered}
  \javaInput[minipage,left=-2ex]{jump-stack.listing}
\end{wrapfigure}

\Cref{Figure:jump} shows the skeleton of type-encoding of a jump-stack whose
elements are drawn from type~\cc{$Γ$}
(\cref{Figure:unary-function}), i.e., either~\cc{$γ$1} or~\cc{$γ$2}.

Just like \cc{Stack} (\cref{Figure:stack-encoding}(b)),
  the generic type \cc{JS} which encodes jump-stacks, takes
  a \cc{Rest} parameter which is the type of a jump-stack after popping.
In addition \cc{JS} takes $k$ type parameters, one for~$γ∈Γ$,
  which is the type encoding of the jump-stack after a~$\textsf{jump}(γ)$
  operation.
In the figure, there are two such parameters: \cc{J\_$γ$1}, and
  \cc{J\_$γ$2}.

Functions defined in \cc{JS} include not only the standard stack opertions: \cc{top},
\cc{pop()}, \cc{$γ1$()} and~\cc{$γ2$()} (encoding a push of~$γᵢ$,~$i=1,2$),
  but also functions \cc{jump\_$γ$1} and \cc{jump\_$γ$2},
  which encode~$\textsf{jump}(γᵢ)$
  thanks to the return type being~\cc{J\_$γ$i},~$i=1,2$.

The type hierarchy rooted at \cc{JS} is similar to that of
\cref{Figure:stack-encoding}(a):
  Two of the specializations are parameterless and are
  almost identical to their \cc{Stack}
  counterparts:
\cc{JS.E} encodes an empty jump-stack; \cc{JS.¤} encodes a jump-stack in error,
e.g., a after popping from \cc{JS.E}.


Type \cc{JS.P} (line 15 in the figure) makes  the third specialization of \cc{JS}, representing 
  a stack with one or more elements.
There are no overriden functions in \cc{JS.P}; it achieves
  it purpose by the parameters it takes and those it passes
  to the type it extends.

\begin{wrapfigure}[10]r{43ex}
  \caption{\label{Figure:jump-stack-push} Type \cc{JS.P} encoding a non-empty jump-stack}
  \javaInput[minipage,width=43ex,left=-2ex]{jump-stack-push.listing}
\end{wrapfigure}

Specifically, \cc{JS.P} takes 
the same \cc{Top} and \cc{Rest} paramters (ll.16--17) as type \cc{Stack.P}:
  as well as $k$ additional paramters:
  \cc{J\_$γ$1} and \cc{J\_$γ$2} (ll.18--18)
which are the types encoding the jump-stack
  after the executation~$\textsf{jump}(γᵢ)$,~$i=1,2$.
Type \cc{JP.P'} passes these four parameters 
to type \cc{Pʹ} which it extends (l.21)
The fifth parameter to \cc{Pʹ} (l.22) is the current incarnation of \cc{P}, i.e., 
  \cc{P<Top, Rest, J\_γ1, J\_γ2>}.

The auxliary type \cc{JS.Pʹ} itself is depicted in \cref{Figure:jump-stack}.
Extending type \cc{JS} and passing the correct \cc{Rest} parameter to it, 
\cc{JS.Pʹ} inherits a correct declaration of function \cc{pop()} (l.6~\cref{Figure:jump}) 



\section{Proof of \Cref{Theorem:Gil-Levy}}
\label{Section:proof}
We now turn to the proof of \cref{Theorem:Gil-Levy}, which will be showing a type encoding for the DPDA that recognizes a certain DCFG language.
The proof is by reduction to type encoding of SDPDAs, which are simpler version of deterministic pushdown automata.

\subsection{Reduction}
The difficulty in type-encoding of general DPDAs is that of~$ε$-moves:
A DPDA is allowed to make a transition into a different state,
  pop one or more stack elements, and then, pop, push or even both pop and push, any number of stack elements,
  all without consuming a single input symbol.

In contrast, our type encoding relies on the encoding of an input string~$α∈Σ^*$, one symbol at a time, as a sequence of method calls.
A step of computation occurs in the domain of \Java types only at method calls. There are no-provisions for encoding~$ε$-moves, however deterministic
  they are.
The following definition is of a deterministic pushdown automata without such moves.

\begin{Definition}
  \label{Definition:SDPDA}
  \slshape
  A \emph{\textbf Simplified \textbf Deterministic \textbf Pushdown \textbf Automaton} (SDPDA) of order~$k$ is
    a quintuple~$⟨Γ,Q,q₀,A,δ⟩$,
  where~$Γ$,~$Q$,~$q₀∈Q$,~$A⊆Q$ are precisely as in \cref{Definition:DPDA}.
  The signature of function~$δ$, the \emph{generalized partial transition function}
  of an SDPDA of order~$k$, is however~$δ: Q⨉\left(Γ∪❴\vdash❵\right)ᵏ⨉Σ↛Q⨉Γ^*$.
  \par
  An SDPDA begins as a DPDA\@. At each step it examines~$σ∈Σ$,
    the next input symbol,~$q∈Q$, the current state,
    and~$γ₁,⋯,γₖ∈Γ$, the~$k$ top most stack elements,
  \par
  If the stack is empty, or~$q∈A$, the automaton stops in accepting.
  If the transition function is undefined, i.e.,~$δ(q,γ₁,…,γₖ,σ)=⊥$ the automaton
    stops in rejection.
  Otherwise, and since there are no~$ε$-moves, there are~$q'∈Q$ and~$ζ∈Γ^*$
    such that~$δ(q,γ₁,…,γₖ,σ)=(q',ζ)$.
  The automaton then pops~$γ₁$ through~$γₖ$, pushes the sequence~$ζ$, and
    moves to state~$q'$.
\end{Definition}

We say that
a \emph{configuration}~$(q,γ)$,~$q∈Q$,~$γ∈Γ$,
  is an \emph{$ε$-configuration} if~$δ(q,γ,ε)=(q',ζ)$ for
    some~$q'∈Q$ and a sequence of stack elements~$ζ$,
Recall that by definition of DPDAs,~$δ(q,γ,σ)=⊥$ for
  all~$σ∈Σ$, i.e., there is precisely one transition
  leading out of an~$ε$-configuration.

Our objective is to remove all~$ε$-configurations,
  replacing these with the non-$ε$ configuration that
  they (eventually) lead to.
This is easy to do when~$|ζ|≥1$, in which
  case the top of the stack is known to be~$γ'$,~$γ'$ being the first
    element of~$ζ$; the configuration at the end of transition
    would be then be~$(q',γ')$, which in its turn may, or may not be,
    an~$ε$-configuration.

However, if~$ζ=ε$, we do not know which element resides at the top
  of the stack.
The transition from the current configuration can thus lead to \emph{multiple} configurations,
  depending on the actual element left at the top of the stack after popping~$γ$.
The definition of a configuration thus must be changed to include the \emph{two} top
  most stack elements.
But, even this would fail if the next configuration is also an~$ε$-configuration that pushes no elements.

To be able to compute the closure of~$ε$-configuration, we shall
  therefore include with the configuration the~$k$ top most stack elements,
  where~$k$ is defined to be large enough so that
  the top of the stack must never be guessed while going
  through~$ε$-configurations.

To compute~$k$ we need to compute the largest possible overdraft
  from the stack along a series of transitions through~$ε$-configuration.
To do so, define the following directed weighted graph:
Nodes in this graph are the~$|Q|·|Γ|$ distinct configurations
  of the form~$(q,γ)$,~$q∈Q$,~$γ∈Γ$.
Let a node~$(q,γ)$ be an~$ε$-configuration for which~$δ(q,γ,ε)=(q',ζ)$.
Then, there is an edge from~$(q,γ)$ to all~$|Γ|$ nodes of the form~$(q',γ')$,~$γ'∈Γ$, i.e.,
  to all possible ‟guesses” of the stack element beneath~$γ$.
Each such edge takes the weight~$1-|ζ|$ to denote the fact
  that the total ‟charge” to the stack is~$1-|ζ|$:
    one stack element (specifically~$γ$) is popped,
    while~$|ζ|$ (which could also be zero), are pushed.
The value of~$k$ is simply the heaviest path in this graph.

In the former graph, cycles might occur.
Sadly, we cannot solve those cycles, since arbitrary number of symbols might be removed from the
  stack, and therefore, the required~$ε$-closure cannot be computed.
As shown by someone~\cite{i:need:to::find:it}, DPDAs with no~$ε$-moves are less expressive than a normal DPDAs,
  and therefore, not all DPDA can be computed.

Now, we shall describe the transition function of an SDPDA in the terms of a DPDA (assuming the former graph is DAG).
Let~$M=⟨Q,Γ,q₀,A,δ⟩$ be a DPDA, our goal is to describe~$Sₘ=⟨Qₘ,Γ,q_{0m},Aₘ,δₘ⟩$
  of order k, the corresponding SDPDA\@.
Notice that the stack elements are the same in both.
The other components of~$Sₘ$ are constructed as follows:
\begin{itemize}
 \item~$Qₘ$ is the set~$Q⨉\left(Γ∪❴\vdash❵\right)ᵏ$
 \item~$q_{0m}$ is the state~$(q₀,\vdash^{k-1}γ)$ where~$γ$ is the stack start element of~$Γ$.
 \item~$Aₘ$ is the set~$❴(q,ζ)∈Qₘ | q∈A❵$
 \item for~$δ(q,γ,σ) = (q',ζ)$
 \begin{itemize}
  \item if~$σ≠ε$ than for all~$(q,ζ'γ')$ such that~$ζ'γ∈Γᵏ$
    we define~$δₘ(q,ζ'γ,σ)=(q',ζ'ζ)$
  \item if~$σ=ε$ then~$δₘ(q,ζ,ε)=(q',ζ')$ when~$(q',ζ')$
    is the~$ε$-closure on~$(q,ζ)$.
    The~$ε$-closure of~$(q,ζ)$ is the series of consecutive~$ε$-transitions from~$δ$
    that end with a single ‟input consuming” transition.
    This computation is possible due to the definition of~$k$, that assures us, that during this ‟static”
    computation on~$δ$, we will at all times know the top of the stack, and therefore, this computation
    is viable.
 \end{itemize}

\end{itemize}

\subsection{Type Encoding}
\begin{Theorem}
  \label{Theorem:SDPDA}
  For every SDPDA~$a$ there exists a set~$J_a$ of \Java type definitions, such that
  the command \[
    \cc{A_= M.build~$\textsf{java}(α)$.\$()};
  \]
  compiles against~$Jₘ$ if an only if~$α$ is the language recognized by~$M$.
\end{Theorem}

The remainder of this section is dedicated to the proof of \cref{Theorem:SDPDA}.
In particular, we describe how~$J_M$ is constructed from the
  specification of~$M$.

We will use the following notation for
The stack contents~$⊥ξ₁ξ₁ξ₂ξ₁$,

\begin{figure}[H]
\begin{JAVA}
class Stack<Head extends ¢$Γ$¢, Rest extends Stack<?,?> > {¢¢
  Head head;
  Rest rest;
  Stack(Head head, Rest rest) {¢¢ this.head = head; this.rest = rest;}
  Rest pop() {¢¢ return rest; };
}
\end{JAVA}
For each state~$ξ∈Q$, we generate a \Java class~$ξ$,
\begin{JAVA}
class ¢$ξ$¢ <T extends S> extends S<T> {¢¢
  ¢$ξ$¢(T t) {¢¢ super(t); }
  // ¢…¢
}
\end{JAVA}
\end{figure}
In addition, we define a special class~$\vdash$ to designate the empty stack.
\begin{JAVA}
class ¢$\vdash$¢ extends Stack<¢$Γ$¢, ¢$\vdash$¢> {¢¢
  ¢$\vdash$ ¢() {¢¢ super(null); }
  ¢$\vdash$ ¢pop() {¢¢ throw new RunTimeException(); }
}
\end{JAVA}
The stack contents~$⊥ξ₁ξ₁ξ₂ξ₁$,
where~$ξ₁,ξ₂∈Q$ are stack elements,
is represented by the following type
\begin{JAVA}
  ¢$ξ₁$¢ < ¢$ξ₂$¢ < ¢$ξ₁$¢ < ¢$⊥$ > > >
\end{JAVA}
This is a general concept for implementing an unbounded stack with \Java's type system,
that will be extended in the future.


\subsection{Encoding states of pushdown automaton with \Java generics}
The generalized transition function \cref{Equation:generalized:transition}
  takes~$k+2$ arguments: a state~$q∈Q$, an input symbol~$σ∈Σ$
    and~$k$ stack elements drawn from~$Γ$.
To capture the full behavior of the automaton, the transition function
  must be able to pass along, in one way or another, the full contents of the stack.

Since the scheme described above can only be applied for binary functions,
  we shall pack together the state~$q$, the stack, and the~$k$ top most
  elements of it into a single type.
The next input symbol,~$σ∈Σ$, the remaining argument of function~$δ$,
  is encoded as a method.

The type part of the encoding is obtained by instantiating a generic type as follows:
Let~\cc{Q} be the abstract class that represents~$Q$, and let~\cc{q} be the concrete class that
  implements~\cc{Q} for an automaton state~$q∈Q$.
Then, to accommodate the extra~$k$ parameters of~$δ$, we add~$k$ generic parameters
  to class~$Q$ and to every class~\cc{q} that implements it.
Yet another such parameter is added for representing stack contents.

Consider the case~$k=2$.
\begin{figure}
  \begin{JAVA}
abstract class Q<S, ¢$γ₁$¢, ¢$γ₂$¢
  \end{JAVA}
\end{figure}

\begin{figure}
  \caption{\label{Figure:SDPDA:hierarchy}%
    Type hierarchy of the type encoding of a simple SDPDA.
  }
  \begin{adjustbox}{}
    \input ../Figures/automaton.tikz
  \end{adjustbox}
\end{figure}

\begin{figure}
  \caption{\label{Figure:SDPDA:example}%
    A type encoding of a simple pushdown automaton.
  }
  \javaInput[minipage,width=\linewidth]{spda.listing}
\end{figure}
As might be expected, this type is obtained
The
The

To encode this function within


\section{Notes on Practical Applicability}
\label{Section:applicability}
\Cref{Theorem:Gil-Levy} and its proof above provide
  a concrete algorithm for converting an EBNF specification of a fluent API into
its realization:
\begin{quote}
  \begin{enumerate}
    \item Convert the specification into a plain BNF form
    \urlref{http://lampwww.epfl.ch/teaching/archive/compilation-ssc/2000/part4/parsing/node3.html}.
    \item Convert this BNF into a definition of a DPDA. If conversion fails,
      then the given specification is not deterministic context-free.
    \item Convert this DPDA into a jDPDA. (Conversion is guaranteed to succeed.)
    \item Apply the proof to generate appropriate \Java type definitions, making sure to
        augment methods with code to maintain the fluent-call-list.
        Parsing the fluent-call-list can done either in each method,
        or lazily, when the product of the fluent API call chain is to
         be used.
  \end{enumerate}
\end{quote}
Although possible, a practical tool that uses the proof directly 
  is a challenge. 
Part of the problem is the complexity of the 
  algorithms used, some of which, e.g., the DPDA and jDPDA equivalance have never been 
  implemented.
Yet another issue that clients of compiler-compiler have grown to expect 
  facililities such as means for resolving ambiguities, manipulation 
  of attributes, etc.
Also, for a fluent API to be elegant and useful, 
  it should support method with parameters whose parameters are also defined by a  fluent API:
these two APIs may mutually recursive and even the same. 
Support of these features through four or so algorithmic abstractions 
  may turn out to be a decent engineering task.

\begin{wrapfigure}[6]r{27ex}
  \caption{\label{Figure:compiler} Encoding of an binary type tree}
  \javaInput[minipage,width=27ex,left=-2ex]{compiler.listing}
\end{wrapfigure}
Yet another challenge is controling the compiler's  
  runtime.
Learning that linear time parsers and lexical analyzers are possible, 
  and being accustomed to seeing these in practice, one 
  may expect the compiler would run in linear, or at least polynomial time. 
As it turns out, this time is exponential in the worst case (at least for \texttt{javac}).
An encoding of an S-expression in type~\cc{Cons} (\cref{Figure:compiler}) 
  is a not terribly complex such worst case.


\begin{wrapfigure}r{43ex}%
  \begin{minipage}{43ex}
  \caption{\label{Figure:compile-empiric} Compilation time
    (sec†{measured on an Intel i5-2520M CPU @ 2.50GHz~$⨉$4, 3.7GB memory, Ubuntu 15.04 64-bit, \texttt{javac} 1.8.0\_66}%
    ) \emph{vs.}
      length of call chain.
}
  \gnuplotloadfile[terminal=pdf,terminaloptions={crop size 2.5in,1.5in color enhanced font ",8" linewidth 1}]{../Figures/kill.gnuplot}
\end{minipage}
\end{wrapfigure}%
Type \cc{Cons} takes two type parameters, \cc{Car} and \cc{Cdr} (denoting left and right branches).
Denote the return type of \cc{d()} by \[
  τ= \cc{Cons< Cons<Car, Cdr>, Cons<Car, Cdr> >}.
\]
Let~$σ$ denote the type of the \kk{this} implicit parameter to~\cc{d}.
Now, since~$τ= \cc{Cons<}σ,σ\cc{>}$, we have~$|τ|≥2|σ|$,
  where the size of a type is measured, e.g., in number of characters in its textual representation.
Therefore, in a chain of~$n$ calls to \cc{d()}
\begin{equation}
  \label{Equation:n}
  \cc{(Cons<?,?>(null)).}\overbrace{\cc{d().}⋯\cc{.d()}}^{\text{$n$ times}}\cc{;}
\end{equation}
the size of the resulting type is~$O(2ⁿ)$.


\Cref{Figure:compile-empiric} shows, on the doubly logarithmic plane, the runtime (on a Lenovo X220)
of the \texttt{javac} compiler (version 1.8.0\_66) in face of a \Java program
  assembled from \cref{Figure:compiler} and \cref{Equation:n} placed as the
  single command of \cc{main()}.
Exponetial growth is demonstrated by the righthand side of the plot,
  in which curve converges on a straight line.
(In fact, a variation of the construction may lead to even super-exponential growth rate of the size of types.)

We believe that this exponential growth is due to a design flaw in the compiler.
Had the compiler used a representation of types that allows sharing of,
  of expression types, compilation time would be linear. 

Still, with current compiler technology, the type encoding scheme demonstated in \cref{Figure:A}
 might not be scaleable.




\section{Discussion and Future Work}
\label{Section:zz}
As should be obvious from \cref{Figure:fluent}, \SELF will be implemented
  in a bootstrapping fashion.
The specification of a BNF, is made using a fluent API.
The BNF for writing BNFs is given in \cref{Figure:BNF:BNF}

\begin{figure}[htbp]
  \scriptsize
  \begin{equation*}
    \def\<#1>{\/⟨\/\text{\textit{#1}}\/⟩\/~}
    \def\|{~|~}
    \let\oldCc=\cc
    \let\oldKk=\kk
    \def\cc#1{{\footnotesize\oldCc{#1}}~}
    \def\cc#1{{\footnotesize\olKk{#1}}~}
    \small
    \begin{aligned}
      \<BNF>              & ::=  \<Notation> \<Body> \<Footer> \hfill⏎
      \<Notation>         & ::=  \<Symbols> \<Terminals> \hfill⏎
      {}                  & \|  \<Terminals> \<Symbols> \hfill⏎
      \<Terminals>        & ::=  \cc{with(Symbols.class)}
      \<Symbols>          & ::=  \cc{with(Terminals.class)}
      \<Body>             & ::= \<Start> \<Rules> \hfill⏎
      \<Start>            & ::=  \cc{with(Class<? \kk{extends} Symbol)} 
      \<Rules>            & ::= \<First-Rule> \<More-Rules> \hfill⏎
      \<More-Rules>       & ::= \<Additional-Rule> \<More-Rules> \hfill⏎
      {}                  & \| ε \hfill⏎
      {}                  & \| \<Lowering-Visitor> \<Down-Visitors> \hfill⏎
      {}                  & \| ε \hfill⏎
      \<Up-Visitor>       & ::= \cc{male()} \cc{urinate()} \hfill⏎
      \<Down-Visitor>     & ::= \cc{female()} \<Action> \hfill⏎
                          & \| \cc{male()} \cc{defecate()} \hfill⏎
      \<Raising-Visitor>  & ::= \cc{male()} \cc{raise()} \cc{urinate()} \hfill⏎
      \<Lowering-Visitor> & ::= \cc{female()} \cc{lower()} \<Action> \hfill⏎
                          & \| \cc{male()} \cc{lower()} \cc{defecate()} \hfill⏎
      \<Activity>         & ::= \cc{urinate()} \hfill⏎
                          & \| \cc{defecate()} \hfill⏎
    \end{aligned}
  \end{equation*}
  \caption{A BNF grammar for the toilette seat problem}
  \label{Figure:BNF:BNF}
\end{figure}


%\textbf{Acknowledgment.}
%Inspiring correspondence with Gilad Bracha is gratefully acknowledged. 

\bibliographystyle{abbrv}\small
\bibliography{author-names,other-shorthands,%
  publishers-abbreviated,%
  conferences-abbreviated,%
  journals-abbreviated,journals-full,%
  yogi-book,yogi-practice,yogi-journal,%
  GPCE,OOPSLA,PLDI,USENIX,%
  00,yogi-confs}
\end{document}


Processing programming languages
\begin{description}
  \item[Lexical analysis] - the first step of the process in which the character strings generated by the
  programmer are aggregated to the abstract tokens defined by the language designer.
  \item[Syntactical analysis (parsing) ] - the second step, in which the processed strings of tokens
  conform to the rules of a formal grammar defined by the language's BNF (or EBNF).
  \item[Semantical analysis] - the next step, usually performed in unison with the previous step,
  in which the legal token sequences are given their semantic meaning.
\end{description}
Specifically, the proposal is that API design of follows the footsteps of
Accordingly, the designer of a fluent API has to follow these three conceptual
steps.
First is the identification of the \emph{vocabulary}, i.e.,
the set of method calls including type arguments that may take part in the
fluent API\@.
In this fluent API example
\begin{JAVA}
allowing (any(Object.class))
  ¢¢.method("get.*")
  ¢¢.withNoArguments();
\end{JAVA}
then, there are three method calls, and the vocabulary has three items in it.
\begin{itemize}
  \item~$ℓ₁ = \cc{any(Class<?>)}$
  \item~$ℓ₂ = \cc{allowing($ℓ₁$)}$
  \item~$ℓ₃ = \cc{method(String)}$
  \item~$ℓ₄ = \cc{withNoArguments()}$
\end{itemize}
