The pattern ‟invoke function on variable \cc{sb}”, specifically with
  a function named \cc{append}, occurs six times in the code in \cref{Figure:chaining}(a), designed
  to format a clock reading, given as integers hours, minutes and
  seconds.

\begin{figure}[H]
  \caption{\label{Figure:chaining}%
    Recurring invocations of the pattern ‟invoke function on the same
      receiver”, before, and after method chaining.
  }%
    \begin{tabular}{@{}cc@{}}%
  \begin{lcode}[minipage,width=44ex,box align=center]{Java}
String time(int hours, int minutes, int seconds) {¢¢
  final StringBuilder sb = new StringBuilder();
  sb.append(hours);
  sb.append(':');
  sb.append(minutes);
  sb.append(':');
  sb.append(seconds);
  return sb.toString();
}\end{lcode}
\hfill
&
\hspace{1ex}
  \begin{lcode}[minipage,width=44ex,box align=center]{Java}
String time(int hours, int minutes, int seconds) {¢¢
    return new StringBuilder()
      ¢¢.append(hours).append(':')
      ¢¢.append(minutes).append(':')
      ¢¢.append(seconds)
      ¢¢.toString();
}\end{lcode}
⏎
\textbf{(a)} before & \textbf{(b)} after
\end{tabular}
\end{figure}

Some languages, e.g., \Smalltalk offer syntactic sugar, called \emph{cascading},
  for abbreviating this pattern.
\emph{Method chaining} is a ‟programmer made” syntactic sugar serving the same purpose:
  If a method~$f$ returns its receiver, i.e., \kk{this},
  then, instead of the series of two commands: \mbox{\cc{o.$f$(); o.$g$();}}, clients can write
  only one: \mbox{\cc{o.$f$().$g$();}}.
  \cref{Figure:chaining}(b) is the method chaining
  (also, shorter and arguably clearer) version of
  \cref{Figure:chaining}(a).
It is made possible thanks to the designer of class \cc{StringBuilder} ensuring that 
  all overloaded variants of
  \cc{append} return their receiver.

The distinction between \emph{fluent API} and method chaining is the identity of the receiver:
In method chaining, all methods are invoked on the same object, whereas in fluent API
the receiver of each method in the chain may be arbitrary.
Fluent APIs are more interesting for this reason.
Consider, e.g., the following \Java code fragment (drawn from JMock~\cite{Freeman:Pryce:06})
\[
  \cc{allowing(any(Object.\kk{class})).method("get.*").withNoArguments();}
\]
Let the return type of function \cc{allowing} (respectively \cc{method}) be denoted by~$τ₁$
(respectively~$τ₂$).
Then, the fact that~$τ₁≠τ₂$ means that the set of methods that can be placed after the dot
in the partial call chain~$\cc{allowing(any(Object.\kk{class})).}$
is not necessarily the same set of methods that can be placed after the 
dot in the partial call chain~$\cc{allowing(any(Object.\kk{class})).method("get.*").}$.
This distinction makes it possible to design expressive and rich fluent APIs, in which a
sequence of ‟chained” calls is not only readable, but also robust, in the sense that the
sequence is type correct only when it makes sense semantically.

There is large body of research on \emph{type-states} 
(See e.g., review articles such
  as~\cite{Aldrich:Sunshine:2009,Bierhoff:Aldrich:2005}).
Informally, an object that belongs to a certain type, has
type-states, if not all methods defined in this object's class are applicable
to the object in all states it may be in.
As it turns out, objects with type states are quite frequent: a recent study~\cite{Beckman:2011} estimates
  that about 7.2% of \Java classes define protocols, that can be interpreted as type-state.

In a sense, type states define the ``language'' of the protocol of an object. 
The protocol of the type-state \cc{Box} class defined in \cref{Figure:box} 
  admits the chain \cc{\kk{new} Box().open().close()} but not the 
  the chain \cc{\kk{new} Box().open().open()}.

\begin{figure}[H]
  \caption{\label{Figure:box}Fluent API of a box object, defined by a DFA and a table}
  \begin{tabular}{cc}
    \hspace{7ex}\parbox[c]{40ex}{%
      \begin{tabular}[align=center]{m{7ex} | m{9ex} @{}| m{9ex}}
        & \cc{open()} & \cc{close()}⏎ \hline
        ‟closed”\ & \color{blue}{\emph{become ‟open”}} & \color{red}{\emph{runtime error}}⏎\hline
        ‟open” & \color{red}{\emph{runtime error}} & \color{blue}{\emph{become ‟closed”}}⏎
      \end{tabular}
    } &
    \hspace{-1ex}\parbox[c]{40ex}{\usetikzlibrary{automata,positioning,topaths}
\tikzstyle{state-style}=[state,every node={draw=black},font=\scriptsize,text width=5ex,align=center,on grid,node distance=13ex]
\begin{tikzpicture}

\node[state-style,accepting] (closed) {closed};
\node[state-style,accepting] (opened) [right=2.9 of closed] {opened};
\node[state-style,text width=4ex] (error) [above right=of closed,red] {runtime error};


\path[->,distance = 2ex,above] 

				(closed) edge[below] node {\cc{open()}} (opened)
				(opened) edge[bend left,below] node {\cc{close()}} (closed)
				(closed) edge[bend left,above left] node {\cc{close()}} (error)
				(opened) edge[bend right,right] node {\cc{open()}} (error);
\draw[<-] (closed) -- node[below left] {start} ++(-5ex,4ex);
\end{tikzpicture}}
    ⏎⏎
    \hspace{0ex}\textbf{(a)} Definition by table & \hspace{-2ex}\textbf{(b)} Definition by DFA
  \end{tabular}
\end{figure}

As mentioned above, tools such as fluflu realize
  type-state based on their finite automaton description.
Our approach is a bit more expressive: examine the language $L$ defined by the type-state, 
  e.g., in the box example,  
        \[
          L = \big(\cc{.open().close()}\big)^*\big(\cc{.open()}\:|\:ε\big).
        \]
If $L$ is deterministic context-free, a fluent API can be made for it. 

To make the proof concrete, consider this example of fluent API definition:
An instance of class \cc{Box} may receive two 
  method invocations: \cc{open()} and \cc{close()}, and can be in either 
  ‟open” or ‟closed” state.
Initially the instance is ‟closed”.
Its behavior henceforth is defined by \cref{Figure:box}.

To realize this definition, we need a type definition by which \cc{\kk{new} Box().open().close()}, more generally
  blue, or accepting states in the figure, type-check.
Conversely, with this type definition, compile time type error should occur in \cc{\kk{new} Box.close()},
  and, more generally, in the red state.

Some skill is required to make this type definition: proper design of class \cc{Box}, perhaps with
  some auxiliary classes extending it, an appropriate method definition here and there, etc.
\endinput
As mentioned above, the process of generating the Fluent API is as follows:

\begin{itemize}
  \item First, consider the language defined by the fluent API\@.
        In the box example, this language is defined by the regular expression
  \item Second, check whether this language is deterministic context-free.
        If it is, the fluent API can be realized, and,
        there is an algorithm to produce the respective type definition.
        In the box example, since language~$L$ is specified by a regular expression,
        it is trivially deterministic context-free.
\end{itemize}

The proof is a construction of a \Java type encoding of
  the \emph{deterministic pushdown automaton} that recognizes
  a given \emph{deterministic context-free language}.
With the generated types and methods, the compilation process of
  any chain of fluent API calls, actually runs the pushdown automaton against the
  specific input string that the chain represents.
When used appropriately, if this run of the automaton ends with an 
  accepting state†{The acceptance of a PDA can also be defined 
  by an empty stack, we will use the accepting state type of PDAs},
  then the fluent API chain type checks correctly.
If however this run ends in a failure, i.e., non-accepting state,
  compile time error will occur.

The Box example discussed above, is a classic example for type-states,
  since a box object can be in one of two states: ``opened'' or ``closed''.

Type-state pose two main challenges to software engineering:
\begin{enumerate}
  \item \emph{\textbf{Identification.}}
    In the typical case, type-state
        receive little to no mention at all in the documentation.
    The identification problem is to find the implicit
    type state in existing \Java: Given an implementation of a class
    (or more generally of a software framework),
    \emph{determine} which sequences of method calls are valid and which violate the
    type state hidden in the \Java.
  \item \emph{\textbf{Maintenance and Enforcement.}}
    Having identified the type-states, the challenge is in automatically flagging out
      illegal sequence of calls that does not conform
      with the type-state, furthermore, with the
      evolution of an API, the challenge is in updating the type-state information,
      and the type checking of \Java of clients.
\end{enumerate}
